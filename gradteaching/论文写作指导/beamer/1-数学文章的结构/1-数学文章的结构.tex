\documentclass[13pt]{ctexbeamer}


\usepackage{amsmath,amssymb,amsthm}             % AMS Math
\usepackage{graphicx}
\usepackage{epstopdf}
\usepackage{tikz}
\linespread{1.3}

\usepackage{mathrsfs}  %花写字母


%%%=== theme ===%%%
%\usetheme{Warsaw}
%\usetheme{Copenhagen}
%\usetheme{Singapore}
\usetheme{Madrid}
%\usefonttheme{professionalfonts}
%\usefonttheme{serif}
% \usefonttheme{structureitalicserif}
%%\useinnertheme{rounded}
%%\useinnertheme{inmargin}
\useinnertheme{circles}
%\useoutertheme{miniframes}
\setbeamertemplate{navigation symbols}{}
%\setbeamertemplate{footline}[page number]
\setbeamertemplate{footline}[frame number]


\titlegraphic{\includegraphics[width=2cm]{tjnu.jpg}}


\usepackage{color}
\definecolor{linkcol}{rgb}{0,0,0.4}
\definecolor{citecol}{rgb}{0.5,0,0}

\usepackage{xcolor}
\newcommand{\red}[1]{\textcolor{red}{#1}}
\newcommand{\blue}[1]{\textcolor{blue}{#1}}
\newcommand{\green}[1]{\textcolor{green}{#1}}


  \usepackage{graphicx}
  \DeclareGraphicsExtensions{.eps}
%   \usepackage[a4paper,pagebackref,hyperindex=true,pdfnewwindow=true]{hyperref}


\begin{document}



\title[]{论文写作指导 --- 数学文章的结构}
\author[]{{\large 张彪} }
\institute[]{{\normalsize
		天津师范大学\\[6pt]
		zhang@tjnu.edu.cn}}

\date{}



\AtBeginSection[]
{
\begin{frame}
	\frametitle{数学文章的结构}
	\tableofcontents[currentsection]
    \setcounter{tocdepth}{1}
\end{frame}
}



\begin{frame}
\maketitle
\end{frame}



\begin{frame}{第一章~数学文章的结构}
一篇数学论文一般由如下几个部分组成:
%\begin{itemize}
%\item {题目}
%\item 摘要
%\item 引言
%\item 主体
%\item 结论
%\item 致谢
%\item 文献
%\item 附录
%\end{itemize}
\setcounter{tocdepth}{1}
	\tableofcontents
\end{frame}


\section{题目}
\begin{frame}{题目}
题目是文章的一句广告语。
\begin{itemize}
	\item 题目不宜太长,也不宜太短。
	\item 题目应具体、明确,反映文章的主要贡献。
	\item 不要泛指一个过大的方向。
\end{itemize}
\end{frame}


\begin{frame}{题目}
	题目是文章的一句广告语。
	\begin{itemize}
		\item 题目不宜太长,也不宜太短。
		\item 题目应具体、明确,反映文章的主要贡献。
		\item 不要泛指一个过大的方向。
	\end{itemize}

题目是文章的脸面。文章的题目需:
\begin{itemize}
	\item 含有一定的信息量,使读者可以通过搜索引擎得到你的文章;
	\item 能够吸引读者的注意力;
	\item  言简意赅地表述文章的研究成果;
	\item 和现有的标题有一定的区别。
\end{itemize}
\end{frame}

\section{摘要}
\begin{frame}{摘要}
摘要是一篇有足够信息的微型文章。	文章的摘要应该:
	\begin{itemize}
		\item 概括文章的主要目的、思想和结果。
		\item 极可能简明扼要,但又要有足够的内容。
		\item 用词精确、意思明确,尽可能让更多的人读懂你的叙述。
	\end{itemize}

摘要是文章的心脏。它一般要回答
	\begin{itemize}
		\item 本文要干什么?本文涉及的问题是什么?
		\item 研究的问题如何解决?研究方法是什么?
		\item 主要的结果是什么?问题彻底解决还是部分解决了?
		\item 研究结果的意义?对科学或对读者有多大帮助?
	\end{itemize}

\end{frame}



\begin{frame}{1.英文摘要写作要素}
    \setlength{\baselineskip}{23pt}
    \begin{itemize}

        \item 摘要本身就是一篇\red{\textbf{minipaper}},概括文章的主要目的、思想、方法、过程和结果
        \item 英文摘要一般\red{\textbf{100—300}}词左右,尽可能简明扼要,但又要\red{\textbf{足够}}的内容,不需要描述细节
        \item 用词\red{\textbf{精准}}、意思明确,尽可能让\red{\textbf{更多的}}人读懂你的叙述
        \item 一般在文章初稿\red{\textbf{写完后}}才撰写,用与文章主体部分\red{\textbf{不同的}}语言结构来概括
    \end{itemize}
    \vspace{20pt}


\end{frame}

\begin{frame}{2.英文摘要常用句型}

    2.1摘要的开头
    \\
    \renewcommand{\baselinestretch}{1.3}
    多数摘要一开始便简明扼要地说明该研究的目的和范围,或者陈述写这篇文章的原因,有的摘要同时又指出该项研究的特点、结果和意义\\

    \begin{itemize}
        \item
        句型:\red{This paper} (report, thesis,work, presentation, document, account, etc.) \red{describes }(reports, explains, outlines, summarizes,
        documents, evaluates, surveys, develops, investigates, discusses,
        focuses on, analyzes, etc. )  \red{the results} (approach, role, framework,
        etc.) \red{of} …
        \item
        例句:\blue{This paper considers }the robust optimal reinsurance-investment strategy selection problem  \blue{with} price jumps and correlated claims for an ambiguity-averse insurer(AAI).

        \item
        译文:本文研究了模糊规避型保险公司的稳健最优再保险投资策略选择问题。

    \end{itemize}

    \vspace{15pt}

\end{frame}


\begin{frame}{2.英文摘要常用句型}


    \renewcommand{\baselinestretch}{1.3}

    \begin{itemize}

        \item
        This paper \red{has three main objectives} …
        \item
        This research project \red{is devoted to} …
        \item \red{Our goal} has been to \red{develop}…
        \item 例句:\blue{The objective of }each insurer \blue{is to } maximize the expected value that synthesizes the discounted utility of his surplus relative to a reference point.
        \item 译文:每个保险公司的目标是最大化其盈余相对于参考点的贴现效用的期望值。
    \end{itemize}

    \vspace{15pt}

\end{frame}
\begin{frame}{2.英文摘要常用句型}
    2.2摘要的过程
    \\
    \renewcommand{\baselinestretch}{1.3}
    这一部分,通常需要描述实验的方法和过程,特别要注重描述新方法。应采用一些连接手段,诸如词汇手段、语法手段和逻辑承接语等,使之前后连贯,过渡自然,易于理解。这一部分不宜写得太繁琐、太长。
    \begin{itemize}
        \item
        句型: \red{is calculated under the}…
        \item
        例句:We assume the reinsurance premium \blue{is calculated under the} generalized variance premium principle.
        \item
        译文:我们假设再保险保费是在广义方差保费原则下计算的。

    \end{itemize}

    \vspace{15pt}

\end{frame}
\begin{frame}{2.英文摘要常用句型}
    2.3摘要的结尾\\
    \renewcommand{\baselinestretch}{1.3}

    结果和结论是科学研究的结晶,是读者普遍关心的问题,对于各种新的发现应尽可能准确详尽地写出来。「结论」可以是这些「结果」的价值、用途与意义,也可以是各种假设或建议。\\文摘中的结果或结论常用下列句型表示。

    \begin{itemize}
        \item
        It has been \red{observed} (shown, proved , etc.) that…
        \item
        These experiments \red{indicate} (\red{reveal, show, demonstrate}, etc.) that…
        \item
        This strategy appeared to \red{be effective in}…
        \item
        These results \red{have direct application} to…
        \item
        This paper also \red{includes a comparison with} other…

    \end{itemize}

    \vspace{15pt}

\end{frame}
\begin{frame}{2.英文摘要常用句型}

    \renewcommand{\baselinestretch}{1.3}


    \begin{itemize}
        \item
        例句:	\blue{Some of our findings include}: the importance of taking advantage of mispricing for medium and long-term investment strategies.
        \item
        译文:我们的一些发现包括:利用中长期投资策略的错误定价的重要性。

    \end{itemize}

    \vspace{15pt}

\end{frame}
\begin{frame}{3.应注意事项}

    3.1摘要的符号使用\\
    \renewcommand{\baselinestretch}{1.3}
    尽量避免在摘要里出现数学符号,特别是一整行的数学公式。尝试\red{\textbf{用叙述性的语言来总结}}你的文章摘要,这对数学的文章尤其重要。\vspace{0.6em}



    \vspace{15pt}

\end{frame}
\begin{frame}{3.应注意事项}
    3.2摘要的人称\\

    \renewcommand{\baselinestretch}{1.3}
    绝大多数英文摘要都使用\red{\textbf{第三人称}}(但不使用 he 或 she),间或出现 the author(s),the writer(s) 和第一人称 we,即使原文献作者只有 1 人,也宜用 \red{\textbf{we}} 而不宜用I\vspace{0.8em}

    \begin{itemize}
        \item
        例句:\red{We} discuss optimal proportional reinsurance–investment problems for an insurer with mispricing and model ambiguity under a complex stochastic environment. \red{ The surplus process }is described by a classical Cramér--Lundberg (C--L) model and the financial market contains a pair of mispriced stocks, a risk-freeasset, and a market index.


    \end{itemize}

    \vspace{15pt}

\end{frame}
%\begin{frame}{3.应注意事项}
%    3.3摘要的语态\\
%
%    \renewcommand{\baselinestretch}{1.3}
%    目前各国出版的英文摘要中,相当一部分用被动语态写成,以减少主观因素,增强客观性。如:
%    \vspace{0.8em}
%    \begin{itemize}
%        \item
%        例句:	Energy balance concepts \red{were used to} determine the amount of energy
%        lost due to damping in a run-arrest fracture event. Possible sources
%        of damping \red{were identified }and experiments \red{were conducted} to
%        determine their relative contribution to the overall damping.
%
%
%    \end{itemize}



\end{frame}


\section{关键词和学科分类}
\begin{frame}{关键词和学科分类}
	关键词keywords
	\begin{itemize}

		\item 关键词的设置是为了让数学论著归档服务机构(如美国《数学评论》)能把你的论文放到其门类齐全的数据库中的正确位置。

		\item 科学引用指标检索工具(SCI)利用关键词给文章分门别类,而读者根据关键词检索与它研究或兴趣相关的文献。
		\item 最好只写单数形式,但是写成复数也无伤大雅
	\end{itemize}
\end{frame}


\begin{frame}{关键词和学科分类}
	数学学科分类 Mathematics Subject Classification  (MSC)
	\begin{itemize}

		\item 分类系统是由美国《数学评论》Math Reviews(MR)和德国《数学文摘》(zbMath)共同制定的。2010,2020

		\item https://mathscinet.ams.org/mathscinet/msc
	\end{itemize}
\end{frame}


\section{引言}
\begin{frame}{引言}
	引言应该比较全面、准确并且客观地介绍文章中将要讨论
	\begin{itemize}
		\item 问题的背景材料和简要发展,
		\item 人们对此已做的相关贡献,
		\item 写这篇文章的动机,
		\item 本文的主要结果。
	\end{itemize}
需要注意:
	\begin{itemize}
	\item 引言的客观性非常重要,切记自我吹嘘。\\
	~~~~~~~~~~~~``文章在自己~~评价在别人''~~~~(华罗庚)
	\item 不要随意贬低别人的研究。
	\item 不要把别人的甚至自己以前的文章句子原封不动地直接移植过来。
\end{itemize}
\end{frame}

\begin{frame}{1~引言的开首}
		\begin{itemize}
		\item ``好的开始是成功的一半''

		\item 最糟糕的开头可能是先给出一堆数学符号和定义。

		\item  最好的引言应该这样开头:
	\begin{itemize}
		\item 先做一些较为通俗浅显的描述,这样会给读者一个容易的起头,以便他们轻松自如、有条不紊地登堂入室,进入角色,并能提高继续读下去的兴趣。
		\item  把你所关心的问题提出来,开门见山、直奔主题,不要不着边际地叙述与文章的中心关系不大的事物。要用直截了当、浅显易懂的语句把问题指出来。
	\end{itemize}
	\end{itemize}
\end{frame}



\begin{frame}

    \begin{itemize}
        \item
        {\small  Title: Optimal multigrid preconditioning}


        {\small Introduction: Among various techniques for solving partial differential equations,  multigrid methods have proven to be one of the most efficient approaches. The efficiency of those methods, however, depends crucially on appropriate underlying multilevel structures. As such multilevel structures are not naturally  available in most unstructured grids, multigrid methods are in general not easy to apply.}

        \pause
        \item {\small  译文 \quad  题目:最优多重网格预处理}

        {\small  引言:在解偏微分方程的各种技术中,多重网格法已被证明是最有效的处理方法之一。然而,这些方法的有效性本质上依赖于合适的基础多层结构。因为在大部分非结构化网格中这样的多层结构不能自然获得,多重网格法一般来说不易使用。}
        \item 这样一开头三两句话就把所要研究的问题提出来了,而且指出了不足之处(困难)---所要改进发展的方面。
    \end{itemize}
\end{frame}

\begin{frame}{引言的开头}

    \begin{itemize}
        \item 句型:In recent years
        \item 例句:\red{In recent years},there has been tremendous interest in developing.
        \item 翻译:近年来在发展...方面已经展现了极大的兴趣。
        \newline
        \item 句型:be concerned with
        \item 例句:We \red{are concerned} in this paper \red{with}  Spectral Collocation Method
        \item 翻译:本文我们关注谱方法配置法。

    \end{itemize}
\end{frame}

\begin{frame}{引言的开头}

    \begin{itemize}
        \item 句型:a controversial issue
        \item 例句:Efficient interpolation method is still  \red{a controversial issue} in the computational Mathematics community.
        \item 翻译:高效的插值方法在计算数学的研究群体中依然是一个争论之处。
        \newline
        \item 句型:explore ... problem
        \item 例句:In this paper,we \red{explore} the following nonlinear singular two-point boundary value \red{problem}.
        \item 翻译:本文研究了如下非线性奇异两点边值问题。
    \end{itemize}
\end{frame}


\begin{frame}{2~引言的中间}
	\begin{itemize}

		\item 定义问题

		\item 解释准备解决什么问题
		\item 总结以前取得的主要成果及不足之处
		\item 简介一下你解决问题主要的手法
		\item 指出写作的动机和目的是必要的。
	\end{itemize}
\end{frame}



\begin{frame}{引言的中间}

\begin{itemize}
    \item
    引言的主体大概包括以下几点:
    \item
    1.定义问题  2.阐述想要解决的问题  3.总结前人的主要贡献以及不足之处  4.介绍自己解决问题的手法
    \item
    这一部分也不要东拉西扯,回顾部分主要讲与本文密切相关的工作和成果,特别应提及下文中将要用到的结果和方法。指出写作的动机和目的,以前的研究的不足和遗漏、方法的处理不够合理有效都可以是写作的理由。
\end{itemize}
\end{frame}

\begin{frame}{引言的中间}

\begin{itemize}
    \item
    例:...However,in spite of quite a number of contributions dealing with these effects,there are no calculations taking into account all influences.To fill this gap, we present a highly accurate numerical method...
    \item 译文: ...然而,尽管在处理这些效果上有相当多的贡献,没有计算把所有影响考虑在内。为填补这些空白,我们推出一个高精度数值方法...

\end{itemize}
\end{frame}

\begin{frame}{引言的中间}

\begin{itemize}
    \item 句型:be limitations to
    \item 例句:There \red{are} some \red{limitations to} this approach to...
    \item 翻译:例如,这在求解....方便已证明了很成功。
    \newline
    \item 句型:the first...the other...
    \item 例句:The objective of this paper is twofold.\red{The first} one is to improve accuracy,\red{The other} is to save time.
    \item 翻译:本文有双重目的,一是提高精度,另一是节省时间。

\end{itemize}
\end{frame}

\begin{frame}{引言的中间}

\begin{itemize}
    \item 句型:besides...also...
    \item 例句:\red{Besides} numercial methods,many semi-analytical methods were \red{also} developed to obtain approximate series solutions.
    \item 翻译:除了数值方法外,还发展了许多半解析方法来获得近似级数解。
    \newline
    \item 句型:As stated above
    \item 例句:\red{As stated above}, the methods for solving singular differential equations usually fall into two categories
    \item 翻译:如上所述,求解奇异微分方程的方法通常分为两类

\end{itemize}
\end{frame}



\begin{frame}{3~引言的结尾}
    \begin{itemize}

        \item 简单叙述文章的组成部分

        \item 对于文章后面的每一节内容写上一句话,概括地告诉读者这一节是干什么的
        \item 用不同的语句给出每一节大意一个简单的小结。
    \end{itemize}
\end{frame}
\\


\begin{frame}{引言的结尾}

\begin{itemize}
    \item
    在引言部分的结尾处一般会简单地叙述本文的的组成部分,一般的做法是对于文章引言后面的每一节的内容写上一句话,概括地告诉读者这一节的目的,一般以:
    \item
    - An outline of this paper is as follows \newline 文章的概要如下\newline或者是\newline-This paper is organized as follows\newline 文章组织如下
    \item
    此部分不应该把题目简单的罗列,最好使用不同的语句概述每一节的大意
\end{itemize}
\end{frame}

\begin{frame}{引言的结尾}

\begin{itemize}
    \item
    例:The remainder of the paper is organized as follows. In Sect. 2, we reformulate the original equation as an equivalent Volterra–Fredholm equation and further translate it into an equation defined on the interval (−1, 1). The existence and uniqueness theorems of the reformulation are also introduced there. In Sect. 3, we introduce some basic properties of the Legendre/Jacobi polynomial interpolations and propose the Legendre spectral collocation method for the reformulated nonlinear Volterra–Fredholm equation (2.6). In Sect. 4, we derive the error bounds of the Legendre collocation method for smooth solutions in the function spaces L2(0,T) and L∞(0,T), respectively. Our theoretical results are verified by the numerical experiments in Sect. 5.

\end{itemize}
\end{frame}

\begin{frame}{引言的结尾}

\begin{itemize}
    \item 译文:论文的其余部分组织如下。第2节我们将原始方程重新表述为等价的Volterra–Fredholm方程,并进一步将其转化为定义在区间(−1, 1)上的方程.文中还介绍了重新表述的存在性和唯一性定理。第3节我们介绍了Legendre/Jacobi多项式插值的一些基本性质,并提出了新的非线性Volterra–Fredholm方程的Legendre谱配置方法。第4节我们推导了函数空间L2(0,T)和L∞(0,T)中光滑解的勒让德配置法的误差界∞(0,T)。第5节中我们的理论结果得到了数值模拟的验证。

\end{itemize}
\end{frame}

\begin{frame}{引言的结尾}

\begin{itemize}
    \item 句型:be organized as
    \item 例句:The remainder of the paper \red{is organized as} follows.
    \item 翻译:论文的其余部分组织如下。
    \newline
    \newline
    \item 句型:reformulate...as...
    \item 例句:we \red{reformulate} the original equation \red{as} an equivalent Volterra–Fredholm equation
    \item 翻译:我们改写了原文作为等效Volterra–Fredholm方程的方程。

\end{itemize}
\end{frame}

\begin{frame}{引言的结尾}

\begin{itemize}
    \item 句型:as an example
    \item 例句:We discuss the convergence of the Chebyshev collocation method by taking the corresponding linear equation \red{as an example}.
    \item 翻译:以相应的线性方程为例,讨论了切比雪夫配置法的收敛性。
    \newline
    \newline
    \item 句型:concise conclusion
    \item 例句:We end with a \red{concise conclusion} in Sect. 7.
    \item 翻译:我们以第7节中的简明结论结束。

\end{itemize}
\end{frame}

\begin{frame}{总结}
引言是文章的四肢,应该注意:
\begin{itemize}
    \item 为什么现在要研究此问题?
    \item 为什么是这个问题?
    \item 为什么用你提出的方法?
    \item 为什么读者会对你的方法或结果感兴趣?
    \end{itemize}

\end{frame}


\section{主体}

\begin{frame}{主体}




	Preliminaries(预备知识)
	\begin{itemize}
		\item  引进符号说明和概念定义。
		\item  有些已知的但对这篇文章有用的结果也需要单独引出来。这样做是为下一节作准备,并能区别什么是你文章的主要新结论,什么是以前的结果。
	\end{itemize}
	Main results(主要结果)
		\begin{itemize}
		\item  1-2个主要定理,最多3-4个
		\item 定理的叙述中要包含完整的条件和结论
	\end{itemize}

Proofs(证明)
\end{frame}


	\setcounter{tocdepth}{2}
\frame{\tableofcontents[currentsection]}
%\frame{\tableofcontents[sectionstyle=shaded , currentsection]}

\subsection{文章的章节标题}
\begin{frame}{文章的章节标题}
章节的标题是文章的骨骼。

对读者来说,文章的结构或小标题

\begin{itemize}
    \item  使读者更容易找到他们感兴趣的部分;
\item 使读者更容易找到作者的主要贡献或文章的亮点;
\item 使读者从逻辑上更易理解作者的写作意图。
\end{itemize}


而对作者来说, 章节小标题
\begin{itemize}
    \item  对相应的章节起到画龙点睛的作用;

\item 为自己写作前提供好的提纲;

\item 把文章划分成相对独立的部分使目的性强的读者直奔所需。
\end{itemize}
\end{frame}




\subsection{定义}
\begin{frame}{定义}
	\begin{itemize}
		\item
		制定一个定义时要考虑三个问题

		``\blue{为什么?}''

		``\blue{放哪里?}''

		``\blue{怎么样?}''
		\item
		首先,自问为什么你要做一个定义: 它是必须的么?
		\item
		不恰当地定义会使表达复杂化并且太多定义会压垮读者,因此设想自己正在为每一个定义支出一大笔花销是明智的。
		\item
		在给定的学科领域中符号是标准的,需要判断定义是否要给出。
		\item
		用多余的单词可以避免潜在的混淆。
	\end{itemize}
\end{frame}


\begin{frame}
	\begin{itemize}
		\item  第二个问题是``\blue{放哪里?}''
		\item 不推荐在一篇文章刚开始的地方放置一长串的定义。
		\item  一个定义应该被放置在该属于首次使用的位置。
		\item 如果定义较早给出,读者将不得不往回查找,这可能失去集中力(或者更糟糕,失去兴趣)。
		\item 尽量缩短定义和它首次被使用位置之间的距离。
	\end{itemize}
\end{frame}


\begin{frame}
	\begin{itemize}
		\item  为了强化几页没有使用的数学符号,你可以重复定义。

		\item 例如,``最佳步长$\alpha^{*}$如下''。

		\item  这个含蓄的再定义或者提醒了读者$\alpha^{*}$是什么, 或者再次确保他们已经准确地记住了它。
	\end{itemize}
\end{frame}


\begin{frame}
	\begin{itemize}
		\item  最后,一个术语如何被定义?
		\item 有可能只有唯一的定义或者有多种可能。
		\item 你应该以一个简短地、用基本性质或者基本思想的术语描述的,并且与相关定义相容的定义为目标。
		\item 举个例子,正规矩阵的标准定义为“矩阵$A\in \mathbb{C}^{n\times n}$ 满足$AA^{*}= A^{*}A$”。 至少有70种不同的方式定义正规性,但是没有一种比$AA^{*}= A^{*}A$简单容易。
	\end{itemize}
\end{frame}


\begin{frame}
	\begin{itemize}
		\item 依照惯例,在定义中if 的意思是 if and only if , 所以不要写

		``The graph is connected  \alert{if and only if }~there is a path from every node in $G$ to every other node in $G$''.

		而写成

		``The graph is connected \alert{if}  there is a path from every node in $G$ to every other node in $G$''.

	\end{itemize}
\end{frame}

\begin{frame}
	\begin{itemize}
		\item 把要定义的文字排版成斜体是惯例:

		``The graph is \emph{connected} {if}  there is a path from every node in $G$ to every other node in $G$~''.

		\item 这种强调也可以写为

		``The graph is defined to be {connected} {if}  \dots''

		或者

		``The graph is said to be {connected} {if}  \dots''
	\end{itemize}

\end{frame}


\subsection{例子}

\begin{frame}{例子}
	\begin{itemize}
		\item
		适应于技术写作(从教学到研究)所有形式的教学策略是在讨论一般情况前都会先讨论\blue{特殊例子}。
		\item
		尤其对于数学家来说,采取反证法是有吸引力的,但是\blue{以例子开头来解释说明是更有效的方法}。
%		\item
		能说明如何\blue{以特定实例开头}的一个很好的例子是Strang的《应用数学导论》第一章内容:

		{\small
			应用数学中最简单的模型是线性方程组,也是目前为止最重要的模型,我们以一个极为简单的例子开始本书的内容:
			\begin{align*}
				2 x_1+ 4 x_2& =2,\\
				4 x_1 +11x_2 & =1.
			\end{align*}
			在一些进一步的开场白之后, Strang继续详细地研究这个$2\times 2$矩阵和一个特殊的$4\times 4$矩阵。仅在数页之后就给出了一般的$n \times n$矩阵。}

		\item 课本上的\alert{练习题}是例子的一种形式。
	\end{itemize}


\end{frame}




\subsection{定理}
\begin{frame}{定理是什么?}


\begin{itemize}
	\item 定理
	\item 引理
	\item 命题
	\item 猜想
	\item 假设
\end{itemize}
\end{frame}

\begin{frame}
\begin{itemize}
\item \blue{定理}、\blue{引理}、\blue{命题}之间的差别是什么?
\item 在某种程度上,答案取决于结论在行文中的位置。
\item
一般地,\blue{定理}是一个具有独立意义的重要结论。
定理的证明通常是非平凡的。


\end{itemize}
\end{frame}




\begin{frame}{引理}

\begin{itemize}
	\item
	\blue{引理}是一个辅助的结果——是迈向定理的一个跳板。
	其证明可能容易也可能困难。

\item 一个明确的独立的值得概括但是不值得冠以定理头衔的结论也可被称为\blue{引理}。
事实上,有很多著名的引理。
\item
	一个结论是否应该被正式地规定为引理或者只是在文中简单地提及取决于你写作的\red{等级}。
\item
\alert{把所有结论都标注为定理是不可取的},因为这样做就无法突出你文章的逻辑结构,
而且也无法将读者的关注点指向最重要的结论。
\item 如果你对一个结论是称为引理还是定理有所迟疑,那就称为\blue{引理}。
\end{itemize}


在一个线性代数的研究论文中,给出一个叙述``对称正定矩阵的特征值是正的'' 的引理是不恰当的,
因为这个权威的结论是众所周知的。

\end{frame}


\begin{frame}{命题}
\begin{itemize}
	\item
\blue{命题}比引理和定理使用范围要小并且他的意义也不尚明确。
它倾向于表示一个次要定理的方法。
\item 讲义和教科书的作者可能认为他温文尔雅的名字是的它比定理显得不那么令人生畏。
 \item 然而,命题并不是学生认为的“一个可能不正确的定理”。
\end{itemize}
\end{frame}

\begin{frame}{推论}
\begin{itemize}
\item \blue{推论}是引理、定理或者命题的直接或很容易推出的结论。
\item 区分推论和一个结论的延伸或者概括是非常重要的。
\item 当心不要过度颂扬一个推论,如给它错误的标注(定理),因为这将是它不恰当地得以突出而且会混淆主要结论(主定理)的地位。
\end{itemize}
\end{frame}

\begin{frame}
\begin{itemize}
\item 有多少结论被正式陈述为引理、定理、命题和推论是个人风格的问题。
\item 一些作者用一系列赋有定义以及评注的结论和定理来阐述他们的观点。
\item 另一种极端是,一些作者会极少正式地陈述结论。
\end{itemize}
\end{frame}

\begin{frame}{猜想}
\begin{itemize}
\item 第五种数学写作声明是\blue{猜想}。
\item 作者认为可能正确但是未曾被证明或反驳的陈述。
\item 作者通常会为了证实陈述的真实性写出一些强有力的证据。
\item  一个著名猜想的例子就是哥德巴赫猜想(1742年),每一个大于2的偶数都可写成两个质数之和,它仍未被证明。
\item 提出一个猜想之后再反驳不一定是个坏事:确定这个猜想旨在回答的问题可能会是一个重要贡献。


\item 一个假设是进一步猜想的基础。
\item 站在自己立场上的假设是罕见的,如黎曼假设和连续统假设。
\end{itemize}
\end{frame}


\subsection{证明}

\begin{frame}{证明}
	\begin{itemize}
		\item 读者更想要知道大纲和关键的想法。
		\item 读者更希望学习一种可以应用在其他情况下的技术或原理。
		\item 当读者要详细研究证明的细节时,他们自然想花费最少的精力去了解它。
		\item 为了在这两种情况下帮助读者,强调一下
		\begin{itemize}
		\item \blue{证明的结构}
		\item \blue{每一步的难易}
		\item \blue{使证明成立的关键想法}
		\end{itemize}
		很重要。

	\end{itemize}
\end{frame}




\section{结论}
\begin{frame}{结论}

	\begin{itemize}
		\item  非常简短地叙述文章的主要贡献
		\item  解释你在引言里提到的问题和疑问。经过整篇文章的论证,也许可以谈论部分答案了
		\item 指出由于某些原因,文章没有考虑的其他方面或更广泛的问题,并说明本文的论证方法是否可以推广到这些情形
		\item  展望下一步,后续研究可以做什么?
		\item 如果合适,讨论一些和本研究相关的猜想
	\end{itemize}
\end{frame}

\section{致谢}
\begin{frame}{致谢}

	\begin{itemize}
		\item  致谢的对象是主要包括给文章写作提供过意见或者帮助的人或机构
		\item  很多研究基金规定获资助者在所发表的\alert{相关}论文中必须清楚表明获其资助,有时需列出所获资基金的号码。
		\item 不必提及匿名审稿人。但如果审稿人对改进文章的质量有很大的共享,应向他们致谢。
	\end{itemize}
\end{frame}

\section{参考文献}
\begin{frame}{参考文献}

	\begin{itemize}
		\item  文献引用
		\item  文献格式
	\end{itemize}
\end{frame}

\section{附录}
\begin{frame}{附录}

	\begin{itemize}
		\item  附录记载的主要是作者不想放在正文内的冗长复杂的定理证明。
		\item  另一种附录防止的是正文定理证明中所要的的一些预备证明或一些标准引理,甚至一些有关概念的定义等。
		\item 附录可以收入任何作者不便或不拟放在文章不要部分、但有不想随意丢掉的东西。
	\end{itemize}
\end{frame}




\end{document}