% !Mode:: "TeX:UTF-8"
\documentclass[fontset=mac]{ctexbeamer}
\usepackage[utf8]{inputenc}


\usepackage{amsmath,amssymb,amsthm}             % AMS Math
%\usepackage[T1]{fontenc}
\usepackage{graphicx}
\usepackage{epstopdf}
\usepackage{tikz}
\linespread{1.3}
\usepackage{amsfonts}


\usepackage{mathrsfs}  %花写字母
 

%%%=== theme ===%%%
% \usetheme{Warsaw}
%\usetheme{Copenhagen}
%\usetheme{Singapore}
\usetheme{Madrid}
%\usefonttheme{professionalfonts}
%\usefonttheme{serif}
% \usefonttheme{structureitalicserif}
%%\useinnertheme{rounded}
%%\useinnertheme{inmargin}
\useinnertheme{circles}
%\useoutertheme{miniframes}
\setbeamertemplate{navigation symbols}{}
%\setbeamertemplate{footline}[page number]
\setbeamertemplate{footline}[frame number] 


\titlegraphic{\includegraphics[width=2cm]{tjnu.jpg}} 


%\usepackage[fontset=mac]{ctex}
%\usepackage{ctex}
\usepackage{xcolor}
\newcommand{\red}[1]{\textcolor{red}{#1}}
\newcommand{\blue}[1]{\textcolor{blue}{#1}}
\newcommand{\green}[1]{\textcolor{green}{#1}}



\setbeamertemplate{theorems}[numbered]
\newtheorem{thm}{定理}
\newtheorem{lem}{引理}
\newtheorem{exa}{例}
\newtheorem*{theo}{定理}
\newtheorem*{conj}{猜想}
\newtheorem*{defi}{定义}
\newtheorem*{coro}{推论}
\newtheorem*{ex}{练习}
\newtheorem*{rem}{注}
\newtheorem*{prop}{性质}
\newtheorem*{qst}{问题}

\def\qed{\nopagebreak\hfill{\rule{4pt}{7pt}}\medbreak}
\def\pf{{\bf 证明~~ }}
\def\sol{{\bf 解~~ }}



\def\R{\mathbb{R}}
\def\Rn{\mathbb{R}^n}
\def\A{\mathscr{A}}
\def\B{\mathscr{B}}
\def\D{\mathscr{D}}
\def\E{\mathscr{E}}
\def\O{\mathscr{O}}

\def\rank{\operatorname{rank}}
\def\dim{\operatorname{dim}}
\def\0{\mathbf{0}}
\def\a{\alpha}
\def\b{\beta}
\def\r{\gamma}


\begin{document}



\title[]{论文写作指导}
\author[]{{\large 张彪} }
\institute[]{{\normalsize
		天津师范大学\\[6pt]
		zhang@tjnu.edu.cn}}

\date{}



\AtBeginSection[]
{
\begin{frame}
	\frametitle{Outline}
	\tableofcontents[currentsection]
\end{frame}
\setcounter{exa}{0}
\setcounter{equation}{0}
}



\begin{frame}
\maketitle
\end{frame}


\section{定理是什么?}
\begin{frame}{定理是什么?}

	
%\begin{itemize}
%	\item 定理
%	\item 引理
%	\item 命题
%	\item 猜想
%	\item 假设
%\end{itemize}
%\end{frame}
%
%
%\begin{frame}
\begin{itemize}
\item \blue{定理}、\blue{引理}、\blue{命题}之间的差别是什么?
\item 在某种程度上,答案取决于结论在行文中的位置。
\item 
一般地,\blue{定理}是一个具有独立意义的重要结论。
定理的证明通常是非平凡的。


\end{itemize}



%在一个线性代数的研究报告中,给出一个叙述``对称正定矩阵的特征值是正的'' 的引理是不恰当的, 
%因为这个权威的结论是众所周知的。
\end{frame}


\begin{frame}{引理}
	
\begin{itemize}
	\item 
	\blue{引理}是一个辅助的结果—是迈向定理的一个跳板。
	其证明可能容易也可能困难。
	
\item 一个明确的独立的值得概括但是不值得冠以定理头衔的结论也可被称为\blue{引理}。
事实上,有很多著名的引理。
\item 
	一个结论是否应该被正式地规定为引理或者只是在文中简单地提及取决于你写作的\red{等级}。
\item 
\alert{把所有结论都标注为定理是不可取的},因为这样做就无法突出你文章的逻辑结构,
而且也无法将读者的关注点指向最重要的结论。
\item 如果你对一个结论是称为引理还是定理有所迟疑,那就称为\blue{引理}。
\end{itemize}
\end{frame}


\begin{frame}{命题}
\begin{itemize}
	\item 
\blue{命题}比引理和定理使用范围要小并且他的意义也不尚明确。
它倾向于表示一个次要定理的方法。
\item 讲义和教科书的作者可能认为他温文尔雅的名字是的它比定理显得不那么令人生畏。
 \item 然而,命题并不是学生认为的“一个可能不正确的定理”。
\end{itemize}
\end{frame}

\begin{frame}{推论}
\begin{itemize}
\item \blue{推论}是引理、定理或者命题的直接或很容易推出的结论。
\item 区分推论和一个结论的延伸或者概括是非常重要的。
\item 当心不要过度颂扬一个推论,如给它错误的标注,因为这将是它不恰当地得以突出而且会混淆原结论的地位。
\end{itemize}
\end{frame}

\begin{frame}
\begin{itemize}
\item 有多少结论被正式陈述为引理、定理、命题和推论是个人风格的问题。
\item 一些作者用一系列赋有定义以及评注的结论和定理来阐述他们的观点。
\item 另一种极端是,一些作者会极少正式地陈述结论。
\end{itemize}
\end{frame}

\begin{frame}{猜想}
\begin{itemize}
\item 第五种数学写作声明是\blue{猜想}。
\item 作者认为可能正确但是未曾被证明或反驳的陈述。
\item 作者通常会为了证实陈述的真实性写出一些强有力的证据。
\item  一个著名猜想的例子就是哥德巴赫猜想(1742年),每一个大于2的偶数都可写成两个质数之和,它仍未被证明。
\item 提出一个猜想之后再反驳不一定是个坏事:确定这个猜想旨在回答的问题可能会是一个重要贡献。

\end{itemize}
\end{frame}

\begin{frame}
\begin{itemize}
\item 一个假设是进一步猜想的基础,通常程序现在证明中——例如,归纳假设。
\item 站在自己立场上的假设是罕见的,如黎曼假设和连续统假设。
\end{itemize}
\end{frame}


\section{证明}

\begin{frame}
	\begin{itemize}
		\item 读者更想要知道大纲和关键的想法。
		\item 读者更希望学习一种可以应用在其他情况下的技术或原理。
		\item 当读者要详细研究证明的细节时,他们自然想花费最少的精力去了解它。
		\item 为了在这两种情况下帮助读者,强调一下
		\begin{itemize}
		\item \blue{证明的结构}
		\item \blue{每一步的难易}
		\item \blue{使证明成立的关键想法}
		\end{itemize}
		很重要。
		
	\end{itemize}
\end{frame}


\begin{frame}
下面是一些可以用的各种各样的短语例子。
\begin{itemize}
\item The aim/idea is to 
\item Our first goal is to show that 
\item Now for the harder part.
\item The trick of the proof is to find
\item \dots ~is the key relation.
\item The only, but crucial use of  \dots~ is that 
\item To obtain \dots ~a little manipulation is needed.
\item THe essential observation is that 
\end{itemize}
\end{frame}


\begin{frame}
当你省略一个证明的一部分是,最好是通过短语表示省略地方的性质和长度,有如下内容:
	\begin{itemize}
		\item It is easy/simple/straightforward to show that 
		\item Some tedious manipulation yields
		\item An easy/obvious induction gives 
		\item After two applications of  \dots ~we find
		\item An argument similar to the one used in  \dots ~shows that 
	\end{itemize}
\end{frame}

\begin{frame}
你也应该努力让读者了解你证明到了那里以及还剩下什么需要证明。有用的短语有
	\begin{itemize}
		\item First, we establish that 
		\item Our task is now to 
		\item Our problem reduces to 
		\item It remains to show that 
		\item We are almost ready to invoke 
		\item Finally, we have to show that 
	\end{itemize}
\end{frame}

\begin{frame}
\begin{itemize}
\item 
一个证明的结束通常用哈尔莫斯符号$\square$标记的。
\item 
有时用缩写QED  (拉丁语: quod erat demonstrandum = 这就是被证明了)代替。
\end{itemize}
\end{frame}


\section{例子的作用}

\begin{frame}
%	{例子的作用}
\begin{itemize}
	\item 
适应于技术写作(从教学到研究)所有形式的教学策略是在讨论一般情况前都会先讨论\blue{特殊例子}。
\item 
尤其对于数学家来说,采取反证法是有吸引力的,但是\blue{以例子开头来解释说明是更有效的方法}。
\item 
能说明如何\blue{以特定实例开头}的一个很好的例子是Strang的《应用数学导论》第一章内容:

{\small
应用数学中最简单的模型是线性方程系统,也是目前为止最重要的模型,我们以一个极为简单的例子开始本书的内容:
\begin{align*}
	2x_1+4x_2& =2,\\
	4x_1+11x_2 & =1.
\end{align*}
在一些进一步的开场白之后, Strang继续详细地研究这个$2\times 2$系统和一个特殊的$4\times 4$系统。仅在数页之后就给出了一般的$n \times n$系统。}

\item 课本上的\alert{练习题}是例子的一种形式。
\end{itemize}


\end{frame}


\section{定义}
\begin{frame}{定义}
\begin{itemize}
\item 
制定一个定义时要考虑三个问题``\blue{为什么?}''  ``\blue{放哪里?}''   ``\blue{怎么样?}'' 
\item 
首先,自问为什么你要做一个定义: 它是必须的么?
\item 
不恰当地定义会使表达复杂化并且太多定义会压垮读者,因此设想自己正在为每一个定理支出一大笔花销是明智的。
\item 
在给定的学科领域中符号是标准的,需要判断定义是否要给出。
\item 
用多余的单词可以避免潜在的混淆。
\end{itemize}
\end{frame}


\begin{frame}
\begin{itemize}
	\item  第二个问题是``\blue{放哪里?}'' 
	\item 不推荐在一篇文章刚开始的地方放置一长串的定义。
	\item  一个定义应该被放置在该属于首次使用的位置。
	\item 如果定义较早给出,读者将不得不往回查找,这可能失去集中力(或者更糟糕,失去兴趣)。
	\item 尽量缩短定义和它首次被使用位置之间的距离。
\end{itemize}
\end{frame}


\begin{frame}
\begin{itemize}
	\item  为了强化几页没有使用的数学符号,你可以重复定义。
	
	\item 例如,``最佳步长$\alpha^{*}$如下''。
	
	\item  这个含蓄的再定义或者提醒了读者$\alpha^{*}$是什么, 或者再次确保他们已经准确地记住了它。
\end{itemize}
\end{frame}


\begin{frame}
	\begin{itemize}
		\item  最后,一个术语如何被定义?
		\item 有可能只有唯一的定义或者有多种可能。
		\item 你应该以一个简短地、用基本性质或者基本思想的术语描述的,并且与相关定义相容的定义为目标。
		\item 举个例子,正规矩阵的标准定义为“矩阵$A\in \mathbb{C}^{n\times n}$ 满足$AA^{*}= A^{*}A$”。 至少有70种不同的方式定义正规性,但是没有一种比$AA^{*}= A^{*}A$简单容易。
	\end{itemize}
\end{frame}


\begin{frame}
\begin{itemize}
\item 依照惯例,在定义中if 的意思是 if and only if , 所以不要写 

``The graph is connected  \alert{if and only if }~there is a path from every node in $G$ to every other node in $G$''.

而写成

``The graph is connected \alert{if}  there is a path from every node in $G$ to every other node in $G$''.

\end{itemize}
\end{frame}

\begin{frame}
\begin{itemize}
\item 把要定义的文字排版成斜体是惯例:

``The graph is \emph{connected} {if}  there is a path from every node in $G$ to every other node in $G$~''.

\item 这种强调也可以写为

``The graph is defined to be {connected} {if}  \dots''

或者

``The graph is said to be {connected} {if}  \dots''
\end{itemize}

\end{frame}


\section{优劣比较}

\begin{frame}{符号放置}
	避免以数学表达式开始一个句子,特别是如果前面的句子以数学表达式结束,否则读者难以分析句子。
	
	例如, ``$A$ is  an ill-conditioned matrix'' 
	(可能与单词``A''混淆)	
	可改为\pause
	
	 \qquad 	 \ ``The matrix $A$ is   ill-conditioned''. 
\vspace{10pt}

如果可能的话,为了同样的理由,用标点符号或者文字将数学符号隔开。

	\begin{itemize}
\item 差: If $x>1 \, f(x)<0$.\pause

\item 中: If $x>1, f(x)<0$. \pause

\item  好: If $x>1$ then $f(x)<0$.
	\end{itemize}

\vspace{10pt}

	\begin{itemize}
	\item 
差: since $p^{-1}+q^{-1}=1,\|\cdot\|_{p}$ and $\|\cdot\|_{q}$ are dual norms. \pause

\item  好: since $p^{-1}+q^{-1}=1,$ the norms $\|\cdot\|_{p}$ and $\|\cdot\|_{q}$ are dual.
	\end{itemize}
\vspace{10pt}
\end{frame}

\begin{frame}
	\begin{itemize}
	\item 
差: It suffices to show that $\|H\|_{p}=n^{1 / p}, 1 \leqslant p \leqslant 2$. \pause
	\item 
好: It suffices to show that $\|H\|_{p}=n^{1 / p}$ for $1 \leqslant p \leqslant 2$. 
\item  
好: It suffices to show that $\|H\|_{p}=n^{1 / p}(1 \leqslant p \leqslant 2)$.
	\end{itemize}
\vspace{10pt}

\begin{itemize}
	\item 
差: For $n=r(2.2)$ holds with $\delta_{r}=0$. \pause
\item 
好: For $n=r,(2.2)$ holds with $\delta_{r}=0 .$ 
\item 
好: For $n=r,$ inequality (2.2) holds with $\delta_{r}=0$.
	\end{itemize}

\end{frame}


\begin{frame}{``The''或``A''}
在数学写作中,当宾语指代(可能)不是独一无二的或者不存在的事物时,使用冠词``the''是不恰当地。
改写句子,或者将冠词改为``a'', 通常能解决问题。
\vspace{10pt}

{
\begin{itemize}
\item 差: Let the Schur decomposition of $A$ be $Q T Q^{*}$. \pause
\item 好: Let a Schur decomposition of $A$ be $Q T Q^{*}$.
\item 好: Let $A$ have the Schur decomposition $Q T Q^{*}$.
	\end{itemize}
}

\vspace{10pt}

\begin{itemize}
\item 差: Under what conditions does the iteration converge to the solution of $f(x)=$ 
$0 ?$  \pause
\item  好: Under what conditions does the iteration converge to a solution of $f(x)=0 ?$
\end{itemize}
\end{frame}
\end{document} 