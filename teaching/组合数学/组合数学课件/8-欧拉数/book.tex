\documentclass[a4paper,11pt]{article}
\usepackage{ctex}



\usepackage{amsmath,amssymb}             % AMS Math
\usepackage[T1]{fontenc}



\usepackage{graphicx}
% \usepackage{epstopdf}
\usepackage{tikz}
\usepackage[left=1.5in,right=1.3in,top=1.1in,bottom=1.1in,includefoot,includehead,headheight=13.6pt]{geometry}
\renewcommand{\baselinestretch}{1.05}



\usepackage{minitoc}
\newtheorem{thm}{定理}[section]
\newtheorem{prop}[thm]{命题}
\newtheorem{coro}[thm]{推论}
\newtheorem{defi}[thm]{定义}
\newtheorem{lem}[thm]{引理}
\newtheorem{exa}[thm]{例}
\newtheorem{ex}[thm]{习题}
\newtheorem{conj}[thm]{猜想}

\def\qed{\nopagebreak\hfill{\rule{4pt}{7pt}}\medbreak}
\def\pf{{\bf 证明~~ }}
\def \sg{\sigma}
\def \asc{\mathrm{asc}}
\def \des{\mathrm{des}}
\def \fix{\mathrm{fix}}
\def \lef{\mathrm{lef}}
\def \one{\mathrm{one}}
\def \Des{\mathrm{Des}}
\def \maj{\mathrm{maj}}
\def \exc{\mathrm{exc}}
\def \inv{\mathrm{inv}}
\def \roots{\mathrm{roots}}
\def \sgn{\mathrm{sgn}}
% Table of contents for each chapter

\usepackage{color}
\definecolor{linkcol}{rgb}{0,0,0.4}
\definecolor{citecol}{rgb}{0.5,0,0}

  \usepackage{graphicx}
  \DeclareGraphicsExtensions{.eps}
  \usepackage[a4paper,pagebackref,hyperindex=true,pdfnewwindow=true]{hyperref}

% \usepackage{chapterbib}
\begin{document}







\section{欧拉数}

欧拉(Euler),瑞士数学家及自然科学家。1707年4月15日出生于瑞士的巴塞尔,1783\\年
9月18日于俄国彼得堡去逝。欧拉出生于牧师家庭,自幼受父亲的教育。
13岁时入读巴塞尔大学,15岁大学毕业,16岁获硕士学位。

欧拉是18世纪数学界最杰出的人物之一,
他不但为数学界作出贡献,更把数学推至几乎整个物理的领域。
他是数学史上最多产的数学家,平均每年写出八百多页的论文,
还写了大量的力学、分析学、几何学、变分法等
的课本,《无穷小分析引论》、《微分学原理》、《积分学原理》
等都成为数学中的经典著作。

欧拉对数学的研究如此广泛,因此在许多数学的分支中也可经常见
到以他的名字
命名的重要常数、公式和定理。 诸如:欧拉函数,欧拉数,
欧拉定理,欧拉常数等等。




\subsection{欧拉数的定义和性质}
排列的各种统计量是组合数学研究的一个重要课题,对排列统计量的研究可以使我们
更清楚的了解排列的内部结构。下面我们就介绍一些在排列上十分熟知的统计量。
对于一个排列$\pi=\pi_1\pi_2\cdots \pi_n$, 
位置$i$($1\leqslant i<n$)称为是$\pi$的一个{下降位}(descent)
如果$\pi_i>\pi_{i+1}$;反之则称为$\pi$的{上升位}(acscent).
定义所有下降位构成的集合

$$\Des(\pi)=\{i|\pi_i>\pi_{i+1}\}$$
为$\pi$的下降集(descent set),
定义该集合的个数为$\des(\pi)=|\Des(\pi)|$为
$\pi$的下降数。由定义$n\notin \Des(\pi)$.


\begin{exa}
	对于$[5]$上的排列$\pi=43521$, 以上的统计量分别为:
	$\Des(\pi)=\{1,3,4\}$, $\des(\pi)=3$.
\end{exa}




设 $A(n,k)$ 为 $n$ 的所有置换中具有$k-1$个下降位的置换个数,我们称
$A(n,k)$ 为欧拉数 (Eulerian number).
本节我们就主要研究欧拉数的一些组合性质。在此之前,我们先给出 $n\leq
6$ 时欧拉数。

\begin{tabular}{c|c|c|c|c|c|c}
 $n \setminus k$    &1 &2 &3  &4  &5  &6\\
\hline $1$       &1  \\
\hline $2$       &1 &1 \\
\hline $3$       &1 &4 &1  \\
\hline $4$       &1 &11 &11  &1 \\
\hline $5$       &1 &26 &66  &26  &1 \\
\hline $6$       &1 &57 &302 &302 &57 &1
\end{tabular}

由欧拉数的组合意义,我们有下面的递推关系。

\begin{prop}\label{p1}
\begin{equation}
A(n,\,k)=kA(n-1,\, k)+(n-k+1)A(n-1,\,k-1)
\end{equation}
\end{prop}

\pf 给定一个 $n-1$ 长的且下降数为 $k-1$ 的排列,则我们把 $n$ 插入这
$k-1$ 个下降位的位置后面不会改变总的下降数的个数。显然如果把 $n$
插在最后一个位置也不会改变下降数的个数。如果在非下降位的后面插入
$n$, 则会使下降位增加一个。所以我们有 $A(n,\,k)=kA(n-1,\,
k)+(n-k+1)A(n-1,\,k-1).$ \qed

由上面的递推关系,我们很容易得到 $A(n,k)$ 的对称性。
\begin{prop}
\begin{equation}
A(n,k+1)=A(n,n-k).
\end{equation}
\end{prop}

当然从组合的观点,我们也可以这样而得。设 $n$ 的置换 $p=p_{1}p_{2}
\cdots p_n$有 $k$ 个下降数,则它的转置 $p^r=p_{n}p_{n-1}\cdots p_1$
有 $n-k-1$ 个降序数。由 $p$ 和 $p^r$ 的一一对应可得。

由\ref{exc_des}知,$\exc$与$\des$是等分布的,所以我们有。
\begin{prop}
在 $[n]$ 的所有置换中具有 $k-1$ 个胜位的置换个数为$A(n,k)$.
\end{prop}


\subsection{与欧拉数有关的等式}
由欧拉数的组合意义,我们还可以得到一些特殊的具有组合意义的式子。
\begin{thm}\label{t1}(\cite{Graham1994})
令 $A(0,0)=1$, 且当 $n>0$ 时,令 $A(n,0)=0$. 则对于所有非负整数 $n$
和实数 $x$ 满足如下等式
\begin{equation}\label{sm}
x^n=\sum_{k=0}^{n}A(n,k){x+n-k \choose n}
\end{equation}
\end{thm}

直接比较 \eqref{Ant}两边$k^n$的系数,很显然定理成立,
这里我们给出它的一个组合证明。定理两边都是关于$t$的$n$次多项式,
我们只需证明它对于$n+1$个不相等的实数成立即可。
这里,我们证明其对于任意正整数成立。

\pf 我们先假设$x$是一个正整数.则等式左边代表的是长度为 $n$
的,且每个分量取自集合 $[x]$
的序列个数.则我们只需说明等式右边也是计算的这种序列的个数。令
$a=a_{1}a_{2}\cdots a_n$
为任意一个这样的序列,重新排列$a$中元素的顺序使其非递降得
$a'=a_{i_{1}}\leq
a_{i_{2}}\leq \cdots \leq a_{i_{n}}$.
如果是相同的数字则在$a'$中的顺序是其在按照它们在$a$中的下
标递增的顺序排列.则$i=i_{1}i_{2}\cdots
i_{n}$ 为$n$ 的由 $a$ 唯一决定的置换,$i_{k}$
代表了$a$中第$i_{k}$大的数字所在的位置.例如 $a=3~1~1~2~4~3$,
重排后得 $a'=1~1~2~3~3~4$,对应的置换为 $i=2~3~4~1~6~5$.
\\
如果我们能说明每一个具有 $k-1$ 个下降数的置换 $i$ 是恰好从 $x+n-k$个
序列 $a$ 而得到的,则我们就完成了证明。
\\
很显然如果 $a_{i_{j}}=a_{i_{j+1}}$, 那么 $i_{j}<i_{j+1}$。
对应的,如果 $j$ 是置换 $p(a)=i_{1}i_{2}\cdots i_{n}$
的一个下降数,则 $a_{i_{j}}<a_{i_{j+1}}$.这就意味着只要 $j$
是一个下降数则序列 $a'$
在此位置是严格递增的。我们可以在上面的例子中验证一下。$i$ 在位置
$3,5$ 是 下降的,确实 $a'$
在这些位置上是严格递增的。那么有多少个序列 $a$ 能得到置换
$i=2~3~4~1~6~5$ 呢?由前面的分析可得,$a$ 中元素必须满足$$1\leq
a_{2}\leq a_{3} \leq a_{4} < a_{1} \leq a_6 <a_5 \leq x
$$严格的不等号是在第三个和第五个位置.上面的不等式链等价为
$$1\leq a_{2}< a_{3}+1
<a_{4}+2 < a_{1}+2 <a_6 +3 <a_5+3 \leq x$$ 因此这种序列的个数为
 ${x+3 \choose 6}$.
 同样的方法,对于任意的 $n$ 和具有 $k-1$ 个下降数的置换 $i$,
 我们得到 $n$ 的具有 $k-1$ 个下降数的置换可从
${x+(n-1)-(k-1)\choose n} $=${x+n-k \choose n}$ 个序列中得到.
\\如果 $x$ 不是一个正整数,由于等式两边都可以看作是关于变量 $x$
的多项式,而它们在无穷多个数值上取值相同,所以它们必须是本身是相等的。\qed


利用上述定理,我们可以讨论正整数前$n$项和的方幂求和的问题,
我们有如下结论:

\begin{prop}
\begin{align}
\sum_{x=1}^mx^n=\sum_{k=1}^{n}A(n,k){k+m\choose n+1}.
\end{align}
\end{prop}

\pf 首先,利用欧拉数的对称性,将\eqref{sm}进行化简。

当$n>0$时 \[t^n=\sum_{k=1}^{n}A(n,k){t+n-k \choose
n}=\sum_{k=1}^{n}A(n,n+1-k){t+n-k \choose
n}=\sum_{k=0}^{n-1}A(n,k+1){t+k \choose n}
\]
上式两边对$t$求和,有
\begin{align*}
\sum_{t=1}^mt^n &=\sum_{k=0}^{n-1}A(n,k+1)\sum_{t=1}^m{t+k \choose n}\\
                &=\sum_{k=0}^{n-1}A(n,k+1)\sum_{t=1}^m \left({t+k+1\choose n+1}-{t+k\choose n+1}\right)\\
                &=\sum_{k=0}^{n-1}A(n,k+1){m+k+1\choose n+1}\\
                &=\sum_{k=1}^{n}A(n,k){m+k\choose n+1}
\end{align*}
\qed



\begin{coro}
\begin{equation}
[x]^n=\sum_{k=0}^{n}A(n,k){x+k-1\choose n}.
\end{equation}
\end{coro}
\pf
 在定理\ref{t1}中用 $-x$ 代替 $x$,我们得
 $$x^n(-1)^n=\sum_{k=0}^{n}A(n,k){-x+n-k\choose n}.$$
 注意到 ${-x+n-k \choose n}$=$\frac{(-x+n-k)(-x+n-k-1)\cdots
(-x+1-k)}{n!}=(-1)^n{x+k-1 \choose
 n}$. 对照这两个等式就得到了结论。\qed

\begin{thm}
对于所有满足 $k\leq n$ 的非负整数 $n,k$, 有
\begin{equation}
A(n,k)=\sum_{i=0}^{k}(-1)^i{n+1 \choose i}(k-i)^n
\end{equation}
\end{thm}

\pf {组合证明}\\
我们先写下 $k-1$ 个竖线,这样就产生了 $k$ 个分间。把 $[n]$
中的每个元素放入任意一个分间内,有 $k^n$
种方法。然后对每个分间中的数字按递增的顺序排列。例如 $k=4,n=9$



那么其中的一个就可以为 $$237||19|4568.$$
忽虑掉那些竖线我们就得到了一个至多有 $k-1$
个下降数的置换(在上例中就是 $2~3~7~1~9~4~5~6~8$).
\\我们需注意以下几种情况:
可能会有空的分间(即分间里面没有放数字);或者相邻的分间之间没有下降数。由此我们就称一个竖线是"多余的",如果
\\(a)
去掉它仍能得到一个符合规定的排列(即在每个分间中的数字是递增的顺序)。例如$4|12|3$中的第二个竖线。
\\(b)此竖线紧接着前面一个竖线(即有空的分间)。例如$2|35||614$中的第三个竖线。
我们的目标是计算没有"多余的竖线"的排列个数,因为这样的排列是与具有$k-1$个下降数的置换一一对应的。
我们利用容斥原理来计算,令$B_i$为至少有$i$个多余的竖线的排列数,$B$为没有多余的竖线的排列数,则
$$
B=k^n-B_1+B_2-B_3+\cdots +(-1)^nB_n.
$$
现在我们来计算这些 $B_i$. $B_1$
指的是至少有一个多余的竖线的排列,我们可以这样得到。先写下 $k-2$
个竖线,再把 $[n]$ 中的数字放入这 $k-1$ 个分间中,然后把一个多余的
竖线插入任意一个数字的左边,或放在末尾,共有 $n+1$
种方法,也就是$B_1={n+1 \choose 1}(k-1)^n$.
类似地,我们可得$B_2={n+1 \choose 2}(k-2)^n$,
这时我们是有$k-2$个分间,再把两个多余的竖线插入。继续这种方法得$$B_i={n+1
\choose i}(k-i)^n$$把这些式子代人 $B$
中就得到了所要证的等式的右边.\qed


\pf {代数证明}\\
 对欧拉多项式 $A_n(x)=\sum\limits^n_{k=1}A(n,k)x^k,$
我们有(见欧拉多项式那节)
$$\sum\limits^{\infty}_{k=1}k^nx^k=\frac{A_n(x)}{(1-x)^{n+1}}.$$
前面已经给出了它的组合证明。由上式可得$$(1-x)^{n+1}\sum\limits^{\infty}_{k=1}k^nx^k=A_n(x).$$
比较两边系数便可得证。\qed

下面我们来看一下欧拉数和第二类 Stirling 数之间的关系。第二类
Stirling 数 $S(n,k)$ 是指把集合 $\{1,2,\cdots,n\}$ 分成 $k$
个互不相交的无序块并的个数。

\begin{thm}
对于任意的正整数 $n,r$, 有
\begin{equation}
S(n,r)=\frac{1}{r!}\sum_{k=1}^{r}A(n,k){n-k\choose r-k}.
\end{equation}
\end{thm}

\pf{组合证明}
 等式两边同乘以
$r!$ 得, $$r!S(n,r)=\sum_{k=1}^{r}A(n,k){n-k\choose r-k}$$
显然左边代表的是集合 $[n]$ 的有序 $r$
划分。我们只要说明右边也是计算的同样的东西。对于 $[n]$ 的具有 $k-1$
个下降数的置换,就产生了 $k$ 个递增的字串,这恰好对应了 把集合 $[n]$
分成 $k$ 个部分。如果 $k=r$, 那么就是我们所要求的。如果 $k<r$,
我们就需要把一些递增字串拆开成若干个更小的串(保持
原来数字的顺序不变),从而能到 $r$ 个递增字串。现在我们已经有了 $k$
个分块,我们还必须增加 $r-k$ 个块。$n$ 个元素的置换除了首末位置共有
$n-1$ 个空隙(相邻两个数字之间),
只要我们不在下降数的位置,就可以把串分成更小的串,这样共有$A(n,k){n-k\choose
r-k}$ 种方法。
\\由上面的方法我们得到了 $\sum_{k=1}^{r}A(n,k){n-k\choose
r-k}$ 个 $[n]$ 的有序 $r$
分划。显然这种分划可由置换唯一决定。反之,给定一个$[n]$的分划,在每个块中的元素按递增的顺序排列,那么一个有序划分,从左到右读就得到了一个
至多具有$r$个递增字串的置换。\qed

\begin{thm}
对于任意的正整数$n,k$,有
\begin{equation}
A(n,k)=\sum_{r=1}^{k}S(n,r)r!{n-r\choose k-r}(-1)^{k-r}.
\end{equation}
\end{thm}

\pf{代数证明}
由上面的性质把$S(n,r)=\frac{1}{r!}\sum_{k=1}^{r}A(n,k){n-k\choose
r-k}$代人要证式子的右边得,
$$\sum_{r=1}^{k}S(n,r)r!{n-r\choose k-r}(-1)^{k-r}=\sum_{r=1}^{k}(-1)^{k-r}{n-r\choose k-r}\sum_{i=1}^{r}A(n,i){n-i\choose
r-i}$$ 改变求和顺序得
$$\sum_{r=1}^{k}S(n,r)r!{n-r\choose k-r}(-1)^{k-r}=\sum_{i=1}^{r}A(n,i){n-i\choose r-i}\sum_{r=1}^{k}(-1)^{k-r}{n-r\choose
k-r}$$
此等式的左边就是我们要证明的式子的右边,所以我们只要说明上式的右边等于$A(n,k)$.
显然上式中$A(n,k)$前的系数为${n-k\choose k-k}=1$,
所以我们能证明对于$i<k$,$A(n,i)$的系数为零就完成了证明。注意到,如果$r<i$,就有${n-i\choose
r-i}=0$, 则对任意的$i<k$, 我们有
$$\sum_{r=i}^{k}{n-i\choose r-i}{n-r\choose k-r}(-1)^{k-r}=\sum_{r=i}^{k}{n-i\choose r-i}{k-n-1\choose k-r}={k-i-1\choose k-i}=0$$
最后第二个等号是由Cauchy's convolution formula \footnote{[Cauchy's
convolution formula] 设$x,y$为实数,$z$为正整数,则有${x+y\choose
z}=\sum_{d=0}^{z}{x\choose d}{y\choose z-d}$}而得到的。\qed






\section{欧拉多项式}
由下降数或胜位出发,我们定义
$$A_n(t)=\sum_{\pi\in S_n}t^{1+\des(\pi)}=
\sum_{\pi\in S_n}t^{1+\exc(\pi)}$$ 为$[n]$上的欧拉多项式(Eulerian
polynomial). 由此定义,则 $A_n(t)$中$t^{k}$的系数为欧拉数$A(n,k)$.
所以欧拉多项式也可写为 \[A_n(t)=\sum_{k\geq 1}A(n,k)t^k,\,n\geq1.\]

特别地,定义$A_0(t)=1$。
本节我们主要研究欧拉多项式的一些基本的性质。在此之前,我们先给出$n\leq
6$时的欧拉多项式。
\begin{align*}
A_1(t) &=t, \\[5pt]
A_2(t) &=t+t^2, \\[5pt]
A_3(t) &=t+4t^2+t^3, \\[5pt]
A_4(t) &=t+11t^2+11t^3+t^4,\\[5pt]
A_5(t) &=t+26t^2+66t^3+26t^4+t^5, \\[5pt]
A_6(t) &=t+57t^2+302t^3+302t^4+57t^5+t^6
\end{align*}

\begin{prop}\label{epd}
欧拉多项式满足下面的微分方程
\begin{equation}
A_{n+1}(t)=t(1-t)A_n'(t)+(n+1)tA_n(t).
\end{equation}
\end{prop}
\pf 由递推关系容易得到
\[
\sum_{k}A(n+1,k)t^k=\sum_{k}kA(n,k)t^k+(n+1)\sum_kA(n,k-1)t^k-(k-1)\sum_{k}a(n,k-1)t^k.\]
由此可得
\[A_{n+1}(t)=tA_n'(t)+t(n+1)A_n(t)-t^2A_n'(t).\]
整理一下上式即可得结论。\qed


由此微分方程,我们可以容易地得到下面这个等式。

\begin{prop}\label{pELdxs}
\begin{equation}
\sum\limits^{\infty}_{k=1}k^nx^k=\frac{A_n(x)}{(1-x)^{n+1}}.
\end{equation}
\end{prop}
\pf 我们利用归纳法来证明。\\
当 $n=1$ 时,左边=
$\sum\limits^{\infty}_{k=1}kx^k=x\frac{1}{1-x}'=\frac{x}{(1-x)^2}.$=右边。\\
假设 $n$ 时也成立,我们来看 $n+1$ 的情况。 由
\[  \sum\limits^{\infty}_{k=1}k^nx^k=\frac{A_n(x)}{(1-x)^{n+1}}.   \]
对上式两边 $x$ 进行微分,得
\[
\sum\limits^{\infty}_{k=1}k^{n+1}x^{k-1}=\frac{(1-x)A_n'(x)+(n+1)A_n(x)}{(1-x)^{n+2}}.\]
要证 \[
\sum\limits^{\infty}_{k=1}k^{n+1}x^k=\frac{A_{n+1}(x)}{(1-x)^{n+2}},\]
则相当于只需证
\[A_{n+1}(x)=x(1-x)A_n(x)+(n+1)xA_n(x).\]
而由性质 \eqref{epd} 上式成立。 \qed




现在我们进一步研究欧拉多项式的指数生成函数。

\begin{thm} \label{ec}
令 $$A(x)=\sum_{n\geq0}A_n(t)\frac{x^n}{n!},$$ 则 $A(x)$ 满足
\begin{equation}
A'(x)=(A(x)-1)A(x)+tA(x).
\end{equation}
\end{thm}
\pf
我们从生成函数的角度来考虑。假定每个排列的降序位包含最后一位,则相应的生成函数仍是
$A(x).$ $A'(x)$
表示在排列中去掉最大元后所得到的排列对应的生成函数。不妨设
$\pi=\pi_1(n+1)\pi_2=a_1a_2\cdots a_i(n+1)a_{i+2}\cdots a_{n+1}\in
S_{n+1},\ 0\leq i\leq n.$

下面分析 $\pi$ 去掉 $n+1$ 后的结构。

如果 $i=0,$ 即 $\pi_1=\emptyset,$ 而 $\pi_2=\emptyset$ 或者
$\pi_2\neq \emptyset.$ 此时,在 $\pi$ 中去掉 $n+1$
后,所得排列降序数减少 $1,$ 所以 $\pi_1=\emptyset$
时,对应生成函数为 $tA(x);$ 如果 $1\leq i\leq n,$ 即 $\pi_1\neq
\emptyset,$ 设 $\des(\pi_1)=k_1,\,\des(\pi_2)=k_2,$ 则 $\pi$ 去掉
$n+1$ 后,所得的两个排列的降序数之和为 $k_1+k_2,$ 而原排列降序中
$a_i(n+1)$ 不对应一个降序,但 $(n+1)a_{i+2}$
一定对应一个降序,即原排列降序数为
$\left(k_1-1\right)+1+k_2=k_1+k_2,$ 则 $\pi_1\neq \emptyset$
时,对应生成函数为 $(A(x)-1)A(x).$

所以有 $$A'(x)=(A(x)-1)A(x)+tA(x).$$ \qed

\begin{coro}
$A_n(t)$ 的生成函数为
$$A(x)=\sum_{n\geq
0}A_n(t)\frac{x^n}{n!}=\frac{1-t}{1-te^{(1-t)x}}.$$
\end{coro}
\pf 根据关系式 \eqref{ec},解如下微分方程, \allowdisplaybreaks
\begin{align*}\frac{d A}{(A-1)A+tA}&=d x,\\[5pt]
\frac{1}{t-1}\left(\frac{1}{A}-\frac{1}{A-1+t}\right)d A&=dx,\\[5pt]
 d\ln\frac{A}{A-1+t}&=(t-1)dx,\\[5pt]
\frac{A}{A-1+t}&=ce^{(t-1)x},
\end{align*}其中 $c$ 为常数,由初值 $A(0)=1,$ 得到 $c=\frac{1}{t},$
所以 $$A(x)=\frac{1-t}{1-te^{(1-t)x}}.$$ \qed

将生成函数展开为$x$和$t$的幂级数,可得
\begin{equation}\label{Ant}
A_n(t)=(1-t)^{n+1}\sum_{k\geq1}k^nt^k (n\geq1).
\end{equation}
即给出了性质\ref{pELdxs}的另一个证明。




下面我们讨论一下欧拉多项式的根的特点,首先我们给出一些关于根的特点的定义。

设 $f$ 是一个度为 $n$ 的,且根全为实数的多项式,定义
$\mathrm{roots}(f)=\left(a_{1},\ldots,\,a_{n}\right),$ 其中
$a_{1}\leqslant a_{2}\leqslant \cdots \leqslant a_{n}$ 为 $f(x)=0$
的根。(注意:如果我们写 $\mathrm{roots}(f)$ 的话,则已假定 $f$
的根全为实根)。

 \begin{defi} 给定多项式 $f,\,g,$
设 $\mathrm{roots}(f)=\left(a_{1},\ldots,\,a_{n}\right),\,
\mathrm{roots}(g)=\left(b_{1},\ldots,\, b_{n}\right),$ 称 $f,\,g$
是严格交错的,如果它们的根满足以下四种关系之一:
\begin{align*}
 a_{1}&<b_{1}<a_{2}<b_{2}< \cdots <a_{n}<b_{n};\\[5pt]
 b_{1}&<a_{1}<b_{2}< a_{2}< \cdots <b_{n};\\[5pt]
 b_{1}&<a_{1}<b_{2}<a_{2}< \cdots <b_{n}<a_{n};\\[5pt]
 a_{1}&<b_{1}<a_{2}< b_{2}< \cdots <a_{n};
\end{align*}
当把不等号 $<$换成$\leq$ 就称 $f,\,g$
是交错的。显然,如果两个多项式交错,则它们的度至多相差 $1$.
\end{defi}

%%%%%%%%%%%%%%%%%%%%%%%%%%%%%%%%%%%%%%%%%%%%%%%%%%%%%%%%%%%%%%
\begin{exa}
Eulerian多项式 $A_{3}(t)$ 和 $A_{4}(t)$ 是交错的。因为
\begin{align*}
A_3(t) &=t+4t^2+t^3 \\
A_4(t) &=t+11t^2+11t^3+t^4.
\end{align*}
所以$\roots\left(A_{3}(t)\right)=\left(-2-\sqrt{3},\,-2+\sqrt{3},\,0\right),\,
\roots\left(A_{4}(t)\right)=\left(-5-2\sqrt{6},\,-1,\,-5+2\sqrt{6},\,0\right).$

显然它们的根满足定义中的关系之一,所以这两个多项式是交错的。
\end{exa}
%%%%%%%%%%%%%%%%%%%%%%%%%%%%%%%%%%%%%%%%%%%%%%%%%%%%%%%%%%%%%%
为了叙述的方便,我们给出符号函数的定义。
 \begin{defi}
定义在实数集上的符号函数 $\sgn(x)$ 为:
$$ \sgn(x)=\left\{ \begin{array}{ll}
+1, \quad &\textrm{if $x>0,$}\\[5pt]
0 , \quad &\textrm{if $x=0,$}\\[5pt]
-1,  \quad &\textrm{if $x<0.$}
\end{array}\right.
$$
 \end{defi}
%%%%%%%%%%%%%%%%%%%%%%%%%%%%%%%%%%%%%%%%%%%%%%%%%%%%%%%%%%%%%%
 \begin{thm}
对于任意给定的 $n,$ 欧拉多项式 $A_{n}(t)=\sum
_{k=1}^{n}A(n,\,k)t^{k}$ 的根全是实根,并且 $A_{n-1}(t)$ 和
$A_{n}(t)$ 是交错的。
\end{thm}

\pf 令$B_n(t)={A_n(t)\over(1-t)^{n+1}}$,由\eqref{Ant},有
$${d\over dt}B_{n-1}(t)={d\over dt}\sum_{k\geq1}k^{n-1}t^k={1\over t}B_n(t)
$$整理得,
$$B_n(t)=t{d\over dt}B_{n-1}(t)
$$
补充定义$A_0(t)=t,$ 对于 $n\geqslant 1,$ 有
\begin{equation}\label{dt}
A_{n}(t)=t(1-t)^{n+1}\frac{d}{dt}(1-t)^{-n}A_{n-1}(t)
\end{equation}

下面我们用归纳法来证明欧拉多项式的根全为实根。当 $n=0$
时,欧拉多项式 $A_0(t)=t$ 的根为 $t=0.$

假设 $A_{n-1}(t)$ 有 $n-1$ 个不同的实根,其中有一个为 $t=0,$
其它全为负根。

从 $A_{n}(t)$ 与 $A_{n-1}(t)$ 的微分关系中,运用罗尔中值定理可知,在
$A_{n-1} (t)$ 的每两个相邻根之间必有一个 $A_{n}(t)$ 的根,而显然 $0$
也是一个根,这样我们就找到了 $A_{n}(t)$ 的 $n- 1$
个根。由于虚根是成对出现的,所以最后一个根也是实根。要证明 $A_{n-1}
(t)$ 和$A_{n}(t)$ 是交错的,只要说明这个根比 $A_{n-1} (t)$
的最小的那个根还要小。

设
$\roots\left(A_{n-1}(t)\right)=\left(r_{1},\,r_2,\ldots,\,r_{n-1}\right),$
则 $\sgn (A_{n-1}'(r_k))=(-1)^k,$ 因为 $A_{n-1}(t)'$
的首项系数为正的。除了 $r_1=0,$ 其余根均为负的,由\eqref{dt}式得
$\sgn (A_{n}(r_k))=(-1)^k,$ 因为 $A_{n}(t)$ 的首项系数也为正,所以
$\sgn (A_{n}(+\infty))=+1,\,  \sgn (A_{n}(-\infty))=(-1)^{n}.$
由此可知 $A_{n}(t)$ 必有一个根在区间 $(-\infty,\,r_{n-1})$ 上,所以
$A_{n-1}(t)$ 和 $A_{n}(t)$ 是交错的。\qed






\end{document} 
