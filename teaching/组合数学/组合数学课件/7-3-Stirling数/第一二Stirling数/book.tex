\documentclass[a4paper,11pt]{article}
\usepackage{ctex}



\usepackage{amsmath,amssymb}             % AMS Math
\usepackage[T1]{fontenc}



\usepackage{graphicx}
% \usepackage{epstopdf}
\usepackage{tikz}
\usepackage[left=1.5in,right=1.3in,top=1.1in,bottom=1.1in,includefoot,includehead,headheight=13.6pt]{geometry}
\renewcommand{\baselinestretch}{1.05}



\usepackage{minitoc}
\newtheorem{thm}{定理}[section]
\newtheorem{prop}[thm]{命题}
\newtheorem{coro}[thm]{推论}
\newtheorem{defi}[thm]{定义}
\newtheorem{lem}[thm]{引理}
\newtheorem{exa}[thm]{例}
\newtheorem{ex}[thm]{习题}
\newtheorem{conj}[thm]{猜想}

\def\qed{\nopagebreak\hfill{\rule{4pt}{7pt}}\medbreak}
\def\pf{{\bf 证明~~ }}
\def \sg{\sigma}
\def \asc{\mathrm{asc}}
\def \des{\mathrm{des}}
\def \fix{\mathrm{fix}}
\def \lef{\mathrm{lef}}
\def \one{\mathrm{one}}
\def \Des{\mathrm{Des}}
\def \maj{\mathrm{maj}}
\def \exc{\mathrm{exc}}
\def \inv{\mathrm{inv}}
\def \roots{\mathrm{roots}}
\def \sgn{\mathrm{sgn}}
% Table of contents for each chapter

\usepackage{color}
\definecolor{linkcol}{rgb}{0,0,0.4}
\definecolor{citecol}{rgb}{0.5,0,0}

  \usepackage{graphicx}
  \DeclareGraphicsExtensions{.eps}
  \usepackage[a4paper,pagebackref,hyperindex=true,pdfnewwindow=true]{hyperref}

% \usepackage{chapterbib}
\begin{document}




%\pagenumbering{roman}
%\cleardoublepage
%\dominitoc
% \tableofcontents
%\mainmatter
%

%\chapter{基本组合数}
%\label{zuhejiegou} \minitoc




%%%%%%%%%%%%%%%%%%%%%%%%%%%%%%%%%%%%%%%%%%%%%%%%%%%%%%%%%%%%%%%%%%%%%%%%%%%%%%%%%%%%%%
\section{Stirling数}

\subsection{Stirling 简介} James
Stirling是一位苏格兰籍的数学家。他1692年5月出生于苏格兰斯特灵郡,1770年10月逝世于爱丁堡。
Stirling数以及Stirling逼近都是以他的名字命名的。James
Stirling18岁进入牛津大学贝利奥尔学院(Balliol College,
Oxford)求学,后被驱逐至威尼斯。在威尼斯期间,James
Stirling在艾萨克.牛顿的帮助下,与皇家科学院取得联系并邮寄了他的一篇论文
Methodus differentials Newtoniana illustrata (Phil.Trans.,
1718)。后又在牛顿的帮助下于1725年回到伦敦。在伦敦的十年时间里,他一直致力于学术工作,
1730年,他最重要的工作 the methodus differentialis, sive tractatus
de summatione et interpolatione serierum infinitarum (4to,
London)发表。

这里我们主要介绍一下他的一个重要工作,Stirling数。Stirling数分为第一类和第二类。
下面先看一下Stirling数的具体定义。

%%%%%%%%%%%%%%%%%%%%%%%%%%%%%%%%%%%%%%%%%%%%%%%%
\subsection{第一类Stirling数}

\begin{defi}
$c(n,k)$为恰好含有$k$个圈的$\pi\in\mathfrak{S}_n$的个数。数
$s(n,k):=(-1)^{n-k}\\c(n,k)$被称为{\bf
第一类Stirling数},\index{第一类斯特林数,Stirling number of the
first kind}而 $c(n,k)$被称为{\bf 无符号的第一类Stirling数}。
\end{defi}

\begin{defi}
{\bf 降阶乘函数}\index{降阶乘函数,falling factorial}$(x)_n$定义如下
$$(x)_n=x(x-1)\cdots(x-n+1).$$
\end{defi}

显然通常的阶乘$n!=(n)_n.$

第一类Stirling数有如下等价定义:

\begin{defi}
{\bf 第一类Stirling数}(记为$s(n,k)$, $1\leq k\leq
n$)恰好为降阶乘\index{降阶乘,falling
factorial}多项式中的各项系数,即
\begin{equation}
(x)_n=x(x-1)(x-2)\cdots(x-n+1)=\sum_{k=1}^n{s(n,k)x^k}.
\end{equation}
\end{defi}

\begin{thm}
$c(n,k)$ 满足如下递推式
$$c(n,k)=(n-1)c(n-1,k)+c(n-1,k-1),\  n,k\geq 1$$
并且初值为$c(0,0)=1$,而对其它的$n\leq 0$或者$k\leq 0$,$c(n,k)=0$。
\end{thm}

{\bf 证明:}选定一个有$k$个圈的排列$\pi\in\mathfrak{S}_{n-1}.$在$
\pi$的不交圈分解中,我们可以将$n$插入数字$1,2,\ldots,n-1$的任一个的后面,
有$n-1$种方法。这样就得到一个排列$\pi^{'}\in\mathfrak{S}_n$的不交圈分解,它具有
$k$个圈并且$n$出现在一个长度$\geq
2$的圈中。于是共有$(n-1)c(n-1,k)$个排列
$\pi^{'}\in\mathfrak{S}_n$具有$k$个圈并且满足$\pi^{'}(n)\neq n.$

另一方面,如果选定一个有$k$个圈的排列$\pi^{'}\in\mathfrak{S}_{n-1},$可以通过定义
  $$\pi^{'}(i)=\left\{\begin{array}{ll}
                          \pi(i), & \mbox{若}i\in[n-1]; \\
                          n, &\mbox{若} i=n.
                        \end{array}\right.
    $$

将其扩充为一个具有$k$个圈的排列$\pi^{'}\in\mathfrak{S}_n,$并且有
$\pi^{'}(n)=n$。因此,共有$c(n-1,k-1)$个排列$\pi^{'}\in\mathfrak{S}_n$具有
$k$个圈并且满足$\pi^{'}(n)=n,$得证。\qed

%%%%%%%%%%%%%%%%%%%%%%%%%%%%%%%%%%%%%%%%%%%%%%%%
\subsection{第二类Stirling数}
\begin{defi}
把$n$个元素构成的集合划分为$k$个非空子集的方法数,称为{\bf
第二类Stirling数}\index{第二类斯特林数,Stirling number of the
second kind},记为$S(n,k)$.
\end{defi}

举例而言,集合$[3]$划分成三个子集合的方法只有一种: $1/2/3$;
划分成两个子集合的方法有三种: $12/3,13/2,1/23$;
划分成一个子集合的方法只有一种: $1\ 2\ 3$.


\begin{thm}
第二类Stirling数有如下的递归关系式:
\begin{equation}S(n,k)=kS(n-1,k)+S(n-1,k-1),\ 1\le k<n,\end{equation}
其中初始条件为$S(n,n)=S(n,1)=1.$
\end{thm}
{\bf 证明:}
我们直接从第二类Stirling数的定义来给出这个递归关系式的证明。
显然,把$n$个元素放在一个集合和$n$个集合里都只有一种放法。
现假设把$n-1$个元素放在$k$个集合里的的方法数为$S(n-1,k).$
现在考虑把$n$个元素 $a_1, a_2, \ldots,
a_n$\\放在$k$个集合里,我们将$a_n$单独拿出来考虑。会有如下两种方式:
\begin{itemize}
\item[1. ]将前$n-1$个元素放入$k$个集合中,
再将$a_n$放入这$k$个集合中的某一个。这样一共有$kS(n-1,k)$种放法。
\item[2. ]将前$n-1$个元素放入$k-1$个集合中,
再将$a_n$单独放在一个新的集合中。这样给出另外$S(n-1,k-1)$种方法。
由此定理证明完毕。\qed
\end{itemize}

现在我们考虑由变量为$x$的多项式组成的向量空间。
对于这个无限维的向量空间最明显的一组基是单项式幂级数$x^n,\  n\geq
0.$ 然而同时,降阶乘函数
$$(x)_n=x(x-1)\cdots(x-n+1),\ n\geq 0$$
也是这个向量空间的一组基,自然能生成幂级数$x^n.$ 其实,
第二类Stirling数就是这两组基之间的过渡矩阵里的元素, 即:
\begin{equation}
x^n=\sum_{k=1}^nS(n,k)(x)_k.\label{1}
\end{equation}
例如,$x^4=x+7x(x-1)+6x(x-1)(x-2)+x(x-1)(x-2)(x-3)$.

{\bf 证明:}现用归纳法证明上述递归关系式\ref{1}。
显然\ref{1}对于$n=1$成立。 \\
假设\ref{1}对于$n$成立。由降阶乘函数的定义我们得到
$$x(x)_k=(x)_{k+1}+k(x)_k,$$
据此我们由归纳假设得到递归关系式
\begin{align*}
x^{n+1}&=\sum_{k=1}^nS(n,k)x(x)_k\\
&=\sum_{k=1}^n S(n,k)[(x)_{k+1}+k(x)_k]\\
&=\sum_{k=1}^{n}[S(n,k-1)+kS(n,k)](x)_k+S(n,n)(x)_{n+1}\\
&=\sum_{k=1}^{n+1}S(n+1,k)(x)_k.
\end{align*}
所以\ref{1}得证。\qed

接下来给出\ref{1}的一个简单的组合证明。考虑映射$f:N\rightarrow
X$,其中$|N|=n$而$|X|=x$。 \ref{1}的左边计数了映射$f:N\rightarrow
X$的总数。而每一个这样的映射都是到$X$的某个满足 $|Y|\leq
n$的子集$Y$的满射,且$Y$唯一。
如果$|Y|=k$,则有$k!S(n,k)$个这样的映射。而满足$|Y|=k$的$X$的子集$Y$有${x\choose
k}$种选择,因此
$$x^n=\sum_{k=0}^nk!S(n,k){x\choose k}=\sum_{k=0}^nS(n,k)(x)_k.$$

现在我们了解了有关于第二类Stirling数的组合解释,以及它作为幂函数与降阶乘函数之间过渡矩阵的元素。
下面我们计算一下$S(n,k)$的具体数值。

在计算的过程中,我们会用到有限差分演算,下面先简单罗列一下我们计算中会用到的一些概念和公式。
给定一个映射$f:\mathbb{N}\rightarrow\mathbb{C}$,定义一个新的映射$\Delta
f$如下,称之为$f$的{\bf 一阶差分},
$$\Delta f(n)=f(n+1)-f(n).$$
$\Delta$称为一阶{\bf
差分算子}。将算子$\Delta$重复$k$次就可以得到$k$阶差分算子。
$$\Delta^k f=\Delta(\Delta^{k-1}f).$$
以上是差分算子的概念,下面的式子在我们的计算中会起到很重要的作用。
\begin{equation}
f(n)=\sum_{k=0}^n{n\choose k}\Delta^k f(0).\label{e2}
\end{equation}

如果将连续两项$f(i),f(i+1)$的差$f(i+1)-f(i)=\Delta
f(i)$写在它们下一行的中间,就得到序列
$$\ldots \Delta f(-2)\ \Delta f(-1)\ \Delta f(0)\ \Delta f(1)\ \Delta f(2)\ldots$$
反复这个过程,就得到映射$f$的差分表,它的第$k$行由$\Delta^k
f(n)$组成。从$f(0)$开始往右下方延伸的对角线则是由0点的差分$\Delta^k
f(0)$构成。例如,令 $f(n)=n^4,$则差分表(从$f(0)$开始)如下


$\begin{array}{cccccccccccc}
  0 &  & 1 &  & 16 &  & 81 &  & 256 &  & 625 & \cdots \\
   & 1 &  & 15 &  & 65 &  & 175 &  & 369 &  &  \\
   &  & 14 &  & 50 &  & 110 &  & 194 &  &  &  \\
   &  &  & 36 &  & 60 &  & 84 &  &  &  &  \\
   &  &  &  & 24 &  & 24 &  &  &  &  &  \\
   &  &  &  &  & 0 &  &  &  &  &  &  \\
   &  &  &  &  &  & \ddots &  &  &  &  &  \\
   &  &  &  &  &  &  &  &  &  &  &
\end{array}$

因此,由\ref{e2}得
$$n^4={n\choose 1}+14{n\choose 2}+36{n\choose 3}+24{n\choose 4}+0{n\choose 5}$$
据此再结合\ref{1}就可以得到$S(n,k)$的具体数值。

\begin{ex} 对所有$m,n\in\mathbb{N}$,成立
$$\sum_{k\geq 0}S(m,n)s(m,n)=\delta_{mn}.$$
\end{ex}

提示:$\left(S(m,n)\right)_{m,n\geq 0},\left(s(m,n)\right)_{m,n\geq
0}$刚好是两组基之间的过渡矩阵。






\end{document} 
