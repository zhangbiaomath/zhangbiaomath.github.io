\documentclass[punct]{ctexbeamer}
\usefonttheme{professionalfonts}   % 数学公式字体

\titlegraphic{\includegraphics[width=2cm]{tjnu.jpg}}

\usepackage{color}
%\lineskip=9pt
\linespread{1.3}\selectfont
\makeatletter
\renewcommand\normalsize{%
	\@setfontsize\normalsize\@xpt\@xiipt
	\abovedisplayskip 3\p@ \@plus3\p@ \@minus3\p@
	\abovedisplayshortskip \z@ \@plus3\p@
	\belowdisplayshortskip 3\p@ \@plus3\p@ \@minus1\p@
	\belowdisplayskip \abovedisplayskip
	\let\@listi\@listI}
\makeatother
\parskip=6pt
%\usepackage{ctex}
%\usepackage[UTF8, heading = false, scheme = plain]{ctex}
%%%=== theme ===%%%
\usetheme{Madrid}
\useinnertheme{circles}
\setbeamertemplate{navigation symbols}{}
%\setbeamertemplate{footline}[page number]
\setbeamertemplate{footline}[frame number]{}
\usepackage{lmodern}
\usepackage{amsmath}
\usepackage{amssymb}
\usepackage{latexsym}
\usepackage{amsthm}
\usepackage{mathrsfs}
\usepackage{tikz}




\setbeamertemplate{theorems}[numbered]
\newtheorem{thm}{定理}[section]
\newtheorem{prop}[thm]{命题}
\newtheorem{cor}[thm]{推论}
\newtheorem{defi}[thm]{定义}
\newtheorem{lem}[thm]{引理}

\newtheorem{quest}[thm]{问题}
\newtheorem{conj}[thm]{猜想}
\newtheorem{ex}{例}[section]
\newtheorem{pr}{性质}

\newcommand{\blue}{\textcolor{blue}}
\def\pf{\noindent {\bf 证明\ }}
\def\sol{\noindent {\bf 解\ }}


\def\multiset#1#2{\ensuremath{\left(\kern-.3em\left(\genfrac{}{}{0pt}{}{#1}{#2}\right)\kern-.3em\right)}}



\begin{document}
	
	\title{组\ 合\ 数\ 学}
	
	\author{张\ 彪}
	\institute[数学科学学院]{\normalsize 天津师范大学}
	%\date[2011年10月13日]{\small 2011年10月13日}
	\date[]{zhang@tjnu.edu.cn}
	\frame[plain]{\titlepage}
	\begin{frame}
		\tableofcontents
	\end{frame}
	\AtBeginSection[]
	{
		\begin{frame}
			\frametitle{Outline}
			\tableofcontents[currentsection]
		\end{frame}
	}
\section{第一类Stirling数及其普通生成函数}
\begin{frame}{第一类Stirling数}
\begin{defi}
	对于正整数$n,k$,定义$c(n,k)$为$n$元对称群$S_n$中恰好由$k$个不交轮换构成的置换个数(不动点也看作一个轮换).
	
	$s(n,k)=(-1)^{n-k}c(n,k)$被称为\bf{第一类Stirling数}, $c(n,k)$也常称为\bf {无符号的第一类Stirling数}.
\end{defi}
规定$c(0,0)=1$以及当$n \geq 1$时,$c(n,0)=c(0,n)=0$,这显然是合理的.
\end{frame}
\begin{frame}{第一类Stirling数的递推关系}
\begin{lem}
	对任意$n\geq 1,k\geq 1$,$c(n,k)$ 满足如下递推式
	$$c(n,k)=(n-1)c(n-1,k)+c(n-1,k-1).$$
\end{lem}
\pause
\begin{proof}
	设置换$\sigma$是$S_n$中恰好有$k$个轮换的置换,若$\sigma{(n)}=n$,则$n$在$\sigma$ 中为一个单独的轮换,从而这样的$\sigma$ 的个数等于$S_{n-1}$中恰有$k-1$个轮换的置换的个数,即$c(n-1,k-1)$个;若$\sigma{(n)}=i,1\leq i\leq n-1$,则将轮换中的$n$与$i$看为一个元素,从而这样的$\sigma$ 的个数等于$S_{n-1}$中恰有$k$个轮换的置换的个数,即$c(n-1,k)$个,因此	$$c(n,k)=(n-1)c(n-1,k)+c(n-1,k-1).$$
	
\end{proof}
\end{frame}
\begin{frame}{第一类Stirling数的普通生成函数}
    \begin{thm}
	$\{c(n,k)\}_{n=1}^{\infty}$满足如下的函数方程
	$$\sum_{k=1}^n{c(n,k)x^k}=x(x+1)(x+2)\cdots(x+n-1).$$
	\end{thm}
\end{frame}
\begin{frame}{第一类Stirling数的普通生成函数}
\begin{proof}
	对$n$用归纳法证明命题.$n=1$时命题即$x=x$,显然成立.
	设对$n \geq 2,n-1$时命题成立,则当$n$时,由归纳假设及$c(n,k)$的递推性质知,任意$1\leq k\leq n$,	
	\[\begin{aligned}
	[x^k]x\cdots(x+n-1)
	=&[x^k]x\cdots(x+n-2)x+(n-1)[x^k]x\cdots(x+n-2)\\
	=&[x^{k-1}]x\cdots(x+n-2)+(n-1)[x^k]x\cdots(x+n-2)\\
	=&c(n-1,k-1)+(n-1)c(n-1,k)\\
	=&c(n,k).
	\end{aligned}
	\]从而
	$$\sum_{k=1}^n{c(n,k)x^k}=x(x+1)(x+2)\cdots(x+n-1),$$
	即命题对$n$成立.

	由归纳原理知命题对一切正整数$n$成立,即$\{c(n,k)\}_{n=1}^{\infty}$满足定理中所述函数方程.
\end{proof}
\end{frame}		



\begin{frame}{第一类Stirling数的普通生成函数}
很多情况下,第一类Stirling数$s(n,k)$往往比无符号的第一类Stirling数$c(n,k)$更容易处理.针对$\{s(n,k)\}_{n=1}^{\infty}$,定理(1.3)相应的等价形式是下面定理.
\begin{thm}
	$\{s(n,k)\}_{n=1}^{\infty}$满足如下的函数方程
	$$\sum_{k=1}^n{s(n,k)x^k}=(x)_n.$$
	
	这里$(x)_n=x(x-1)(x-2)\cdots(x-n+1).$
\end{thm}
\end{frame}	
\begin{frame}{第一类Stirling数的普通生成函数}
	
	{\bf 定理1.4}
		$\{s(n,k)\}_{n=1}^{\infty}$满足如下的函数方程
		$$\sum_{k=1}^n{s(n,k)x^k}=(x)_n.$$
	这里$(x)_n=x(x-1)(x-2)\cdots(x-n+1).$
    \begin{proof}
	在定理(1.3)中用$-x$代替$x$得
	$$\sum_{k=1}^n{c(n,k)(-x)^k}=-x(-x+1)(-x+2)\cdots(-x+n-1),$$
	两边同时乘以$(-1)^n$得
	$$\sum_{k=1}^n{(-1)^nc(n,k)(-x)^k}=x(x-1)(x-2)\cdots(x-n+1),$$
	即$$\sum_{k=1}^n{s(n,k)x^k}=(x)_n.$$
    \end{proof}
\end{frame}
\begin{frame}{第一类Stirling数的普通生成函数}

{\bf 证明}
   		现在我们用另外一种方法证明定理(1.4).
   		回顾引理(1.2)中无符号的第一类Stirling数$c(n,k)$满足的递推关系及初始条件,将其转化成第一类Stirling数$s(n,k)$,且当$n\ge 1$时,满足如下递推式
   		$$s(n,k)=s(n-1,k-1)-(n-1)s(n-1,k).$$
   		我们希望找到生成函数$f_n(x)=\sum_{k\ge 0}{s(n,k)x^k}.$
   		在上述递推式左右两边同乘以$x^k$,并对$k,k\ge 1$求和,得
   		   	$$\sum_{k\ge 1}{s(n,k)x^k}=\sum_{k\ge 1}{s(n-1,k-1)x^k}-(n-1)\sum_{k\ge 1}{s(n-1,k)x^k}.$$
   		 接下来使用$f_n(x)$表示上述等式,得
   		 $$f_n(x)-s(n,0)x^0=f_n(x)=xf_{n-1}(x)-(n-1)f_{n-1}(x),$$
   		 即
   		 $$f_n(x)=(x-n+1)f_{n-1}(x).$$
   		 这就得到生成函数序列$f_n(x)$的递推关系.由初始条件$s(0,k)$得$f_0(x)=1$.现在容易猜出$f_n(x)$的表达式,只需写出前面一些值然后用归纳法证明即可.
   		这样就证明了当$n,k\ge 1$时定理成立.

\end{frame}
\section{第二类Stirling数及其普通生成函数}
\begin{frame}{第二类Stirling数}
	\begin{defi}
		对于正整数$n,k$,定义$S(n,k)$为把$[n]$分成$k$个非空子集的划分的个数,称为\bf{第二类Stirling数}
	\end{defi}
	规定$S(0,0)=1$以及当$n \geq 1$时,$S(n,0)=S(0,n)=0$,这显然是合理的.
\end{frame}
\begin{frame}{第二类Stirling数的递推关系}
	%第二类Stirling数有着与第一类Stirling数对偶的递推关系.
	\begin{lem}
		对任意$n\geq 1,k\geq 1$,第二类Stirling数$S(n,k)$ 满足如下递推式
		$$S(n,k)=kS(n-1,k)+S(n-1,k-1).$$
	\end{lem}
\pause
    \begin{proof}
    	把$[n]$分成$k$个非空子集的划分,简称为把$[n]$的一个$k-$划分.设$P$是$[n]$的一个$k-$划分.若$n$在$P$中为一个单独的子集,则这样的$P$的个数等于$[n-1]$的$(k-1)-$划分个数,即$S(n-1,k-1)$.若$n$在$P$中不是一个单独的子集,则从$P$中去掉$n$可以得到一个$[n-1]$的$k-$划分,而把$n$插入任意一个$[n-1]$的$k-$划分可得到$k$个不同的$[n]$的$k-$划分,从而这样的$P$的个数等于$[n-1]$的$k-$划分个数的$k$倍,即$kS(n-1,k)$.因此
    	$$S(n,k)=kS(n-1,k)+S(n-1,k-1).$$
    \end{proof}
\end{frame}
\begin{frame}{第二类Stirling数}
   \begin{thm}
   	$\{S(n,k)\}_{n=1}^{\infty}$满足如下的函数方程
   	$$\sum_{k=1}^n{S(n,k)(x)_k}=x^n.$$
   	
   	这里$(x)_k=x(x-1)(x-2)\cdots(x-k+1).$
   \end{thm}
\end{frame}	
\begin{frame}{第二类Stirling数}
	
   {\bf 证明}
   	对$n$用归纳法证明命题.$n=1$时命题即$x=x$,显然成立.
   	设对$n \geq 2,n-1$时命题成立,则当$n$时,由归纳假设及$S(n,k)$的递推性质知
   		\[\begin{aligned}
   		x^n &=x^{n-1} x \\
   		&=\sum_{k=1}^{n-1} S(n-1, k)(x)_k x \\
   		&=\sum_{k=1}^{n-1} S(n-1, k)(x)_k(x-k+k) \\
   		&=\sum_{k=1}^{n-1} S(n-1, k)(x)_{k+1}+\sum_{k=1}^{n-1} k S(n-1, k)(x)_k \\
   		&=\sum_{k=2}^n S(n-1, k-1)(x)_k+\sum_{k=1}^{n-1} k S(n-1, k)(x)_k \\
      \end{aligned}
        \]
    
\end{frame}		
\begin{frame}{第二类Stirling数}
		\[\begin{aligned}
		&=\sum_{k=1}^n S(n-1, k-1)(x)_k+\sum_{k=1}^n k S(n-1, k)(x)_k \\
		&=\sum_{k=1}^n S(n, k)(x)_k .
		\end{aligned}\]
		即命题对$n$成立.
		
		由归纳原理知命题对一切正整数$n$成立,即$\{S(n,k)\}_{n=1}^{\infty}$满足定理中所述函数方程.\qed
		
		{\bf 注:}
		设$x$为一个正整数,则有$x^n$个从$[n]$到$[x]$的映射,对$[x]$的每个$k-$子集$Y$,有$k ! S(n, k)$个从$[n]$到$Y$的满射,所以
		$$
		x^n=\sum_{k=1}^n\left(\begin{array}{l}
		x \\
		k
		\end{array}\right) k ! S(n, k)=\sum_{k=1}^n S(n, k)(x)_k.
		$$
	\end{frame}
\begin{frame}{第二类Stirling数的普通生成函数}
	\begin{thm}
		$\{S(n,k)\}_{k=1}^{\infty}$满足如下的函数方程
		$$\sum_{n=1}^k{S(n,k)x^n}=\frac{x^k}{(1-x)(1-2x)\cdots(1-kx)}.$$
	\end{thm}
	
	
\end{frame}		
\begin{frame}{第二类Stirling数的普通生成函数}
	
	{\bf 证明}
	回顾引理(2.2)中第二类Stirling数$S(n,k)$满足如下递推式
	$$S(n,k)=kS(n-1,k)+S(n-1,k-1).$$
	我们希望找到生成函数$f_k(x)=\sum_{n\ge 0}{S(n,k)x^n}.$
	在上述递推式左右两边同乘以$x^n$,并对$n,n\ge 1$求和,得
	$$\sum_{n\ge 1}{S(n,k)x^n}=\sum_{n\ge 1}{kS(n-1,k)x^n}+\sum_{n\ge 1}{S(n-1,k-1)x^n}.$$
	接下来使用$f_k(x)$表示上述等式,得
	$$f_k(x)=kxf_{k}(x)+xf_{k-1}(x),$$
	即
	$$f_k(x)=\frac{x}{1-kx}f_{k-1}(x).$$
	这就得到生成函数序列$f_k(x)$的递推关系.由初始条件$S(n,0)$得$f_0(x)=1$.现在容易猜出$f_k(x)$的表达式,只需写出前面一些值然后用归纳法证明即可.
	这样就证明了当$n,k\ge 1$时定理成立.
	
\end{frame}		
   	

\section{第二类Stirling数的指数型生成函数} 
\begin{frame}{$S(n, k)$的显式公式}
	在求第二类Stirling数的指数型生成函数之前,我们先得到$S(n, k)$的显式公式.
	\begin{thm}
			对于任意的正整数$n,k$,
		$$
		S(n, k)=\frac{1}{k!}\sum_{j=0}^k\left(\begin{array}{l}
		k \\
		j
		\end{array}\right) j^n(-1)^{k-j}.
		$$
	\end{thm}
{\bf 证明}
构造$k$个有标记的“篮子”,将$[n]$中的元素分到这$k$个有区别的篮子里(有些篮子可能分到了零个元素),用$S$表示所有这样的分法组成的集合.显然$ \mid S \mid=k^n$.对任意$1\le i\le k$,定义$P_i$为“第$i$个篮子是空”的性质,$A_i$为$S$中满足性质$P_i$的分法组成的集合,$\mathcal{P}$为所有这些性质组成的集合,则
$$S(n, k) =\frac{\mid\{A \in S: A \text { 不满足 } \mathcal{P} \text { 中的任何性质}\}\mid}{k !}=\frac{\left|\overline{A_1} \cap \overline{A_2} \cap \cdots \cap \overline{A_k}\right|}{k !} . $$

注意到对于任意$1 \leq i_1<i_2<\cdots<i_s \leq k, A_{i_1} \cap A_{i_2} \cap \cdots \cap A_{i_s} $表示的意义是$S$中满足性质$P_{i_1},\cdots,P_{i_s}$的分法组成的集合.在这些分法中,标号

\end{frame}	
\begin{frame}{$S(n, k)$的显式公式}	
	为${i_1},\cdots,{i_s}$的篮子为空,所有元素只能放进其余$k-s$个篮子中,从而$\mid A_{i_1}\cap A_{i_2}\cap\cdots\cap A_{i_s}\mid=(k-s)^n$.由容斥原理得
	$$
	\begin{aligned}
	k ! S(n, k) 
	&=|S|-\sum_i\left|A_i\right|+\sum_{1 \leq i<j \leq k}\left|A_i \cap A_j\right|-\sum_{1 \leq i<j<t \leq k}\left|A_i \cap A_j \cap A_t\right|+\cdots \\
	&+(-1)^k\left|A_1 \cap A_2 \cap \cdots \cap A_k\right| \\
	&=\sum_{r=0}^k\left(\begin{array}{c}
	k \\
	r
	\end{array}\right)(k-r)^n(-1)^r \\
	&=\sum_{j=0}^k\left(\begin{array}{c}
	k \\
	j
	\end{array}\right) j^n(-1)^{k-j} . \\
	\end{aligned}
	$$\qed
	
{\bf 注:}容易看到,$k ! S(n, k) $就是容斥原理章节中例题2.4里面所讨论的从$[n]$到$[k]$的满射的个数.
	
	
\end{frame}	

\begin{frame}{第二类Stirling数的指数型生成函数}
	\begin{thm}
		$\{S(n,k)\}_{n=1}^{\infty}$的指数型生成函数为
			$$\sum_{n=0}^{\infty} \frac{S(n, k) x^n}{n !}=\frac{\left(e^x-1\right)^k}{k !} . $$
	\end{thm}
\end{frame}	

\begin{frame}{第二类Stirling数的指数型生成函数}
	
	{\bf 证明}
	$$
	\begin{aligned}
	\sum_{n=0}^{\infty} \frac{S(n, k) x^n}{n !}&=\sum_{n=0}^{\infty} \frac{1}{k !} \sum_{j=0}^k\left(\begin{array}{c}
		k \\
		j
	\end{array}\right) j^n(-1)^{k-j} \frac{x^n}{n !} \\
    &=\sum_{j=0}^k(-1)^{k-j} \frac{1}{k !}\left(\begin{array}{c}
    	k \\
    	j
    \end{array}\right) \sum_{n=0}^{\infty} j^n \frac{x^n}{n !} \\
   &=\sum_{j=0}^k(-1)^{k-j} \frac{1}{k !}\left(\begin{array}{c}
	k \\
	j
   \end{array}\right) e^{j x} \\
   &=\frac{1}{k !} \sum_{j=0}^k\left(\begin{array}{c}
   k \\
   j
   \end{array}\right) (e^{x})^j (-1)^{k-j}\\
   &=\frac{1}{k !}(e^{x}-1)^k.\\\end{aligned}$$\qed

相比之下,尽管也有第一类Stirling数的递推关系,但关于其生成函数的推导要更复杂,为了求$s(n, k)$的指数型生成函数,需要借助两类Stirling数之间的关系.

\end{frame}	

\section{两类Stirling数的联系}	
\begin{frame}{两类Stirling数的联系}
	\begin{thm}
		由两类Stirling数, 定义 $n$ 级矩阵 $A=\left(a_{i j}\right)_{n \times n}:=(s(i, j))_{n \times n}$及$B=\left(b_{i j}\right)_{n \times n}:=(S(i, j))_{n \times n}.$
		则
		$$AB=BA=I.$$
	\end{thm}
	\pause
	\begin{proof}
		考虑复数域上次数小于$n+1$ 且常数项为零的多项式关于加法和数量乘法构成的线性空间
		$$
		\left\{\sum_{i=1}^n \lambda_i x^i \mid \lambda_i \in \mathbb{C}\right\},
		$$
		
		$\left\{x, x^2, \ldots, x^n\right\}$ 与 $\left\{(x)_1,(x)_2, \ldots,(x)_n\right\}$ 是它的两组基.记$X=(x, x^2, \ldots, x^n)^{'}$,
		$Y=((x)_1,(x)_2, \ldots,(x)_n)^{'}$,则$X=BY,Y=AX$.即$A,B$恰好是这两组基之间的过渡矩阵.固$$AB=BA=I.$$
		
	\end{proof}
	
	于是又立即有下面的推论.
\end{frame}
\begin{frame}{两类Stirling数的联系}
	\begin{cor}
		$$
		\begin{aligned}
		&\sum_{l=1}^n s(i, l) S(l, j)=\delta(i, j), \\
		&\sum_{l=1}^n S(i, l) s(l, j)=\delta(i, j) .
		\end{aligned}
		$$
	\end{cor}
\end{frame}
\begin{frame}{两类Stirling数的联系}
	\begin{thm}
		令 $A(x), B(x)$ 分别表示数列 $\left\{a_n\right\}_{n=0}^{\infty},\left\{b_n\right\}_{n=0}^{\infty}$ 的指数型生成函数. 则下列 三个命题等价:
		
		(i) 对任意 $n \geq 0, b_n=\sum_{i=0}^n S(n, i) a_i$;
		
		(ii) 对任意 $n \geq 0, a_n=\sum_{i=0}^n s(n, i) b_i$;
		
		(iii) $B(x)=A\left(e^x-1\right)$, 也即 $A(x)=B(\ln (1+x))$.
	\end{thm}
\end{frame}
\begin{frame}{两类Stirling数的联系}
	
	{\bf 证明}
若 (ii) 成立, 由推论 $4.2$ 有
$$
\begin{aligned}
\sum_{j=0}^n S(n, j) a_j &=\sum_{j=0}^n S(n, j) \sum_{i=0}^j s(j, i) b_i \\
&=\sum_{i=0}^n b_i \sum_{j=i}^n S(n, j) s(j, i) \\
&=\sum_{i=0}^n b_i \sum_{j=1}^n S(n, j) s(j, i) \\
&=\sum_{i=0}^n b_i \delta(n, i) \\
&=b_n
\end{aligned}
$$
即 (i) 成立. 同理, 若 (i) 成立, 由推论 $4.2$ 可得 (ii) 成立, 从而命题 (ii) 与 (i) 等价.
\end{frame}
\begin{frame}{两类Stirling数的联系}

若 (i) 成立, 由定义, 有
$$
\begin{aligned}
B(x) &=\sum_{n=0}^{\infty} b_n \frac{x^n}{n !} \\
&=\sum_{n=0}^{\infty} \sum_{i=0}^n S(n, i) a_i \frac{x^n}{n !} \\
&=\sum_{i=0}^{\infty} a_i \sum_{n \geq i} S(n, i) \frac{x^n}{n !} \\
&=\sum_{i=0}^{\infty} a_i \sum_{n \geq 0} S(n, i) \frac{x^n}{n !} \\
&=\sum_{i=0}^{\infty} a_i \frac{\left(e^x-1\right)^i}{i !} \\
&=A\left(e^x-1\right)
\end{aligned}
$$
即 (iii) 成立.易见推导过程可逆,从而命题 (iii) 与 (i) 等价.

综上知命题(i),(ii)与(iii)相互等价.
\end{frame}
\section{第一类Stirling数的指数型生成函数} 
\begin{frame}{第一类Stirling数的指数型生成函数}
	
	\begin{cor}
		$\{s(n,k)\}_{n=1}^{\infty}$的指数型生成函数是
		$$\frac{(\ln (1+x))^k}{k!}.$$
	\end{cor}
\pause
  \begin{proof}
  	借助两类Stirling数的联系进行证明.
  	对于 $n \in \mathbb{N}$, 令 $b_n=\delta(n, k), a_n=s(n, k)$, 则 $a_n=\sum_{i=0}^n s(n, i) b_i$. 令 $A(x), B(x)$ 分别表 示数列 $\left\{a_n\right\}_{n=0}^{\infty},\left\{b_n\right\}_{n=0}^{\infty}$ 的指数型生成函数. 显然 $B(x)=\frac{x^k}{k !}$, 由定理4.3知 $\left\{a_n\right\}_{n=0}^{\infty}$ 的指 数型生成函数为
  	$$
  	A(x)=B(\ln (1+x))=\frac{(\ln (1+x))^k}{k !}
  	$$
  \end{proof}
\end{frame}






	
	
	
	
	











   
   
   
   


\end{document}