\documentclass[mathserif]{beamer}  
 %%%=====For Chinese=====%%%
\usepackage{pptgeshi}  

%%%=====显示指令=====%%%
%\beamerdefaultoverlayspecification{<+->} %使用该命令,许多环境都会自动逐段显示


%%%=====标题信息=====%%%
\title
{第二章 \qquad 数列极限}
\author{}
\date{\sihao \S 2\qquad 收敛数列的性质}

%%%=====内容=====%%%
\begin{document}
	
%%%=============公式和文字之间的间距
	\setlength\abovedisplayskip{2pt}
	\setlength\belowdisplayskip{2pt}

%%%%%%%%%%%%%%%%%%%%%%%%%%%%%%%%%%%%%%%%标题
\begin{frame}
\Background
\titlepage  %标题页
\end{frame}



%%%%%%%%%%%%%%%%%%%%%%%%%%%%%%%%%%%%%%%%%%%目录
\begin{frame}
\tableofcontents
\end{frame}




\section{\S 2收敛数列的性质}


%%%%%%%%%%%%%%%%%%%%%%%%%%%%%%%%%%%%%%%%%%% 唯 一 性
\subsection{唯一性}
\begin{frame}
\frametitle{唯一性}
	\begin{thm}
		\suojin 若 $\left\{a_n\right\}$ 收敛,则它只有一个极限.
	\end{thm}
\pause 
\begin{proofs}
	\suojin 设 $a$ 是 $\left\{a_n\right\}$ 的一个极限.\ 下面证明对于任何定数 $b \neq a, b$ 不能是 $\left\{a_n\right\}$ 的极限.\\
	\suojin 若 $a, b$ 都是 $\left\{a_n\right\}$ 的极限,\ 则对于任何正数 $\varepsilon>0$,\ $\exists N_1$,\ 当 $n>N_1$ 时,\ 有
	$$\left|a_n-a\right|<\varepsilon.$$
	$\exists N_2$, 当 $n>N_2$ 时, 有
	$$\left|a_n-b\right|<\varepsilon . $$
\end{proofs}	
	
\end{frame}

\begin{frame}{唯一性}
\begin{proofs}
	\suojin 令 $N=\max \left\{N_1, N_2\right\}$, 当 $n>N$ 时 (1), (2) 同时成立, 从而有
	$$
	|a-b| \leq\left|a_n-a\right|+\left|a_n-b\right|<2 \varepsilon .
	$$
	\suojin 因为 $\varepsilon$ 是任意的, 所以 $a=b$. 
\end{proofs}  
\end{frame}

%%%%%%%%%%%%%%%%%%%%%%%--------------------   有 界 性   ----------------%
\subsection{有界性}

\begin{frame}
\frametitle{有界性}
	\begin{thm}
		\suojin 若数列 $\left\{a_n\right\}$ 收敛,\ 则 $\left\{a_n\right\}$ 为有界数列,\ 即存在 $M>0$,\ 使得 $\left|a_n\right| \leq M,\ n=1,2, \cdots$.
		\end{thm}
	\pause 
	\begin{proofs}
		\suojin 设 $\lim _{n \rightarrow \infty} a_n=a$, 对于正数 $\varepsilon=1, \exists N, n>N$ 时, 有\\ $\left|a_n-a\right|<1$, 从而 $|a_n|<|a|+1$.\\
	\suojin 若令 $M=\max \left\{\left|a_1\right|,\left|a_2\right|, \cdots,\left|a_N\right|,|a|+1\right\}$,\\
	则对一切正整数 $n$, 都有 $\left|a_n\right| \leq M$.
    \end{proofs}
    \pause 
	\begin{alertblock}{}
			\suojin \liang{注} 数列 $\left\{(-1)^n\right\}$ 是有界的, 但却不收敛. 这就说 明有界只是数列收敛的必要条件, 而不是充分条 件.
		\end{alertblock}
\end{frame}




%%%%%%%%%%%%%%%%%%%%%%%--------------------   保 号 性   ----------------%
\subsection{保号性}

\begin{frame}
\frametitle{保号性}
	\begin{thm}
		\suojin 设 $\lim _{n \rightarrow \infty} a_n=a$,\ 对于任意两个实数 $b, c$,\ $b<a<c$,\ 则存在 $N$,\ 当 $n>N$ 时,\ $b<a_n<c$.
		\end{thm}
	\pause 
   \begin{proofs}
	\suojin 取 $\varepsilon=\min \{a-b, c-a\}>0, \exists N$, 当 $n>N$ 时, $b \leq a-\varepsilon<a_n<a+\varepsilon \leq c$, 故 $b<a_n<c$. 
    \end{proofs}
    \pause 
	\begin{alertblock}{}
		\suojin \liang{注} 若 $a>0$ (或 $a<0)$, 我们可取 $b=\frac{a}{2}$ (或 $c=\frac{a}{2}$ ), 则 $a_n>\frac{a}{2}>0$ (或 $a_n<\frac{a}{2}<0$).
	\end{alertblock}
	这也是为什么称该定理为保号性定理的原因.  
\end{frame}






\begin{frame}{例子}%%%%
	\begin{ex}
			\suojin 证明 $\lim _{n \rightarrow \infty} \frac{1}{\sqrt[n]{n !}}=0$.
		\end{ex}
	\pause 
	\begin{proofs}
		\suojin 对任意正数 $\varepsilon$, 因为 $\lim _{n \rightarrow \infty} \frac{(1 / \varepsilon)^n}{n !}=0$, 所以由 定理 2.4, $\exists N>0$, 当 $n>N$ 时,
	$$
	\frac{(1 / \varepsilon)^n}{n !}<1, \text { 即 } \frac{1}{\sqrt[n]{n !}}<\varepsilon . 
	$$
	这就证明了 $\lim _{n \rightarrow \infty} \frac{1}{\sqrt[n]{n !}}=0$.
    \end{proofs}
\end{frame}





%%%%%%%%%%%%%%%%%%%%%%%--------------------   保不等式性   ----------------%
\subsection{保不等式性}

\begin{frame}
\frametitle{保不等式性}
	\begin{thm}
		\suojin 设 $\left\{a_n\right\},\left\{b_n\right\}$ 均为收敛数列,如果存在正数 $N_0$,当 $n>N_0$ 时, 有 $a_n \leq b_n$,则 $\lim _{n \rightarrow \infty} a_n \leq \lim _{n \rightarrow \infty} b_n$.
		\end{thm}
	\pause 
	\begin{proofs}
		\suojin 设 $\lim _{n \rightarrow \infty} a_n=a,\ \lim _{n \rightarrow \infty} b_n=b$. 若 $b<a$,取 $\varepsilon=\frac{a-b}{2}$,由保号性定理,存在 $N>N_0$,当 $n>N$ 时,
	$$
	a_n>a-\frac{a-b}{2}=\frac{a+b}{2},~ b_n<b+\frac{a-b}{2}=\frac{a+b}{2},
	$$
	故 $a_n>b_n$,导致矛盾. \\
	\suojin 所以 $a \leqslant b$.   
    \end{proofs}
\end{frame}



\begin{frame}{注释}%%%%
	\begin{alertblock}{}
	\suojin \liang{注} 若将定理 2.5 中的条件 $a_n \leq b_n$ 改为 $a_n<b_n$,也只能得到 $\lim _{n \rightarrow \infty} a_n \leq \lim _{n \rightarrow \infty} b_n$.
	\end{alertblock}
	\suojin {这就是说,即使条件是严格不等式,结论却不一定是严格不等式.}\\
	\vskip 0.5\baselineskip
	\suojin 例如:虽然 $\frac{1}{n}<\frac{2}{n}$, 但 $\lim _{n \rightarrow \infty} \frac{1}{n}=\lim _{n \rightarrow \infty} \frac{2}{n}=0$.
	
\end{frame}



%%%%%%%%%%%%%%%%%%%%%%%--------------------   迫敛性   ----------------%
\subsection{迫敛性}

\begin{frame}
\frametitle{迫敛性(夹逼原理)}
	\begin{thm}
	    \suojin 设数列 $\left\{a_n\right\},\ \left\{b_n\right\}$ 都以 $a$ 为极限,数列 $\left\{c_n\right\}$ 满足:\\
	    \suojin 存在 $N_0$,当 $n>N_0$ 时,有 $a_n \leq c_n \leq b_n$,则 $\left\{c_n\right\}$ 收敛,且 $\lim _{n \rightarrow \infty} c_n=a$.
		\end{thm}
\end{frame}



\begin{frame}
	\frametitle{迫敛性(夹逼原理)的证明}
	\begin{proofs}
		\suojin   对任意正数 $\varepsilon$,因为 $\lim _{n \rightarrow \infty} a_n=\lim _{\substack{n \rightarrow \infty}} b_n=a$,所以分 别存在 $N_1, N_2$,使得
		\begin{itemize}
			\item[] 当 $n>N_1$ 时,$a-\varepsilon<a_n$;
			\item[] 当 $n>N_2$ 时,$b_n<a+\varepsilon$. 
		\end{itemize}
		\suojin 取 $N=\max \left\{N_0, N_1, N_2\right\}$,当 $n>N$ 时,
		$$a-\varepsilon<a_n \leq c_n \leq b_n<a+\varepsilon.$$
		\suojin 这就证得  
		$$
		\lim _{n \rightarrow \infty} c_n=a. 
		$$
	\end{proofs}
\end{frame}





\begin{frame}{例子}%%%%
	\begin{ex}
		\suojin 求数列 $\{\sqrt[n]{n}\}$ 的极限.   
	\end{ex}
\begin{jie}
	\suojin  设 $h_n=\sqrt[n]{n}-1 \geqslant 0$,则有
	$$
	n=\left(1+h_n\right)^n \geq \frac{n(n-1)}{2} h_n^2(n \geq 2),
	$$
	故 $1 \leq \sqrt[n]{n}=1+h_n \leq 1+\sqrt{\frac{2}{n-1}}$.\ 又因
	$$
	\lim _{n \rightarrow \infty} 1=\lim _{n \rightarrow \infty}\left(1+\sqrt{\frac{2}{n-1}}\right)=1 \text {,}
	$$
	所以由迫敛性,求得 $\lim _{n \rightarrow \infty} \sqrt[n]{n}=1$. 
\end{jie}
\end{frame}





%%%%%%%%%%%%%%%%%%%%%%%--------------------   四则运算法则   ----------------%
\subsection{四则运算法则}

\begin{frame}[label=szys]
\frametitle{四则运算法则\hfill\hyperlink{szyszm<1>}{\beamergotobutton{证明过程}}}
	\begin{thm}
	\suojin 若 $\left\{a_n\right\}$ 与 $\left\{b_n\right\}$ 为收敛数列,则数列 $$\left\{a_n+b_n\right\},\ \left\{a_n-b_n\right\},\ \left\{a_n \cdot b_n\right\}$$ 
	也都是收敛数列,且有
	\begin{enumerate}
		\item[{\color{black}$(1)$}] $\lim _{n \rightarrow \infty}\left(a_n \pm b_n\right)=\lim _{n \rightarrow \infty} a_n \pm \lim _{n \rightarrow \infty} b_n$;
		\item[{\color{black}$(2)$}]  $\lim _{n \rightarrow \infty}\left(a_n \cdot b_n\right)=\lim _{n \rightarrow \infty} a_n \cdot \lim _{n \rightarrow \infty} b_n$,当 $b_n$ 为常数 $c$ 时,$\lim _{n \rightarrow \infty} c b_n=c \lim _{n \rightarrow \infty} b_n;$
		\item[{\color{black}$(3)$}] 若 $b_n \neq 0, \lim _{n \rightarrow \infty} b_n \neq 0$,则 $\left\{\frac{a_n}{b_n}\right\}$ 也收敛,且 $\lim _{n \rightarrow \infty} \frac{a_n}{b_n}=\lim _{n \rightarrow \infty} a_n / \lim _{n \rightarrow \infty} b_n$. 
	\end{enumerate}
	\end{thm}
\end{frame}



\begin{frame}{例子}%%%%
	\begin{ex}
		\suojin 用四则运算法则计算
		$$
		\lim _{n \rightarrow \infty} \frac{a_m n^m+a_{m-1} n^{m-1}+\cdots+a_1 n+a_0}{b_k n^k+b_{k-1} n^{k-1}+\cdots+b_1 n+b_0}
		$$
		其中 $m \leq k,\  a_m b_k \neq 0$.
	\end{ex}
	
\end{frame}



\begin{frame}{例子解答}%%%%
	\begin{jie} 
		\suojin 依据 $\lim _{n \rightarrow \infty} \frac{1}{n^\alpha}=0(\alpha>0)$,分别得出:\\
		\suojin 
		(1) 当 $m=k$ 时,有  
		$$
		\begin{aligned}
			& \lim _{n \rightarrow \infty} \frac{a_m n^m+a_{m-1} n^{m-1}+\cdots+a_1 n+a_0}{b_k n^k+b_{k-1} n^{k-1}+\cdots+b_1 n+b_0} \\
			& \quad=\lim _{n \rightarrow \infty} \frac{a_m+a_{m-1} \frac{1}{n}+\cdots+a_1 \frac{1}{n^{m-1}}+a_0 \frac{1}{b^m}}{b_m+b_{m-1} \frac{1}{n}+\cdots+b_1 \frac{1}{n^{m-1}}+b_0 \frac{1}{n^m}} \\
			& \quad=\frac{a_m}{b_m} .
		\end{aligned}$$		
	\end{jie}
	
\end{frame}


\begin{frame}{例子解答}%%%%
	\begin{jie}
		\suojin (2) 当 $m<k$ 时,有  
		$$
		\begin{aligned}
			& \lim _{n \rightarrow \infty} \frac{a_m n^m+a_{m-1} n^{m-1}+\cdots+a_1 n+a_0}{b_k n^k+b_{k-1} n^{k-1}+\cdots+b_1 n+b_0} \\
			& =\lim _{n \rightarrow \infty} \frac{1}{n^{k-m}} \cdot \lim _{n \rightarrow \infty} \frac{a_m+a_{m-1} \frac{1}{n}+\cdots+a_1 \frac{1}{n^{m-1}}+a_0 \frac{1}{n^m}}{b_k+b_{k-1} \frac{1}{n}+\cdots+b_1 \frac{1}{n^{k-1}}+b_0 \frac{1}{n^k}} \\
			& =0 \cdot \frac{a_m}{b_k}=0 .
		\end{aligned}
		$$
		\suojin 所以
		$$
		\text { 原式 }=\left\{
			\begin{aligned}
				\frac{a_m}{b_m}, & m=k, \\
				0, & m<k.
			\end{aligned}
		\right.
		$$  
		\end{jie}
	\end{frame}






%--------------------   例子   --------------------------%
\subsection{其他运算法则}
\begin{frame}[label=jsbuchong]
\frametitle{运算的补充(例子)}
\begin{prop}
\suojin 数列$\{a_n\}$收敛,则$\{|a_n|\}$也收敛,且
$$\lim_{n\rightarrow\infty}|a_n|=|\lim_{n\rightarrow\infty}a_n|\ .$$
\end{prop}

\hfill\hyperlink{jdzjszm<1>}{\beamergotobutton{证明过程}}
\jiange
  
\begin{prop}
\suojin 数列$\{a_n\}(a_n\geqslant 0)$收敛,则$\{\sqrt{a_n}\}$也收敛,且
$$\lim_{n\rightarrow\infty}\sqrt{a_n}=\sqrt{\lim_{n\rightarrow\infty}a_n}\ .$$
\end{prop}

 \hfill\hyperlink{kgjszm<1>}{\beamergotobutton{证明过程}}
\end{frame}




%---------------------   例题   ---------------------------%
\subsection*{一些例子}

\begin{frame}{一些例子}%%%%
	\begin{ex}
		\suojin 设 $a_n \geq 0$,$\lim _{n \rightarrow \infty} a_n=a>0$,求证 $\lim _{n \rightarrow \infty} \sqrt[n]{a_n}=1$.
		\end{ex}
	\pause 
	\begin{proofs}
		\suojin 因为 $\lim _{n \rightarrow \infty} a_n=a>0$,根据极限的保号性,存在 $N$, 当 $n>N$ 时,有 $\frac{a}{2}<a_n<\frac{3 a}{2}$,即
	$$
	\sqrt[n]{\frac{a}{2}}<\sqrt[n]{a_n}<\sqrt[n]{\frac{3 a}{2}}.
	$$
	\suojin 又因为 $\lim _{n \rightarrow \infty} \sqrt[n]{\frac{a}{2}}=\lim _{n \rightarrow \infty} \sqrt[n]{\frac{3 a}{2}}=1$,所以由极限的迫敛性,证得 $\lim _{n \rightarrow \infty} \sqrt[n]{a_n}=1$.  
\end{proofs} 
\end{frame}



\begin{frame}{一些例子}%%%%
	\begin{ex}
		\suojin 求极限 $\lim _{n \rightarrow \infty} \frac{a^n}{1+a^n}(a \neq-1)$.
	\end{ex} 
\pause 
	\begin{jie}
		\begin{itemize}
			\item[(1)]  当$|a|<1$,因为 $\lim _{n \rightarrow \infty} a^n=0$,所以由极限四则运算法则,得 $\lim _{n \rightarrow \infty} \frac{a^n}{1+a^n}=\frac{\lim _{n \rightarrow \infty} a^n}{1+\lim _{n \rightarrow \infty} a^n}=0$.
			\item[(2)] 当$a=1, \lim _{n \rightarrow \infty} \frac{a^n}{1+a^n}=\lim _{n \rightarrow \infty} \frac{1}{2}=\frac{1}{2}$.
			\item[(3)] 当$|a|>1$,因 $\lim _{n \rightarrow \infty}\left(1 / a^n\right)=0$,故得
			$$
			\lim _{n \rightarrow \infty} \frac{a^n}{1+a^n}=\lim _{n \rightarrow \infty} \frac{1}{1+1 / a^n}=\frac{1}{1+\lim _{n \rightarrow \infty}\left(1 / a^n\right)}=1 .
			$$   
		\end{itemize}
	\end{jie}
\end{frame}




\begin{frame}{一些例子}%%%%
	\begin{ex}
		\suojin 设 $a_1, a_2, \cdots, a_m$ 为 $m$ 个正数, 证明
			$$
			\lim _{n \rightarrow \infty} \sqrt[n]{a_1{ }^n+a_2{ }^n+\cdots+a_m{ }^n}=\max \left\{a_1, a_2, \cdots, a_m\right\} .
			$$
	\end{ex}
\pause
	\begin{proofs}    
	\suojin 设 $a=\max \left\{a_1, a_2, \cdots, a_m\right\}$.\ 由
	$$
	\begin{gathered}
			a \leq \sqrt[n]{a_1{ }^n+a_2{ }^n+\cdots+a_m{ }^n} \leq \sqrt[n]{m} a, \\
			\lim _{n \rightarrow \infty} \sqrt[n]{m} a=\lim _{n \rightarrow \infty} a=a,
		\end{gathered}
	$$
	以及极限的迫敛性,可得
	$$
	\lim _{n \rightarrow \infty} \sqrt[n]{a_1{ }^n+a_2{ }^n+\cdots+a_m{ }^n}=a=\max \left\{a_1, a_2, \cdots, a_m\right\} .
	$$
	\end{proofs}
\end{frame}


%--------------------   子列   ----------------%
\subsection{子列}

\begin{frame}
\frametitle{子列}
	\begin{dfn}
		\suojin 设 $\left\{a_n\right\}$ 为数列,$\left\{n_k\right\}$ 为 $\N_{+}$的无限子集,且
			$$
			n_1<n_2<\cdots<n_k<\cdots,
			$$
			则数列
			$$
			a_{n_1}, a_{n_2}, \cdots, a_{n_k}, \cdots
			$$
			称为 $\left\{a_n\right\}$ 的子列,简记为 $\left\{a_{n_k}\right\}$.
	\end{dfn}
\pause
	\begin{alertblock}{}
		\suojin \liang{注} 由定义,$\left\{a_n\right\}$ 的子列 $\left\{a_{n_k}\right\}$ 的各项均选自 $\left\{a_n\right\}$,且保持这些项在 $\left\{a_n\right\}$ 中的先后次序.\\ 
		\suojin $\left\{a_{n_k}\right\}$ 中的第 $k$ 项是 $\left\{a_n\right\}$ 中的第 $n_k$ 项,故总有 $n_k \geqslant k$. 
	\end{alertblock}
	
\end{frame}




\begin{frame}{子列的性质}%%%%
	\begin{thm}
		\suojin 若数列 $\left\{a_n\right\}$ 收敛到 $a$,则 $\left\{a_n\right\}$ 的任意子列 $\left\{a_{n_k}\right\}$ 也收敛到 $a$.
		\end{thm}  
	\pause 
	\begin{proofs}
		\suojin 设 $\lim _{n \rightarrow \infty} a_n=a$. 则 $\forall \varepsilon>0,\ \exists N$,当 $n>N,\left|a_n-a\right|<\varepsilon$\jh \\
	\suojin 设 $\left\{a_{n_k}\right\}$ 是 $\left\{a_n\right\}$ 的任意一个子列\jh 
	由于 $n_k \geqslant k$,因此 $k>N$ 时,$n_k \geqslant k>N$,亦有 $\left|a_{n_k}-a\right|<\varepsilon$. 这就证明了
	$$
	\lim _{k \rightarrow \infty} a_{n_k}=a . 
	$$
	\end{proofs}
\pause
	\begin{alertblock}{}
		\suojin \liang{注} 由定理 $2. 8$ 可知,若一个数列的两个子列收敛于不同的值,则此数列必发散.
	\end{alertblock}
\end{frame}





\begin{frame}{例子}%%%%
	\begin{ex}
		\suojin 求证 $\lim _{n \rightarrow \infty} a_n=a$ 的充要条件是
			$$
			\lim _{n \rightarrow \infty} a_{2 n-1}=\lim _{n \rightarrow \infty} a_{2 n}=a . 
			$$
	\end{ex}
	\pause 
	\begin{proofs}[必要性]
		\suojin 设 $\lim _{n \rightarrow \infty} a_n=a$,则 $\forall \varepsilon>0, \exists N, n>N$ 时,
	\benas
	\left|a_n-a\right|<\varepsilon
	\eenas
	因为 $2 n>N,\  2 n-1 \geq N$,所以
	\benas
	\left|a_{2 n-1}-a\right|<\varepsilon, \quad\left|a_{2 n}-a\right|<\varepsilon .
	\eenas
	从而必要性得证.
    \end{proofs}
	
\end{frame}




\begin{frame}{例子}%%%%
	\begin{proofs}[充分性]
	\suojin 设 $\lim _{k \rightarrow \infty} a_{2 k+1}=\lim _{k \rightarrow \infty} a_{2 k}=a$,则 $\forall \varepsilon>0, \exists N$,当 $k>N$ 时,    $$
	\left|a_{2 k-1}-a\right|<\varepsilon, \quad\left|a_{2 k}-a\right|<\varepsilon .
	$$
	\suojin 令 $N=2 K$,当 $n>N$ 时,则有
	$$
	\left|a_n-a\right|<\varepsilon
	$$
	所以 $\lim _{n \rightarrow \infty} a_n=a$.
   \end{proofs}
\end{frame}



\begin{frame}{例子}%%%%
	\begin{ex}
		\suojin 若 $a_n=(-1)^n\left(1-\frac{1}{n}\right)$\jh 证明数列 $\left\{a_n\right\}$ 发散.
	\end{ex}
	\pause 
\begin{proofs}
\suojin 显然
	$$
	\begin{gathered}
			\lim _{k \rightarrow \infty} a_{2 k-1}=\lim _{k \rightarrow \infty}-\left(1-\frac{1}{2 k-1}\right)=-1 ; \\
			\lim _{k \rightarrow \infty} a_{2 k}=\lim _{k \rightarrow \infty}\left(1-\frac{1}{2 k}\right)=1 .
		\end{gathered}
	$$
	因此,数列 $\left\{a_n\right\}$ 发散.  
\end{proofs} 
\end{frame}



%
%\begin{frame}{ 复习思考题}%%%%
%	\begin{itemize}
%		\item[1.\ ] 极限的保号性与保不等式性有什么不同?
%		\item[2.\ ] 仿效例题的证法,证明: 若 $\left\{a_n\right\}$ 为正有界数列,则
%		$$
%		\lim _{n \rightarrow \infty} \sqrt[n]{a_1^n+a_2^n+\cdots+a_n^n}=\sup \left\{a_n\right\} .
%		$$   
%	\end{itemize}
%\end{frame}




%%%%%%%%%%%%%%%%%%%%%%%-----------------------------作业
\subsection*{作业}

\begin{frame}
	\frametitle{作业:}
	\begin{itemize}
		\item P31\quad 习题2.2
		\item[]
		\begin{center}1,2,3,4,9,10.\end{center}
	\end{itemize}
\end{frame}






%%%%%%%%%%%%%%%%%%%%%%%%%%%%%%%%%%%%%%定理和例题的证明过程
\subsection*{定理和例题的证明过程}
%---------------------   四则运算法则的证明   ---------------------%

\begin{frame}[label=szyszm]{四则运算法则的证明\hfill\hyperlink{szys<1>}{\beamergotobutton{返回定理}}}
	\suojin \zheng \liang{(1)} 设 $\lim _{n \rightarrow \infty} a_n=a, \lim _{n \rightarrow \infty} b_n=b, \forall \varepsilon>0$,存在 $N$,当 $n>N$ 时,有 $\left|a_n-a\right|<\varepsilon,\left|b_n-b\right|<\varepsilon$,所以
	$$
	\left|a_n \pm b_n-(a \pm b)\right| \leq\left|a_n-a\right|+\left|b_n-b\right|<2 \varepsilon,
	$$
	由 $\varepsilon$ 的任意性,得到
	$$
	\lim _{n \rightarrow \infty}\left(a_n \pm b_n\right)=a \pm b=\lim _{n \rightarrow \infty} a_n \pm \lim _{n \rightarrow \infty} b_n .
	$$
	\suojin \liang{(2)}  因 $\left\{b_n\right\}$ 收敛,故 $\left\{b_n\right\}$ 有界,设 $\left|b_n\right| \leq M$.
	对于任意 $\varepsilon>0$,当 $n>N$ 时,有
	$$
	\left|a_n-a\right|<\frac{\varepsilon}{M+1},\left|b_n-b\right|<\frac{\varepsilon}{|a|+1},
	$$
	
\end{frame}


\begin{frame}{四则运算法则的证明\hfill\hyperlink{szys<1>}{\beamergotobutton{返回定理}}}%%%%
	于是
	$$
	\begin{aligned}
		\left|a_n b_n-a b\right| & =\left|a_n b_n-a b_n+a b_n-a b\right| \\
		& \leq\left|b_n\right|\left|a_n-a\right|+|a|\left|b_n-b\right|<2 \varepsilon
	\end{aligned}
	$$
	由 $\varepsilon$ 的任意性,证得
	$$
	\lim _{n \rightarrow \infty} a_n b_n=a b=\lim _{n \rightarrow \infty} a_n \lim _{n \rightarrow \infty} b_n .
	$$
	\suojin  \liang{(3)} 因为 $\frac{a_n}{b_n}=a_n \cdot \frac{1}{b_n}$,由(2),只要证明
	$$
	\lim _{n \rightarrow \infty} \frac{1}{b_n}=\frac{1}{\lim\limits_{n \rightarrow \infty} b_n} .
	$$
\end{frame}


\begin{frame}{四则运算法则的证明\hfill\hyperlink{szys<1>}{\beamergotobutton{返回定理}}}%%%%
	\suojin 由于 $b \neq 0$,据保号性,$\exists N_1$,当 $n>N_1$ 时, 
	$$
	\left|b_n\right|>\frac{|b|}{2} \text {. }
	$$
	又因为 $\lim _{n \rightarrow \infty} b_n=b, \exists N_2$,当 $n>N_2$ 时,
	$$
	\left|b_n-b\right|<\frac{|b|^2}{2} \varepsilon
	$$
	取 $N=\max \left\{N_1, N_2\right\}$,当 $n>N$ 时,
	$$
	\left|\frac{1}{b_n}-\frac{1}{b}\right|=\left|\frac{b_n-b}{b_n b}\right| \leq \frac{2}{|b|^2}\left|b_n-b\right| \leq \varepsilon,
	$$
	即 $\lim _{n \rightarrow \infty} \frac{1}{b_n}=\frac{1}{b}$. 所以 $\lim _{n \rightarrow \infty} \frac{a_n}{b_n}=\frac{\lim \limits_{n \rightarrow \infty} a_n}{\lim \limits_{n \rightarrow \infty} b_n}$.   
\end{frame}




%--------------------------  计算法则补充证明  ------------------------%


\begin{frame}[label=jdzjszm]
	\frametitle{绝对值性质的证明\hfill\hyperlink{jsbuchong<1>}{\beamergotobutton{返回性质}}}
	\begin{proofs}
		\suojin 设$a=\lim_{n\rightarrow\infty}a_n$\jh 则有对于任意的$\varepsilon>0$,存在$N\in\N$,使得当$n>N$,有
		$$\left|a_{n}-a\right|<\varepsilon,$$
		此时,有
		$$\left|\left|a_{n}\right|-\left|a\right|\right|\leqslant \left|a_{n}-a\right|<\varepsilon.$$
		从而可知
		$$\lim_{n\rightarrow\infty}|a_n|=\left|a\right|=|\lim_{n\rightarrow\infty}a_n|\ .$$
	\end{proofs}
\end{frame}




\begin{frame}[label=kgjszm]
	\frametitle{开根性质的证明\hfill\hyperlink{jsbuchong<1>}{\beamergotobutton{返回性质}}}
	\begin{proofs}
		\suojin 设$a=\lim_{n\rightarrow\infty}a_n$\jh 由于 $a_n \geqslant 0$,根据极限的保不等式性,有 $a \geqslant 0$. \\
		\suojin 对于任意 $\varepsilon>0,\ \exists N$,当 $n>N$ 时,$\left|a_n-a\right|<\varepsilon$. 于是可得:
		\begin{itemize}
			\item[(1)] $a=0$ 时,有 $\left|\sqrt{a_n}-\right|=\sqrt{a_n}<\sqrt{\varepsilon}$;
			\item[(2)] $a>0$ 时,有
			$$
			\left|\sqrt{a_n}-\sqrt{a}\right|=\frac{\left|a_n-a\right|}{\sqrt{a_n}+\sqrt{a}} \leq \frac{\left|a_n-a\right|}{\sqrt{a}} \leq \frac{\varepsilon}{\sqrt{a}} .
			$$
		\end{itemize}
		
		故 $\lim _{n \rightarrow \infty} \sqrt{a_n}=\sqrt{a}$ 得证. 
	\end{proofs}
\end{frame}





\end{document} 

