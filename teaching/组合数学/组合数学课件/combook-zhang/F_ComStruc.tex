\chapter{基本组合数}
\label{zuhejiegou} \minitoc



%%%%%%%%%%%%%%%%%%%%%%%%%%%%%%%%%%%%%%%%%%%%%%%%%%%%%%%%%%%%%%%%%%%%%%%%%%%%%%%%%%
\section{二项式系数}

二项式系数${n \choose
k}$计数了$n$元集合的$k$元子集的个数,在组合中具有十分重要的作用。
它的很多好的性质表现在各个组合恒等式中,
例如二项式定理,也正是源于此,它得名二项式系数。在本章中,我们将讨
论一下关于它的一些基本性质和恒等式的证明。

\subsection{帕斯卡(Pascal)公式}
对于非负整数$n$, $k$, 二项式系数${n \choose
k}$计数了$n$元集合的$k$元子集的个数,于是若 $k>n$, 则${n \choose
k}=0$,且对任意的$n$,均有${n \choose 0}=1$,若$n>0$,$1\leq k \leq
n$,则
\begin{equation}\label{e1}
{n \choose k}=\frac{n!}{k!(n-k)!}=\frac{n(n-1)\cdots
(n-k+1)}{k(k-1)\cdots 1}.
\end{equation}
\begin{thm}
(帕斯卡公式). 对于所有满足$1\leq k\leq n-1$ 的整数 $n$及$k$,均有
\begin{equation}\label{e2}
{n\choose k}={n-1 \choose k}+{n-1\choose k-1}.
\end{equation}
\end{thm}
\noindent{\textbf{证明}:}
首先可直接利用公式(\ref{e1})展开等式的两边即可得到,读者可以自己验算一下。
下面我们介绍一种组合方法,设$S$是一个$n$元集合,$x$为其中的一个元素,下面我们将集合$S$的$k(k\leq
n-1)$元 子集$X$分两种情况进行讨论。

情形1: $x\in
X$,则还需在除$x$外的$n-1$元集合中取$k-1$元子集作为$X$中的元素,共有${n-1
\choose k-1}$种方法;

情形2:$x\overline{\in}
X$,则需在除$x$外的$n-1$元集合中取$k$元子集作为$X$中的元素,共有${n-1
\choose k}$种方法。

从而,由加法原理,$S$ 的$k(k\leq n-1)$元子集 $X$ 共有${n-1 \choose
k-1}+{n-1 \choose k}$个,即
$${n\choose k}={n-1 \choose k}+{n-1\choose k-1}.$$
\qed
\begin{exa}
令$n=4$,$k=3$,$S=\{x,a,b,c\}$,于是属于情形$1$的$3$元子集为$$\{x,a,b\},\
\{x,a,c\},\
\{x,b,c\}.$$此可视为集合$\{a,b,c\}$的$2$元子集。属于情形$2$的$3$元子集为$$\{a,b,c\}.$$
此可视为集合集合$\{a,b,c\}$的$3$元子集。从而$${4 \choose
3}=4=3+1={3\choose 2}+{3\choose 3}.$$
\end{exa}

由递推公式(\ref{e2}),及初始条件$${n \choose 0}=1,\ {n\choose n}=1,
\ \ \ \ \ \ (n\geq
0)$$我们不需要利用公式(\ref{e1})即可计算出二项式系数。由此种方法,
我们在计算二项式系数的过程中经常以帕斯卡三角的形式显示出来,如下图所示:

\begin{figure}[ht] \begin{picture}(17,38)(-200,0)
\setlength{\unitlength}{0.4cm}
\put(-14.5,0){\line(1,0){27}}\put(-14.5,1.5){\line(1,0){27}}
\put(-14.5,-16.5){\line(1,0){27}} \put(-12,1.5){\line(0,-1){18}}
\put(-10,1.5){\line(0,-1){18}} \put(-8,1.5){\line(0,-1){18}}
\put(-6,1.5){\line(0,-1){18}} \put(-4,1.5){\line(0,-1){18}}
\put(-2,1.5){\line(0,-1){18}} \put(0,1.5){\line(0,-1){18}}
\put(2,1.5){\line(0,-1){18}} \put(4,1.5){\line(0,-1){18}}
\put(6,1.5){\line(0,-1){18}} \put(8,1.5){\line(0,-1){18}}
\put(-14,0.2){$n/k$}\put(-11.3,0.2){$0$}\put(-9.3,0.2){$1$}\put(-7.3,0.2){$2$}\put(-5.3,0.2){$3$}\put(-3.3,0.2){$4$}
\put(-1.3,0.2){$5$}\put(0.7,0.2){$6$}\put(2.7,0.2){$7$}\put(4.7,0.2){$8$}\put(6.7,0.2){$9$}\put(9.7,0.2){$\cdots$}
\put(-13.3,-1.3){$0$}\put(-11.3,-1.3){$1$}
\put(-13.3,-2.8){$1$}\put(-11.3,-2.8){$1$}\put(-9.3,-2.8){$1$}
\put(-13.3,-4.3){$2$}\put(-11.3,-4.3){$1$}\put(-9.3,-4.3){$2$}\put(-7.3,-4.3){$1$}
\put(-13.3,-5.8){$3$}\put(-11.3,-5.8){$1$}\put(-9.3,-5.8){$3$}\put(-7.3,-5.8){$3$}\put(-5.3,-5.8){$1$}
\put(-13.3,-7.3){$4$}\put(-11.3,-7.3){$1$}\put(-9.3,-7.3){$4$}\put(-7.3,-7.3){$6$}\put(-5.3,-7.3){$4$}
\put(-3.3,-7.3){$1$}
\put(-13.3,-8.8){$5$}\put(-11.3,-8.8){$1$}\put(-9.3,-8.8){$5$}\put(-7.3,-8.8){$10$}\put(-5.3,-8.8){$10$}
\put(-3.3,-8.8){$5$}\put(-1.3,-8.8){$1$}
\put(-13.3,-10.3){$6$}\put(-11.3,-10.3){$1$}\put(-9.3,-10.3){$6$}\put(-7.3,-10.3){$15$}\put(-5.3,-10.3){$20$}
\put(-3.3,-10.3){$15$}\put(-1.3,-10.3){$6$}\put(0.7,-10.3){$1$}

\put(-13.3,-11.8){$7$}\put(-11.3,-11.8){$1$}\put(-9.3,-11.8){$7$}\put(-7.3,-11.8){$21$}\put(-5.3,-11.8){$35$}
\put(-3.3,-11.8){$35$}\put(-1.3,-11.8){$21$}\put(0.7,-11.8){$7$}\put(2.7,-11.8){$1$}
\put(-13.3,-13.3){$8$}\put(-11.3,-13.3){$1$}\put(-9.3,-13.3){$8$}\put(-7.3,-13.3){$28$}\put(-5.3,-13.3){$56$}
\put(-3.3,-13.3){$70$}\put(-1.3,-13.3){$56$}\put(0.7,-13.3){$28$}\put(2.7,-13.3){$8$}\put(4.7,-13.3){$1$}
\put(-13.2,-15.3){$\vdots$}\put(-11.2,-15.3){$\vdots$}\put(-9.2,-15.3){$\vdots$}\put(-7.2,-15.3){$\vdots$}
\put(-5.2,-15.3){$\vdots$}\put(-3.2,-15.3){$\vdots$}\put(-1.2,-15.3){$\vdots$}\put(0.8,-15.3){$\vdots$}
\put(2.8,-15.3){$\vdots$}\put(4.8,-15.3){$\vdots$}\put(6.8,-15.3){$\vdots$}
\end{picture}
\vspace{6.5cm} \caption{帕斯卡三角.} \label{pascal}
\end{figure}

图中除了最左侧一列的$1$以外,其余的值都可以通过上行中同列及其左邻的值之和。例如,对$n=6$,我们有
$${6 \choose 3}=20=10+10={5 \choose 3}+{5 \choose 2}.$$

二项式系数的许多性质和恒等式均可以通过帕斯卡三角得到,将帕斯卡三角某行的元素加起来可发现,
$${n \choose 0}+{n\choose 1}+\cdots+{n\choose n}=2^n.$$

下面给出帕斯卡三角的另一种组合解释,令$n$,$k$为满足$0\leq k\leq
n$的非负整数,定义$p(n,k)$计数了帕斯卡三角中左上角(数值${0 \choose
0}=1$)到数值${n\choose
k}$的路的条数,其中路的每一步均为向南走一个单位或向东南方向走一个单位,即按向量方向$(0,-1)$和$(1,-1)$走一步。
\begin{figure}[ht]
\begin{picture}(20,20)(20,0)
\setlength{\unitlength}{0.5cm} \thicklines
\put(10,0){\line(0,-1){3}} \put(10,0){\circle*{0.2}}
\put(10,-3){\circle*{0.2}} \put(10,-1,4){\vector(0,-1){0.4}}
\put(18,0){\circle*{0.2}}\put(21,-3){\circle*{0.2}}\put(21,0){\circle*{0.2}}
\put(18,0){\line(1,-1){3}} \put(19.4,-1,4){\vector(1,-1){0.4}}
\end{picture}
\vspace{2cm} \caption{步子}\label{path}
\end{figure}

我们约定$p(0,0)=1$,且对任意的非负整数$n$,均有
$p(n,0)=1$,(每一步都必须朝下走一直到${n\choose
0}$)及$p(n,n)=1$。(每一步都不需沿对角线走直到${n\choose n}$)
注意到每一条从${0\choose 0}$到${n\choose k}$的路均可看为是

(i)从${0\choose 0}$到${n-1\choose k}$的路再加上一个竖直步子,

或

(ii)从${0\choose 0}$到${n-1\choose k-1}$的路再加上一个对角步子。

从而,由加法原理,我们有递推关系
$$p(n,k)=p(n-1,k)+p(n-1,k-1).$$
观察到$p(n,k)$与二项式系数有相同的初始条件及递推关系,故而易知对任意满足$0\leq
k\leq n$的非负整数$n$,$k$,有
$$p(n,k)={n\choose k}.$$
于是帕斯卡三角的各数值也表示从左上角到该数值的路的条数。这也给了二项式系数以新的组合解释。


%%%%%%%%%%%%%%%%%%%%%%%%%%%%%%%%%%%%%%%%%%%%%%%%%%
\subsection{二项式定理}
二项式系数是从二项式定理中得名的,本节中将介绍有关二项式定理的恒等式,它作为代数恒等式我们在高中就已经接触过。
\begin{thm}
设$n$是正整数,则对任意的实数$x$,$y$,均有
\begin{equation}\label{e3}
(x+y)^n=\sum_{k=0}^n{n\choose k}x^{n-k}y^k.
\end{equation}
\end{thm}
\noindent{\textbf{证明}:}(方法1)将$(x+y)^n$写成$n$个因子的乘积$$(x+y)(x+y)\cdots(x+y).$$
由乘法的分配律,展开乘积并合并同类项,由于每一个因子$(x+y)$,我们都有$x$,$y$两种选择,所以展开式中共有$2^n$项,
且每项都是$x^{n-k}y^k$($k=0,1,2,\ldots,n$)的形式。项$x^{n-k}y^k$是在$n$个因子中$k$个选取$y$,其余$n-k$个因子中取$x$,
于是$x^{n-k}y^k$在展开式中出现的次数为$x^{n-k}y^k$,即展开式中项$x^{n-k}y^k$的系数为${n\choose
k}$,从而
$$(x+y)^n=\sum_{k=0}^n{n\choose k}x^{n-k}y^k.$$
\qed
\noindent{\textbf{证明}:}(方法2)对$n$进行数学归纳法。对$n=1$,式(\ref{e3})为$$(x+y)^1=\sum_{k=0}^1{1\choose
k}x^{1-k}y^k=x+y,$$
显然成立。假设式(\ref{e3})对整数$n$成立,即$(x+y)^n=\sum_{k=0}^n{n\choose
k}x^{n-k}y^k$,下考虑$n+1$的情形,
\begin{align*}
(x+y)^{n+1}&=(x+y)(x+y)^n=(x+y)\left(\sum_{k=0}^n{n\choose k}x^{n-k}y^k\right)\\
&=x\left(\sum_{k=0}^n{n\choose k}x^{n-k}y^k\right)+y\left(\sum_{k=0}^n{n\choose k}x^{n-k}y^k\right)\\
&=\sum_{k=0}^n{n\choose k}x^{n+1-k}y^k+\sum_{k=0}^n{n\choose k}x^{n-k}y^{k+1}\\
&={n\choose 0}x^{n+1}+\sum_{k=1}^n{n\choose
k}x^{n+1-k}y^{k}+\sum_{k=0}^{n-1}{n\choose
k}x^{n-k}y^{k+1}+{n\choose n}y^{n+1}
\end{align*}
在后一个连加项中,用$k-1$代替$k$,得到
$$\sum_{k=1}^n{n\choose k-1}x^{n-k+1}y^k.$$
从而$$(x+y)^{n+1}=x^{n+1}+\sum_{k=1}^n\left[{n\choose k}+{n\choose
k-1}\right]x^{n+1-k}y^k+y^{n+1},$$
利用帕斯卡公式,上式等价于$$(x+y)^{n+1}=x^{n+1}+\sum_{k=1}^n{n+1\choose
k}x^{n+1-k}y^k+y^{n+1}=\sum_{k=0}^{n+1}{n+1\choose
k}x^{n+1-k}y^k,$$恰满足式(\ref{e3}),由归纳法原理定理得证。 \qed

一般情况下,我们常常利用到一种特殊情形,即当$y=1$时,我们有如下推论。
\begin{coro}
令$n$是正整数,则对任意的实数$x$均有$$(1+x)^n=\sum_{k=0}^n{n\choose
k}x^k=\sum_{k=0}^n{n\choose n-k}x^k.$$
\end{coro}

%%%%%%%%%%%%%%%%%%%%%%%%%%%%%%%%%%%%%%%%%%%%%%%
\subsection{恒等式}
本节中,我们将考虑一些关于二项式系数的恒等式并给出它们的组合解释。首先由二项式系数的代数展开式,很快地,我们有
\begin{equation}\label{e4}
k{n\choose k}=n{n-1 \choose k-1}.
\end{equation}
作为代数式,我们很容易根据二项式的表达式证明,但是作为组合恒等式,式(\ref{e4})又有怎么样的组合意义呢?

我们考虑这样一个实际问题,从$n$个人中选出$k$个人组成一个足球队,并选出队长,完成这项事件共有多少种选择方案?
一种计数方法是我们先选出足球队,则有${n\choose
k}$种,再在这选出的$k$个人中选出队长,有$k$种,于是由乘法原理完成这项事件共有$k{n\choose
k}$种选择方案。
另一种计数方法是先从$n$个人中选出队长,有$n$种选择,再在剩下的$n-1$个人中选出$k-1$个人与队长一起组成足球队,有${n-1\choose
k-1}$种选择,于是由乘法原理完成这项事件共有$n{n-1 \choose
k-1}$种选择方案。
于是式(\ref{e4})两边计数的是同一事件的选择方案,故而等式成立。

在二项式定理中若同时取$x=1$,$y=1$,则可得到恒等式
\begin{equation}\label{e5}
{n\choose 0}+{n\choose 1}+\cdots+{n\choose n}=2^n.
\end{equation}
若取$x=1$,$y=-1$,则可得到恒等式
$${n\choose 0}-{n\choose 1}+{n\choose 2}-\cdots+(-1)^n{n\choose n}=0,\ \ \ (n\geq 1).
$$
也即
\begin{equation}\label{e6}
{n\choose 0}+{n\choose 2}+\cdots={n\choose 1}+{n\choose 3}+\cdots,\
\ \ (n\geq 1).
\end{equation}
由式(\ref{e5}),要证明式(\ref{e6}),只需证明
$${n\choose 0}+{n\choose 2}+\cdots=2^{n-1}.$$
下面我们将给出该式的组合解释。令$S={x_1,x_2,\ldots,x_n}$为$n$元集合,我们需要计数的是$S$的偶子集$X$的个数,
我们对$S$中元素逐个考虑,首先考虑$x_1$,我们有放不放入$X$中两种选择,考虑$x_2$,我们也有放不放入$X$中两种选择,接着
考虑$x_3$,一直到$x_{n-1}$,均有放不放入$X$中两种选择,最后对于$x_n$,若前面放入$X$的元素个数为奇,则将$x_n$放入$X$中,
否则将$x_n$不放入$X$中。所以在整个事件中,前$n-1$步均有两种选择,最后一步只有一种选择,于是由乘法原理,
$S$的偶子集的个数为$2^{n-1}$,又${n\choose 0}+{n\choose
2}+\cdots$也计数了$n$元集合的偶子集的个数,故而
$${n\choose 0}+{n\choose 2}+\cdots=2^{n-1}.$$
\qed

同样的道理,我们可以证明$${n\choose 1}+{n\choose
3}+\cdots=2^{n-1}.$$

利用恒等式(\ref{e4})和(\ref{e5}),我们可得到
\begin{equation}\label{e7}
1{n\choose 1}+2{n\choose 2}+\cdots+n{n \choose n}=n2^n,\ \ (n\geq
1).
\end{equation}
式(\ref{e7})也可以根据二项式定理而得到,在二项式公式
$$(1+x)^n={n\choose 0}+{n\choose 1}x+{n\choose 2}x^2+\cdots+{n\choose n}x^n$$
两边同时对$x$求导,我们有
$$n(1+x)^{n-1}={n\choose 1}+2{n\choose 2}x+\cdots+n{n\choose n}x^{n-1},$$
最后,令$x=1$,即可得到式(\ref{e7})。





%%%%%%%%%%%%%%%%%%%%%%%%%%%%%%%%%%%%%%%%%%%%%%%%%%%%%%%%%%%%%%%%%%%%%%%%%%%%%%%%%%%%%%
\section{Stirling数}

\subsection{Stirling 简介} James
Stirling是一位苏格兰籍的数学家。他1692年5月出生于苏格兰斯特灵郡,1770年10月逝世于爱丁堡。
Stirling数以及Stirling逼近都是以他的名字命名的。James
Stirling18岁进入牛津大学贝利奥尔学院(Balliol College,
Oxford)求学,后被驱逐至威尼斯。在威尼斯期间,James
Stirling在艾萨克.牛顿的帮助下,与皇家科学院取得联系并邮寄了他的一篇论文
Methodus differentials Newtoniana illustrata (Phil.Trans.,
1718)。后又在牛顿的帮助下于1725年回到伦敦。在伦敦的十年时间里,他一直致力于学术工作,
1730年,他最重要的工作 the methodus differentialis, sive tractatus
de summatione et interpolatione serierum infinitarum (4to,
London)发表。

这里我们主要介绍一下他的一个重要工作,Stirling数。Stirling数分为第一类和第二类。
下面先看一下Stirling数的具体定义。

%%%%%%%%%%%%%%%%%%%%%%%%%%%%%%%%%%%%%%%%%%%%%%%%
\subsection{第一类Stirling数}

\begin{defi}
$c(n,k)$为恰好含有$k$个圈的$\pi\in\mathfrak{S}_n$的个数。数
$s(n,k):=(-1)^{n-k}\\c(n,k)$被称为{\bf
第一类Stirling数},\index{第一类斯特林数,Stirling number of the
first kind}而 $c(n,k)$被称为{\bf 无符号的第一类Stirling数}。
\end{defi}

\begin{defi}
{\bf 降阶乘函数}\index{降阶乘函数,falling factorial}$(x)_n$定义如下
$$(x)_n=x(x-1)\cdots(x-n+1).$$
\end{defi}

显然通常的阶乘$n!=(n)_n.$

第一类Stirling数有如下等价定义:

\begin{defi}
{\bf 第一类Stirling数}(记为$s(n,k)$, $1\leq k\leq
n$)恰好为降阶乘\index{降阶乘,falling
factorial}多项式中的各项系数,即
\begin{equation}
(x)_n=x(x-1)(x-2)\cdots(x-n+1)=\sum_{k=1}^n{s(n,k)x^k}.
\end{equation}
\end{defi}

\begin{thm}
$c(n,k)$ 满足如下递推式
$$c(n,k)=(n-1)c(n-1,k)+c(n-1,k-1),\  n,k\geq 1$$
并且初值为$c(0,0)=1$,而对其它的$n\leq 0$或者$k\leq 0$,$c(n,k)=0$。
\end{thm}

{\bf 证明:}选定一个有$k$个圈的排列$\pi\in\mathfrak{S}_{n-1}.$在$
\pi$的不交圈分解中,我们可以将$n$插入数字$1,2,\ldots,n-1$的任一个的后面,
有$n-1$种方法。这样就得到一个排列$\pi^{'}\in\mathfrak{S}_n$的不交圈分解,它具有
$k$个圈并且$n$出现在一个长度$\geq
2$的圈中。于是共有$(n-1)c(n-1,k)$个排列
$\pi^{'}\in\mathfrak{S}_n$具有$k$个圈并且满足$\pi^{'}(n)\neq n.$

另一方面,如果选定一个有$k$个圈的排列$\pi^{'}\in\mathfrak{S}_{n-1},$可以通过定义
  $$\pi^{'}(i)=\left\{\begin{array}{ll}
                          \pi(i), & \mbox{若}i\in[n-1]; \\
                          n, &\mbox{若} i=n.
                        \end{array}\right.
    $$

将其扩充为一个具有$k$个圈的排列$\pi^{'}\in\mathfrak{S}_n,$并且有
$\pi^{'}(n)=n$。因此,共有$c(n-1,k-1)$个排列$\pi^{'}\in\mathfrak{S}_n$具有
$k$个圈并且满足$\pi^{'}(n)=n,$得证。\qed

%%%%%%%%%%%%%%%%%%%%%%%%%%%%%%%%%%%%%%%%%%%%%%%%
\subsection{第二类Stirling数}
\begin{defi}
把$n$个元素构成的集合划分为$k$个非空子集的方法数,称为{\bf
第二类Stirling数}\index{第二类斯特林数,Stirling number of the
second kind},记为$S(n,k)$.
\end{defi}

举例而言,集合$[3]$划分成三个子集合的方法只有一种: $1/2/3$;
划分成两个子集合的方法有三种: $12/3,13/2,1/23$;
划分成一个子集合的方法只有一种: $1\ 2\ 3$.


\begin{thm}
第二类Stirling数有如下的递归关系式:
\begin{equation}S(n,k)=kS(n-1,k)+S(n-1,k-1),\ 1\le k<n,\end{equation}
其中初始条件为$S(n,n)=S(n,1)=1.$
\end{thm}
{\bf 证明:}
我们直接从第二类Stirling数的定义来给出这个递归关系式的证明。
显然,把$n$个元素放在一个集合和$n$个集合里都只有一种放法。
现假设把$n-1$个元素放在$k$个集合里的的方法数为$S(n-1,k).$
现在考虑把$n$个元素 $a_1, a_2, \ldots,
a_n$\\放在$k$个集合里,我们将$a_n$单独拿出来考虑。会有如下两种方式:
\begin{itemize}
\item[1. ]将前$n-1$个元素放入$k$个集合中,
再将$a_n$放入这$k$个集合中的某一个。这样一共有$kS(n-1,k)$种放法。
\item[2. ]将前$n-1$个元素放入$k-1$个集合中,
再将$a_n$单独放在一个新的集合中。这样给出另外$S(n-1,k-1)$种方法。
由此定理证明完毕。\qed
\end{itemize}

现在我们考虑由变量为$x$的多项式组成的向量空间。
对于这个无限维的向量空间最明显的一组基是单项式幂级数$x^n,\  n\geq
0.$ 然而同时,降阶乘函数
$$(x)_n=x(x-1)\cdots(x-n+1),\ n\geq 0$$
也是这个向量空间的一组基,自然能生成幂级数$x^n.$ 其实,
第二类Stirling数就是这两组基之间的过渡矩阵里的元素, 即:
\begin{equation}
x^n=\sum_{k=1}^nS(n,k)(x)_k.\label{1}
\end{equation}
例如,$x^4=x+7x(x-1)+6x(x-1)(x-2)+x(x-1)(x-2)(x-3)$.

{\bf 证明:}现用归纳法证明上述递归关系式\ref{1}。
显然\ref{1}对于$n=1$成立。 \\
假设\ref{1}对于$n$成立。由降阶乘函数的定义我们得到
$$x(x)_k=(x)_{k+1}+k(x)_k,$$
据此我们由归纳假设得到递归关系式
\begin{align*}
x^{n+1}&=\sum_{k=1}^nS(n,k)x(x)_k\\
&=\sum_{k=1}^n S(n,k)[(x)_{k+1}+k(x)_k]\\
&=\sum_{k=1}^{n}[S(n,k-1)+kS(n,k)](x)_k+S(n,n)(x)_{n+1}\\
&=\sum_{k=1}^{n+1}S(n+1,k)(x)_k.
\end{align*}
所以\ref{1}得证。\qed

接下来给出\ref{1}的一个简单的组合证明。考虑映射$f:N\rightarrow
X$,其中$|N|=n$而$|X|=x$。 \ref{1}的左边计数了映射$f:N\rightarrow
X$的总数。而每一个这样的映射都是到$X$的某个满足 $|Y|\leq
n$的子集$Y$的满射,且$Y$唯一。
如果$|Y|=k$,则有$k!S(n,k)$个这样的映射。而满足$|Y|=k$的$X$的子集$Y$有${x\choose
k}$种选择,因此
$$x^n=\sum_{k=0}^nk!S(n,k){x\choose k}=\sum_{k=0}^nS(n,k)(x)_k.$$

现在我们了解了有关于第二类Stirling数的组合解释,以及它作为幂函数与降阶乘函数之间过渡矩阵的元素。
下面我们计算一下$S(n,k)$的具体数值。

在计算的过程中,我们会用到有限差分演算,下面先简单罗列一下我们计算中会用到的一些概念和公式。
给定一个映射$f:\mathbb{N}\rightarrow\mathbb{C}$,定义一个新的映射$\Delta
f$如下,称之为$f$的{\bf 一阶差分},
$$\Delta f(n)=f(n+1)-f(n).$$
$\Delta$称为一阶{\bf
差分算子}。将算子$\Delta$重复$k$次就可以得到$k$阶差分算子。
$$\Delta^k f=\Delta(\Delta^{k-1}f).$$
以上是差分算子的概念,下面的式子在我们的计算中会起到很重要的作用。
\begin{equation}
f(n)=\sum_{k=0}^n{n\choose k}\Delta^k f(0).\label{e2}
\end{equation}

如果将连续两项$f(i),f(i+1)$的差$f(i+1)-f(i)=\Delta
f(i)$写在它们下一行的中间,就得到序列
$$\ldots \Delta f(-2)\ \Delta f(-1)\ \Delta f(0)\ \Delta f(1)\ \Delta f(2)\ldots$$
反复这个过程,就得到映射$f$的差分表,它的第$k$行由$\Delta^k
f(n)$组成。从$f(0)$开始往右下方延伸的对角线则是由0点的差分$\Delta^k
f(0)$构成。例如,令 $f(n)=n^4,$则差分表(从$f(0)$开始)如下


$\begin{array}{cccccccccccc}
  0 &  & 1 &  & 16 &  & 81 &  & 256 &  & 625 & \cdots \\
   & 1 &  & 15 &  & 65 &  & 175 &  & 369 &  &  \\
   &  & 14 &  & 50 &  & 110 &  & 194 &  &  &  \\
   &  &  & 36 &  & 60 &  & 84 &  &  &  &  \\
   &  &  &  & 24 &  & 24 &  &  &  &  &  \\
   &  &  &  &  & 0 &  &  &  &  &  &  \\
   &  &  &  &  &  & \ddots &  &  &  &  &  \\
   &  &  &  &  &  &  &  &  &  &  &
\end{array}$

因此,由\ref{e2}得
$$n^4={n\choose 1}+14{n\choose 2}+36{n\choose 3}+24{n\choose 4}+0{n\choose 5}$$
据此再结合\ref{1}就可以得到$S(n,k)$的具体数值。

\begin{ex} 对所有$m,n\in\mathbb{N}$,成立
$$\sum_{k\geq 0}S(m,n)s(m,n)=\delta_{mn}.$$
\end{ex}

提示:$\left(S(m,n)\right)_{m,n\geq 0},\left(s(m,n)\right)_{m,n\geq
0}$刚好是两组基之间的过渡矩阵。

%%%%%%%%%%%%%%%%%%%%%%%%%%%%%%%%%%%%%%%%%%%%%%%%%%%%%%%%%%%%%%%
\subsection{第二类Stirling数的正态分布性质}

在这一节里,我们来研究第二类Stirling数的概率性质。这个性质是由Harper\cite{Harper1967}
首先给出的。回忆一下$S(m,j)$满足如下递推关系:
\begin{equation}\label{St}
S(m,j)=S(m-1,j-1)+jS(m-1,j).
\end{equation}
其中,当$m\geq
1$时,如果$j>m$,则$S(m,j)=0$。称$m$个元素的集合$X$上的所有分拆的个数为$m$-阶Bell数,即
当$m\geq 1$时,$B_m=\sum\limits_{j=1}^{m}S(m,j)$。


为了研究$S(m,j)$的分布性质,我们假定具有$m$个元素的集合$X$上的所有分拆是等可能分布的,
即每个分拆出现的可能性都是 $B_m^{-1}$。在这个假定下,如果我们
记$\xi_m$为具有$m$个元素的集合$X$上的一个随机分拆所含有的块
个数,则易知当$0\leq k\leq m$时,
\begin{equation}\label{xik}
P\{\xi_m=j\}=\frac{S(m,j)}{B_m}.
\end{equation}
下面我们来研究当$m\rightarrow \infty$时,随机变量
$\xi_m$所满足的分布。我们有如下定理:

\begin{thm}\label{secstirling}
记
\begin{equation}
\eta_m=\frac{\xi_m-E\xi_m}{\sqrt {Var\xi_m}},
\end{equation}
其中$E\xi_m$为随机变量$\xi_m$的数学期望,$Var\xi_m$为随机变量$\xi_m$的方差,
则当$m\rightarrow
\infty$,随机变量$\eta_m$的分布收敛于标准正态分布,即
\[
lim_{m\rightarrow \infty}P\{\eta_m
<x\}=\frac{1}{\sqrt{2\pi}}\int_{-\infty}^{x}e^{-u^2/2}du
\]
\end{thm}

为了证明这个定理,我们给出如下三个引理,为简单起见,我们只给出第二个引理的证明,
关于其他两个引理的证明,读者可参考文献\cite{Sachkov1997}。
\begin{lem}\label{EVar}
对于上述定义的随机变量$\xi_m$,其数学期望
\begin{equation}
E\xi_m=\frac{B_{m+1}}{B_m}-1,
\end{equation}
其方差
\begin{equation}
Var\xi_m=\frac{B_{m+2}}{B_m}-(\frac{B_{m+1}}{B_m})^2-1,
\end{equation}
进一步的,我们有
\begin{equation}
lim_{m\rightarrow \infty}Var\xi_m=\infty.
\end{equation}
\end{lem}
\noindent {\bf{证明:}} 略。\qed

下面我们利用罗尔(Roll)引理来讨论与第二类Stirling数有关的多项式的实根性。
\begin{lem}\label{Fmroot}
记
\begin{equation}
F_m(x)=\sum\limits_{j=0}^{m}S(m,j)x^j,
\end{equation}
则对任意的$m\geq 1$,多项式$F_m(x)$有$m$个不同的非正实根。
\end{lem}
\noindent
{\bf{证明:}}我们用归纳法来证明。$F_1(x)=1$,$F_2(x)=x(x+1)$。
假设引理对所有多项式$F_n(x)$,其中$n\leq m-1$都成立。
我们来证明多项式$F_m(x)$有$m$个不同的非正实根。

对$F_m(x)$关于$x$求导,并利用$S(m,j)$所满足的递推关系式(\ref{St}),我们
得到
\begin{equation}\label{Fmrec}
F_m(x)=x\left[F_{m-1}(x)+\frac{d}{dx}F_{m-1}(x)\right].
\end{equation}
根据归纳假设,$F_{m-1}(x)$有$m-1$个不同的非正实根,在$(-\infty,\infty)$上,定义
函数
\[
H_m(x)=F_m(x)e^x,
\]
则$H_m(x)$与多项式$F_m(x)$有相同的实根。根据(\ref{Fmrec})式,我们得到
\[
F_m(x)e^x=x\left[F_{m-1}(x)e^x+\left(\frac{d}{dx}F_{m-1}(x)\right)e^x\right],
\]
因此,
\begin{equation}\label{Hm}
H_m(x)=x\frac{d}{dx}H_{m-1}(x).
\end{equation}
观察到
\[
lim_{x\rightarrow -\infty}H_{m-1}(x)=lim_{x\rightarrow
-\infty}e^xF_{m-1}(x)=0,
\]
所以当$x\in [-\infty,0]$,函数$H_{m-1}(x)$有$m$个根。根据Roll定理,
在实轴上每两个相邻$H_{m-1}(x)$的根所构成的区间内,包含一个点使得
$\frac{d}{dx}H_{m-1}(x)=0$。 所以$\frac{d}{dx}H_{m-1}(x)$在实轴上有
$m$个根,其中一个为$x=-\infty$。
故根据(\ref{Hm})式,$H_m(x)$在实轴上有$m+1$个非正根,其中一个为$x=-\infty$。
综上,多项式$F_m(x)$有$m$个非正实根。\qed

最后,我们给出如下引理,这个引理实际上式著名的李雅普诺夫(Lyapunov)定理的一个推论。
我们先来引出一些概念。
考虑如下独立的随机变量$\xi_{kn}$,其中$k=1,2,\ldots,n$,$n=1,2,\ldots$,使得
\begin{align*}
P\{\xi_{kn}=1\}&=p_k,\\
P\{\xi_{kn}=0\}&=q_k,
\end{align*}
其中$p_k=p_k(n)$,$q_k=q_k(n)$,并且$p_k+q_k=1$。
我们称序列$\xi_{kn}$为一个Poisson序列。
\begin{lem}\label{Poissona}
记
\begin{align*}
T_n^2&=\sum\limits_{k=1}^{n}p_kq_k,\\
\eta_n&=T_n^{-1}\sum\limits_{k=1}^{n}(\xi_{kn}-p_k)
\end{align*}
如果当$n\rightarrow \infty$时,$T_n\rightarrow
\infty$,则序列$\{\eta_n\}$渐进正态分布。
\end{lem}
\noindent {\bf{证明:}} 略。\qed

下面,我们就来利用以上的三个引理来证明我们的定理。

 \noindent
{\bf{定理(\ref{secstirling})证明:}}根据引理(\ref{Fmroot}),
记$-\alpha_1,-\alpha_2,\ldots,-\alpha_{m-1}$为多项式$F_m(x)$除去0以外的根,则
\begin{equation}\label{F}
F_m(x)=x(x+\alpha_1)(x+\alpha_2)\cdots(x+\alpha_{m-1}).
\end{equation}
记
\begin{equation*}
P_m(x)=\sum\limits_kP\{\xi_m=k\}x^k,
\end{equation*}
为随机变量$\xi_m$的分布函数,根据定义\ref{xik},
\begin{equation*}
P_m(x)=\frac{1}{B_m}\sum\limits_kS(m,k)x^k=\frac{F_m(x)}{F_m(1)},
\end{equation*}
结合(\ref{F})式,
\begin{equation}\label{Pm}
P_m(x)=x\left(\frac{x}{1+\alpha_1}+\frac{\alpha_1}{1+\alpha_1}\right)\cdots
\left(\frac{x}{1+\alpha_{m-1}}+\frac{\alpha_{m-1}}{1+\alpha_{m-1}}\right),
\end{equation}
考虑如下独立的取值为0和1的随机变量$\xi_{m1}$,$\xi_{m2}$,$\ldots$,$\xi_{m,m-1}$,
满足
\begin{align*}
P\{\xi_{mi}=1\}&=\frac{1}{1+\alpha_i},\\
P\{\xi_{mi}=0\}&=\frac{\alpha_i}{1+\alpha_i}.
\end{align*}
其中$i=1,2,\ldots,m-1$。 记
\begin{equation*}
P_{mi}(x)=\sum\limits_kP\{\xi_{mi}=k\}x^k,
\end{equation*}
为随机变量$\xi_{mi}$的分布函数,则
\begin{equation}\label{mi}
P_{mi}(x)=\frac{x}{1+\alpha_i}+\frac{\alpha_i}{1+\alpha_i},
\end{equation}
结合(\ref{Pm})式和(\ref{mi})式,我们得到
\begin{equation}\label{ximsum}
\xi_m=\xi_{m1}+\xi_{m2}+\cdots+\xi_{m,m-1}+1.
\end{equation}
若我们定义
\begin{equation*}
\eta_{mi}=\frac{\xi_{mi}-E\xi_{mi}}{\sqrt{Var\xi_m}},
\end{equation*}
其中$i=1,2,\ldots,m-1$, 其中
\[
E\xi_{mi}=\sum\limits_kP\{\xi_{mi}=k\}=\frac{1}{1+\alpha_i},
\]
则定理叙述中的$\eta_m$满足
\begin{align*}
\eta_m&=\frac{1}{\sqrt{Var\xi_m}}\left(\xi_m-E\xi_m\right)\\
&=\frac{1}{\sqrt{Var\xi_m}}\sum\limits_{i=1}^{m-1}\left(\xi_{mi}-E\xi_{mi}\right)\\
&=\eta_{m1}+\eta_{m2}+\cdots+\eta_{m,m-1}.
\end{align*}
这样,相互独立的随机变量$\xi_{m1}$,$\xi_{m2}$,$\ldots$,$\xi_{m,m-1}$是一个
Poisson序列,满足$p_k=1/(1+\alpha_k)$,其中$k=1,2,\ldots,m-1$。因为诸
$\xi_{mi}$是相互独立的随机变量,由(\ref{ximsum})式,我们得到
\[
Var\xi_m=\sum\limits_{i=1}^{m-1}Var\xi_{mi}=\sum\limits_{i=1}^{m-1}\frac{\alpha_i}
{(1+\alpha_i)^2}=\sum\limits_{i=1}^{m-1}p_iq_i.
\]
根据引理(\ref{EVar}),$lim_{m\rightarrow
\infty}Var\xi_m=\infty$,故满足引理(\ref{Poissona})中的条件,所以$\eta_m$渐进正态分布。
\qed










% \bibliographystyle{cfcbook}
\begin{thebibliography}{99}



\bibitem{Sloane} N. J. A. Sloane, The On-Line Encyclopedia of Integer Sequences A000110.

\bibitem{Book Richard } Richard P. Stanley, Enumerative Combinatorics,
 volume 2, Cambridge
University Press, Cambridge, 1999.

\bibitem{webderange} Available in:\\ {\tt
http://math.ucr.edu/home/baez/qg-winter2004/derangement.pdf}.

\bibitem{webstirling} Available in:\\ {\tt
planetmath.org/encyclopedia/StirlingNumbersOfTheFirstKind.html}.

\bibitem{Andre1887} D. Andr\'e,
Solution directe du probl\`eme r\'esolu par M. Bertrand, CR Acad.
Sci. Paris 105 (1887), 436--437.

\bibitem{Chen1990} W.Y.C. Chen,
\href{ref/A general bijective algorithm for trees.pdf}{A general
bijective algorithm for trees}, Proc. Natl. Acad. Sci. USA. 87
(1990), 9635--9639.

\bibitem{Feller1950} W. Feller,
An Introduction to Probability Theory and Its Applications, vol 1,
John Wiley and Sons, New York, 1950.

\bibitem{Harper1967} L.H. Harper, \href{ref/Stirling behavior is asymptotically
 normal.pdf}
{Stirling behavior is asymptotically normal}, Ann. Math. Statist. 38
(1967), 410--414.

\bibitem{Mohanty1979} S.G. Mohanty, Lattice Path Counting and Applications,
Academic Press, New York, 1979.

\bibitem{Narayana1979} T.V. Narayana,
Lattice Path Combinatorics with Statistical Applications,
Mathematical Expositions no. 23, University of Toronto Press,
Toronto, 1979.

\bibitem{Sachkov1997} V.N. Sachkov,
Probabilistic methods in combinatorial analysis, Cambridge
university press, 1997.

\bibitem{Stanley1999} R.P. Stanley,
Enumerative Combinatorics, vol. 2, Cambridge University, Cambridge,
1999.

\bibitem{Takacs1967} L. Takacs,
Combinatorial Methods in the Theory of Stochastic Processes, John
Wiley and Sons, New York, 1967.


\bibitem{Ayoub1974} R. Ayoub, \href{ref/Euler-and-the-zeta-fuction.pdf}{Euler and the zeta function}, Amer. Math. Monthly, 81(10), (1974),1067-1086.

\bibitem{Sury2003} B. Sury, \href{ref/Bernoulli-numbers-and-the-Riemann-zeta-function.pdf}{Bernoulli numbers and the Riemann zeta function}, Resonance, 8, (2003),54-62.


\bibitem{barbara}Barbara H. Margolius, Permutations with Inversions, Journal of Integer Sequences, Vol.4(2001),1--12.


\bibitem{woon} S.C. Woon, A tree for generating Bernoulli numbers,
Math. Mag. 70(1997), 51-56.

\bibitem{fuchs} P. Fuchs, Bernoulli numbers and binary trees, Tatra
Mt. Math. Publ. 20(2000), 111-117.

\bibitem{Euler} http://en.wikipedia.org/wiki/Leonhard\_Euler

\bibitem{Stanley2004}R.P.Stanley, 计数组合学(卷1),机械工业出版社,北京,2004.11

\bibitem{Graham1994}Graham, Knuth, Patashnik, Concrete Mathematics: A Foundation for Computer Science, Second Edition. Addison-Wesley, 1994


\bibitem{Benjamin-Quinn2003} A.T. Benjamin and J.J. Quinn, Proofs that Really Count The Art of Combinatorial Proof,
The Mathematical Association of America, 2003.

\end{thebibliography}




