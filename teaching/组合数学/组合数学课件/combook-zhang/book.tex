\documentclass[a4paper,11pt,twoside]{book}
\usepackage{ctex}



\usepackage{amsmath,amssymb}             % AMS Math
\usepackage[T1]{fontenc}



\usepackage{graphicx}
% \usepackage{epstopdf}
\usepackage{tikz}
\usepackage[left=1.5in,right=1.3in,top=1.1in,bottom=1.1in,includefoot,includehead,headheight=13.6pt]{geometry}
\renewcommand{\baselinestretch}{1.05}



\usepackage{minitoc}
\newtheorem{thm}{定理}[section]
\newtheorem{prop}[thm]{命题}
\newtheorem{coro}[thm]{推论}
\newtheorem{defi}[thm]{定义}
\newtheorem{lem}[thm]{引理}
\newtheorem{exa}[thm]{例}
\newtheorem{ex}[thm]{习题}
\newtheorem{conj}[thm]{猜想}

\def\qed{\nopagebreak\hfill{\rule{4pt}{7pt}}\medbreak}
\def\pf{{\bf 证明~~ }}
\def \sg{\sigma}
\def \asc{\mathrm{asc}}
\def \des{\mathrm{des}}
\def \fix{\mathrm{fix}}
\def \lef{\mathrm{lef}}
\def \one{\mathrm{one}}
\def \Des{\mathrm{Des}}
\def \maj{\mathrm{maj}}
\def \exc{\mathrm{exc}}
\def \inv{\mathrm{inv}}
\def \roots{\mathrm{roots}}
\def \sgn{\mathrm{sgn}}
% Table of contents for each chapter

\usepackage{color}
\definecolor{linkcol}{rgb}{0,0,0.4}
\definecolor{citecol}{rgb}{0.5,0,0}

  \usepackage{graphicx}
  \DeclareGraphicsExtensions{.eps}
  \usepackage[a4paper,pagebackref,hyperindex=true,pdfnewwindow=true]{hyperref}

% \usepackage{chapterbib}
\begin{document}

\begin{titlepage}
	\begin{center}
		\vspace*{5cm}
		\noindent \Huge  \textbf{组\quad 合\quad 数\quad 学} \\
		\vspace*{1cm}
		\noindent \Huge \textbf{Combinatorics} \\
		\vspace*{2cm}
		\noindent \LARGE \textbf{张\quad 彪} 
	\end{center}
\end{titlepage}


\pagenumbering{roman}
\cleardoublepage
\dominitoc
 \tableofcontents
\mainmatter

\chapter{基本原理}
\label{chap1} \minitoc

\section{加法原理与乘法原理}
在计数问题中,加法原理与乘法原理是两个最基本也是最常用的原理。以下假设$A$和$B$是两类不同的、互不关联的事件。
\subsection{加法原理}
\textbf{加法原理}\
设事件$A$有$m$种选取方式,事件$B$有$n$种选取方式,则选$A$或$B$共有$m+n$种方式。

例如,大于$0$小于$10$的偶数有$4$个,即$2,4,6,8$;大于$0$小于$10$的奇数有$5$个,即$1,3,5,7,9$,则大于$0$小于$10$的整数有
$9$个,即$1,2,3,4,5,6,7,8,9.$\
这里,事件$A$指的是大于$0$小于$10$的偶数,事件$B$指的是大于$0$小于$10$的奇数。而大于$0$小于$10$的整数不外乎是偶数或奇数,
即属于$A$或$B$.

用集合的语言可将加法原理叙述成以下定理:
\begin{thm}
设$A$, $B$为有限集,$A\cap B=\varnothing$, 则$$|A\cup B|=|A|+|B|.$$
\end{thm}
\pf 当$A$, $B$中有一个是空集时,定理的结论是平凡的。

设$A\neq \varnothing,\ B\neq \varnothing$, 记$$A=\{a_1, a_2, \ldots,
a_m\},$$
$$B=\{b_1, b_2, \ldots,
b_n\},$$ 并作映射$$\varphi :\ a_i\rightarrow i\ (1\leq i\leq m),$$
$$\ \ \ \ \ \ \ \ \ \ \ b_j\rightarrow m+j\ (1\leq j\leq n).$$
因为$$\varphi(a_i)\neq \varphi (b_j)\ (1\leq i\leq m,\ 1\leq j\leq n
),$$ 所以$\varphi$是从$A\cup
B$到集合$\{1,2,\ldots,m+n\}$上的一一映射,因而定理成立。
\begin{coro}
设$n$个有限集合$A_1,A_2,\ldots,A_n$满足$$A_i\cup A_j=\varnothing\
(1\leq i \neq j\leq n),$$ 则$$|\bigcup_{i=1}^n A_i|=\sum_{i=1}^n
|A_i|.$$
\end{coro}

\begin{exa}
在所有六位二进制数中,至少有连续$4$位是$1$的有多少个?
\end{exa}

\textbf{解} 把所有满足要求的二进制数分为如下$3$类:

(i)恰有$4$位连续的$1$. 它们可能是$*01111, 011110, 11110*$,
其中“$*$”可能取$0$或$1$. 故此种情况共有$5$个不同的六位二进制数。

(ii)恰有$5$位连续的$1$, 它们可能是$011111, 111110$, 共有$2$个。

(iii)恰有$6$位连续的$1$, 即$111111$, 只有$1$个。

于是综合以上分析,由加法原理知共有$5+2+1=8$个满足题意要求的六位二进制数。

\subsection{乘法原理}

\textbf{乘法原理}
设事件$A$有$m$种选取方式,事件$B$有$n$种选取方式,那么选取$A$以后再选取$B$共有$m\cdot
n$种方式。

用集合论的语言可将上述乘法原理叙述成如下的定理:
\begin{thm}
设$A,\ B$是两个有限集合,$|A|=m,\ |B|=n,$ 则$$|A\times
B|=|A|\times|B|=m\cdot n.$$
\end{thm}
\pf 若$m=0$或$n=0$, 则上面的等式两边均为$0$, 故等式成立。

设$m>0,n>0$, 并且记$$A=\{a_1, a_2, \ldots, a_m\},$$
$$B=\{b_1, b_2, \ldots,
b_n\}.$$ 定义映射$$\varphi:\ (a_i,b_j)\rightarrow (i-1)n+j\ (1\leq
i\leq m,\ 1\leq j\leq n),$$ 则$\varphi$是$A\times
B$到集合$\{1,2,\ldots,mn-1,mn\}$上的一一映射,所以等式成立。
\begin{coro}
设$A_1,A_2,\ldots,A_n$为$n$个有限集合,则$$|A_1\times A_2\times
\cdots \times A_n|=|A_1|\times |A_2|\times \cdots \times |A_n|.$$
\end{coro}

\begin{exa}
设从$A$到$B$有$3$条不同的道路,从$B$到$C$有$2$条不同的道路,如下图所示,则从$A$经$B$到$C$的道路数为$$n=3\times
2=6.$$ \vspace*{0.2cm} \hspace*{5.5cm}
\begin{picture}(20,20)
\setlength{\unitlength}{2.5mm}
\put(-3,0){\circle*{0.3}}\put(5,0){\circle*{0.3}}\put(13,0){\circle*{0.3}}
\put(-3,0){\line(1,0){8}} \qbezier(-3,0)(0.6,4)(5,0)
\qbezier(-3,0)(0.6,-4)(5,0) \qbezier(5,0)(8.6,4)(13,0)
\qbezier(5,0)(8.6,-4)(13,0)
\put(-4.5,-2){$A$}\put(4,-2){$B$}\put(13.5,-2){$C$}
\end{picture}

\end{exa}
\begin{exa}
从$5$位先生、$6$位女士、$2$位男孩、$4$位女孩中选取$1$位先生、$1$位女士、$1$位男孩、$1$位女孩,共有$5\times
6\times 2\times
4=240$种方式(由乘法原理)。而从中选取一个人的方式共有$5+ 6+ 2+
4=17$种方式(由加法原理)。
\end{exa}

\begin{exa}
在$1000$到$9999$之间有多少个各位数字不同的奇数?
\end{exa}
\textbf{解}
将一个千位数的千位、百位、十位、个位分别记为第$1$、$2$、$3$、$4$位,则满足条件的数字第$4$位必须是奇数,可取$1,3,5,7,9$,
共有$5$种选择。第$1$位不能取$0$,也不能取第$4$位已选定的数字,所以在第$4$位选定后第$1$位有$8$种选择。
第$2$位不能取第$1$位和第$4$位已选定的数字,共有$8$种选择。类似地,第$3$位有$7$种选择。从而由乘法原理,满足题意的
数共有$5\times 8\times 8\times 7=2240$个。

\section{鸽巢原理}
\subsection{引言}

鸽巢原理又称抽屉原理,这个原理最早是由Dirichlet提出的。鸽巢原理是解决组合论中一些存在性问题的基本而又有力的工具。
它是组合数学中最简单也是最基本的原理之一,从这个原理出发,可以导出许多有趣的结果,而这些结果常常是令人惊奇的。

Ramsey理论对组合数学发展产生过重要的影响。$1928$年,年仅$24$岁的英国杰出数学家Ramsey发表了著名的论文《论形式逻辑中的一个问题》,
他在这篇文论中,提出并证明了关于集合论的一个重大研究成果,现称为Ramsey定理。尽管两年后他不幸去世,但他开拓的这一新领域至今
仍十分活跃,而且近年来在科技领域获得了成功的应用。

本章主要介绍鸽巢原理、Ramsey数及性质、Ramsey定理。
\subsection{鸽巢原理}
\begin{thm}
若有$n+1$只鸽子飞回$n$个鸽巢,则至少有两个鸽子飞入了同一鸽巢。
这个原理也可以表述为:\\如果把$n+1$件东西放入$n$个盒子中,则至少有一个盒子里面有不少于两件的东西。
\end{thm}
这个原理的证明非常容易,只要使用反证法马上就可以得到结论。

鸽巢原理不能用来寻找究竟是哪个盒子含有两件或更多的东西。该原理
只能证明某种安排或某种现象存在,而并未指出怎样构造这种安排或怎样寻找这种现象出现的场合。利用鸽巢原理解决实际问题关键是要
看出这是一个鸽巢问题,建立“鸽巢”寻找“鸽子”。下面我们给出几个利用鸽巢原理解决问题的例子。
\begin{exa}
如果鞋架上放$10$双鞋,从总任意取11只,其中至少有两只恰好是配对的。
\end{exa}
\begin{exa}
从整数$1,2,\ldots,100$中任选$51$个数,证明在所选的数中间必然存在两个整数,其中之一可以被另一个整除。

\pf 对于任何一个整数$x$,总可以把$x$写成$x=2^n\cdot
a$的形式,其中$a$是奇数,$n\geq 0$.
知$1$到$100$之间共有$50$个奇数,由所选的$51$个数利用上述表达方式可以得到$51$个奇数,由鸽巢原理知,其中必然有两个奇数相同,
设对应这两个奇数的整数为$x=2^ra,\ y=2^sa$,如果$r\leq s$那么$x|y$;如果$r>s$那么$y|x$.\\
注:本例中:鸽子$=$去掉$2$因子所得到的奇数;鸽巢$=$$1$到$100$之间的奇数。这个例子可以推广到从$1,2,\ldots,2n$中任意取
$n+1$个数,其中必然存在两个数,其中一个整除另外一个,证法类似。
\end{exa}

利用鸽巢原理解决问题的关键在于:辨认问题,建立鸽巢,寻找鸽子。
\begin{exa}
在\{1,2,\ldots,2n\}中任取$n+2$个数,其中必有两个数,其和为$2n$.
\\分析:鸽子为集合中的元素,鸽巢为$$\{1,2n-1\},\{2,2n-2\},\ldots,\{n-1,n+1\},\{n\},\{2n\}.$$
\end{exa}

\begin{exa}
任取$11$个正整数,其中必有两个数,其差被$10$整除。
\\分析:鸽子为选出的$11$个数,鸽巢为$A_i=\{\mbox{个位数为}i\},\ i=0,1,\ldots,9.$
\end{exa}

\subsection{一般形式的鸽巢原理}

已经知道了简单的鸽巢原理,现在我们来了解一下推广的鸽巢原理。
\begin{thm}
设$m_1,m_2,\ldots,m_n$均为正整数,如果有$m_1+m_2+\cdots+m_n-n+1$只鸽子飞回$n$个鸽巢,则或者第$1$个鸽巢至少有$m_1$只鸽子,
或者第$2$个鸽巢至少有$m_2$只鸽子,$\ldots$,或者第$n$个鸽巢至少有$m_n$只鸽子。
\end{thm}
\pf
用反证法。假若第$1$鸽巢少于$m_1$只鸽子,第$2$鸽巢少于$m_2$只鸽子,$\ldots$,第$n$鸽巢少于$m_n$只鸽子,则鸽子
总数之多为:$(m_1-1)+(m_2-1)+\cdots+(m_n-1)=m_1+m_2+\cdots+m_n-n$,
这比假定的鸽子数少了一个,矛盾。\qed

从定理二可以得到以下推论:
\begin{prop}
如果$m_1=m_2=\cdots=m_n=r$,
即若将$n(r-1)+1$个球放入$n$个盒子里,则至少有一个盒子含有不少于$r$个球。
\end{prop}
\begin{prop}
如果$n$个正整数$m_1,m_2,\cdots,m_n$的平均数$(m_1+m_2+\cdots+m_n)/n>r-1$,
则$m_1,m_2,\ldots,m_n$中至少有一个正整数不会小于$r$.
\end{prop}
\begin{prop}
有$m$个球放入$n$个盒子,则至少有一个盒子中有不少于$[(m-1)/n]+1$个球。
\end{prop}

下面我们来看一些利用这些定理和推论的例子。
\begin{exa}
随意给一个正十边形的十个顶点标上号码$1,2,\ldots,10$,求证:必然有一个顶点,该顶点及与之相邻的两个顶点的标号之和不小于$17$.
\\ \pf 设$v_1,v_2,\ldots,v_{10}$是正十边形的$10$个顶点,$a_i$表示顶点$v_i$及与$v_i$相邻的两个顶点的标号之和,则
$a_1+a_2+\cdots+a_{10}=(1+2+\cdots+10)\times 3=165>(17-1)\times
10+1$, 这样必然有某个$a_k\geq 17$.
\end{exa}

\begin{exa}
将$N_5=\{1,2,3,4,5\}$分为两组,则必有某个数,它是同组中的一个数的$2$倍,或者是同一组中另两个数之和。\\
\pf 若存在一个分组方法,$A\bigcup B=N_5,\ A\bigcap B=\emptyset $,
使得性质不成立,即$a,b\in A \Rightarrow a-b,b-a\in A$和$a,b\in B
\Rightarrow a-b,b-a \in
B$都不成立。由鸽巢原理,不妨设$a_1,a_2,a_3\in A, a_1>a_2>a_3$,
由反证法,
$$\left.\begin{array}{c}
  b_1 = a_1-a_2\notin A \\
  b_2 = a_1-a_3\notin A
\end{array}\right\}\Rightarrow\left. \begin{array}{c}
                                  b_1\in B \\
                                    b_2 \in B
                                \end{array}\right\}\Rightarrow
   b_2-b_1 \notin B \Rightarrow b_2-b_1\in A$$
但 $b_2-b_1=(a_1-a_3)-(a_1-a_2)=a_2-a_3\notin A$矛盾。
\end{exa}

通过上面的定理我们还可以得到以下结论:
\begin{thm}{(Erd\"{o}s)}
由$n_2-1$个不同实数构成的序列中,至少存在由$n+1$个实数组成一个单调递增子序列和单调递减子序列。
\end{thm}
\pf
设原序列为:$a_1,a_2,\ldots,a_{n_2-1}$,令$m_i$表示从$a_i$开始最长递增子序列的长度,若有某个$m_i\geq
n+1$,
则定理得证。因为给定的序列有$n^2+1$个实数,顾可产生$n^2+1$个长度:$m_1,m_2,\ldots,m_{n^2+1}$.
如果全部的$m_i<n+1$,
则这些整数必定在$1$到$n$之间,相对于把$n^2+1$个球放入$n$个盒子。由定理$2$的推论$1$可知,这是$r=n+1$的特殊情况,这$n^2+1$
个$m_i$中至少有$n+1$个数
相等。不妨设$m_{i_1}=m_{i_2}=\cdots=m_{i_{n+1}}=m$. 且$1\leq
i_1<i_2<\cdots<i_{n+1}\leq n^2+1$,
则可以得到下面的长度为$n+1$的递减序列:$a_{i_1}>a_{i_2}>\cdots>a_{i_{n+1}}$.
\qed

\section{容斥原理}
粗略地讲,组合计数学中的筛法是一种首先从一个较大的集合入手,
以某种方式减掉或删去一些不需要的元素,从而计算出集合 基数的方法。

容斥原理是组合计数学的主要工具之一。从抽象的角度来讲,容斥原理
就是计算某个矩阵的逆,因此,它只不过是线性代数中一个
很平凡的结果,但这个原理的优美之处远不在结果本身,而在于它应用的
广泛性。我们将给出一些可以利用容斥原理解决的问题的例子,其中一些
采用了相当巧妙的构思。在此之前,我们用最抽象的形式来陈述这个原理:

\subsection{容斥原理}
\begin{thm}
\label{theorem2.11} 设$S$是一个$n$元集,$V$是一个由函数$f\colon
2^S\rightarrow
K$组成的$2^n$维向量空间$(K$为某个域$)$。定义线性变换$\phi\colon
V\rightarrow V$如下:

\begin{equation}\phi f(T)=\sum_{Y\supseteq T}
f(Y),\quad \mbox{对任意的$T\subseteq S$},
\end{equation}

 则$\phi^{-1}$存在,且

\begin{equation} \phi^{-1}f(T)=\sum_{Y\supseteq T}(-1)^{|Y-T|}f(Y),\quad
\mbox{对任意的$T\subseteq S$}.\end{equation}
\end{thm}

{\bf 证明:}定义$\psi\colon V\rightarrow V$为$\psi
f(T)=\sum_{Y\supseteq T}
(-1)^{|Y-T|} f(Y).$则我们只需要证明$\phi\circ\psi$为恒等变换。\\
由定义,\allowdisplaybreaks
\begin{align*}
\phi\circ\psi f(T)&=\sum_{Y\supseteq T} (-1)^{|Y-T|}\phi f(Y)\\[5pt]
&=\sum_{Y\supseteq T} (-1)^{|Y-T|}\sum_{Z\supseteq Y} f(Z)\\[5pt]
&=\sum_{Z\supseteq T}\left(\sum_{Z\supseteq Y\supseteq
T}(-1)^{|Y-T|}\right)f(Z)
\end{align*}
设$m=|Z-T|,$则:
\[\sum_{Z\supseteq Y\supseteq T \atop
\mbox{\tiny{$Z,\,T$固定}}}(-1)^{|Y-T|}=\sum_{i=0}^m (-1)^i {m\choose
i}=\delta_{0m},
\]
所以,$\phi\circ\psi f(T)=f(T),$因此,$\phi\circ\psi
f=f.$从而可证$\psi=\phi^{-1}.$\qed

容斥原理通常的组合描述为:设$S$为由不同性质组成的集合,
$A$为一些元素的集合,且这些元素或者有或者没有$S$中的某些性质。
对$S$的任一个子集$T$,令
$f_=(T)$表示$A$中恰好具有$T$中性质的元素个数(易知这些元素
没有$\bar{T}=S-T$中的性质)。(更一般的,若$\omega\colon A\rightarrow
K$是$A$上的赋权函数,且赋权值在数域(或Abel群)$K$中,
则可令$f_{=}(T)=\sum_x \omega(x),$
这里$x$取遍$A$中恰好具有$T$中性质的元素。)
令$f_{\geq}(T)$表示$A$中至少具有$T$中性质的元素的个数,于是
\begin{equation}\label{equation3}
f_{\geq}(T)=\sum_{Y\supseteq T}f_{=} (Y).
\end{equation}
从而由定理\ref{theorem2.11}易知,
\begin{equation}\label{equation4}
f_{=}(T)=\sum_{Y\supseteq T}(-1)^{|Y-T|}f_{\geq}(Y).
\end{equation}
特别的,不具有$S$中任何性质的元素个数为:
\begin{equation}\label{equation5}
f_{=}(\emptyset)=\sum_Y (-1)^{|Y|} f_{\geq}(Y),
\end{equation}

接下来我们来了解一下容斥原理的运用。

\subsection{错排}

\begin{defi}
$[n]$上的置换$\pi=\pi_1,\pi_2,\ldots,
\pi_n$称为错排,是指对$\forall\  i\in [n]$均满足$\pi_i\ne
i$。即任何一个数字都不在其自然序时所处的位置上。同时我们称满足$\pi_i=i$的点为不动点。
例如:$[3]$上的错排有$2,3,1$和$3,1,2$。
\end{defi}


Pierre de Montmort在1708年首先提出了如何计算$[n]$上错排的个数问题,
然后在1713年由他本人解决了这一问题。Nicholas
Bernoulli几乎在同时运用容斥原理解决了这一问题。


\begin{defi}
令$a_0,a_1,a_2,\ldots$为一无穷序列,则
$$f(x)=\sum_{n\geq0}a_n\frac{x^n}{n!}$$
称为该无穷序列的指数型生成函数。例如:$\{0!,1!,2!,3!,\ldots\}$的指数型生成函数为:
$$f(x)=\sum_{n\ge0}n!\frac{x^n}{n!}=\frac{1}{1-x}.$$
$\{1,1,1,\ldots\}$的指数型生成函数为:
$$f(x)=\sum_{n\ge0}\frac{x^n}{n!}=e^x.$$
\end{defi}
由定义可得:

\begin{thm}
$f(x)=\sum a_n\frac{x^n}{n!}$和$g(x)=\sum b_n\frac{x^n}{n!}$
是两个生成函数,则:
$$f(x)g(x)=\sum c_n\frac{x^n}{n!},$$
其中$c_n=\sum{n\choose k}a_kb_{n-k}.$
\end{thm}

生成函数的乘法有严格的组合意义。假设$a_n$
所计数的组合结构称为A-结构,$b_n$
计数的是B-结构,那么$c_n$计数的则是由部分A-结构和部分B-结构组合而成的结构。


我们已经给出了错排及生成函数的定义,现在我们具体讨论一下。设$d_n$计数了$[n]$上的错排个数。现在讨论关于$d_n$
的计算及其生成函数。在这里会用到组合中一个常用方法:容斥原理。

\begin{thm}
\label{theorem2.11} 设$S$是一个$n$元集,$V$是一个由函数$f\colon
2^S\rightarrow
K$组成的$2^n$维向量空间$(K$为某个域$)$。定义线性变换$\phi\colon
V\rightarrow V$如下:

\begin{equation}\phi f(T)=\sum_{Y\supseteq T}
f(Y),\quad \mbox{对任意的$T\subseteq S$},
\end{equation}

 则$\phi^{-1}$存在,且

\begin{equation} \phi^{-1}f(T)=\sum_{Y\supseteq T}(-1)^{|Y-T|}f(Y),\quad
\mbox{对任意的$T\subseteq S$}.\end{equation}
\end{thm}

{\bf 证明:}定义$\psi\colon V\rightarrow V$为$\psi
f(T)=\sum_{Y\supseteq T}
(-1)^{|Y-T|} f(Y).$则我们只需要证明$\phi\circ\psi$为恒等变换。\\
由定义,\allowdisplaybreaks
\begin{align*}
\phi\circ\psi f(T)&=\sum_{Y\supseteq T} (-1)^{|Y-T|}\phi f(Y)\\[5pt]
&=\sum_{Y\supseteq T} (-1)^{|Y-T|}\sum_{Z\supseteq Y} f(Z)\\[5pt]
&=\sum_{Z\supseteq T}\left(\sum_{Z\supseteq Y\supseteq
T}(-1)^{|Y-T|}\right)f(Z)
\end{align*}
设$m=|Z-T|,$则:
\[\sum_{Z\supseteq Y\supseteq T \atop
\mbox{\tiny{$Z,\,T$固定}}}(-1)^{|Y-T|}=\sum_{i=0}^m (-1)^i {m\choose
i}=\delta_{0m},
\]
所以,$\phi\circ\psi f(T)=f(T),$因此,$\phi\circ\psi
f=f.$从而可证$\psi=\phi^{-1}.$\qed

容斥原理通常的组合描述为:设$S$为由不同性质组成的集合,
$A$为一些元素的集合,且这些元素或者有或者没有$S$中的某些性质。
对$S$的任一个子集$T$,令
$f_=(T)$表示$A$中恰好具有$T$中性质的元素个数(易知这些元素
没有$\bar{T}=S-T$中的性质)。(更一般的,若$\omega\colon A\rightarrow
K$是$A$上的赋权函数,且赋权值在数域(或Abel群)$K$中,
则可令$f_{=}(T)=\sum_x \omega(x),$
这里$x$取遍$A$中恰好具有$T$中性质的元素。)
令$f_{\geq}(T)$表示$A$中至少具有$T$中性质的元素的个数,于是
\begin{equation}\label{equation3}
f_{\geq}(T)=\sum_{Y\supseteq T}f_{=} (Y).
\end{equation}
从而由定理\ref{theorem2.11}易知,
\begin{equation}\label{equation4}
f_{=}(T)=\sum_{Y\supseteq T}(-1)^{|Y-T|}f_{\geq}(Y).
\end{equation}
特别的,不具有$S$中任何性质的元素个数为:
\begin{equation}\label{equation5}
f_{=}(\emptyset)=\sum_Y (-1)^{|Y|} f_{\geq}(Y),
\end{equation}

从抽象的角度来讲,容斥原理
就是计算某个矩阵的逆,因此,它只不过是线性代数中一个
很平凡的结果,但这个原理的优美之处远不在结果本身,而在于它应用的
广泛性。现在我们用容斥原理来计算一下$d_n$的大小。

\begin{thm}
设$d_n$计数了$[n]$上的错排个数,则$$d_n=\sum_{i=0}^n{n\choose
i}(-1)^{n-i}i!.$$
\end{thm}
{\bf
证明:}将条件$\pi(i)=i$视为第$i$个条件。不动点集至少包含集合$T\subseteq
[n]$的排列的个数等于 $f_{\geq}(T)=b(n-i)=(n-i)!,$其中$|T|=i$
(固定$T$中的元素,其余$n-i$个元素任意排列)。因此通过\ref{equation5}可以得到没有不动点的排列的个数
$f_{=}(\emptyset)=a(n)=d_n$为$\sum_{i=0}^n{n\choose
i}(-1)^{n-i}i!.$\qed

由$d_n$的计数可以很容易得到其生成函数如下:
\begin{thm}
设$d_n$计数了~$[n]$上的错排个数,则
$$\sum_{n\ge0}d_n\frac{x^n}{n!}=\frac{e^{-x}}{1-x}.$$
\end{thm}


\noindent{\bf 练习:}证明如下等式
$$d_n=nd_{n-1}+(-1)^n,$$

提示:设$\pi_n=i,$由于$\pi$为错排,$i\neq
n,$考虑$\pi_i$的不同取值就可以得到上述递推关系式。

$$n!=\sum_{k=0}^n{n\choose k}d_k.$$

提示:反着运用一下容斥原理即可。

\subsection{下降集}
对于$\pi\in\mathfrak{G}_n$,
如果$\pi=\pi_1\pi_2\ldots\pi_n$那么定义下降集为:
$$D(\pi)=\{i:\pi_i>\pi_{i+1}\}.$$
如果$S\subseteq[n-1],$
那么用$\alpha(S)$表示下降集包含在$S$中的排列$\pi\in\mathfrak{G}_n$的个数,用
$\beta(S)$表示下降集等于$S$的排列的个数。用符号表示,即为:
\begin{eqnarray*}
\alpha(S)=card\{\pi\in\mathfrak{G}_n:D(\pi)\subseteq S\};\\
\beta(S)=card\{\pi\in\mathfrak{G}_n:D(\pi)=S\}.
\end{eqnarray*}
显然
$$\alpha(S)=\sum_{T\subseteq S}\beta(S).$$
\begin{prop}
设$S=\{s_1,s_2,\ldots,s_k\}_<\subseteq[n-1],$ 则
$$\alpha(S)={n\choose s_1,s_2-s_1,\ldots,n-s_k}.$$
\end{prop}
\pf 为了得到一个满足$D{\pi}\subseteq
S$的排列$\pi=\pi_1\pi_2\ldots\pi_n$,
首先选择$\pi_1<\pi_2<\cdots<\pi_{s_1},$
 有${n\choose s_1}$种选择,然后选择$\pi_{s_1+1}<\pi_{s_1+2}<\cdots<\pi_{s_2},$ 有${n-s_1\choose s_2-s_1}$种选择,依此类推,可得:
 \begin{eqnarray*}
 \alpha(S)&=&{n\choose s_1}{n-s_1\choose s_2-s_1}\cdots{n-s_k\choose n-s_k}\\
 &=&{n\choose s_1,s_2-s_1,\ldots,n-s_k}
 \end{eqnarray*}

 接下来我们利用一下容斥原理计算$\beta(S)$
由于
$$\alpha(S)=\sum_{T\subseteq S}\beta(S),$$ 所以
$$\beta(S)=\sum_{T\subseteq S}(-1)^{|S-T|}\alpha(T).$$
而我们已知$\alpha(S)={n\choose s_1,s_2-s_1,\ldots,n-s_k},$ 因此
$$\beta(S)=\sum_{1\leq i_1\leq i_2\leq\cdots \leq i_j\leq k}(-1)^{k-j}{n\choose s_{i_1},s_{i_2}-s_{i_1},\ldots,n-s_{i_j}}.$$
通过变化上述式子可以转换为
$$\beta(S)=n!\det[1/(s_{j+1}-s_i)!].$$




\chapter{生成函数}
\label{chap1} \minitoc

\section{生成函数}
生成函数(generating function)方法最初是由Laplace和Euler引进的,是组合计数中一个很有效的方法。
在了解生成函数的具体定义之前,我们首先从我们熟知的Fibonacci数列开始,了解一下生成函数的运用。

意大利数学家Fibonacci在13世纪提出了如下的一个问题:

最初有一对小兔子(雌雄各一),这对兔子从第二个月开始每月都产下一对雌雄各一的小兔,每对新生小兔间隔一个月后也开始每月都产下
一对雌雄各一的小兔。假定兔子都不死亡,最终会有多少对兔子。

著名的Fibonacci数列由此而得名。若设$F_n$表示第$n$个月所有的兔子对数,则我们不难得出如下递推关系式:
$$F_0=F_1=1,\ F_{n+2}=F_{n+1}+F_n\ \ (n\geq 0).$$

先给出生成函数的一个粗略的定义:令$a_0,a_1,a_2,\ldots$为一无穷序列,则
$f(x)=\sum_{n\geq0}a_nx^n$
称为该无穷序列的生成函数。


由上述定义,我们现在计算一下Fibonacci数列的生成函数$F(x).$

\begin{eqnarray*}
F(x)&=&\sum_{n=0}^{\infty}F_nx^n=1+x+\sum_{n=2}^{\infty}F_nx^n\\
&=&1+x+\sum_{n=2}^{\infty}(F_{n-1}+F_{n-2})x^n\\
&=&1+x+x\sum_{n=2}^{\infty}F_{n-1}x^{n-1}+x^2\sum_{n=2}^{\infty}F_{n-2}x^{n-2}\\
&=&1+xF(x)+x^2F(x)
\end{eqnarray*}
由此可得$F(x)=(1-x-x^2)^{-1}.$

此即Fibonacci数列的生成函数,因$1-x-x^2$的两根为
$$\alpha=\frac{1+\sqrt{5}}{2}\ \ \beta=\frac{1-\sqrt{5}}{2}.$$
于是
\begin{eqnarray*}
(1-x-x^2)^{-1}&=&(1-\alpha x)^{-1}(1-\beta x)^{-1}\\
&=&\frac{\alpha/(1-\alpha x)-\beta/(1-\beta x)}{\alpha-\beta}\\
&=&\sum_{n=0}^{\infty}\frac{\alpha^{n+1}-\beta^{n+1}}{\alpha-\beta}x^n
\end{eqnarray*}
因此
\begin{equation}
F_n=\frac{\alpha^{n+1}-\beta^{n+1}}{\alpha-\beta}=\frac{1}{\sqrt{5}}(\alpha^{n+1}-\beta^{n+1}).
\end{equation}
通过上述的例子,我们得知通过生成函数的方法可以求解一些计数上的问题。通过生成函数的变换我们还可以得知一些
数列的性质。我们同样以Fabonacci数列为例来进行说明。

\begin{prop}
Fibonacci数列满足如下恒等式:
$$\sum_{i=0}^nF_i=F_{n+2}-1.$$
\end{prop}
\pf 注意到对任意级数$\sum_{n}a_nx^n,$ 有$(1-x)^{-1}\sum_{n}a_nx^n=\sum_n(\sum_{i=0}^na_i)x^n$成立,于是从
$$1=F(x)(1-x-x^2)=F(x)(2-x-x^2)-F(x)$$
得
$$F(x)=F(x)(2-x-x^2)-1=F(x)(2+x)(1-x)-1$$
所以
$$(1-x)^{-1}F(x)=F(x)(2+x)-(1-x)^{-1}.$$
比较两边的系数,就得到上式。\qed

通过上述我们熟知的例子,我们对生成函数的方法有个一个大致的了解。接下来我们详细给出生成函数的严格定义以及一些常用的
类型。

\begin{defi}
设$g_i(x)(i=0,1,2,\ldots)$线性无关,则称
\begin{equation}\label{ge1}
G(x)=\sum_{i=0}^{\infty}a_ig_i(x)
\end{equation}
为$a_i(i=0,1,2,\ldots)$的生成函数。
\end{defi}


$(\ref{ge1})$式称为关于未定元$x$的形式幂级数。一般情况下,形式幂级数中的$x$只是一个抽象符号,并不需要对$x$赋予具体的数值。
进而也就不需要讨论级数收敛性的问题。

$\mathbb{R}$上关于未定元$x$的形式幂级数的全体记为$\mathbb{R}(x)$。在集合$\mathbb{R}(x)$中适当定义加法$(+)$和乘法
$(\cdot)$, 则$(\mathbb{R}(x),+,\cdot)$构成环。

设$A(x)=\sum_{i=0}^{\infty}a_ig_i(x),\ B(x)=\sum_{i=0}^{\infty}b_ig_i(x)$. 定义
\begin{eqnarray*}
A(x)+B(x)=\sum_{i=0}^{\infty}(a_i+b_i)g_i(x).
\end{eqnarray*}

以上是形式幂级数的加法,与$g_i(x)$的具体形式无关。而形式幂级数的乘法在定义的时候会依据$g_i(x)$的不同而有细微的变化。
接下来我们先了解一下生成函数的一些常见形式。

\begin{defi}
取$g_i(x)=x^i$, 则有
$$f(x)=\sum_{i\geq0}a_ix^i$$
称为$a_i$的普通型生成函数。例如:$\{1,1,1,1,\ldots\}$的普通型生成函数为:
$$f(x)=\sum_{i\ge0}x^i=\frac{1}{1-x}.$$
\end{defi}

\begin{defi}
取$g_i(x)=\frac{x^i}{i!}$, 则有
$$f(x)=\sum_{i\geq0}a_i\frac{x^i}{i!}$$
称为$a_i$的指数型生成函数。例如:$\{0!,1!,2!,3!,\cdots\}$的指数型生成函数为:
$$f(x)=\sum_{i\ge0}i!\frac{x^i}{i!}=\frac{1}{1-x}.$$
$\{1,1,1,\ldots\}$的指数型生成函数为:
$$f(x)=\sum_{i\ge0}\frac{x^i}{i!}=e^x.$$
\end{defi}

以上是常用的两种生成函数的形式,下来我们就依据两者的具体形式给出生成函数乘法的定义。
对于普通性生成函数而言:
\begin{thm}
$f(x)=\sum a_1x^i$和$g(x)=\sum b_ix^i$
是两个生成函数,则:
$$f(x)g(x)=\sum c_ix^i,$$
其中$c_i=\sum_{k=0}^i a_kb_{i-k}.$
\end{thm}

对于指数型生成函数而言:
\begin{thm}
$f(x)=\sum a_i\frac{x^i}{i!}$和$g(x)=\sum b_i\frac{x^i}{i!}$
是两个生成函数,则:
$$f(x)g(x)=\sum c_i\frac{x^i}{i!},$$
其中$c_i=\sum_{k=0}^i{i\choose k}a_kb_{n-k}.$
\end{thm}

生成函数的乘法有严格的组合意义。假设$a_i$
所计数的组合结构称为A-结构,$b_i$
计数的是B-结构,那么$c_i$计数的则是由部分A-结构和部分B-结构组合而成的结构。

\section{生成函数的计算}
在了解完生成函数的具体定义之后,我们现在具体看一下怎么利用生成函数进行计算。
如果我们已知了$a_i,b_i$之间的关系,如何推出$A(x)=\sum_{i=0}^{\infty}a_i$与
$B(x)=\sum_{i=0}^{\infty}b_i$之间的关系。

如下我们列出一些常见的关系:
$$b_k=\sum_{i=0}^ka_i\Rightarrow B(x)=\frac{A(x)}{1-x}.$$
\pf 由假设可得:
\begin{align}
b_0&=a_0\notag\\
b_1x&=(a_0+a_1)x\notag\\
b_2x^2&=(a_0+a_1+a_2)x^2\notag\\
&\cdots\notag\\
b_nx^n&=(a_0+a_1+a_2+\cdots+a_n)x^n\notag\\
&\cdots\notag
\end{align}
等式左端相加为$B(x).$ 等式右端相加,得
\begin{align}
&a_0+(a_0+a_1)x+(a_0+a_1+a_2)x^2+\cdots\notag\\
&=a_0\sum_{i=0}^{\infty}x^i+a_1x\sum_{i=0}^{\infty}x^i+a_2x^2\sum_{i=0}^{\infty}x^i+\cdots\notag\\
&=\sum_{i=0}^{\infty}a_i\sum_{i=0}^{\infty}x^i\notag\\
&=\frac{A(x)}{1-x}.\notag
\end{align}
因此
$$B(x)=\frac{A(x)}{1-x}.$$

用类似的方法还可以证明如下几个等式:
\begin{align}
&\mbox{若}\sum_{i=0}^{\infty}a_i\mbox{收敛,且}
b_k=\sum_{i=k}^{\infty}\Rightarrow B(x)=\frac{A(x)-xA(x)}{1-x}\notag\\
&b_k=ka_k\Rightarrow B(x)=xA(x)^{'}\notag\\
&b_k=\frac{a_k}{1+k}\Rightarrow B(x)=\frac{1}{x}\int_0^1A(x)dx\notag
\end{align}

\section{生成函数的运用}
与组合相关的很多计数问题都会用到生成函数这一工具。现在我们看一下有关二叉树的例子。
设$c_n$表示有$n$个结点的不同的二叉树的个数。则有$c_0=1.$ 在$n>0$时,二叉树由一个根节点和$n-1$个儿子结点,设左子树
$T_l$有$k$个结点,则右子树$T_r$有$n-1-k$个结点,从而
$$c_n=\sum_{k=0}^{n-1}c_kc_{n-1-k},\ \ \ n>0.$$
设$c_n$的生成函数为$B(x)=\sum_{i=0}^{\infty}c_ix^i,$ 于是$B(x)$满足如下方程:
$$xB(x)^2=B(x)-1,\ \ B(0)=1.$$
解方程得
\begin{align}
B(x)&=\frac{1-\sqrt{1-4x}}{2x}\notag\\
&=\frac{1-\sum_{n\geq 0}{1/2\choose n}(-4x)^n}{2x}\notag\\
&=\sum_{n\geq 0}{1/2\choose n+1}(-1)^n2^{2n+1}x^n\notag
\end{align}
因此
$$c_n={1/2\choose n+1}(-1)^n2^{2n+1}=\frac{1}{n+1}{2n\choose n}$$
$c_n$常称为Catalan数。是在组合计数中常见的数,可以用来计数很多组合物体。



\chapter{基本组合数}
\label{zuhejiegou} \minitoc



%%%%%%%%%%%%%%%%%%%%%%%%%%%%%%%%%%%%%%%%%%%%%%%%%%%%%%%%%%%%%%%%%%%%%%%%%%%%%%%%%%
\section{二项式系数}

二项式系数${n \choose
k}$计数了$n$元集合的$k$元子集的个数,在组合中具有十分重要的作用。
它的很多好的性质表现在各个组合恒等式中,
例如二项式定理,也正是源于此,它得名二项式系数。在本章中,我们将讨
论一下关于它的一些基本性质和恒等式的证明。

\subsection{帕斯卡(Pascal)公式}
对于非负整数$n$, $k$, 二项式系数${n \choose
k}$计数了$n$元集合的$k$元子集的个数,于是若 $k>n$, 则${n \choose
k}=0$,且对任意的$n$,均有${n \choose 0}=1$,若$n>0$,$1\leq k \leq
n$,则
\begin{equation}\label{e1}
{n \choose k}=\frac{n!}{k!(n-k)!}=\frac{n(n-1)\cdots
(n-k+1)}{k(k-1)\cdots 1}.
\end{equation}
\begin{thm}
(帕斯卡公式). 对于所有满足$1\leq k\leq n-1$ 的整数 $n$及$k$,均有
\begin{equation}\label{e2}
{n\choose k}={n-1 \choose k}+{n-1\choose k-1}.
\end{equation}
\end{thm}
\noindent{\textbf{证明}:}
首先可直接利用公式(\ref{e1})展开等式的两边即可得到,读者可以自己验算一下。
下面我们介绍一种组合方法,设$S$是一个$n$元集合,$x$为其中的一个元素,下面我们将集合$S$的$k(k\leq
n-1)$元 子集$X$分两种情况进行讨论。

情形1: $x\in
X$,则还需在除$x$外的$n-1$元集合中取$k-1$元子集作为$X$中的元素,共有${n-1
\choose k-1}$种方法;

情形2:$x\overline{\in}
X$,则需在除$x$外的$n-1$元集合中取$k$元子集作为$X$中的元素,共有${n-1
\choose k}$种方法。

从而,由加法原理,$S$ 的$k(k\leq n-1)$元子集 $X$ 共有${n-1 \choose
k-1}+{n-1 \choose k}$个,即
$${n\choose k}={n-1 \choose k}+{n-1\choose k-1}.$$
\qed
\begin{exa}
令$n=4$,$k=3$,$S=\{x,a,b,c\}$,于是属于情形$1$的$3$元子集为$$\{x,a,b\},\
\{x,a,c\},\
\{x,b,c\}.$$此可视为集合$\{a,b,c\}$的$2$元子集。属于情形$2$的$3$元子集为$$\{a,b,c\}.$$
此可视为集合集合$\{a,b,c\}$的$3$元子集。从而$${4 \choose
3}=4=3+1={3\choose 2}+{3\choose 3}.$$
\end{exa}

由递推公式(\ref{e2}),及初始条件$${n \choose 0}=1,\ {n\choose n}=1,
\ \ \ \ \ \ (n\geq
0)$$我们不需要利用公式(\ref{e1})即可计算出二项式系数。由此种方法,
我们在计算二项式系数的过程中经常以帕斯卡三角的形式显示出来,如下图所示:

\begin{figure}[ht] \begin{picture}(17,38)(-200,0)
\setlength{\unitlength}{0.4cm}
\put(-14.5,0){\line(1,0){27}}\put(-14.5,1.5){\line(1,0){27}}
\put(-14.5,-16.5){\line(1,0){27}} \put(-12,1.5){\line(0,-1){18}}
\put(-10,1.5){\line(0,-1){18}} \put(-8,1.5){\line(0,-1){18}}
\put(-6,1.5){\line(0,-1){18}} \put(-4,1.5){\line(0,-1){18}}
\put(-2,1.5){\line(0,-1){18}} \put(0,1.5){\line(0,-1){18}}
\put(2,1.5){\line(0,-1){18}} \put(4,1.5){\line(0,-1){18}}
\put(6,1.5){\line(0,-1){18}} \put(8,1.5){\line(0,-1){18}}
\put(-14,0.2){$n/k$}\put(-11.3,0.2){$0$}\put(-9.3,0.2){$1$}\put(-7.3,0.2){$2$}\put(-5.3,0.2){$3$}\put(-3.3,0.2){$4$}
\put(-1.3,0.2){$5$}\put(0.7,0.2){$6$}\put(2.7,0.2){$7$}\put(4.7,0.2){$8$}\put(6.7,0.2){$9$}\put(9.7,0.2){$\cdots$}
\put(-13.3,-1.3){$0$}\put(-11.3,-1.3){$1$}
\put(-13.3,-2.8){$1$}\put(-11.3,-2.8){$1$}\put(-9.3,-2.8){$1$}
\put(-13.3,-4.3){$2$}\put(-11.3,-4.3){$1$}\put(-9.3,-4.3){$2$}\put(-7.3,-4.3){$1$}
\put(-13.3,-5.8){$3$}\put(-11.3,-5.8){$1$}\put(-9.3,-5.8){$3$}\put(-7.3,-5.8){$3$}\put(-5.3,-5.8){$1$}
\put(-13.3,-7.3){$4$}\put(-11.3,-7.3){$1$}\put(-9.3,-7.3){$4$}\put(-7.3,-7.3){$6$}\put(-5.3,-7.3){$4$}
\put(-3.3,-7.3){$1$}
\put(-13.3,-8.8){$5$}\put(-11.3,-8.8){$1$}\put(-9.3,-8.8){$5$}\put(-7.3,-8.8){$10$}\put(-5.3,-8.8){$10$}
\put(-3.3,-8.8){$5$}\put(-1.3,-8.8){$1$}
\put(-13.3,-10.3){$6$}\put(-11.3,-10.3){$1$}\put(-9.3,-10.3){$6$}\put(-7.3,-10.3){$15$}\put(-5.3,-10.3){$20$}
\put(-3.3,-10.3){$15$}\put(-1.3,-10.3){$6$}\put(0.7,-10.3){$1$}

\put(-13.3,-11.8){$7$}\put(-11.3,-11.8){$1$}\put(-9.3,-11.8){$7$}\put(-7.3,-11.8){$21$}\put(-5.3,-11.8){$35$}
\put(-3.3,-11.8){$35$}\put(-1.3,-11.8){$21$}\put(0.7,-11.8){$7$}\put(2.7,-11.8){$1$}
\put(-13.3,-13.3){$8$}\put(-11.3,-13.3){$1$}\put(-9.3,-13.3){$8$}\put(-7.3,-13.3){$28$}\put(-5.3,-13.3){$56$}
\put(-3.3,-13.3){$70$}\put(-1.3,-13.3){$56$}\put(0.7,-13.3){$28$}\put(2.7,-13.3){$8$}\put(4.7,-13.3){$1$}
\put(-13.2,-15.3){$\vdots$}\put(-11.2,-15.3){$\vdots$}\put(-9.2,-15.3){$\vdots$}\put(-7.2,-15.3){$\vdots$}
\put(-5.2,-15.3){$\vdots$}\put(-3.2,-15.3){$\vdots$}\put(-1.2,-15.3){$\vdots$}\put(0.8,-15.3){$\vdots$}
\put(2.8,-15.3){$\vdots$}\put(4.8,-15.3){$\vdots$}\put(6.8,-15.3){$\vdots$}
\end{picture}
\vspace{6.5cm} \caption{帕斯卡三角.} \label{pascal}
\end{figure}

图中除了最左侧一列的$1$以外,其余的值都可以通过上行中同列及其左邻的值之和。例如,对$n=6$,我们有
$${6 \choose 3}=20=10+10={5 \choose 3}+{5 \choose 2}.$$

二项式系数的许多性质和恒等式均可以通过帕斯卡三角得到,将帕斯卡三角某行的元素加起来可发现,
$${n \choose 0}+{n\choose 1}+\cdots+{n\choose n}=2^n.$$

下面给出帕斯卡三角的另一种组合解释,令$n$,$k$为满足$0\leq k\leq
n$的非负整数,定义$p(n,k)$计数了帕斯卡三角中左上角(数值${0 \choose
0}=1$)到数值${n\choose
k}$的路的条数,其中路的每一步均为向南走一个单位或向东南方向走一个单位,即按向量方向$(0,-1)$和$(1,-1)$走一步。
\begin{figure}[ht]
\begin{picture}(20,20)(20,0)
\setlength{\unitlength}{0.5cm} \thicklines
\put(10,0){\line(0,-1){3}} \put(10,0){\circle*{0.2}}
\put(10,-3){\circle*{0.2}} \put(10,-1,4){\vector(0,-1){0.4}}
\put(18,0){\circle*{0.2}}\put(21,-3){\circle*{0.2}}\put(21,0){\circle*{0.2}}
\put(18,0){\line(1,-1){3}} \put(19.4,-1,4){\vector(1,-1){0.4}}
\end{picture}
\vspace{2cm} \caption{步子}\label{path}
\end{figure}

我们约定$p(0,0)=1$,且对任意的非负整数$n$,均有
$p(n,0)=1$,(每一步都必须朝下走一直到${n\choose
0}$)及$p(n,n)=1$。(每一步都不需沿对角线走直到${n\choose n}$)
注意到每一条从${0\choose 0}$到${n\choose k}$的路均可看为是

(i)从${0\choose 0}$到${n-1\choose k}$的路再加上一个竖直步子,

或

(ii)从${0\choose 0}$到${n-1\choose k-1}$的路再加上一个对角步子。

从而,由加法原理,我们有递推关系
$$p(n,k)=p(n-1,k)+p(n-1,k-1).$$
观察到$p(n,k)$与二项式系数有相同的初始条件及递推关系,故而易知对任意满足$0\leq
k\leq n$的非负整数$n$,$k$,有
$$p(n,k)={n\choose k}.$$
于是帕斯卡三角的各数值也表示从左上角到该数值的路的条数。这也给了二项式系数以新的组合解释。


%%%%%%%%%%%%%%%%%%%%%%%%%%%%%%%%%%%%%%%%%%%%%%%%%%
\subsection{二项式定理}
二项式系数是从二项式定理中得名的,本节中将介绍有关二项式定理的恒等式,它作为代数恒等式我们在高中就已经接触过。
\begin{thm}
设$n$是正整数,则对任意的实数$x$,$y$,均有
\begin{equation}\label{e3}
(x+y)^n=\sum_{k=0}^n{n\choose k}x^{n-k}y^k.
\end{equation}
\end{thm}
\noindent{\textbf{证明}:}(方法1)将$(x+y)^n$写成$n$个因子的乘积$$(x+y)(x+y)\cdots(x+y).$$
由乘法的分配律,展开乘积并合并同类项,由于每一个因子$(x+y)$,我们都有$x$,$y$两种选择,所以展开式中共有$2^n$项,
且每项都是$x^{n-k}y^k$($k=0,1,2,\ldots,n$)的形式。项$x^{n-k}y^k$是在$n$个因子中$k$个选取$y$,其余$n-k$个因子中取$x$,
于是$x^{n-k}y^k$在展开式中出现的次数为$x^{n-k}y^k$,即展开式中项$x^{n-k}y^k$的系数为${n\choose
k}$,从而
$$(x+y)^n=\sum_{k=0}^n{n\choose k}x^{n-k}y^k.$$
\qed
\noindent{\textbf{证明}:}(方法2)对$n$进行数学归纳法。对$n=1$,式(\ref{e3})为$$(x+y)^1=\sum_{k=0}^1{1\choose
k}x^{1-k}y^k=x+y,$$
显然成立。假设式(\ref{e3})对整数$n$成立,即$(x+y)^n=\sum_{k=0}^n{n\choose
k}x^{n-k}y^k$,下考虑$n+1$的情形,
\begin{align*}
(x+y)^{n+1}&=(x+y)(x+y)^n=(x+y)\left(\sum_{k=0}^n{n\choose k}x^{n-k}y^k\right)\\
&=x\left(\sum_{k=0}^n{n\choose k}x^{n-k}y^k\right)+y\left(\sum_{k=0}^n{n\choose k}x^{n-k}y^k\right)\\
&=\sum_{k=0}^n{n\choose k}x^{n+1-k}y^k+\sum_{k=0}^n{n\choose k}x^{n-k}y^{k+1}\\
&={n\choose 0}x^{n+1}+\sum_{k=1}^n{n\choose
k}x^{n+1-k}y^{k}+\sum_{k=0}^{n-1}{n\choose
k}x^{n-k}y^{k+1}+{n\choose n}y^{n+1}
\end{align*}
在后一个连加项中,用$k-1$代替$k$,得到
$$\sum_{k=1}^n{n\choose k-1}x^{n-k+1}y^k.$$
从而$$(x+y)^{n+1}=x^{n+1}+\sum_{k=1}^n\left[{n\choose k}+{n\choose
k-1}\right]x^{n+1-k}y^k+y^{n+1},$$
利用帕斯卡公式,上式等价于$$(x+y)^{n+1}=x^{n+1}+\sum_{k=1}^n{n+1\choose
k}x^{n+1-k}y^k+y^{n+1}=\sum_{k=0}^{n+1}{n+1\choose
k}x^{n+1-k}y^k,$$恰满足式(\ref{e3}),由归纳法原理定理得证。 \qed

一般情况下,我们常常利用到一种特殊情形,即当$y=1$时,我们有如下推论。
\begin{coro}
令$n$是正整数,则对任意的实数$x$均有$$(1+x)^n=\sum_{k=0}^n{n\choose
k}x^k=\sum_{k=0}^n{n\choose n-k}x^k.$$
\end{coro}

%%%%%%%%%%%%%%%%%%%%%%%%%%%%%%%%%%%%%%%%%%%%%%%
\subsection{恒等式}
本节中,我们将考虑一些关于二项式系数的恒等式并给出它们的组合解释。首先由二项式系数的代数展开式,很快地,我们有
\begin{equation}\label{e4}
k{n\choose k}=n{n-1 \choose k-1}.
\end{equation}
作为代数式,我们很容易根据二项式的表达式证明,但是作为组合恒等式,式(\ref{e4})又有怎么样的组合意义呢?

我们考虑这样一个实际问题,从$n$个人中选出$k$个人组成一个足球队,并选出队长,完成这项事件共有多少种选择方案?
一种计数方法是我们先选出足球队,则有${n\choose
k}$种,再在这选出的$k$个人中选出队长,有$k$种,于是由乘法原理完成这项事件共有$k{n\choose
k}$种选择方案。
另一种计数方法是先从$n$个人中选出队长,有$n$种选择,再在剩下的$n-1$个人中选出$k-1$个人与队长一起组成足球队,有${n-1\choose
k-1}$种选择,于是由乘法原理完成这项事件共有$n{n-1 \choose
k-1}$种选择方案。
于是式(\ref{e4})两边计数的是同一事件的选择方案,故而等式成立。

在二项式定理中若同时取$x=1$,$y=1$,则可得到恒等式
\begin{equation}\label{e5}
{n\choose 0}+{n\choose 1}+\cdots+{n\choose n}=2^n.
\end{equation}
若取$x=1$,$y=-1$,则可得到恒等式
$${n\choose 0}-{n\choose 1}+{n\choose 2}-\cdots+(-1)^n{n\choose n}=0,\ \ \ (n\geq 1).
$$
也即
\begin{equation}\label{e6}
{n\choose 0}+{n\choose 2}+\cdots={n\choose 1}+{n\choose 3}+\cdots,\
\ \ (n\geq 1).
\end{equation}
由式(\ref{e5}),要证明式(\ref{e6}),只需证明
$${n\choose 0}+{n\choose 2}+\cdots=2^{n-1}.$$
下面我们将给出该式的组合解释。令$S={x_1,x_2,\ldots,x_n}$为$n$元集合,我们需要计数的是$S$的偶子集$X$的个数,
我们对$S$中元素逐个考虑,首先考虑$x_1$,我们有放不放入$X$中两种选择,考虑$x_2$,我们也有放不放入$X$中两种选择,接着
考虑$x_3$,一直到$x_{n-1}$,均有放不放入$X$中两种选择,最后对于$x_n$,若前面放入$X$的元素个数为奇,则将$x_n$放入$X$中,
否则将$x_n$不放入$X$中。所以在整个事件中,前$n-1$步均有两种选择,最后一步只有一种选择,于是由乘法原理,
$S$的偶子集的个数为$2^{n-1}$,又${n\choose 0}+{n\choose
2}+\cdots$也计数了$n$元集合的偶子集的个数,故而
$${n\choose 0}+{n\choose 2}+\cdots=2^{n-1}.$$
\qed

同样的道理,我们可以证明$${n\choose 1}+{n\choose
3}+\cdots=2^{n-1}.$$

利用恒等式(\ref{e4})和(\ref{e5}),我们可得到
\begin{equation}\label{e7}
1{n\choose 1}+2{n\choose 2}+\cdots+n{n \choose n}=n2^n,\ \ (n\geq
1).
\end{equation}
式(\ref{e7})也可以根据二项式定理而得到,在二项式公式
$$(1+x)^n={n\choose 0}+{n\choose 1}x+{n\choose 2}x^2+\cdots+{n\choose n}x^n$$
两边同时对$x$求导,我们有
$$n(1+x)^{n-1}={n\choose 1}+2{n\choose 2}x+\cdots+n{n\choose n}x^{n-1},$$
最后,令$x=1$,即可得到式(\ref{e7})。





%%%%%%%%%%%%%%%%%%%%%%%%%%%%%%%%%%%%%%%%%%%%%%%%%%%%%%%%%%%%%%%%%%%%%%%%%%%%%%%%%%%%%%
\section{Stirling数}

\subsection{Stirling 简介} James
Stirling是一位苏格兰籍的数学家。他1692年5月出生于苏格兰斯特灵郡,1770年10月逝世于爱丁堡。
Stirling数以及Stirling逼近都是以他的名字命名的。James
Stirling18岁进入牛津大学贝利奥尔学院(Balliol College,
Oxford)求学,后被驱逐至威尼斯。在威尼斯期间,James
Stirling在艾萨克.牛顿的帮助下,与皇家科学院取得联系并邮寄了他的一篇论文
Methodus differentials Newtoniana illustrata (Phil.Trans.,
1718)。后又在牛顿的帮助下于1725年回到伦敦。在伦敦的十年时间里,他一直致力于学术工作,
1730年,他最重要的工作 the methodus differentialis, sive tractatus
de summatione et interpolatione serierum infinitarum (4to,
London)发表。

这里我们主要介绍一下他的一个重要工作,Stirling数。Stirling数分为第一类和第二类。
下面先看一下Stirling数的具体定义。

%%%%%%%%%%%%%%%%%%%%%%%%%%%%%%%%%%%%%%%%%%%%%%%%
\subsection{第一类Stirling数}

\begin{defi}
$c(n,k)$为恰好含有$k$个圈的$\pi\in\mathfrak{S}_n$的个数。数
$s(n,k):=(-1)^{n-k}\\c(n,k)$被称为{\bf
第一类Stirling数},\index{第一类斯特林数,Stirling number of the
first kind}而 $c(n,k)$被称为{\bf 无符号的第一类Stirling数}。
\end{defi}

\begin{defi}
{\bf 降阶乘函数}\index{降阶乘函数,falling factorial}$(x)_n$定义如下
$$(x)_n=x(x-1)\cdots(x-n+1).$$
\end{defi}

显然通常的阶乘$n!=(n)_n.$

第一类Stirling数有如下等价定义:

\begin{defi}
{\bf 第一类Stirling数}(记为$s(n,k)$, $1\leq k\leq
n$)恰好为降阶乘\index{降阶乘,falling
factorial}多项式中的各项系数,即
\begin{equation}
(x)_n=x(x-1)(x-2)\cdots(x-n+1)=\sum_{k=1}^n{s(n,k)x^k}.
\end{equation}
\end{defi}

\begin{thm}
$c(n,k)$ 满足如下递推式
$$c(n,k)=(n-1)c(n-1,k)+c(n-1,k-1),\  n,k\geq 1$$
并且初值为$c(0,0)=1$,而对其它的$n\leq 0$或者$k\leq 0$,$c(n,k)=0$。
\end{thm}

{\bf 证明:}选定一个有$k$个圈的排列$\pi\in\mathfrak{S}_{n-1}.$在$
\pi$的不交圈分解中,我们可以将$n$插入数字$1,2,\ldots,n-1$的任一个的后面,
有$n-1$种方法。这样就得到一个排列$\pi^{'}\in\mathfrak{S}_n$的不交圈分解,它具有
$k$个圈并且$n$出现在一个长度$\geq
2$的圈中。于是共有$(n-1)c(n-1,k)$个排列
$\pi^{'}\in\mathfrak{S}_n$具有$k$个圈并且满足$\pi^{'}(n)\neq n.$

另一方面,如果选定一个有$k$个圈的排列$\pi^{'}\in\mathfrak{S}_{n-1},$可以通过定义
  $$\pi^{'}(i)=\left\{\begin{array}{ll}
                          \pi(i), & \mbox{若}i\in[n-1]; \\
                          n, &\mbox{若} i=n.
                        \end{array}\right.
    $$

将其扩充为一个具有$k$个圈的排列$\pi^{'}\in\mathfrak{S}_n,$并且有
$\pi^{'}(n)=n$。因此,共有$c(n-1,k-1)$个排列$\pi^{'}\in\mathfrak{S}_n$具有
$k$个圈并且满足$\pi^{'}(n)=n,$得证。\qed

%%%%%%%%%%%%%%%%%%%%%%%%%%%%%%%%%%%%%%%%%%%%%%%%
\subsection{第二类Stirling数}
\begin{defi}
把$n$个元素构成的集合划分为$k$个非空子集的方法数,称为{\bf
第二类Stirling数}\index{第二类斯特林数,Stirling number of the
second kind},记为$S(n,k)$.
\end{defi}

举例而言,集合$[3]$划分成三个子集合的方法只有一种: $1/2/3$;
划分成两个子集合的方法有三种: $12/3,13/2,1/23$;
划分成一个子集合的方法只有一种: $1\ 2\ 3$.


\begin{thm}
第二类Stirling数有如下的递归关系式:
\begin{equation}S(n,k)=kS(n-1,k)+S(n-1,k-1),\ 1\le k<n,\end{equation}
其中初始条件为$S(n,n)=S(n,1)=1.$
\end{thm}
{\bf 证明:}
我们直接从第二类Stirling数的定义来给出这个递归关系式的证明。
显然,把$n$个元素放在一个集合和$n$个集合里都只有一种放法。
现假设把$n-1$个元素放在$k$个集合里的的方法数为$S(n-1,k).$
现在考虑把$n$个元素 $a_1, a_2, \ldots,
a_n$\\放在$k$个集合里,我们将$a_n$单独拿出来考虑。会有如下两种方式:
\begin{itemize}
\item[1. ]将前$n-1$个元素放入$k$个集合中,
再将$a_n$放入这$k$个集合中的某一个。这样一共有$kS(n-1,k)$种放法。
\item[2. ]将前$n-1$个元素放入$k-1$个集合中,
再将$a_n$单独放在一个新的集合中。这样给出另外$S(n-1,k-1)$种方法。
由此定理证明完毕。\qed
\end{itemize}

现在我们考虑由变量为$x$的多项式组成的向量空间。
对于这个无限维的向量空间最明显的一组基是单项式幂级数$x^n,\  n\geq
0.$ 然而同时,降阶乘函数
$$(x)_n=x(x-1)\cdots(x-n+1),\ n\geq 0$$
也是这个向量空间的一组基,自然能生成幂级数$x^n.$ 其实,
第二类Stirling数就是这两组基之间的过渡矩阵里的元素, 即:
\begin{equation}
x^n=\sum_{k=1}^nS(n,k)(x)_k.\label{1}
\end{equation}
例如,$x^4=x+7x(x-1)+6x(x-1)(x-2)+x(x-1)(x-2)(x-3)$.

{\bf 证明:}现用归纳法证明上述递归关系式\ref{1}。
显然\ref{1}对于$n=1$成立。 \\
假设\ref{1}对于$n$成立。由降阶乘函数的定义我们得到
$$x(x)_k=(x)_{k+1}+k(x)_k,$$
据此我们由归纳假设得到递归关系式
\begin{align*}
x^{n+1}&=\sum_{k=1}^nS(n,k)x(x)_k\\
&=\sum_{k=1}^n S(n,k)[(x)_{k+1}+k(x)_k]\\
&=\sum_{k=1}^{n}[S(n,k-1)+kS(n,k)](x)_k+S(n,n)(x)_{n+1}\\
&=\sum_{k=1}^{n+1}S(n+1,k)(x)_k.
\end{align*}
所以\ref{1}得证。\qed

接下来给出\ref{1}的一个简单的组合证明。考虑映射$f:N\rightarrow
X$,其中$|N|=n$而$|X|=x$。 \ref{1}的左边计数了映射$f:N\rightarrow
X$的总数。而每一个这样的映射都是到$X$的某个满足 $|Y|\leq
n$的子集$Y$的满射,且$Y$唯一。
如果$|Y|=k$,则有$k!S(n,k)$个这样的映射。而满足$|Y|=k$的$X$的子集$Y$有${x\choose
k}$种选择,因此
$$x^n=\sum_{k=0}^nk!S(n,k){x\choose k}=\sum_{k=0}^nS(n,k)(x)_k.$$

现在我们了解了有关于第二类Stirling数的组合解释,以及它作为幂函数与降阶乘函数之间过渡矩阵的元素。
下面我们计算一下$S(n,k)$的具体数值。

在计算的过程中,我们会用到有限差分演算,下面先简单罗列一下我们计算中会用到的一些概念和公式。
给定一个映射$f:\mathbb{N}\rightarrow\mathbb{C}$,定义一个新的映射$\Delta
f$如下,称之为$f$的{\bf 一阶差分},
$$\Delta f(n)=f(n+1)-f(n).$$
$\Delta$称为一阶{\bf
差分算子}。将算子$\Delta$重复$k$次就可以得到$k$阶差分算子。
$$\Delta^k f=\Delta(\Delta^{k-1}f).$$
以上是差分算子的概念,下面的式子在我们的计算中会起到很重要的作用。
\begin{equation}
f(n)=\sum_{k=0}^n{n\choose k}\Delta^k f(0).\label{e2}
\end{equation}

如果将连续两项$f(i),f(i+1)$的差$f(i+1)-f(i)=\Delta
f(i)$写在它们下一行的中间,就得到序列
$$\ldots \Delta f(-2)\ \Delta f(-1)\ \Delta f(0)\ \Delta f(1)\ \Delta f(2)\ldots$$
反复这个过程,就得到映射$f$的差分表,它的第$k$行由$\Delta^k
f(n)$组成。从$f(0)$开始往右下方延伸的对角线则是由0点的差分$\Delta^k
f(0)$构成。例如,令 $f(n)=n^4,$则差分表(从$f(0)$开始)如下


$\begin{array}{cccccccccccc}
  0 &  & 1 &  & 16 &  & 81 &  & 256 &  & 625 & \cdots \\
   & 1 &  & 15 &  & 65 &  & 175 &  & 369 &  &  \\
   &  & 14 &  & 50 &  & 110 &  & 194 &  &  &  \\
   &  &  & 36 &  & 60 &  & 84 &  &  &  &  \\
   &  &  &  & 24 &  & 24 &  &  &  &  &  \\
   &  &  &  &  & 0 &  &  &  &  &  &  \\
   &  &  &  &  &  & \ddots &  &  &  &  &  \\
   &  &  &  &  &  &  &  &  &  &  &
\end{array}$

因此,由\ref{e2}得
$$n^4={n\choose 1}+14{n\choose 2}+36{n\choose 3}+24{n\choose 4}+0{n\choose 5}$$
据此再结合\ref{1}就可以得到$S(n,k)$的具体数值。

\begin{ex} 对所有$m,n\in\mathbb{N}$,成立
$$\sum_{k\geq 0}S(m,n)s(m,n)=\delta_{mn}.$$
\end{ex}

提示:$\left(S(m,n)\right)_{m,n\geq 0},\left(s(m,n)\right)_{m,n\geq
0}$刚好是两组基之间的过渡矩阵。

%%%%%%%%%%%%%%%%%%%%%%%%%%%%%%%%%%%%%%%%%%%%%%%%%%%%%%%%%%%%%%%
\subsection{第二类Stirling数的正态分布性质}

在这一节里,我们来研究第二类Stirling数的概率性质。这个性质是由Harper\cite{Harper1967}
首先给出的。回忆一下$S(m,j)$满足如下递推关系:
\begin{equation}\label{St}
S(m,j)=S(m-1,j-1)+jS(m-1,j).
\end{equation}
其中,当$m\geq
1$时,如果$j>m$,则$S(m,j)=0$。称$m$个元素的集合$X$上的所有分拆的个数为$m$-阶Bell数,即
当$m\geq 1$时,$B_m=\sum\limits_{j=1}^{m}S(m,j)$。


为了研究$S(m,j)$的分布性质,我们假定具有$m$个元素的集合$X$上的所有分拆是等可能分布的,
即每个分拆出现的可能性都是 $B_m^{-1}$。在这个假定下,如果我们
记$\xi_m$为具有$m$个元素的集合$X$上的一个随机分拆所含有的块
个数,则易知当$0\leq k\leq m$时,
\begin{equation}\label{xik}
P\{\xi_m=j\}=\frac{S(m,j)}{B_m}.
\end{equation}
下面我们来研究当$m\rightarrow \infty$时,随机变量
$\xi_m$所满足的分布。我们有如下定理:

\begin{thm}\label{secstirling}
记
\begin{equation}
\eta_m=\frac{\xi_m-E\xi_m}{\sqrt {Var\xi_m}},
\end{equation}
其中$E\xi_m$为随机变量$\xi_m$的数学期望,$Var\xi_m$为随机变量$\xi_m$的方差,
则当$m\rightarrow
\infty$,随机变量$\eta_m$的分布收敛于标准正态分布,即
\[
lim_{m\rightarrow \infty}P\{\eta_m
<x\}=\frac{1}{\sqrt{2\pi}}\int_{-\infty}^{x}e^{-u^2/2}du
\]
\end{thm}

为了证明这个定理,我们给出如下三个引理,为简单起见,我们只给出第二个引理的证明,
关于其他两个引理的证明,读者可参考文献\cite{Sachkov1997}。
\begin{lem}\label{EVar}
对于上述定义的随机变量$\xi_m$,其数学期望
\begin{equation}
E\xi_m=\frac{B_{m+1}}{B_m}-1,
\end{equation}
其方差
\begin{equation}
Var\xi_m=\frac{B_{m+2}}{B_m}-(\frac{B_{m+1}}{B_m})^2-1,
\end{equation}
进一步的,我们有
\begin{equation}
lim_{m\rightarrow \infty}Var\xi_m=\infty.
\end{equation}
\end{lem}
\noindent {\bf{证明:}} 略。\qed

下面我们利用罗尔(Roll)引理来讨论与第二类Stirling数有关的多项式的实根性。
\begin{lem}\label{Fmroot}
记
\begin{equation}
F_m(x)=\sum\limits_{j=0}^{m}S(m,j)x^j,
\end{equation}
则对任意的$m\geq 1$,多项式$F_m(x)$有$m$个不同的非正实根。
\end{lem}
\noindent
{\bf{证明:}}我们用归纳法来证明。$F_1(x)=1$,$F_2(x)=x(x+1)$。
假设引理对所有多项式$F_n(x)$,其中$n\leq m-1$都成立。
我们来证明多项式$F_m(x)$有$m$个不同的非正实根。

对$F_m(x)$关于$x$求导,并利用$S(m,j)$所满足的递推关系式(\ref{St}),我们
得到
\begin{equation}\label{Fmrec}
F_m(x)=x\left[F_{m-1}(x)+\frac{d}{dx}F_{m-1}(x)\right].
\end{equation}
根据归纳假设,$F_{m-1}(x)$有$m-1$个不同的非正实根,在$(-\infty,\infty)$上,定义
函数
\[
H_m(x)=F_m(x)e^x,
\]
则$H_m(x)$与多项式$F_m(x)$有相同的实根。根据(\ref{Fmrec})式,我们得到
\[
F_m(x)e^x=x\left[F_{m-1}(x)e^x+\left(\frac{d}{dx}F_{m-1}(x)\right)e^x\right],
\]
因此,
\begin{equation}\label{Hm}
H_m(x)=x\frac{d}{dx}H_{m-1}(x).
\end{equation}
观察到
\[
lim_{x\rightarrow -\infty}H_{m-1}(x)=lim_{x\rightarrow
-\infty}e^xF_{m-1}(x)=0,
\]
所以当$x\in [-\infty,0]$,函数$H_{m-1}(x)$有$m$个根。根据Roll定理,
在实轴上每两个相邻$H_{m-1}(x)$的根所构成的区间内,包含一个点使得
$\frac{d}{dx}H_{m-1}(x)=0$。 所以$\frac{d}{dx}H_{m-1}(x)$在实轴上有
$m$个根,其中一个为$x=-\infty$。
故根据(\ref{Hm})式,$H_m(x)$在实轴上有$m+1$个非正根,其中一个为$x=-\infty$。
综上,多项式$F_m(x)$有$m$个非正实根。\qed

最后,我们给出如下引理,这个引理实际上式著名的李雅普诺夫(Lyapunov)定理的一个推论。
我们先来引出一些概念。
考虑如下独立的随机变量$\xi_{kn}$,其中$k=1,2,\ldots,n$,$n=1,2,\ldots$,使得
\begin{align*}
P\{\xi_{kn}=1\}&=p_k,\\
P\{\xi_{kn}=0\}&=q_k,
\end{align*}
其中$p_k=p_k(n)$,$q_k=q_k(n)$,并且$p_k+q_k=1$。
我们称序列$\xi_{kn}$为一个Poisson序列。
\begin{lem}\label{Poissona}
记
\begin{align*}
T_n^2&=\sum\limits_{k=1}^{n}p_kq_k,\\
\eta_n&=T_n^{-1}\sum\limits_{k=1}^{n}(\xi_{kn}-p_k)
\end{align*}
如果当$n\rightarrow \infty$时,$T_n\rightarrow
\infty$,则序列$\{\eta_n\}$渐进正态分布。
\end{lem}
\noindent {\bf{证明:}} 略。\qed

下面,我们就来利用以上的三个引理来证明我们的定理。

 \noindent
{\bf{定理(\ref{secstirling})证明:}}根据引理(\ref{Fmroot}),
记$-\alpha_1,-\alpha_2,\ldots,-\alpha_{m-1}$为多项式$F_m(x)$除去0以外的根,则
\begin{equation}\label{F}
F_m(x)=x(x+\alpha_1)(x+\alpha_2)\cdots(x+\alpha_{m-1}).
\end{equation}
记
\begin{equation*}
P_m(x)=\sum\limits_kP\{\xi_m=k\}x^k,
\end{equation*}
为随机变量$\xi_m$的分布函数,根据定义\ref{xik},
\begin{equation*}
P_m(x)=\frac{1}{B_m}\sum\limits_kS(m,k)x^k=\frac{F_m(x)}{F_m(1)},
\end{equation*}
结合(\ref{F})式,
\begin{equation}\label{Pm}
P_m(x)=x\left(\frac{x}{1+\alpha_1}+\frac{\alpha_1}{1+\alpha_1}\right)\cdots
\left(\frac{x}{1+\alpha_{m-1}}+\frac{\alpha_{m-1}}{1+\alpha_{m-1}}\right),
\end{equation}
考虑如下独立的取值为0和1的随机变量$\xi_{m1}$,$\xi_{m2}$,$\ldots$,$\xi_{m,m-1}$,
满足
\begin{align*}
P\{\xi_{mi}=1\}&=\frac{1}{1+\alpha_i},\\
P\{\xi_{mi}=0\}&=\frac{\alpha_i}{1+\alpha_i}.
\end{align*}
其中$i=1,2,\ldots,m-1$。 记
\begin{equation*}
P_{mi}(x)=\sum\limits_kP\{\xi_{mi}=k\}x^k,
\end{equation*}
为随机变量$\xi_{mi}$的分布函数,则
\begin{equation}\label{mi}
P_{mi}(x)=\frac{x}{1+\alpha_i}+\frac{\alpha_i}{1+\alpha_i},
\end{equation}
结合(\ref{Pm})式和(\ref{mi})式,我们得到
\begin{equation}\label{ximsum}
\xi_m=\xi_{m1}+\xi_{m2}+\cdots+\xi_{m,m-1}+1.
\end{equation}
若我们定义
\begin{equation*}
\eta_{mi}=\frac{\xi_{mi}-E\xi_{mi}}{\sqrt{Var\xi_m}},
\end{equation*}
其中$i=1,2,\ldots,m-1$, 其中
\[
E\xi_{mi}=\sum\limits_kP\{\xi_{mi}=k\}=\frac{1}{1+\alpha_i},
\]
则定理叙述中的$\eta_m$满足
\begin{align*}
\eta_m&=\frac{1}{\sqrt{Var\xi_m}}\left(\xi_m-E\xi_m\right)\\
&=\frac{1}{\sqrt{Var\xi_m}}\sum\limits_{i=1}^{m-1}\left(\xi_{mi}-E\xi_{mi}\right)\\
&=\eta_{m1}+\eta_{m2}+\cdots+\eta_{m,m-1}.
\end{align*}
这样,相互独立的随机变量$\xi_{m1}$,$\xi_{m2}$,$\ldots$,$\xi_{m,m-1}$是一个
Poisson序列,满足$p_k=1/(1+\alpha_k)$,其中$k=1,2,\ldots,m-1$。因为诸
$\xi_{mi}$是相互独立的随机变量,由(\ref{ximsum})式,我们得到
\[
Var\xi_m=\sum\limits_{i=1}^{m-1}Var\xi_{mi}=\sum\limits_{i=1}^{m-1}\frac{\alpha_i}
{(1+\alpha_i)^2}=\sum\limits_{i=1}^{m-1}p_iq_i.
\]
根据引理(\ref{EVar}),$lim_{m\rightarrow
\infty}Var\xi_m=\infty$,故满足引理(\ref{Poissona})中的条件,所以$\eta_m$渐进正态分布。
\qed






\clearpage




%-------------------------------------------------------------
\section{Catalan数}
\subsection{格路}
\begin{defi}
平面上从 $(0, 0)$ 到 $(2n, 0)$ 的一条路径,如果每步只能是向 $(1, 1)$
方向或向 $(1, -1)$ 方向前进(只走格点),并且保证不穿越到 $x$
轴的下方,这样的路径被称为 Dyck path。从 $(0, 0)$ 到 $(2n, 0)$ 的
Dyck path 可简记为 $n$-Dyck path。
\end{defi}
我们知道,$n$-Dyck path 的总数是第 $n$ 项 Catalan 数
$C_{n}=\frac{1}{n+1}{2n\choose n}$。容易计算,$C_{2}=2, C_{3}=5,
C_{4}=14, C_{5}=42$。Catalan
数在组合计数中有着十分广泛的应用,许多计数问题都可以直接或间接地用
Catalan 数解决。比如乘法排序问题 ($multiplication\ orderings$):将
$n+1$
个数通过不同的顺序乘起来的方式数,注意要求保持数本身的顺序不变。

\begin{ex}
将 $xyz$ 按照不同顺序乘起来有 $C_{2}=2$ 种方式:$((xy)z), (x(yz))$。

将 $xyzw$ 按照不同顺序乘起来有 $C_{3}=5$ 种方式:$((xy)(zw)),
(((xy)z)w), \\((x(yz))w), (x((yz)w)), (x(y(zw)))$。
\end{ex}
%-------------------------------------------------------------
\subsection{有禁置换}

\begin{defi}
若置换 $\pi=\pi_1\pi_2\cdots\pi_n\in\mathfrak{S}_n$ 中不存在 $1\le
i<j<k\le n$ 使得 $\pi_i\pi_j\pi_k$ 是模式 (pattern) 312(即
$\pi_j<\pi_k<\pi_i$),则称置换 $\pi=\pi_1\pi_2\cdots\pi_n$ 是
312-禁排置换。类似可定义 123-禁排置换,321-禁排置换。
\end{defi}

由 312-禁排置换的定义容易得到下面的引理:

\begin{lem}
置换 $\pi\in\mathfrak{S}_n$ 是 312-禁排的当且仅当对 $\forall\
i\in[n]$,$i$ 右边比 $i$ 小的所有元素构成的子列是递减的。
\end{lem}

\begin{thm}
$\mathfrak{S}_n$ 中 312-禁排置换数目是第 $n$ 项 Catalan 数
$C_n=\frac{1}{n+1}{2n\choose n}$。
\end{thm}
{\bf{证明}} 我们知道 $n$-Dyck path 的数目是第 $n$ 项 Catalan 数
$C_{n}$,所以一个比较自然的想法就是试图在 $n$-Dyck paths 和
$\mathfrak{S}_n$ 中
312-禁排置换间构造一个双射。为此,我们介绍一种标号方法,这里称之为
312-禁排标号。对任意一条 $n$-Dyck path
$D$,我们按如下方式来标号:将向 (1, 1) 方向走的 $n$ 步用
$1,2,\ldots,n$ 从左到右依次标号;对向 $(1, -1)$ 方向走的 $n$
步我们这样来处理:对任意一个峰不妨设为 $ud$ (这里 $u$ 是向(1, 1)
方向走的一步,$d$ 是紧接的向 $(1, -1)$ 方向走的一步),对 $d$ 标与
$u$ 相同的标号;对剩下的尚未标号的向 $(1, -1)$ 方向走的每步
$d$,我们用 $max\{l^u\backslash l^d\}$ 从左到右依次给它标号,这里
$l^u(l^d)$ 是 $d$ 左边向(1, 1)($(1, -1)$)
方向走的每步标号的集合。从左到右依次记下向 $(1, -1)$ 方向走的 $n$
步的标号得与 $D$ 对应的置换记为
$L_{312}(D)$,由我们的标号原则不难看出 $L_{312}(D)$ 是
312-禁排置换。反之,对任意一个 312-禁排置换
$\pi$,据之前的原则很容易构造出与之对应的 $n$-Dyck
path。从而,我们就得到了 $n$-Dyck paths 和 $\mathfrak{S}_n$ 中
312-禁排置换之间的一个双射。\qed

\begin{ex}
对如下图所示的 Dyck path 用上面所叙述的原则来标号得到对应的
$\pi=324651798$。
\end{ex}


\begin{picture}(320,50)
\put(5,0){\line(1,1){50}}
\put(3,-2){$\bullet$}%
\put(8,10){1}%
\put(20,14){$\bullet$}%
\put(22,26){2}%
\put(37,31){$\bullet$}%
\put(40,42){3}%
\put(68,42){3}%
\put(83,27){2}%
\put(94,26){4}%
\put(120,25){4}%
\put(130,25){5}%
\put(145,42){6}%
\put(171,40){6}%
\put(186,25){5}%
\put(200,10){1}%
\put(213,10){7}%
\put(235,9){7}%
\put(248,10){8}%
\put(263,25){9}%
\put(290,24){9}%
\put(307,10){8}%
\put(54,48){$\bullet$}%
\put(56,51){\line(1,-1){35}}%
\put(71,31){$\bullet$}%
\put(88,14){$\bullet$}%
\put(90,17){\line(1,1){17}}%
\put(107,32){$\bullet$}%
\put(110,35){\line(1,-1){18}}%
\put(125,14){$\bullet$}%
\put(125,15){\line(1,1){35}}%
\put(142,31){$\bullet$}%
\put(157,46){$\bullet$}%
\put(160,49){\line(1,-1){50}}%
\put(172,31){$\bullet$}%
\put(190,14){$\bullet$}%
\put(206,-2){$\bullet$}%
\put(208,0){\line(1,1){17}}%
\put(224,15){$\bullet$}%
\put(226,17){\line(1,-1){17}}%
\put(242,-2){$\bullet$}%
\put(244,0){\line(1,1){35}}%
\put(260,15){$\bullet$}%
\put(277,31){$\bullet$}%
\put(280,33){\line(1,-1){35}}%
\put(294,15){$\bullet$}%
\put(313,-5){$\bullet$}%
\put(123,-25){图 1}
\end{picture}
\\
\\



若将置换 $\pi=\pi_1\pi_2\cdots\pi_n\in\mathfrak{S}_n$ 看作一个字
(word),我们可以定义它的子字 (subword)
$\sigma=\pi_{i_1}\cdots\pi_{i_k}$($1\le i_1<\cdots< i_k\le n$)。记
$\pi-\sigma$ 为子字 (subword)
$\pi_{j_1}\cdots\pi_{j_{n-k}}$,这里,$\{i_1,\ldots,i_k\}\cup\{j_1,\cdots,j_{n-k}\}=[n]$
且 $\{i_1,\ldots,i_k\}\cap\{j_1,\cdots,j_{n-k}\}=\phi$。

由 321-禁排置换的定义我们容易得到下面的引理:

\begin{lem}
置换 $\pi\in\mathfrak{S}_n$ 是 321-禁排的当且仅当 $lrm(\pi) =
\{\pi_i | i=1$ 或 $\pi_i>\pi_j (j<i)\}$ 和 $[n]\backslash lrm(\pi)$
都是递增的。
\end{lem}

\begin{thm}
$\mathfrak{S}_n$ 中 312-禁排置换的数目与 123-禁排置换的数目相等。
\end{thm}
{\bf{证明}} 我们这里仍然给出一个类似于定理 2.3
的组合证明,但这次介绍的是 321-禁排标号。对任意一条 $n$-Dyck path
$D$,我们按如下方式来标号:将向 (1, 1) 方向走的 $n$ 步用
$1,2,\ldots,n$ 从左到右依次标号;对向 $(1, -1)$ 方向走的 $n$
步我们这样来处理:对任意一个峰不妨设为 $ud$ (这里 $u$ 是向(1, 1)
方向走的一步,$d$ 是紧接的向 $(1, -1)$ 方向走的一步),对 $d$ 标与
$u$ 相同的标号;对剩下的尚未标号的向 $(1, -1)$ 方向走的每步
$d$,我们用 $min\{l^u\backslash l^d\}$ 从左到右依次给它标号,这里
$l^u(l^d)$ 是 $d$ 左边向(1, 1)($(1, -1)$)
方向走的每步标号的集合。从左到右依次记下向 $(1, -1)$ 方向走的 $n$
步的标号得与 $D$ 对应的置换记为 $L_{321}(D)$。


\begin{ex}
对如下图所示的 Dyck path 用上面所叙述的原则来标号得到对应的
$\pi=314625798$。
\end{ex}

\begin{picture}(320,50)
\put(5,0){\line(1,1){50}}%
\put(3,-2){$\bullet$}%
\put(8,10){1}%
\put(20,14){$\bullet$}%
\put(22,26){2}%
\put(37,31){$\bullet$}%
\put(40,42){3}%
\put(68,42){3}%
\put(83,27){1}%
\put(94,26){4}%
\put(120,25){4}%
\put(130,25){5}%
\put(145,42){6}%
\put(171,40){6}%
\put(186,25){2}%
\put(200,10){5}%
\put(213,10){7}%
\put(235,9){7}%
\put(248,10){8}%
\put(263,25){9}%
\put(290,24){9}%
\put(307,10){8}%
\put(54,48){$\bullet$}%
\put(56,51){\line(1,-1){35}}%
\put(71,31){$\bullet$}%
\put(88,14){$\bullet$}%
\put(90,17){\line(1,1){17}}%
\put(107,32){$\bullet$}%
\put(110,35){\line(1,-1){18}}%
\put(125,14){$\bullet$}%
\put(125,15){\line(1,1){35}}%
\put(142,31){$\bullet$}%
\put(157,46){$\bullet$}%
\put(160,49){\line(1,-1){50}}%
\put(172,31){$\bullet$}%
\put(190,14){$\bullet$}%
\put(206,-2){$\bullet$}%
\put(208,0){\line(1,1){17}}%
\put(224,15){$\bullet$}%
\put(226,17){\line(1,-1){17}}%
\put(242,-2){$\bullet$}%
\put(244,0){\line(1,1){35}}%
\put(260,15){$\bullet$}%
\put(277,31){$\bullet$}%
\put(280,33){\line(1,-1){35}}%
\put(294,15){$\bullet$}%
\put(313,-5){$\bullet$}%
\put(123,-25){图 2}
\end{picture}
\\
\\
\\




下面我们来证明刚刚的 321-禁排标号事实上给出了$n$-Dyck paths 和
$\mathfrak{S}_n$ 中 321-禁排置换间的一个双射。设 $A$
是由从左至右的峰中向 $(1, -1)$ 方向走的那些步的标号所构成的字,$B$
是由剩余的从左至右的向 $(1, -1)$
方向走的那些步的标号所构成的字。需要注意的是 $A$ 和 $B$
都是递增的。考虑到向 $(1, -1)$
方向走的任意三步的标号中必定有两步的标号或者包含于 $A$ 或者包含于
$B$ 中,从而易知 $L_{321}(D)$ 是 321-禁排置换。反之,对任意一个
321-禁排置换 $\pi$,作如下分解:$\pi=a_1b_1a_2b_2\cdots
a_lb_l$,这里 $1\le l\le n$,$a_i$($1\le i\le l$) 是 $\pi$
中左到右最大的元素(left-to-right maximum),$b_j$($1\le j\le l-1$) 是
$a_j$ 和 $a_{j+1}$ 间的元素构成的子字,$b_l$ 是 $a_l$
右边的元素所构成的子字。让 $a_i$ ($1\le i\le l$) 对应向(1, 1)
方向走的 $a_i-a_{i-1}$ 步(习惯上令 $a_0=0$),让 $b_i$ ($1\le i\le
l$) 对应向 $(1, -1)$ 方向走的 $|b_i|+1$ 步,结合引理 2.5
知最后所得到的路是一条 $n$-Dyck path。另外,很容易得到
$\mathfrak{S}_n$ 中 321-禁排置换与 123-禁排置换间的一一对应。\qed

%-------------------------------------------------------------
\subsection{Catalan 数}

\begin{thm}第 $n$ 项 Catalan 数
$$C_n={2n\choose n}-{2n\choose n-1}.$$
\end{thm}
{\bf{证明}} 定理中这个等式可以借助 Dyck path
通过组合方法来证明。这里,我们给出的是一种应用了反射原理的证明方法。

令 $A$ 是平面上从 (0, 0) 到 $(2n, 0)$ 且每步只能取 (1, 1) 或 $(1,
-1)$ 的路的集合。易知 $|A|={2n\choose n}$。令 $B$ 是 $n$-Dyck path
的集合,则 $|B|=C_n$ (第 $n$ 项 Catalan 数)。令 $C$ 是包含于 $A$
且穿越了 $x$ 轴的那些路的集合,则 $|C|=|A|-|B|$。令 $D$ 是平面上从
(0, 0) 到 $(2n, -2)$ 且每步只能取 (1, 1) 或 $(1, -1)$
的路的集合。易知 $|D|={2n\choose n-1}$。因此我们只需要说明 $|C|=|D|$
即可。对 $C$ 中任意一条路 $p$,假定 $(2i-1, -1)$ 是 $p$ 与 $y=-1$
所交的第一个点,容易知道,将 $p$ 中从 $(2i-1, -1)$ 到 $(2n, 0)$
的那段关于 $y=-1$ 作反射,$(0, 0)$ 到 $(2i-1,
-1)$那段不变所得到的是从 $(0, 0)$ 到 $(2n, -2)$ 属于 $D$ 的路
$p'$,很容易证明这样的反射事实上给出了集合 $C$ 和 $D$
间的一个一一对应。\qed

\begin{thm}
设 $G(x)$ 是 Catalan 数的生成函数即
$G(x)=\sum\limits_{n=0}^{\infty}C_nx^n$,则
$G(x)=\frac{1-\sqrt{1-4x}}{2x}$。
\end{thm}
{\bf{证明}} 令 $u$($d$) 表示向 (1, 1)($(1,-1)$)
方向走的一步,这样我们就可以把每条 Dyck path 对应于 $\{u, d\}$
上的一个字。比如,图 1 中的 Dyck path 对应于字
$uuudduduuddduduudd$。可以观察到每条非空的 Dyck path
可以被唯一的分解成 $\delta=u\alpha d\beta$,这里 $\alpha$ 和 $\beta$
都是 Dyck path。假定 $\delta$ 和 $\alpha$ 分别长为 $2n$ 和
$2i$($0\le i\le n-1$),从而易知 $\beta$ 长为 $2n-2i-2$,因此有
$C_n=\sum\limits_{i=0}^{n-1}C_iC_{n-1-i}$,所以
\begin{align*}
G(x)-1%
&=\sum_{n\ge1}C_nx^n\\
&=\sum_{n\ge1}\left(\sum_{i=0}^{n-1}C_iC_{n-1-i}\right)x^n\\%
&=x\sum_{n\ge0}\left(\sum_{i=0}^{n}C_iC_{n-i}\right)x^{n}\\
&=xG^2(x).%
\end{align*}

解上面的函数方程得

\[
G(x)=\frac{1-\sqrt{1-4x}}{2x}.
\]

注意我们在应用求根公式时取了负号,这是因为
$\frac{1+\sqrt{1-4x}}{2x}$ 的展开式中包含了项
$\frac{2}{2x}=x^{-1}$。\qed






\chapter{排列}

排列的研究在组合数学中占有重要的地位。本章中我们就主要介绍一下排列这种组合结构,
包括它的计数,算法,其上的统计量以及与杨表的关系。

\section{集合的排列}% A 上的置换的组合性质}

给定某个含有不同的元素集合$S$,我们把它的元素排成一个线性序,使得每个元素恰好出现一次,
叫做该集合的一个排列 (permutation). 以$[n]$表示$n$个正整数构成的集合$\{1,2,\cdots,n\}$, 那么$[n]$上的一个排列可以看成是$[n]$到自身的
一个双射。则$[n]$上所有排列对于映射的复合构成一个群,称
为$[n]$上的置换群,记作$S_n$。此时对于一个排列,我们可以用一行$\pi=\pi_1\pi_2
\cdots \pi_n$来表示,其中$\pi_i$表示$i$的像。
我们首先来看一下$[n]$上的排列的个数。

\begin{thm}\label{perT1}
集合$[n]$的全部排列的个数为$n!=n\times (n-1) \times (n-2)\times \cdots \times 2\times 1$, 即$|S_n|=n!$.
\end{thm}

\pf 在构建$n$元素集得一个排列$\pi=\pi_1\pi_2 \cdots
\pi_n$时,我们有$n$种选择方法来放置第一个元素$\pi_1$,
即可以从$\{1,2,\ldots,n\}$中任意选。对于$\pi_2$,由于我们
要求排列中元素互不相同,所以$\pi_2$不能取已放在第一个位置上的元素,
这时$\pi_2$有 $n-1$种选择。以此类推,我们得到$\pi_3$有$n-2$种选择,
到了最后一个元素$\pi_n$, 由于前面已经被选掉了$n-1$个元素,
所以$\pi_n$只有唯一的一种选择方法。所以根据乘法原理,
我们可得$[n]$上的排列总共有$n!$个。 \qed


注意,为方便起见,我们令$0!=1$.

\begin{exa}
若$n=3$, 则集合$[3]$上的排列个数为$3!=6$. 它们分别为
\[ 1\,2\,3,~~ 1\,3\,2,~~2\,1\,3,~~2\,3\,1,~~ 3\,1\,2,~~ 3\,2\,1.\]
\end{exa}

令$r$为正整数,从含有$n$个元素的集合$S$中取出$r$个元素排成一个线性序,叫做一个$r$-排列。
我们用$P(n,r)$表示$n$个元素集合的$r$-排列的数目。如果$r>n$, 则$P(n,r)=0$.

\begin{thm}
对于正整数$n$和$r$, $r\leq n$, 有
\[ P(n,r)=n\times (n-1) \times (n-2)\times \cdots \times (n-r+1).\]
\end{thm}

\pf 同定理\ref{perT1}的证明方法,此时第一个位置有$n$种选择方法, 即可以从$S$中任意选一个。对于第二个位置,我们有
$n-1$种选择。以此类推,对于第$r$个位置,我们还剩有$n-r+1$种选择。
所以总共有$n\times (n-1) \times (n-2)\times \cdots \times (n-r+1)$个$r$-排列。 \qed

\begin{exa}
若$S=\{a,b,c\}$, 则集合$S$上的$1$-排列为
\[ a,~~b,~~c\]
$S$上的$2$-排列为
\[ ab,~~ac,~~ba,~~bc,~~ca,~~cb.\]
\end{exa}


\begin{exa}
在中国采用公历之前,人们长期用干支纪法来记年、月、日,如甲子年、丙寅年、戊辰年等。实际
上这里也包含了排列的思想。
在中国古代的历法中,甲、乙、丙、丁、戊、己、庚、辛、壬、癸被称为“十天干”。
子、丑、寅、卯、辰、巳、午、未、申、酉、戌、亥叫作“十二地支”。其中甲、丙、戊、庚、壬
为阳干,乙、丁、己、辛、癸为阴干。
子、寅、辰、午、申、戌为阳支,丑、卯、巳、未、酉、亥为阴支。
以一个天干和一个地支相配,排列起来,天干在前,地支在后,天干由甲起,
地支由子起,
阳干对阳支,阴干对阴支(阳干不配阴支,阴干不配阳支)
得到六十年一周期的甲子回圈。
称为“六十甲子”或“花甲子”。
\end{exa}

\section{排列的生成算法}
实际工作中有时要在计算机上对各种排列状态下出现的情况加以分析,下面介绍若干排列的
生成算法。

\subsection{递归法}
要生成一个$[n]$上的排列,我们可以从$[n-1]$上的排列,插入$n$而得。具体的说就是
给定$\pi=\pi_1\pi_2
\cdots \pi_{n-1}$, 我们可以把$n$插入$\pi_i$ $\leq n-1$ 前面而得到一个$[n]$上的排列$\pi_1\pi_2
\cdots n \pi_i \cdots \pi_{n-1}$, 或者插入到最后一个位置后面得到$\pi_1\pi_2
\cdots \pi_{n-1}n$. 由此递归关系,我们只要已知了小于$n$个数上的集合的排列,就能得出
具有$n$个元素的集合的排列。下面给出了在maple里实现的代码,感兴趣的同学可以自己尝试一下。
\\
> p:=proc(n)\\
> local i,j,k,T:\\
> option remember:\\
> if n<1 then return "Error input!":\\
> elif n=1 then p(n):=[[1]]:\\
> elif n>1 then\\
> T:=[]:\\
> for i from 1 to nops(p(n-1))\\
> do T:=[op(T),[seq(p(n-1)[i][k],k=1..n-1),n]]:\\
> for j from 2 to n-1\\
> do T:=[op(T),[seq(p(n-1)[i][k],k=1..j-1),n,seq(p(n-1)[i][k],k=j..n-1)]]:od:\\
> T:=[op(T),[n,seq(p(n-1)[i][k],k=1..n-1)]]: od:\\
> p(n):=T:\\
> fi:\\
> end:\\

这里我们给出的是集合$[n]$上的排列。例如,当$n=4$时,运行$p(4)$之后,我们就得到了
24个排列。

> p(4);

  [[1, 2, 3, 4], [1, 4, 2, 3], [1, 2, 4, 3], [4, 1, 2, 3],

        [1, 3, 2, 4], [1, 4, 3, 2], [1, 3, 4, 2], [4, 1, 3, 2],

        [3, 1, 2, 4], [3, 4, 1, 2], [3, 1, 4, 2], [4, 3, 1, 2],

        [2, 1, 3, 4], [2, 4, 1, 3], [2, 1, 4, 3], [4, 2, 1, 3],

        [2, 3, 1, 4], [2, 4, 3, 1], [2, 3, 4, 1], [4, 2, 3, 1],

        [3, 2, 1, 4], [3, 4, 2, 1], [3, 2, 4, 1], [4, 3, 2, 1]]

\subsection{直接法}
下面我们给出直接生成所有排列的算法,由于不需要储存之前小于所求数字上的所有
排列,所以该算法的效率比较高。详细思想可见Stanton和White的书\cite{SW}. 以下是
直接生成排列的Maple程序。
\\
>PermutationList:=proc(n)\\
> local i,j,m,d,T,TINV,A,Done,res:
> if n=1 then return [1]; end if:  \\
> T:=[seq(i-1,i=1..n+2)]:TINV:=[seq(i+1,i=1..n)]:d:=[seq(-1,i=1..n)]:\\
> T[1]:=n+1:T[n+2]:=n+1:\\
> A:={seq(i,i=2..n)}:\\
> res:=[]:\\
>\\
> Done:=true:\\
> while Done do\\
>     res:=[op(res),T[2..n+1]]:\\
>     if nops(A)<>0 then\\
>         m:=op(nops(A),A):\\
>         j:=TINV[m]:\\
>         T[j]:=T[j+d[m]]:\\
>         T[j+d[m]]:=m:\\
>         TINV[m]:=TINV[m]+d[m]:\\
>         TINV[T[j]]:=j:\\
>         if m<T[j+2*d[m]] then\\
>            d[m]:=-1*d[m]:\\
>            A:=A minus {m}:\\
>         end if:\\
>         A:=A union {seq(i,i=m+1..n)}:\\
>     else\\
>         Done:=false:\\
>     end if:\\
> end do:\\
> return res;\\
> end:\\


例如,当$n=4$时,运行PermutationList(4)之后,我们就得到了
如下24个排列。

>PermutationList(4);

  [[1, 2, 3, 4], [1, 2, 4, 3], [1, 4, 2, 3], [4, 1, 2, 3],

        [4, 1, 3, 2], [1, 4, 3, 2], [1, 3, 4, 2], [1, 3, 2, 4],

        [3, 1, 2, 4], [3, 1, 4, 2], [3, 4, 1, 2], [4, 3, 1, 2],

        [4, 3, 2, 1], [3, 4, 2, 1], [3, 2, 4, 1], [3, 2, 1, 4],

        [2, 3, 1, 4], [2, 3, 4, 1], [2, 4, 3, 1], [4, 2, 3, 1],

        [4, 2, 1, 3], [2, 4, 1, 3], [2, 1, 4, 3], [2, 1, 3, 4]]

\section{基本统计量}% 上的置换上的一些统计量的定义和性质}

排列的各种统计量是组合数学研究的一个重要课题,对排列统计量的研究可以使我们
更清楚的了解排列的内部结构。下面我们就介绍一些在排列上十分熟知的统计量。

位置$i$($1\leqslant i<n$)称为是$\pi$的一个{下降位}(descent)
如果$\pi_i>\pi_{i+1}$;反之则称为$\pi$的{上升位}(acscent).
定义所有下降位构成的集合

$$\Des(\pi)=\{i|\pi_i>\pi_{i+1}\}$$
为$\pi$的下降集(descent set),
定义该集合的个数为$\des(\pi)=|\Des(\pi)|$为
$\pi$的下降数。由定义$n\notin \Des(\pi)$.
同时我们定义一个排列的主指标(major index)为
\[\maj(\pi)=\sum \limits_{i \in \Des(\pi)}i.\]

如果位置$i$满足$\pi_i>i$, 则称$i$是一个{胜位}(excedance), 若$i$满足
$\pi_i\geq i$, 则称$i$是{弱胜位}(weak excedance). 我们记 $\pi$
的所有胜位的个 数为 $\exc(\pi)$.

一对元素$(i,j)$
称为是一个{逆序}(inversion),如果满足$i<j$且$\pi_i>\pi_j$, 称$\pi$的
所有逆序的个数为$\pi$的逆序数,记作$\inv(\pi)$.

对于排列$\pi=\pi_1\pi_2\cdots \pi_n$, 定义其逆为其作为映射的逆,即
$\pi^{-1}=\pi^{-1}(1)\pi^{-1}(2)\cdots \pi^{-1}(n)$;
定义其反为$\pi^r=\pi_n\pi_{n-1}\ldots\pi_1$; 定义
其补为$\pi^c=(n+1-\pi_1)(n+1-\pi_2)\cdots(n+1-\pi_n)$,
显然它们三个都是$S_n$上自然的一一映射。

\begin{exa}
对于$[5]$上的排列$\pi=43521$, 以上的统计量分别为:
$\Des(\pi)=\{1,3,4\}$, $\des(\pi)=3$, $\maj(\pi)=1+3+4=8$,
$\exc(\pi)=3$, $\inv(\pi)=7$.
\end{exa}

\subsection{下降数与胜位的等分布性质}
我们称两个统计量$u,v$在某个集合$S$上是{等分布的}(equidistribute),
若对于任意的自然数$k$, 有
$\#\{x\in S|u(x)=k\}=\#\{x\in S|v(x)=k\}$.

\begin{thm} \label{exc_des}
$\exc$ 与 $\des$在$S_n$上是等分布的。
\end{thm}

一般而言,证明两个统计量的等分布性有两个主要的思路:
一个是组合证明,即寻找所在集合的一个到
自身的双射;另一个是代数证明,即证明二者有相同的生成函数。

\pf 组合证明:

在证明之前,先引入排列的另一种表示形式——圈表示。对于任意$x\in
[n]$, 考虑 序列$x,\pi(x),\pi^2(x),\ldots$,
最终一定形成一个圈(因为$\pi$是双射且$[n]$是有限集)。对所有的元素寻
找这样的圈,我们可以把排列$\pi$写成若干个不交圈的并的形式。
这种形式显然不是唯一的,首先,圈之间的顺
序可以任意,其次,圈内部的圈排列也有不同的表示。
为保证其唯一性,我们定义如下标准圈表示形式:

\begin{itemize}
   \item [a.]每个圈的最大元素放在首位;
   \item [b.]圈按照其最大元从小到大排列。
\end{itemize}


可以证明,以上的标准圈表示形式存在且唯一的。

对于任意一个排列$\pi\in S_n$,我们考虑其标准圈表示,
并将标准圈表示的圈去掉,这样就得到$[n]$上的一个新
的排列$\pi'$,可以证明$\pi\rightarrow \pi'$必然是$S_n$上的双射。
事实上,对于任意$\pi\in S_n$, 取其自左
向右极大元(即满足对于任意$j<i$, $\pi_j>\pi_i$的元素$\pi_i$)。
在相应位置加括号就可以得到上述映射的逆映射。

我们利用以上映射证明我们的结论,只需要证明对于任意的$\pi\in S_n$,
$\exc(\pi)=\des(\pi')$.
事实上,考虑$\pi$的补排列$\pi^c$的标准圈表示形式,
$\pi$的每一个胜位恰好对应到$(\pi^c)'$的一个下降位。命题得证。\qed

一般地,称与$\des$在$S_n$上等分布的统计量为Eulerian的.


\subsection{逆序数与主指标}

首先我们用代数的方法来给出逆序数的生成函数。

\begin{thm}
\begin{equation}
\sum_{\pi\in
S_n}q^{\inv(\pi)}=(1+q)(1+q+q^2)\cdots(1+q+q^2+\cdots+q^{n-1}).
\end{equation}
\end{thm}

\pf 对任意的 $\pi \in S_n$, 定义其对应的逆序表(inversion table)为
$I(\pi)=(a_1,a_2,\cdots,a_n)$,
其中$a_i$为在$i$左边且比$i$大的元素的个数。
例如 $\pi=417396285$, 则$I(\pi)=(1,5,2,0,4,2,0,1,0)$.
由定义容易看出
\[I(n)=\{(a_1,a_2,\cdots,a_n): 0\leq a_i \leq n-i\}=
[0,n-1]\times [0,n-2] \times [0,1]\times [0,n].\]
且$I(n)$与$S_n$是一一对应的。

因此从上面的分析可知
\begin{eqnarray*}
\sum_{\pi\in S_n}q^{\inv(\pi)}
&=&\sum_{a_1=0}^{n-1}\sum_{a_2=0}^{n-2}\cdots
\sum_{a_n=0}^{0}q^{a_1+a_2+\cdots+a_n}\\
&=& \sum_{a_1=0}^{n-1}q^{a_1}\sum_{a_2=0}^{n-2}q^{a_2} \cdots
\sum_{a_n=0}^{0}q^{a_n}\\
&=&(1+q)(1+q+q^2)\cdots(1+q+q^2+\cdots+q^{n-1}).
\end{eqnarray*}
\qed

下面是置换与其逆之间的逆序数的一个关系。
\begin{prop}
对任意的 $\pi \in S_n$, 我们有 $\inv(\pi)=\inv(\pi^{-1})$.
\end{prop}

逆序数的生成函数从它的定义中就很容易得到,然而另一个定义方式截然不同的统计量——
主指标却和它有着非常紧密的联系,下面的定理告诉我们,二者是同分布的。

\begin{thm}
\begin{equation}
\sum_{\pi\in S_n}q^{\inv(\pi)}=\sum_{\pi\in S_n}q^{\maj(\pi)}.
\end{equation}
\end{thm}

\pf
我们寻找$S_n$到自身的一个双射来证明它。下面我们就给出由
Foata给出的这个经典的双射,一般称为Foata双射。

双射$\varphi$是递归的定义的。对$w=w_1w_2\cdots w_n \in S_n$,
我们首先令$r_1=w_1$. 现在假设$r_k$($k\leq 1$)已经定义了,
则$r_{k+1}$的定义是这样的:

如果$r_k$的最后一个字母大于(或小于)$w_{k+1}$,
则我们就在$r_k$中每个大于(或小于)$w_{k+1}$ 的字母后面画一条竖线,
这样就把$r_k$中的元素分成了一些块,
然后我们对每个块中的字母向右循环移动
一位,此时每个块中的最后一个元素就变成该块中第一个元素了,
最后我们再把$w_{k+1}$接到变换后的序
列后面,就得到了$r_{k+1}$. 令$\varphi(w)=r_n$.

由$\varphi$的构造可知在每一步变换后都能保证
$\maj(w_1w_2\cdots w_k)=\inv(r_k)$. 

要说明$\varphi$是双射,我们只需给出其逆映射。
从$\varphi$的定义我们可以类似的定义$\varphi^{-1}$如下:

假设$\sigma=\varphi(w)$, 则$\varphi^{-1}$ 的定义为:
若$\sg_n>\sg_1$, 则在小于$\sg_n$的数字之前加一 条竖线,并且在$\sg_n$
的前面也加;若$\sg_n<\sg_1$,
则在大于$\sg_n$的数字之前加一条竖线,并且在 $\sg_n$
的前面也加。然后我们把每个块中的元素向左循环移动一位,
去掉竖线就得到了一个新的置换,此时
我们就把最后一个元素固定下来作为$\varphi^{-1}$的最后一个元素。
接下来用同样的方法确定最后第二个元
素,$n$步以后就得到了$\varphi^{-1}(\sg)$,
且有$\inv(\sg)=\maj(\varphi^{-1}(\sg)$. \qed


我们给出一个例子以便读者更好的理解。

\begin{exa}
若 $w=417396285$, 我们有:
\begin{eqnarray*}
r_1  &=& w_1=4;\\
r_2  &=& 4|1;\\
r_3  &=& 4|1|7;\\
r_4  &=& 4|71|3;\\
r_5  &=& 4|7|1|3|9;\\
r_6  &=& 74|913|6;\\
r_7  &=& 7|4|9|31|6|2;\\
r_8  &=& 7|4|39|1|6|2|8;\\
r_9  &=& 7|934|61|82|5.
\end{eqnarray*}
且 $\maj(w)=1+3+5+6+8=23, \inv(\varphi(w))=23$.
\end{exa}

一般地,与$\mathrm{maj}$在$S_n$上等分布的统计量称为Mohonian的。




\section{欧拉数}

欧拉(Euler),瑞士数学家及自然科学家。1707年4月15日出生于瑞士的巴塞尔,1783\\年
9月18日于俄国彼得堡去逝。欧拉出生于牧师家庭,自幼受父亲的教育。
13岁时入读巴塞尔大学,15岁大学毕业,16岁获硕士学位。

欧拉是18世纪数学界最杰出的人物之一,
他不但为数学界作出贡献,更把数学推至几乎整个物理的领域。
他是数学史上最多产的数学家,平均每年写出八百多页的论文,
还写了大量的力学、分析学、几何学、变分法等
的课本,《无穷小分析引论》、《微分学原理》、《积分学原理》
等都成为数学中的经典著作。

欧拉对数学的研究如此广泛,因此在许多数学的分支中也可经常见
到以他的名字
命名的重要常数、公式和定理。 诸如:欧拉函数,欧拉数,
欧拉定理,欧拉常数等等。




\subsection{欧拉数的定义和性质}
设 $A(n,k)$ 为 $n$ 的所有置换中具有$k-1$个下降位的置换个数,我们称
$A(n,k)$ 为欧拉数 (Eulerian number).
本节我们就主要研究欧拉数的一些组合性质。在此之前,我们先给出 $n\leq
6$ 时欧拉数。

\begin{tabular}{c|c|c|c|c|c|c}
 $n \setminus k$    &1 &2 &3  &4  &5  &6\\
\hline $1$       &1  \\
\hline $2$       &1 &1 \\
\hline $3$       &1 &4 &1  \\
\hline $4$       &1 &11 &11  &1 \\
\hline $5$       &1 &26 &66  &26  &1 \\
\hline $6$       &1 &57 &302 &302 &57 &1
\end{tabular}

由欧拉数的组合意义,我们有下面的递推关系。

\begin{prop}\label{p1}
\begin{equation}
A(n,\,k)=kA(n-1,\, k)+(n-k+1)A(n-1,\,k-1)
\end{equation}
\end{prop}

\pf 给定一个 $n-1$ 长的且下降数为 $k-1$ 的排列,则我们把 $n$ 插入这
$k-1$ 个下降位的位置后面不会改变总的下降数的个数。显然如果把 $n$
插在最后一个位置也不会改变下降数的个数。如果在非下降位的后面插入
$n$, 则会使下降位增加一个。所以我们有 $A(n,\,k)=kA(n-1,\,
k)+(n-k+1)A(n-1,\,k-1).$ \qed

由上面的递推关系,我们很容易得到 $A(n,k)$ 的对称性。
\begin{prop}
\begin{equation}
A(n,k+1)=A(n,n-k).
\end{equation}
\end{prop}

当然从组合的观点,我们也可以这样而得。设 $n$ 的置换 $p=p_{1}p_{2}
\cdots p_n$有 $k$ 个下降数,则它的转置 $p^r=p_{n}p_{n-1}\cdots p_1$
有 $n-k-1$ 个降序数。由 $p$ 和 $p^r$ 的一一对应可得。

由\ref{exc_des}知,$\exc$与$\des$是等分布的,所以我们有。
\begin{prop}
在 $[n]$ 的所有置换中具有 $k-1$ 个胜位的置换个数为$A(n,k)$.
\end{prop}


\subsection{与欧拉数有关的等式}
由欧拉数的组合意义,我们还可以得到一些特殊的具有组合意义的式子。
\begin{thm}\label{t1}(\cite{Graham1994})
令 $A(0,0)=1$, 且当 $n>0$ 时,令 $A(n,0)=0$. 则对于所有非负整数 $n$
和实数 $x$ 满足如下等式
\begin{equation}\label{sm}
x^n=\sum_{k=0}^{n}A(n,k){x+n-k \choose n}
\end{equation}
\end{thm}

直接比较 \eqref{Ant}两边$k^n$的系数,很显然定理成立,
这里我们给出它的一个组合证明。定理两边都是关于$t$的$n$次多项式,
我们只需证明它对于$n+1$个不相等的实数成立即可。
这里,我们证明其对于任意正整数成立。

\pf 我们先假设$x$是一个正整数.则等式左边代表的是长度为 $n$
的,且每个分量取自集合 $[x]$
的序列个数.则我们只需说明等式右边也是计算的这种序列的个数。令
$a=a_{1}a_{2}\cdots a_n$
为任意一个这样的序列,重新排列$a$中元素的顺序使其非递降得
$a'=a_{i_{1}}\leq
a_{i_{2}}\leq \cdots \leq a_{i_{n}}$.
如果是相同的数字则在$a'$中的顺序是其在按照它们在$a$中的下
标递增的顺序排列.则$i=i_{1}i_{2}\cdots
i_{n}$ 为$n$ 的由 $a$ 唯一决定的置换,$i_{k}$
代表了$a$中第$i_{k}$大的数字所在的位置.例如 $a=3~1~1~2~4~3$,
重排后得 $a'=1~1~2~3~3~4$,对应的置换为 $i=2~3~4~1~6~5$.
\\
如果我们能说明每一个具有 $k-1$ 个下降数的置换 $i$ 是恰好从 $x+n-k$个
序列 $a$ 而得到的,则我们就完成了证明。
\\
很显然如果 $a_{i_{j}}=a_{i_{j+1}}$, 那么 $i_{j}<i_{j+1}$。
对应的,如果 $j$ 是置换 $p(a)=i_{1}i_{2}\cdots i_{n}$
的一个下降数,则 $a_{i_{j}}<a_{i_{j+1}}$.这就意味着只要 $j$
是一个下降数则序列 $a'$
在此位置是严格递增的。我们可以在上面的例子中验证一下。$i$ 在位置
$3,5$ 是 下降的,确实 $a'$
在这些位置上是严格递增的。那么有多少个序列 $a$ 能得到置换
$i=2~3~4~1~6~5$ 呢?由前面的分析可得,$a$ 中元素必须满足$$1\leq
a_{2}\leq a_{3} \leq a_{4} < a_{1} \leq a_6 <a_5 \leq x
$$严格的不等号是在第三个和第五个位置.上面的不等式链等价为
$$1\leq a_{2}< a_{3}+1
<a_{4}+2 < a_{1}+2 <a_6 +3 <a_5+3 \leq x$$ 因此这种序列的个数为
 ${x+3 \choose 6}$.
 同样的方法,对于任意的 $n$ 和具有 $k-1$ 个下降数的置换 $i$,
 我们得到 $n$ 的具有 $k-1$ 个下降数的置换可从
${x+(n-1)-(k-1)\choose n} $=${x+n-k \choose n}$ 个序列中得到.
\\如果 $x$ 不是一个正整数,由于等式两边都可以看作是关于变量 $x$
的多项式,而它们在无穷多个数值上取值相同,所以它们必须是本身是相等的。\qed


利用上述定理,我们可以讨论正整数前$n$项和的方幂求和的问题,
我们有如下结论:

\begin{prop}
\begin{align}
\sum_{x=1}^mx^n=\sum_{k=1}^{n}A(n,k){k+m\choose n+1}.
\end{align}
\end{prop}

\pf 首先,利用欧拉数的对称性,将\eqref{sm}进行化简。

当$n>0$时 \[t^n=\sum_{k=1}^{n}A(n,k){t+n-k \choose
n}=\sum_{k=1}^{n}A(n,n+1-k){t+n-k \choose
n}=\sum_{k=0}^{n-1}A(n,k+1){t+k \choose n}
\]
上式两边对$t$求和,有
\begin{align*}
\sum_{t=1}^mt^n &=\sum_{k=0}^{n-1}A(n,k+1)\sum_{t=1}^m{t+k \choose n}\\
                &=\sum_{k=0}^{n-1}A(n,k+1)\sum_{t=1}^m \left({t+k+1\choose n+1}-{t+k\choose n+1}\right)\\
                &=\sum_{k=0}^{n-1}A(n,k+1){m+k+1\choose n+1}\\
                &=\sum_{k=1}^{n}A(n,k){m+k\choose n+1}
\end{align*}
\qed



\begin{coro}
\begin{equation}
[x]^n=\sum_{k=0}^{n}A(n,k){x+k-1\choose n}.
\end{equation}
\end{coro}
\pf
 在定理\ref{t1}中用 $-x$ 代替 $x$,我们得
 $$x^n(-1)^n=\sum_{k=0}^{n}A(n,k){-x+n-k\choose n}.$$
 注意到 ${-x+n-k \choose n}$=$\frac{(-x+n-k)(-x+n-k-1)\cdots
(-x+1-k)}{n!}=(-1)^n{x+k-1 \choose
 n}$. 对照这两个等式就得到了结论。\qed

\begin{thm}
对于所有满足 $k\leq n$ 的非负整数 $n,k$, 有
\begin{equation}
A(n,k)=\sum_{i=0}^{k}(-1)^i{n+1 \choose i}(k-i)^n
\end{equation}
\end{thm}

\pf {组合证明}\\
我们先写下 $k-1$ 个竖线,这样就产生了 $k$ 个分间。把 $[n]$
中的每个元素放入任意一个分间内,有 $k^n$
种方法。然后对每个分间中的数字按递增的顺序排列。例如 $k=4,n=9$



那么其中的一个就可以为 $$237||19|4568.$$
忽虑掉那些竖线我们就得到了一个至多有 $k-1$
个下降数的置换(在上例中就是 $2~3~7~1~9~4~5~6~8$).
\\我们需注意以下几种情况:
可能会有空的分间(即分间里面没有放数字);或者相邻的分间之间没有下降数。由此我们就称一个竖线是"多余的",如果
\\(a)
去掉它仍能得到一个符合规定的排列(即在每个分间中的数字是递增的顺序)。例如$4|12|3$中的第二个竖线。
\\(b)此竖线紧接着前面一个竖线(即有空的分间)。例如$2|35||614$中的第三个竖线。
我们的目标是计算没有"多余的竖线"的排列个数,因为这样的排列是与具有$k-1$个下降数的置换一一对应的。
我们利用容斥原理来计算,令$B_i$为至少有$i$个多余的竖线的排列数,$B$为没有多余的竖线的排列数,则
$$
B=k^n-B_1+B_2-B_3+\cdots +(-1)^nB_n.
$$
现在我们来计算这些 $B_i$. $B_1$
指的是至少有一个多余的竖线的排列,我们可以这样得到。先写下 $k-2$
个竖线,再把 $[n]$ 中的数字放入这 $k-1$ 个分间中,然后把一个多余的
竖线插入任意一个数字的左边,或放在末尾,共有 $n+1$
种方法,也就是$B_1={n+1 \choose 1}(k-1)^n$.
类似地,我们可得$B_2={n+1 \choose 2}(k-2)^n$,
这时我们是有$k-2$个分间,再把两个多余的竖线插入。继续这种方法得$$B_i={n+1
\choose i}(k-i)^n$$把这些式子代人 $B$
中就得到了所要证的等式的右边.\qed


\pf {代数证明}\\
 对欧拉多项式 $A_n(x)=\sum\limits^n_{k=1}A(n,k)x^k,$
我们有(见欧拉多项式那节)
$$\sum\limits^{\infty}_{k=1}k^nx^k=\frac{A_n(x)}{(1-x)^{n+1}}.$$
前面已经给出了它的组合证明。由上式可得$$(1-x)^{n+1}\sum\limits^{\infty}_{k=1}k^nx^k=A_n(x).$$
比较两边系数便可得证。\qed

下面我们来看一下欧拉数和第二类 Stirling 数之间的关系。第二类
Stirling 数 $S(n,k)$ 是指把集合 $\{1,2,\cdots,n\}$ 分成 $k$
个互不相交的无序块并的个数。

\begin{thm}
对于任意的正整数 $n,r$, 有
\begin{equation}
S(n,r)=\frac{1}{r!}\sum_{k=1}^{r}A(n,k){n-k\choose r-k}.
\end{equation}
\end{thm}

\pf{组合证明}
 等式两边同乘以
$r!$ 得, $$r!S(n,r)=\sum_{k=1}^{r}A(n,k){n-k\choose r-k}$$
显然左边代表的是集合 $[n]$ 的有序 $r$
划分。我们只要说明右边也是计算的同样的东西。对于 $[n]$ 的具有 $k-1$
个下降数的置换,就产生了 $k$ 个递增的字串,这恰好对应了 把集合 $[n]$
分成 $k$ 个部分。如果 $k=r$, 那么就是我们所要求的。如果 $k<r$,
我们就需要把一些递增字串拆开成若干个更小的串(保持
原来数字的顺序不变),从而能到 $r$ 个递增字串。现在我们已经有了 $k$
个分块,我们还必须增加 $r-k$ 个块。$n$ 个元素的置换除了首末位置共有
$n-1$ 个空隙(相邻两个数字之间),
只要我们不在下降数的位置,就可以把串分成更小的串,这样共有$A(n,k){n-k\choose
r-k}$ 种方法。
\\由上面的方法我们得到了 $\sum_{k=1}^{r}A(n,k){n-k\choose
r-k}$ 个 $[n]$ 的有序 $r$
分划。显然这种分划可由置换唯一决定。反之,给定一个$[n]$的分划,在每个块中的元素按递增的顺序排列,那么一个有序划分,从左到右读就得到了一个
至多具有$r$个递增字串的置换。\qed

\begin{thm}
对于任意的正整数$n,k$,有
\begin{equation}
A(n,k)=\sum_{r=1}^{k}S(n,r)r!{n-r\choose k-r}(-1)^{k-r}.
\end{equation}
\end{thm}

\pf{代数证明}
由上面的性质把$S(n,r)=\frac{1}{r!}\sum_{k=1}^{r}A(n,k){n-k\choose
r-k}$代人要证式子的右边得,
$$\sum_{r=1}^{k}S(n,r)r!{n-r\choose k-r}(-1)^{k-r}=\sum_{r=1}^{k}(-1)^{k-r}{n-r\choose k-r}\sum_{i=1}^{r}A(n,i){n-i\choose
r-i}$$ 改变求和顺序得
$$\sum_{r=1}^{k}S(n,r)r!{n-r\choose k-r}(-1)^{k-r}=\sum_{i=1}^{r}A(n,i){n-i\choose r-i}\sum_{r=1}^{k}(-1)^{k-r}{n-r\choose
k-r}$$
此等式的左边就是我们要证明的式子的右边,所以我们只要说明上式的右边等于$A(n,k)$.
显然上式中$A(n,k)$前的系数为${n-k\choose k-k}=1$,
所以我们能证明对于$i<k$,$A(n,i)$的系数为零就完成了证明。注意到,如果$r<i$,就有${n-i\choose
r-i}=0$, 则对任意的$i<k$, 我们有
$$\sum_{r=i}^{k}{n-i\choose r-i}{n-r\choose k-r}(-1)^{k-r}=\sum_{r=i}^{k}{n-i\choose r-i}{k-n-1\choose k-r}={k-i-1\choose k-i}=0$$
最后第二个等号是由Cauchy's convolution formula \footnote{[Cauchy's
convolution formula] 设$x,y$为实数,$z$为正整数,则有${x+y\choose
z}=\sum_{d=0}^{z}{x\choose d}{y\choose z-d}$}而得到的。\qed






\section{欧拉多项式}
由下降数或胜位出发,我们定义
$$A_n(t)=\sum_{\pi\in S_n}t^{1+\des(\pi)}=
\sum_{\pi\in S_n}t^{1+\exc(\pi)}$$ 为$[n]$上的欧拉多项式(Eulerian
polynomial). 由此定义,则 $A_n(t)$中$t^{k}$的系数为欧拉数$A(n,k)$.
所以欧拉多项式也可写为 \[A_n(t)=\sum_{k\geq 1}A(n,k)t^k,\,n\geq1.\]

特别地,定义$A_0(t)=1$。
本节我们主要研究欧拉多项式的一些基本的性质。在此之前,我们先给出$n\leq
6$时的欧拉多项式。
\begin{align*}
A_1(t) &=t, \\[5pt]
A_2(t) &=t+t^2, \\[5pt]
A_3(t) &=t+4t^2+t^3, \\[5pt]
A_4(t) &=t+11t^2+11t^3+t^4,\\[5pt]
A_5(t) &=t+26t^2+66t^3+26t^4+t^5, \\[5pt]
A_6(t) &=t+57t^2+302t^3+302t^4+57t^5+t^6
\end{align*}

\begin{prop}\label{epd}
欧拉多项式满足下面的微分方程
\begin{equation}
A_{n+1}(t)=t(1-t)A_n'(t)+(n+1)tA_n(t).
\end{equation}
\end{prop}
\pf 由递推关系容易得到
\[
\sum_{k}A(n+1,k)t^k=\sum_{k}kA(n,k)t^k+(n+1)\sum_kA(n,k-1)t^k-(k-1)\sum_{k}a(n,k-1)t^k.\]
由此可得
\[A_{n+1}(t)=tA_n'(t)+t(n+1)A_n(t)-t^2A_n'(t).\]
整理一下上式即可得结论。\qed


由此微分方程,我们可以容易地得到下面这个等式。

\begin{prop}\label{pELdxs}
\begin{equation}
\sum\limits^{\infty}_{k=1}k^nx^k=\frac{A_n(x)}{(1-x)^{n+1}}.
\end{equation}
\end{prop}
\pf 我们利用归纳法来证明。\\
当 $n=1$ 时,左边=
$\sum\limits^{\infty}_{k=1}kx^k=x\frac{1}{1-x}'=\frac{x}{(1-x)^2}.$=右边。\\
假设 $n$ 时也成立,我们来看 $n+1$ 的情况。 由
\[  \sum\limits^{\infty}_{k=1}k^nx^k=\frac{A_n(x)}{(1-x)^{n+1}}.   \]
对上式两边 $x$ 进行微分,得
\[
\sum\limits^{\infty}_{k=1}k^{n+1}x^{k-1}=\frac{(1-x)A_n'(x)+(n+1)A_n(x)}{(1-x)^{n+2}}.\]
要证 \[
\sum\limits^{\infty}_{k=1}k^{n+1}x^k=\frac{A_{n+1}(x)}{(1-x)^{n+2}},\]
则相当于只需证
\[A_{n+1}(x)=x(1-x)A_n(x)+(n+1)xA_n(x).\]
而由性质 \eqref{epd} 上式成立。 \qed




现在我们进一步研究欧拉多项式的指数生成函数。

\begin{thm} \label{ec}
令 $$A(x)=\sum_{n\geq0}A_n(t)\frac{x^n}{n!},$$ 则 $A(x)$ 满足
\begin{equation}
A'(x)=(A(x)-1)A(x)+tA(x).
\end{equation}
\end{thm}
\pf
我们从生成函数的角度来考虑。假定每个排列的降序位包含最后一位,则相应的生成函数仍是
$A(x).$ $A'(x)$
表示在排列中去掉最大元后所得到的排列对应的生成函数。不妨设
$\pi=\pi_1(n+1)\pi_2=a_1a_2\cdots a_i(n+1)a_{i+2}\cdots a_{n+1}\in
S_{n+1},\ 0\leq i\leq n.$

下面分析 $\pi$ 去掉 $n+1$ 后的结构。

如果 $i=0,$ 即 $\pi_1=\emptyset,$ 而 $\pi_2=\emptyset$ 或者
$\pi_2\neq \emptyset.$ 此时,在 $\pi$ 中去掉 $n+1$
后,所得排列降序数减少 $1,$ 所以 $\pi_1=\emptyset$
时,对应生成函数为 $tA(x);$ 如果 $1\leq i\leq n,$ 即 $\pi_1\neq
\emptyset,$ 设 $\des(\pi_1)=k_1,\,\des(\pi_2)=k_2,$ 则 $\pi$ 去掉
$n+1$ 后,所得的两个排列的降序数之和为 $k_1+k_2,$ 而原排列降序中
$a_i(n+1)$ 不对应一个降序,但 $(n+1)a_{i+2}$
一定对应一个降序,即原排列降序数为
$\left(k_1-1\right)+1+k_2=k_1+k_2,$ 则 $\pi_1\neq \emptyset$
时,对应生成函数为 $(A(x)-1)A(x).$

所以有 $$A'(x)=(A(x)-1)A(x)+tA(x).$$ \qed

\begin{coro}
$A_n(t)$ 的生成函数为
$$A(x)=\sum_{n\geq
0}A_n(t)\frac{x^n}{n!}=\frac{1-t}{1-te^{(1-t)x}}.$$
\end{coro}
\pf 根据关系式 \eqref{ec},解如下微分方程, \allowdisplaybreaks
\begin{align*}\frac{d A}{(A-1)A+tA}&=d x,\\[5pt]
\frac{1}{t-1}\left(\frac{1}{A}-\frac{1}{A-1+t}\right)d A&=dx,\\[5pt]
 d\ln\frac{A}{A-1+t}&=(t-1)dx,\\[5pt]
\frac{A}{A-1+t}&=ce^{(t-1)x},
\end{align*}其中 $c$ 为常数,由初值 $A(0)=1,$ 得到 $c=\frac{1}{t},$
所以 $$A(x)=\frac{1-t}{1-te^{(1-t)x}}.$$ \qed

将生成函数展开为$x$和$t$的幂级数,可得
\begin{equation}\label{Ant}
A_n(t)=(1-t)^{n+1}\sum_{k\geq1}k^nt^k (n\geq1).
\end{equation}
即给出了性质\ref{pELdxs}的另一个证明。




下面我们讨论一下欧拉多项式的根的特点,首先我们给出一些关于根的特点的定义。

设 $f$ 是一个度为 $n$ 的,且根全为实数的多项式,定义
$\mathrm{roots}(f)=\left(a_{1},\ldots,\,a_{n}\right),$ 其中
$a_{1}\leqslant a_{2}\leqslant \cdots \leqslant a_{n}$ 为 $f(x)=0$
的根。(注意:如果我们写 $\mathrm{roots}(f)$ 的话,则已假定 $f$
的根全为实根)。

 \begin{defi} 给定多项式 $f,\,g,$
设 $\mathrm{roots}(f)=\left(a_{1},\ldots,\,a_{n}\right),\,
\mathrm{roots}(g)=\left(b_{1},\ldots,\, b_{n}\right),$ 称 $f,\,g$
是严格交错的,如果它们的根满足以下四种关系之一:
\begin{align*}
 a_{1}&<b_{1}<a_{2}<b_{2}< \cdots <a_{n}<b_{n};\\[5pt]
 b_{1}&<a_{1}<b_{2}< a_{2}< \cdots <b_{n};\\[5pt]
 b_{1}&<a_{1}<b_{2}<a_{2}< \cdots <b_{n}<a_{n};\\[5pt]
 a_{1}&<b_{1}<a_{2}< b_{2}< \cdots <a_{n};
\end{align*}
当把不等号 $<$换成$\leq$ 就称 $f,\,g$
是交错的。显然,如果两个多项式交错,则它们的度至多相差 $1$.
\end{defi}

%%%%%%%%%%%%%%%%%%%%%%%%%%%%%%%%%%%%%%%%%%%%%%%%%%%%%%%%%%%%%%
\begin{exa}
Eulerian多项式 $A_{3}(t)$ 和 $A_{4}(t)$ 是交错的。因为
\begin{align*}
A_3(t) &=t+4t^2+t^3 \\
A_4(t) &=t+11t^2+11t^3+t^4.
\end{align*}
所以$\roots\left(A_{3}(t)\right)=\left(-2-\sqrt{3},\,-2+\sqrt{3},\,0\right),\,
\roots\left(A_{4}(t)\right)=\left(-5-2\sqrt{6},\,-1,\,-5+2\sqrt{6},\,0\right).$

显然它们的根满足定义中的关系之一,所以这两个多项式是交错的。
\end{exa}
%%%%%%%%%%%%%%%%%%%%%%%%%%%%%%%%%%%%%%%%%%%%%%%%%%%%%%%%%%%%%%
为了叙述的方便,我们给出符号函数的定义。
 \begin{defi}
定义在实数集上的符号函数 $\sgn(x)$ 为:
$$ \sgn(x)=\left\{ \begin{array}{ll}
+1, \quad &\textrm{if $x>0,$}\\[5pt]
0 , \quad &\textrm{if $x=0,$}\\[5pt]
-1,  \quad &\textrm{if $x<0.$}
\end{array}\right.
$$
 \end{defi}
%%%%%%%%%%%%%%%%%%%%%%%%%%%%%%%%%%%%%%%%%%%%%%%%%%%%%%%%%%%%%%
 \begin{thm}
对于任意给定的 $n,$ 欧拉多项式 $A_{n}(t)=\sum
_{k=1}^{n}A(n,\,k)t^{k}$ 的根全是实根,并且 $A_{n-1}(t)$ 和
$A_{n}(t)$ 是交错的。
\end{thm}

\pf 令$B_n(t)={A_n(t)\over(1-t)^{n+1}}$,由\eqref{Ant},有
$${d\over dt}B_{n-1}(t)={d\over dt}\sum_{k\geq1}k^{n-1}t^k={1\over t}B_n(t)
$$整理得,
$$B_n(t)=t{d\over dt}B_{n-1}(t)
$$
补充定义$A_0(t)=t,$ 对于 $n\geqslant 1,$ 有
\begin{equation}\label{dt}
A_{n}(t)=t(1-t)^{n+1}\frac{d}{dt}(1-t)^{-n}A_{n-1}(t)
\end{equation}

下面我们用归纳法来证明欧拉多项式的根全为实根。当 $n=0$
时,欧拉多项式 $A_0(t)=t$ 的根为 $t=0.$

假设 $A_{n-1}(t)$ 有 $n-1$ 个不同的实根,其中有一个为 $t=0,$
其它全为负根。

从 $A_{n}(t)$ 与 $A_{n-1}(t)$ 的微分关系中,运用罗尔中值定理可知,在
$A_{n-1} (t)$ 的每两个相邻根之间必有一个 $A_{n}(t)$ 的根,而显然 $0$
也是一个根,这样我们就找到了 $A_{n}(t)$ 的 $n- 1$
个根。由于虚根是成对出现的,所以最后一个根也是实根。要证明 $A_{n-1}
(t)$ 和$A_{n}(t)$ 是交错的,只要说明这个根比 $A_{n-1} (t)$
的最小的那个根还要小。

设
$\roots\left(A_{n-1}(t)\right)=\left(r_{1},\,r_2,\ldots,\,r_{n-1}\right),$
则 $\sgn (A_{n-1}'(r_k))=(-1)^k,$ 因为 $A_{n-1}(t)'$
的首项系数为正的。除了 $r_1=0,$ 其余根均为负的,由\eqref{dt}式得
$\sgn (A_{n}(r_k))=(-1)^k,$ 因为 $A_{n}(t)$ 的首项系数也为正,所以
$\sgn (A_{n}(+\infty))=+1,\,  \sgn (A_{n}(-\infty))=(-1)^{n}.$
由此可知 $A_{n}(t)$ 必有一个根在区间 $(-\infty,\,r_{n-1})$ 上,所以
$A_{n-1}(t)$ 和 $A_{n}(t)$ 是交错的。\qed




\section{排列与杨表的对应}

\subsection{杨表}
杨表(Young tableau)是由杨(R.A.
Young)在1901年研究不变量理论时引入的,它在组合数学、群表示论、数学物理
等领域中都有重要应用。通常情况下,杨表是指定义在
杨图上的半标准杨表。

给定一个整数分拆$\lambda=(\lambda_1,\lambda_2,
\ldots,\lambda_k)$,与$\lambda$对应的
杨图(Young Diagram)定义为平面上一些左对齐的$k$行方块的集合,
使得第$i$行恰有$\lambda_i$个方块。例如,分拆$(4,2,1)$对应的杨图为
\begin{figure}[h]
\setlength{\unitlength}{0.5mm}
\begin{center}
\begin{picture}(50,30)
\put(0,30){\line(1,0){40}}\put(0,20){\line(1,0){40}}
\put(0,20){\line(1,0){20}}\put(0,10){\line(1,0){20}}
\put(0,0){\line(1,0){10}}\put(0,0){\line(0,1){30}}
\put(10,0){\line(0,1){30}}\put(20,30){\line(0,-1){20}}
\put(30,30){\line(0,-1){10}}\put(40,30){\line(0,-1){10}}
\end{picture}
\end{center}
\end{figure}

在$\lambda$对应的杨图中,用正整数填充图中的每个方块得到一个阵列$T$,
若其每行递增每列严格递增,则称$T$为具有形状$\lambda$的
半标准杨表SSYT(semistandard Young
tableau),并记$\mathrm{sh}(T)=\lambda$。如果$T$中含有$\alpha_i$个$i$,那么称$(\alpha_1,\alpha_2,\ldots)$为
$T$的 类型(type)。例如,下面半标准杨表的类型为$(1,2,2,1,1)$。
\begin{figure}[ht]
\setlength{\unitlength}{0.5mm}
\begin{center}
\begin{picture}(50,30)
\put(0,30){\line(1,0){40}}\put(0,20){\line(1,0){40}}
\put(0,20){\line(1,0){20}}\put(0,10){\line(1,0){20}}
\put(0,0){\line(1,0){10}}\put(0,0){\line(0,1){30}}
\put(10,0){\line(0,1){30}}\put(20,30){\line(0,-1){20}}
\put(30,30){\line(0,-1){10}}\put(40,30){\line(0,-1){10}}

 \put(4,22){1}  \put(14,22){2}
\put(24,22){3}\put(34,22){3} \put(4,12){2}
 \put(14,12){5}
\put(4,2){4}
\end{picture}
\end{center}
\end{figure}

设$\lambda$是$n$的一个分拆。
若用${1,2,\ldots,n}$填充$\lambda$对应的杨图使得每个数字恰好出现一次,
并且每行每列递增,则称这样的阵列为具有形状$\lambda$的
标准杨表SYT(standard Young tableau)。
例如下图是一个形状为$(4,2,1)$的标准杨表。

\begin{figure}[ht]
\setlength{\unitlength}{0.5mm}
\begin{center}
\begin{picture}(50,30)
\put(0,30){\line(1,0){40}}\put(0,20){\line(1,0){40}}
\put(0,20){\line(1,0){20}}\put(0,10){\line(1,0){20}}
\put(0,0){\line(1,0){10}}\put(0,0){\line(0,1){30}}
\put(10,0){\line(0,1){30}}\put(20,30){\line(0,-1){20}}
\put(30,30){\line(0,-1){10}}\put(40,30){\line(0,-1){10}}
\put(4,22){1}  \put(14,22){3} \put(24,22){6}\put(34,22){7}
\put(4,12){2}
 \put(14,12){5}
\put(4,2){4}
\end{picture}
\end{center}
\end{figure}


\subsection{RSK算法}

RSK算法是根据罗宾森 (G.B. Robinson),森斯特德(C.E.
Schensted),克努斯 (D.E. Knuth)
的名字命名的,它是对称函数领域里一个优美的组合对应。
RSK算法最初是在试图证明
李特尔伍德-里查德森法则时作为一个工具由罗宾森于1938年提出的,
后又被森斯特德于1961年在研究排列的最长递增和递降子序列时重新发现,而
克努斯于1970年将RSK算法从排列推广到了广义置换(generalized
permutaition)上。
关于更多的RSK算法的介绍可见Stanley书\cite{Stanley1999}.


RSK算法的基本运算是行插入算法,即把整数$i$插入行递增和列严格递增的一个
表格$T$中,即$T$是一个半标准杨表。
把$i$插入$T$后,我们就得到了一个新的表格,记为$T\leftarrow i$,
仍旧满足行递增和列严格递增。如果$S$是$T$中的值组成的集合,则$S \cup
\{i\}$就是$T\leftarrow
i$中的值组成的集合。现在我们就来介绍如何递归地定义$T\leftarrow i$.

\begin{itemize}
  \item 如果$T$的第一行是空行或者第一行上的最大值小于$i$,
  则把$i$插入到第一行的末尾。
  \item 否则,找到第一行中第一个满足大于$i$的数字$j$, 用$i$取代$j$,
  然后用同样的原则把$j$插入第二行。继续这种算法,直到某个数插入
  某一行的末尾即停止(或者在原来的表格中新增了一行)。
\end{itemize}

同时,我们可以定义与之对应的$insertion\
 path$,即$T$中所有的那些在这些行插入的过程中改变过的位置的集合,
 记为$I(T\leftarrow i)$。

\begin{exa}
下面是相应的一个简单的例子:
\end{exa}

\begin{picture}(250,120)(-20,5)
\put(45,110){i=5} \put(0,70){T=} \put(30,90){2} \put(45,90){3}
\put(60,90){3} \put(75,90){6} \put(90,90){7} \put(30,75){4}
\put(45,75){5} \put(60,75){5} \put(75,75){7} \put(30,60){6}
\put(45,60){6} \put(60,60){8} \put(30,45){9}
\put(140,70){$T\leftarrow i=$} \put(200,90){2} \put(215,90){3}
\put(230,90){3} \put(245,90){\textbf{5}} \put(260,90){7}
\put(200,75){4} \put(215,75){5} \put(230,75){5}
\put(245,75){\textbf{6}} \put(200,60){6} \put(215,60){6}
\put(230,60){\textbf{7}} \put(200,45){\textbf{8}}
\put(200,30){\textbf{9}} \put(0,10){$I(T\leftarrow i)= \{(1,4),
(2,4), (3,3), (4,1), (5,1)\}$}
\end{picture}

关于更多的RSK算法的介绍可见Sagan书\cite{Sagan}第三章
和Stanley书\cite{Stanley1999}第七章。

令$w=a_1a_2\cdots a_n \in S_n$, 令$\emptyset$为空表格。定义
\[ P_i=P_i(w)=(\cdots ((\emptyset \leftarrow a_1)\leftarrow a_2)
\leftarrow \cdots \leftarrow a_i).\]
也就是说,$P_i$是从空表格出发,依次插入$a_1,a_2,\ldots,a_i$而得的。
此时$P_i$可以看成是一个标准杨表,除了它的值可以为任意的不同整数,
而不是
仅仅限制为$1,2,\ldots,n$. 记$P=P(w)=P_n(w)$. 定义$Q_0=\emptyset$,
当$Q_{i-1}$确定好之后,定义$ Q_i=Q_i(w)$为是在$Q_{i-1}$中插入$i$,
使得$Q_i$和$P_i$具有相同的形状,且不改变$Q_{i-1}$中任何元素的位置和值。
记$Q=Q(w)=Q_n(w)$,
最后定义RSK算法作用在$w$上后的输出值为一对杨表$(P,Q)$, 记作$w
\xrightarrow[]{\text{RSK}}(P,Q)$.



例如,我们对排列$\pi=256384197$运用RSK算法,对应的$(P,Q)$可有如下过程生成。
\newpage
\begin{picture}(100,50)
\put(35,40){\line(1,0){15}}\put(35,40){\line(0,-1){15}}\put(35,25){\line(1,0){15}}\put(50,40){\line(0,-1){15}}
\put(40,29){2}\put(40,50){$P_i$}

\put(285,40){\line(1,0){15}}\put(285,40){\line(0,-1){15}}\put(285,25){\line(1,0){15}}\put(300,40){\line(0,-1){15}}
\put(290,29){1}\put(290,50){$Q_i$}

\put(28,0){\line(1,0){30}}\put(28,-15){\line(1,0){30}}\put(28,0){\line(0,-1){15}}\put(43,0){\line(0,-1){15}}\put(58,0){\line(0,-1){15}}
\put(33,-11){2}\put(48,-11){5}

\put(278,0){\line(1,0){30}}\put(278,-15){\line(1,0){30}}\put(278,0){\line(0,-1){15}}\put(293,0){\line(0,-1){15}}\put(308,0){\line(0,-1){15}}
\put(283,-11){1}\put(298,-11){2}

\put(20,-40){\line(1,0){45}} \put(20,-55){\line(1,0){45}}
\put(20,-40){\line(0,-1){15}}\put(35,-40){\line(0,-1){15}}\put(50,-40){\line(0,-1){15}}\put(65,-40){\line(0,-1){15}}
\put(25,-51){2}\put(40,-51){5}\put(55,-51){6}

\put(270,-40){\line(1,0){45}} \put(270,-55){\line(1,0){45}}
\put(270,-40){\line(0,-1){15}}\put(285,-40){\line(0,-1){15}}\put(300,-40){\line(0,-1){15}}\put(315,-40){\line(0,-1){15}}
\put(275,-51){1}\put(290,-51){2}\put(305,-51){3}

\put(20,-80){\line(1,0){45}} \put(20,-95){\line(1,0){45}}
\put(20,-110){\line(1,0){15}}\put(20,-80){\line(0,-1){30}}\put(35,-80){\line(0,-1){30}}\put(50,-80){\line(0,-1){15}}
\put(65,-80){\line(0,-1){15}}\put(25,-91){2}\put(40,-91){3}\put(55,-91){6}\put(25,-106){5}

\put(270,-80){\line(1,0){45}} \put(270,-95){\line(1,0){45}}
\put(270,-110){\line(1,0){15}}\put(270,-80){\line(0,-1){30}}\put(285,-80){\line(0,-1){30}}\put(300,-80){\line(0,-1){15}}
\put(315,-80){\line(0,-1){15}}\put(275,-91){1}\put(290,-91){2}\put(305,-91){3}\put(275,-106){4}

\put(13,-135){\line(1,0){60}}\put(13,-150){\line(1,0){60}}\put(13,-165){\line(1,0){15}}\put(13,-135){\line(0,-1){30}}
\put(28,-135){\line(0,-1){30}}\put(43,-135){\line(0,-1){15}}\put(58,-135){\line(0,-1){15}}\put(73,-135){\line(0,-1){15}}
\put(18,-146){2}\put(33,-146){3}\put(48,-146){6}\put(63,-146){8}\put(18,-161){5}

\put(263,-135){\line(1,0){60}}\put(263,-150){\line(1,0){60}}\put(263,-165){\line(1,0){15}}\put(263,-135){\line(0,-1){30}}
\put(278,-135){\line(0,-1){30}}\put(293,-135){\line(0,-1){15}}\put(308,-135){\line(0,-1){15}}\put(323,-135){\line(0,-1){15}}
\put(268,-146){1}\put(283,-146){2}\put(298,-146){3}\put(313,-146){5}\put(268,-161){4}

\put(13,-190){\line(1,0){60}}\put(13,-205){\line(1,0){60}}\put(13,-220){\line(1,0){30}}\put(13,-190){\line(0,-1){30}}
\put(28,-190){\line(0,-1){30}}\put(43,-190){\line(0,-1){30}}\put(58,-190){\line(0,-1){15}}\put(73,-190){\line(0,-1){15}}
\put(18,-201){2}\put(33,-201){3}\put(48,-201){4}\put(63,-201){8}\put(18,-216){5}\put(33,-216){6}

\put(263,-190){\line(1,0){60}}\put(263,-205){\line(1,0){60}}\put(263,-220){\line(1,0){30}}\put(263,-190){\line(0,-1){30}}
\put(278,-190){\line(0,-1){30}}\put(293,-190){\line(0,-1){30}}\put(308,-190){\line(0,-1){15}}\put(323,-190){\line(0,-1){15}}
\put(268,-201){1}\put(283,-201){2}\put(298,-201){3}\put(313,-201){5}\put(268,-216){4}\put(283,-216){6}
\end{picture}
\\
\\
\\
\\
\\
\\
\\
\\
\\
\\
\\
\\
\\
\\
\\
\\
\\


\begin{picture}(100,50)
\put(13,50){\line(1,0){60}}\put(13,35){\line(1,0){60}}\put(13,20){\line(1,0){30}}\put(13,5){\line(1,0){15}}
\put(13,50){\line(0,-1){45}}\put(28,50){\line(0,-1){45}}\put(43,50){\line(0,-1){30}}\put(58,50){\line(0,-1){15}}
\put(73,50){\line(0,-1){15}}\put(18,39){1}\put(33,39){3}\put(48,39){4}\put(63,39){8}\put(18,24){2}\put(33,24){6}\put(18,9){5}

\put(263,50){\line(1,0){60}}\put(263,35){\line(1,0){60}}\put(263,20){\line(1,0){30}}\put(263,5){\line(1,0){15}}
\put(263,50){\line(0,-1){45}}\put(278,50){\line(0,-1){45}}\put(293,50){\line(0,-1){30}}\put(308,50){\line(0,-1){15}}
\put(323,50){\line(0,-1){15}}\put(268,39){1}\put(283,39){2}\put(298,39){3}\put(313,39){5}\put(268,24){4}\put(283,24){6}\put(268,9){7}

\put(5,-20){\line(1,0){75}}\put(5,-35){\line(1,0){75}}\put(5,-50){\line(1,0){30}}\put(5,-65){\line(1,0){15}}
\put(5,-20){\line(0,-1){45}}\put(20,-20){\line(0,-1){45}}\put(35,-20){\line(0,-1){30}}\put(50,-20){\line(0,-1){15}}
\put(65,-20){\line(0,-1){15}}\put(80,-20){\line(0,-1){15}}\put(10,-31){1}\put(25,-31){3}\put(40,-31){4}\put(55,-31){8}
\put(70,-31){9}\put(10,-46){2}\put(25,-46){6}\put(10,-61){5}

\put(255,-20){\line(1,0){75}}\put(255,-35){\line(1,0){75}}\put(255,-50){\line(1,0){30}}\put(255,-65){\line(1,0){15}}
\put(255,-20){\line(0,-1){45}}\put(270,-20){\line(0,-1){45}}\put(285,-20){\line(0,-1){30}}\put(300,-20){\line(0,-1){15}}
\put(315,-20){\line(0,-1){15}}\put(330,-20){\line(0,-1){15}}\put(260,-31){1}\put(275,-31){2}\put(290,-31){3}\put(305,-31){5}
\put(320,-31){8}\put(260,-46){4}\put(275,-46){6}\put(260,-61){7}


\put(5,-90){\line(1,0){75}}\put(5,-105){\line(1,0){75}}\put(5,-120){\line(1,0){45}}\put(5,-135){\line(1,0){15}}
\put(5,-90){\line(0,-1){45}}\put(20,-90){\line(0,-1){45}}\put(35,-90){\line(0,-1){30}}\put(50,-90){\line(0,-1){30}}
\put(65,-90){\line(0,-1){15}}\put(80,-90){\line(0,-1){15}}
\put(10,-101){1}\put(25,-101){3}\put(40,-101){4}\put(55,-101){7}\put(70,-101){9}\put(10,-116){2}\put(25,-116){6}
\put(40,-116){8}\put(10,-131){5}

\put(255,-90){\line(1,0){75}}\put(255,-105){\line(1,0){75}}\put(255,-120){\line(1,0){45}}\put(255,-135){\line(1,0){15}}
\put(255,-90){\line(0,-1){45}}\put(270,-90){\line(0,-1){45}}\put(285,-90){\line(0,-1){30}}\put(300,-90){\line(0,-1){30}}
\put(315,-90){\line(0,-1){15}}\put(330,-90){\line(0,-1){15}}
\put(260,-101){1}\put(275,-101){2}\put(290,-101){3}\put(305,-101){5}\put(320,-101){8}\put(260,-116){4}\put(275,-116){6}
\put(290,-116){9}\put(260,-131){7}
\end{picture}
\\
\\
\\
\\
\\
\\
\\
\\
\\
\\
\\
\begin{thm}
令$\mathcal{T}_{n}=\{(T_{1},
T_{2})|T_{1}$和$T_{2}$是两个形状均为$\lambda\mapsto
n$的标准杨表$\}$,则
RSK算法给出了$[n]$上的排列集合$S_n$和集合$\mathcal{T}_{n}$
之间的一一对应。
\end{thm}
\pf 记映射\[\varphi: S\mapsto \mathcal{T}_{n}, w=a_{1}a_{2}\cdots
a_{n}\mapsto (T_{1}, T_{2})\]为RSK算法后所对应的那个映射。

下面来考虑$\varphi^{-1}$。对任一$(P, Q)=(P(n), Q(n))\in
\mathcal{T}_{n}$,假定$Q_{rs}=n$,即$n$在$Q$的第$r$行第$s$列的位置。则令$Q(n-1)=Q(n)\setminus
n$。不难知道,$P_{rs}$是将$\pi_{n}$插入$P(n-1)$后对应的$insertion\
 path$中最后一个位置的元素。事实上,不难得到$P(n-1)\leftarrow \pi_{n}$的逆过程:$P_{rs}$一定是被$P$的第$r-1$
 行中最靠右的比$P_{rs}$小的元素不妨设为$P_{r-1, t}$挤入第$r$行的。因此,从$P$中移去$P_{rs}$所在的方格,将$P_{r-1, t}$
 用$P_{rs}$代换,然后继续将第$r-2$行中最靠右的比$P_{r-1, t}$小的元素用$P_{r-1, t}$代换,$\ldots$,最后,必然从$P$中挤出了某元素即为$\pi_{n}$。至此,我们由$(P(n), Q(n))$唯一得到了$(n,
\pi_{n})$和$(P(n-1),
Q(n-1))$,如此继续,最终可得置换$\left(\begin{array}{cccc}
1&2&\cdots&n\\\pi_1&\pi_2&\cdots&\pi_n\end{array}\right)$。\qed

如果$w \xrightarrow[]{\text{RSK}}(P,Q)$,
且$P,Q$具有相同的形状$\lambda$, 则我们称$w$具有形状$\lambda$,
记为$\lambda=\mathrm{sh}(w)$.
$\lambda$的共轭分拆$\lambda'=(\lambda_1',\lambda_2',\ldots)$,
其对应的表格是把$\lambda$的表格翻转而得。等价地说,$j$在$\lambda'$中出现
的次数为$\lambda_j-\lambda_{j+1}$.
记$l(\lambda)$为分拆$\lambda$中非零部分的个数,所以$l(\lambda)=\lambda_1'$.

$w$的一个递增子序列是指一个子序列$a_{i_1},a_{i_2},\ldots,a_{i_k}$满足
$a_{i_1}<a_{i_2}<\cdots<a_{i_k}$,
类似地可以定义递减子序列。例如,若$w=5\,6\,4\,2\,7\,1\,3$,
则$5\,6\,7$是一个递增子序列,$5\,4\,3$是一个递减子序列。令$\mathrm{is}(w)
 (\mathrm{ds}(w))$为$w$中的最长递增(递减)子序列的长度。对于如上的$w$, 我们
 有$\mathrm{is}(w) = 3$ (对应于$5\,6\,7$), $\mathrm{ds}(w) =
4$ (对应于序列$5\,4\,2\,1$或$6\,4\,2\,1$).
由RSK算法,我们可以发现$T_{1}$中第一行的方格总数即为$a=a_{1}a_{2}\cdots
a_{n}$的最长递增子列的长度;$T_{1}$中第一列的方格总数即为$a=a_{1}a_{2}\cdots
a_{n}$的最长递减子列的长度。所以我们可以得到以下关于
排列中的最长的递增或递减子序列的长度与对于杨表之间的
关系。
\begin{thm}
令$w \in S_n$, 且$\mathrm{sh}(w)=\lambda$, 则
\begin{eqnarray}
\mathrm{is}(w)&=&\lambda_1,\\
 \mathrm{ds}(w)&=&\lambda_1'.
\end{eqnarray}
\end{thm}


下面我们考虑置换$\pi$与其逆置换$\pi^{-1 }$在RSK算法下所对应的杨表
之间的相互关系。


\begin{thm} 若 $\pi\in S_n$ 并且
$\pi\xrightarrow{\text{RSK}}(P, Q)$,则对 $\pi$ 的逆 $\pi^{-1 }$,有
$\pi^{-1 }\xrightarrow{\text{RSK}}(Q, P)$。
\end{thm}

\pf 设 $\pi=\left(\begin{array}{cccc}
1&2&\cdots&n\\v_1&v_2&\cdots&v_n\end{array}\right)$=
$\left(\begin{array}{c}
u\\v\end{array}\right)$, $\pi^{-1 }=\left(\begin{array}{c}
v\\u\end{array}\right)_{sorted}$ (即适当排列使得
$\left(\begin{array}{c} v\\u\end{array}\right)_{sorted}$
中第一行元素递增)。按如下方式定义 inversion poset
$I=I\left(\begin{array}{c} u\\v\end{array}\right)$:

$\left(\begin{array}{c} u\\v\end{array}\right)$ 的每列定义为 $I$
中的每个点,若 $a<c, b<d$,则在 $I$ 中 $ab<cd$ (为方便起见,将
$\begin{array}{c} a\\b\end{array}$ 记为 $ab$)。

由 $I(A)$ 的定义容易得到下面的引理。


\begin{lem}  映射 $\varphi: I\left(\begin{array}{c}
u\\v\end{array}\right)\rightarrow I\left(\begin{array}{c}
v\\u\end{array}\right)$ \ \ $\varphi(ab)=ba$ 是
$I\left(\begin{array}{c} u\\v\end{array}\right)$ 到
$I\left(\begin{array}{c} v\\u\end{array}\right)$ 的同构。\end{lem}


定义 $I_1$ 是 $I$ 中最小元的集合,$I_2$ 是 $I-I_1$
中最小元的集合,$I_3$ 是 $I-I_1-I_2$
中最小元的集合,$\cdots$。注意,易知 $I_i$ 是 $I$
中的反链即它的元素可以记为:$(u_{i1}, v_{i1}), (u_{i2}, v_{i2}),
\ldots, (u_{in_i}, v_{in_i})$ 使得 $u_{i1}<u_{i2}<\cdots<u_{in_i}$
且
$v_{i1}>v_{i2}>\cdots>v_{in_i}(n_i=|I_i|)$。
我们假定以下反链中的元素都作如此标记。


\begin{lem}\label{ya}  若 $I_1, I_2, \ldots, I_d$ 是如上标记的 $I$ 的非空反链,则 $P$ 的第一行元素是 $v_{1n_1}v_{2n_2}\cdots
v_{dn_d}$,$Q$ 的第一行元素是 $u_{11}u_{21}\cdots u_{d1}$。并且,若
$(u_k, v_k)\in I_i$,则 $v_k$ 在 RSK 算法执行过程中被插入 $P(k-1)$
第一行的第 $i$ 个位置。
\end{lem}


\noindent {\bf{引理\ref{ya}的证明}} 通过对 $n$ 归纳来证此引理。$n=1$
时显然结果是平凡的。假设对 $n-1$ 时此引理成立,并且令
$\left(\begin{array}{c}
u\\v\end{array}\right)=\left(\begin{array}{cccc}
1&2&\cdots&n\\v_1&v_2&\cdots&v_n\end{array}\right)$,
$\left(\begin{array}{c}
\tilde{u}\\\tilde{v}\end{array}\right)=\left(\begin{array}{cccc}
1&2&\cdots&n-1\\v_1&v_2&\cdots&v_{n-1}\end{array}\right)$。 设
$(P(n-1), Q(n-1))$ 是插入 $v_1, \ldots, v_{n-1}$ 后所得到的
tableau,$I_i^{'} :=I_i\left(\begin{array}{c}
\tilde{u}\\\tilde{v}\end{array}\right)=\{(\tilde{u}_{i_1},
\tilde{v}_{i_1}),  (\tilde{u}_{i_2}, \tilde{v}_{i_2}), \ldots,
(\tilde{u}_{im_i}, \tilde{v}_{im_i})\},\\ 1\leq i\leq e(e=d-1$ 或
$e=d)$。由归纳假设易知,$P(n-1)$ 的第一行是
$\tilde{v}_{1m_1}\tilde{v}_{2m_2}\cdots\tilde{v}_{em_e}$,$Q(n-1)$
的第一行是
$\tilde{u}_{11}\tilde{u}_{21}\cdots\tilde{u}_{e1}$。现在将 $v_n$
插入 $P(n-1)$。若 $\tilde{v}_{im_i}>v_n$,则 $I_i^{'}\cup (u_n,
v_n)$ 是 $I\left(\begin{array}{c} u\\v\end{array}\right)$
的一个反链。因此易知若 $i$ 是最小的满足 $\tilde{v}_{im_i}>v_n$
的下标,则 $(u_n, v_n)\in I_i\left(\begin{array}{c}
u\\v\end{array}\right)$;若不存在这样的 $i$,则 $I_d=\{(u_n,
v_n)\}$。从而此引理得证。

 若记 $I_i\left(\begin{array}{c}
u\\v\end{array}\right)=\{(u_{i1},v_{i1}), (u_{i2},v_{i2}),\ldots,
(u_{im_i},v_{im_i})\}$ 且其中有 $u_{i1}<u_{i2}<\cdots<u_{im_i},
v_{i1}>v_{i2}>\cdots>v_{im_i}$ 成立,则 $I_i\left(\begin{array}{c}
v\\u\end{array}\right)=\{(v_{im_i}, u_{im_i}), \ldots,\\ (v_{i2},
u_{i2}), (v_{i1}, u_{i1})\}$ 且 $v_{im_i}<\cdots<v_{i2}<v_{i1},
u_{im_i}>\cdots>u_{i2}>u_{i1}$。因此若 $\pi^{-1
}\xrightarrow{\text{RSK}}(P^{'}, Q^{'})$,则由引理 1.3 知 $P^{'}$
的第一行元素是 $u_{11}u_{21}\cdots u_{d1}$,$Q^{'}$ 的第一行元素是
$v_{1m_1}v_{2m_2}\cdots v_{dm_d}$。即 $P^{'}, Q^{'}$ 的第一行元素与
$Q, P$ 的第一行元素分别相等。

易知,在 RSK 算法执行过程中,$v_{ij}(1\leq j<m_i)$ 比 $v_{rs}(1\leq
s<m_r)$ 先被挤入第二行当且仅当 $u_{i, j+1}<u_{r, s+1}$。设
$\overline{P}, \overline{Q}$ 分别表示移去 $P, Q$ 第一行后所得的
tableau,从而有

\begin{align*}
\left(\begin{array}{c} a\\b\end{array}\right)
&:=\left(\begin{array}{cccccccccc}
u_{12}&\cdots&u_{1m_1}&u_{22}&\cdots&u_{2m_2}&\cdots&u_{d2}&\cdots&u_{dm_d}\\v_{11}&\cdots&v_{1,
m_1-1}&v_{21}&\cdots&v_{2, m_2-1}&\cdots&v_{d1}&\cdots&v_{d,
m_d-1}\end{array}\right)_{sorted}\\&\xrightarrow{\text{RSK}}(\overline{P}
, \overline{Q}).
\end{align*}



设 $\overline{P}^{'}, \overline{Q}^{'}$ 分别表示移去 $P^{'}, Q^{'}$
第一行后所得的 tableau,对 $\left(\begin{array}{c}
v\\u\end{array}\right)$ 进行类似讨论有:


\begin{align*}
 \left(\begin{array}{c}
a^{'}\\b^{'}\end{array}\right) &:=\left(\begin{array}{cccccccccc}
v_{1, m_1-1}&\cdots&v_{11}&v_{2, m_2-1}&\cdots&v_{21}&\cdots&v_{d,
m_d-1}&\cdots&v_{d1}\\u_{1m_1}&\cdots&u_{12}&u_{2m_2}&\cdots&u_{22}&\cdots&u_{dm_d}&\cdots&u_{d2}\end{array}\right)
_{sorted}\\&\xrightarrow{\text{RSK}}(\overline{P}^{'} ,
\overline{Q}^{'}).\end{align*}


 由于 $\left(\begin{array}{c}
a\\b\end{array}\right)=\left(\begin{array}{c}
b^{'}\\a^{'}\end{array}\right)_{sorted}$, 从而由对行数的归纳易知
$({\overline{P}}^{'}, \overline{Q}^{'})=(\overline{Q},
\overline{P})$。 \qed





% \bibliographystyle{cfcbook}
\begin{thebibliography}{99}



\bibitem{Sagan}
B. Sagan, The Symmetric Group, second ed. Graduate Texts in
Mathematics 203, Springer-Verlag, New York, 2001.


\bibitem{Stanley86}
R.P. Stanley, Enumerative Combinatorics, vol. 1. Wadsworth and
Brooks/Cole, Pacific Grove, CA, 1986; second printing, Cambridge
University Press, New York/Cambridge, 1996.

\bibitem{Stanley1999}
R.P. Stanley, Enumerative Combinatorics, vol. 2. Cambridge
University Press, New York/Cambridge, 1999.

\bibitem{SW}
D. Stanton and D. White, Constructive Combinatorics, Springer, New York (1986).


\end{thebibliography}





\end{document} 
