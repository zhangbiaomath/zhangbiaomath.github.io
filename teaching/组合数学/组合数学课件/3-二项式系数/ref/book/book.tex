\documentclass[a4paper,11pt]{article}
\usepackage{ctex}



\usepackage{amsmath,amssymb}             % AMS Math
\usepackage[T1]{fontenc}



\usepackage{graphicx}
% \usepackage{epstopdf}
\usepackage{tikz}
\usepackage[left=1.5in,right=1.3in,top=1.1in,bottom=1.1in,includefoot,includehead,headheight=13.6pt]{geometry}
\renewcommand{\baselinestretch}{1.05}



\usepackage{minitoc}
\newtheorem{thm}{定理}[section]
\newtheorem{prop}[thm]{命题}
\newtheorem{coro}[thm]{推论}
\newtheorem{defi}[thm]{定义}
\newtheorem{lem}[thm]{引理}
\newtheorem{exa}[thm]{例}
\newtheorem{ex}[thm]{习题}
\newtheorem{conj}[thm]{猜想}

\def\qed{\nopagebreak\hfill{\rule{4pt}{7pt}}\medbreak}
\def\pf{{\bf 证明~~ }}
\def \sg{\sigma}
\def \asc{\mathrm{asc}}
\def \des{\mathrm{des}}
\def \fix{\mathrm{fix}}
\def \lef{\mathrm{lef}}
\def \one{\mathrm{one}}
\def \Des{\mathrm{Des}}
\def \maj{\mathrm{maj}}
\def \exc{\mathrm{exc}}
\def \inv{\mathrm{inv}}
\def \roots{\mathrm{roots}}
\def \sgn{\mathrm{sgn}}
% Table of contents for each chapter

\usepackage{color}
\definecolor{linkcol}{rgb}{0,0,0.4}
\definecolor{citecol}{rgb}{0.5,0,0}

  \usepackage{graphicx}
  \DeclareGraphicsExtensions{.eps}
  \usepackage[a4paper,pagebackref,hyperindex=true,pdfnewwindow=true]{hyperref}

% \usepackage{chapterbib}
\begin{document}




%\pagenumbering{roman}
%\cleardoublepage
%\dominitoc
% \tableofcontents
%\mainmatter
%

%\chapter{基本组合数}
%\label{zuhejiegou} \minitoc



%%%%%%%%%%%%%%%%%%%%%%%%%%%%%%%%%%%%%%%%%%%%%%%%%%%%%%%%%%%%%%%%%%%%%%%%%%%%%%%%%%
\section{二项式系数}

二项式系数${n \choose
k}$计数了$n$元集合的$k$元子集的个数,在组合中具有十分重要的作用。
它的很多好的性质表现在各个组合恒等式中,
例如二项式定理,也正是源于此,它得名二项式系数。在本章中,我们将讨
论一下关于它的一些基本性质和恒等式的证明。

\subsection{帕斯卡(Pascal)公式}
对于非负整数$n$, $k$, 二项式系数${n \choose
k}$计数了$n$元集合的$k$元子集的个数,于是若 $k>n$, 则${n \choose
k}=0$,且对任意的$n$,均有${n \choose 0}=1$,若$n>0$,$1\leq k \leq
n$,则
\begin{equation}\label{e1}
{n \choose k}=\frac{n!}{k!(n-k)!}=\frac{n(n-1)\cdots
(n-k+1)}{k(k-1)\cdots 1}.
\end{equation}
\begin{thm}
(帕斯卡公式). 对于所有满足$1\leq k\leq n-1$ 的整数 $n$及$k$,均有
\begin{equation}\label{e2}
{n\choose k}={n-1 \choose k}+{n-1\choose k-1}.
\end{equation}
\end{thm}
\noindent{\textbf{证明}:}
首先可直接利用公式(\ref{e1})展开等式的两边即可得到,读者可以自己验算一下。
下面我们介绍一种组合方法,设$S$是一个$n$元集合,$x$为其中的一个元素,下面我们将集合$S$的$k(k\leq
n-1)$元 子集$X$分两种情况进行讨论。

情形1: $x\in
X$,则还需在除$x$外的$n-1$元集合中取$k-1$元子集作为$X$中的元素,共有${n-1
\choose k-1}$种方法;

情形2:$x\overline{\in}
X$,则需在除$x$外的$n-1$元集合中取$k$元子集作为$X$中的元素,共有${n-1
\choose k}$种方法。

从而,由加法原理,$S$ 的$k(k\leq n-1)$元子集 $X$ 共有${n-1 \choose
k-1}+{n-1 \choose k}$个,即
$${n\choose k}={n-1 \choose k}+{n-1\choose k-1}.$$
\qed
\begin{exa}
令$n=4$,$k=3$,$S=\{x,a,b,c\}$,于是属于情形$1$的$3$元子集为$$\{x,a,b\},\
\{x,a,c\},\
\{x,b,c\}.$$此可视为集合$\{a,b,c\}$的$2$元子集。属于情形$2$的$3$元子集为$$\{a,b,c\}.$$
此可视为集合集合$\{a,b,c\}$的$3$元子集。从而$${4 \choose
3}=4=3+1={3\choose 2}+{3\choose 3}.$$
\end{exa}

由递推公式(\ref{e2}),及初始条件$${n \choose 0}=1,\ {n\choose n}=1,
\ \ \ \ \ \ (n\geq
0)$$我们不需要利用公式(\ref{e1})即可计算出二项式系数。由此种方法,
我们在计算二项式系数的过程中经常以帕斯卡三角的形式显示出来,如下图所示:

\begin{figure}[ht] \begin{picture}(17,38)(-200,0)
\setlength{\unitlength}{0.4cm}
\put(-14.5,0){\line(1,0){27}}\put(-14.5,1.5){\line(1,0){27}}
\put(-14.5,-16.5){\line(1,0){27}} \put(-12,1.5){\line(0,-1){18}}
\put(-10,1.5){\line(0,-1){18}} \put(-8,1.5){\line(0,-1){18}}
\put(-6,1.5){\line(0,-1){18}} \put(-4,1.5){\line(0,-1){18}}
\put(-2,1.5){\line(0,-1){18}} \put(0,1.5){\line(0,-1){18}}
\put(2,1.5){\line(0,-1){18}} \put(4,1.5){\line(0,-1){18}}
\put(6,1.5){\line(0,-1){18}} \put(8,1.5){\line(0,-1){18}}
\put(-14,0.2){$n/k$}\put(-11.3,0.2){$0$}\put(-9.3,0.2){$1$}\put(-7.3,0.2){$2$}\put(-5.3,0.2){$3$}\put(-3.3,0.2){$4$}
\put(-1.3,0.2){$5$}\put(0.7,0.2){$6$}\put(2.7,0.2){$7$}\put(4.7,0.2){$8$}\put(6.7,0.2){$9$}\put(9.7,0.2){$\cdots$}
\put(-13.3,-1.3){$0$}\put(-11.3,-1.3){$1$}
\put(-13.3,-2.8){$1$}\put(-11.3,-2.8){$1$}\put(-9.3,-2.8){$1$}
\put(-13.3,-4.3){$2$}\put(-11.3,-4.3){$1$}\put(-9.3,-4.3){$2$}\put(-7.3,-4.3){$1$}
\put(-13.3,-5.8){$3$}\put(-11.3,-5.8){$1$}\put(-9.3,-5.8){$3$}\put(-7.3,-5.8){$3$}\put(-5.3,-5.8){$1$}
\put(-13.3,-7.3){$4$}\put(-11.3,-7.3){$1$}\put(-9.3,-7.3){$4$}\put(-7.3,-7.3){$6$}\put(-5.3,-7.3){$4$}
\put(-3.3,-7.3){$1$}
\put(-13.3,-8.8){$5$}\put(-11.3,-8.8){$1$}\put(-9.3,-8.8){$5$}\put(-7.3,-8.8){$10$}\put(-5.3,-8.8){$10$}
\put(-3.3,-8.8){$5$}\put(-1.3,-8.8){$1$}
\put(-13.3,-10.3){$6$}\put(-11.3,-10.3){$1$}\put(-9.3,-10.3){$6$}\put(-7.3,-10.3){$15$}\put(-5.3,-10.3){$20$}
\put(-3.3,-10.3){$15$}\put(-1.3,-10.3){$6$}\put(0.7,-10.3){$1$}

\put(-13.3,-11.8){$7$}\put(-11.3,-11.8){$1$}\put(-9.3,-11.8){$7$}\put(-7.3,-11.8){$21$}\put(-5.3,-11.8){$35$}
\put(-3.3,-11.8){$35$}\put(-1.3,-11.8){$21$}\put(0.7,-11.8){$7$}\put(2.7,-11.8){$1$}
\put(-13.3,-13.3){$8$}\put(-11.3,-13.3){$1$}\put(-9.3,-13.3){$8$}\put(-7.3,-13.3){$28$}\put(-5.3,-13.3){$56$}
\put(-3.3,-13.3){$70$}\put(-1.3,-13.3){$56$}\put(0.7,-13.3){$28$}\put(2.7,-13.3){$8$}\put(4.7,-13.3){$1$}
\put(-13.2,-15.3){$\vdots$}\put(-11.2,-15.3){$\vdots$}\put(-9.2,-15.3){$\vdots$}\put(-7.2,-15.3){$\vdots$}
\put(-5.2,-15.3){$\vdots$}\put(-3.2,-15.3){$\vdots$}\put(-1.2,-15.3){$\vdots$}\put(0.8,-15.3){$\vdots$}
\put(2.8,-15.3){$\vdots$}\put(4.8,-15.3){$\vdots$}\put(6.8,-15.3){$\vdots$}
\end{picture}
\vspace{6.5cm} \caption{帕斯卡三角.} \label{pascal}
\end{figure}

图中除了最左侧一列的$1$以外,其余的值都可以通过上行中同列及其左邻的值之和。例如,对$n=6$,我们有
$${6 \choose 3}=20=10+10={5 \choose 3}+{5 \choose 2}.$$

二项式系数的许多性质和恒等式均可以通过帕斯卡三角得到,将帕斯卡三角某行的元素加起来可发现,
$${n \choose 0}+{n\choose 1}+\cdots+{n\choose n}=2^n.$$

下面给出帕斯卡三角的另一种组合解释,令$n$,$k$为满足$0\leq k\leq
n$的非负整数,定义$p(n,k)$计数了帕斯卡三角中左上角(数值${0 \choose
0}=1$)到数值${n\choose
k}$的路的条数,其中路的每一步均为向南走一个单位或向东南方向走一个单位,即按向量方向$(0,-1)$和$(1,-1)$走一步。
\begin{figure}[ht]
\begin{picture}(20,20)(20,0)
\setlength{\unitlength}{0.5cm} \thicklines
\put(10,0){\line(0,-1){3}} \put(10,0){\circle*{0.2}}
\put(10,-3){\circle*{0.2}} \put(10,-1,4){\vector(0,-1){0.4}}
\put(18,0){\circle*{0.2}}\put(21,-3){\circle*{0.2}}\put(21,0){\circle*{0.2}}
\put(18,0){\line(1,-1){3}} \put(19.4,-1,4){\vector(1,-1){0.4}}
\end{picture}
\vspace{2cm} \caption{步子}\label{path}
\end{figure}

我们约定$p(0,0)=1$,且对任意的非负整数$n$,均有
$p(n,0)=1$,(每一步都必须朝下走一直到${n\choose
0}$)及$p(n,n)=1$。(每一步都不需沿对角线走直到${n\choose n}$)
注意到每一条从${0\choose 0}$到${n\choose k}$的路均可看为是

(i)从${0\choose 0}$到${n-1\choose k}$的路再加上一个竖直步子,

或

(ii)从${0\choose 0}$到${n-1\choose k-1}$的路再加上一个对角步子。

从而,由加法原理,我们有递推关系
$$p(n,k)=p(n-1,k)+p(n-1,k-1).$$
观察到$p(n,k)$与二项式系数有相同的初始条件及递推关系,故而易知对任意满足$0\leq
k\leq n$的非负整数$n$,$k$,有
$$p(n,k)={n\choose k}.$$
于是帕斯卡三角的各数值也表示从左上角到该数值的路的条数。这也给了二项式系数以新的组合解释。


%%%%%%%%%%%%%%%%%%%%%%%%%%%%%%%%%%%%%%%%%%%%%%%%%%
\subsection{二项式定理}
二项式系数是从二项式定理中得名的,本节中将介绍有关二项式定理的恒等式,它作为代数恒等式我们在高中就已经接触过。
\begin{thm}
设$n$是正整数,则对任意的实数$x$,$y$,均有
\begin{equation}\label{e3}
(x+y)^n=\sum_{k=0}^n{n\choose k}x^{n-k}y^k.
\end{equation}
\end{thm}
\noindent{\textbf{证明}:}(方法1)将$(x+y)^n$写成$n$个因子的乘积$$(x+y)(x+y)\cdots(x+y).$$
由乘法的分配律,展开乘积并合并同类项,由于每一个因子$(x+y)$,我们都有$x$,$y$两种选择,所以展开式中共有$2^n$项,
且每项都是$x^{n-k}y^k$($k=0,1,2,\ldots,n$)的形式。项$x^{n-k}y^k$是在$n$个因子中$k$个选取$y$,其余$n-k$个因子中取$x$,
于是$x^{n-k}y^k$在展开式中出现的次数为$x^{n-k}y^k$,即展开式中项$x^{n-k}y^k$的系数为${n\choose
k}$,从而
$$(x+y)^n=\sum_{k=0}^n{n\choose k}x^{n-k}y^k.$$
\qed
\noindent{\textbf{证明}:}(方法2)对$n$进行数学归纳法。对$n=1$,式(\ref{e3})为$$(x+y)^1=\sum_{k=0}^1{1\choose
k}x^{1-k}y^k=x+y,$$
显然成立。假设式(\ref{e3})对整数$n$成立,即$(x+y)^n=\sum_{k=0}^n{n\choose
k}x^{n-k}y^k$,下考虑$n+1$的情形,
\begin{align*}
(x+y)^{n+1}&=(x+y)(x+y)^n=(x+y)\left(\sum_{k=0}^n{n\choose k}x^{n-k}y^k\right)\\
&=x\left(\sum_{k=0}^n{n\choose k}x^{n-k}y^k\right)+y\left(\sum_{k=0}^n{n\choose k}x^{n-k}y^k\right)\\
&=\sum_{k=0}^n{n\choose k}x^{n+1-k}y^k+\sum_{k=0}^n{n\choose k}x^{n-k}y^{k+1}\\
&={n\choose 0}x^{n+1}+\sum_{k=1}^n{n\choose
k}x^{n+1-k}y^{k}+\sum_{k=0}^{n-1}{n\choose
k}x^{n-k}y^{k+1}+{n\choose n}y^{n+1}
\end{align*}
在后一个连加项中,用$k-1$代替$k$,得到
$$\sum_{k=1}^n{n\choose k-1}x^{n-k+1}y^k.$$
从而$$(x+y)^{n+1}=x^{n+1}+\sum_{k=1}^n\left[{n\choose k}+{n\choose
k-1}\right]x^{n+1-k}y^k+y^{n+1},$$
利用帕斯卡公式,上式等价于$$(x+y)^{n+1}=x^{n+1}+\sum_{k=1}^n{n+1\choose
k}x^{n+1-k}y^k+y^{n+1}=\sum_{k=0}^{n+1}{n+1\choose
k}x^{n+1-k}y^k,$$恰满足式(\ref{e3}),由归纳法原理定理得证。 \qed

一般情况下,我们常常利用到一种特殊情形,即当$y=1$时,我们有如下推论。
\begin{coro}
令$n$是正整数,则对任意的实数$x$均有$$(1+x)^n=\sum_{k=0}^n{n\choose
k}x^k=\sum_{k=0}^n{n\choose n-k}x^k.$$
\end{coro}

%%%%%%%%%%%%%%%%%%%%%%%%%%%%%%%%%%%%%%%%%%%%%%%
\subsection{恒等式}
本节中,我们将考虑一些关于二项式系数的恒等式并给出它们的组合解释。首先由二项式系数的代数展开式,很快地,我们有
\begin{equation}\label{e4}
k{n\choose k}=n{n-1 \choose k-1}.
\end{equation}
作为代数式,我们很容易根据二项式的表达式证明,但是作为组合恒等式,式(\ref{e4})又有怎么样的组合意义呢?

我们考虑这样一个实际问题,从$n$个人中选出$k$个人组成一个足球队,并选出队长,完成这项事件共有多少种选择方案?
一种计数方法是我们先选出足球队,则有${n\choose
k}$种,再在这选出的$k$个人中选出队长,有$k$种,于是由乘法原理完成这项事件共有$k{n\choose
k}$种选择方案。
另一种计数方法是先从$n$个人中选出队长,有$n$种选择,再在剩下的$n-1$个人中选出$k-1$个人与队长一起组成足球队,有${n-1\choose
k-1}$种选择,于是由乘法原理完成这项事件共有$n{n-1 \choose
k-1}$种选择方案。
于是式(\ref{e4})两边计数的是同一事件的选择方案,故而等式成立。

在二项式定理中若同时取$x=1$,$y=1$,则可得到恒等式
\begin{equation}\label{e5}
{n\choose 0}+{n\choose 1}+\cdots+{n\choose n}=2^n.
\end{equation}
若取$x=1$,$y=-1$,则可得到恒等式
$${n\choose 0}-{n\choose 1}+{n\choose 2}-\cdots+(-1)^n{n\choose n}=0,\ \ \ (n\geq 1).
$$
也即
\begin{equation}\label{e6}
{n\choose 0}+{n\choose 2}+\cdots={n\choose 1}+{n\choose 3}+\cdots,\
\ \ (n\geq 1).
\end{equation}
由式(\ref{e5}),要证明式(\ref{e6}),只需证明
$${n\choose 0}+{n\choose 2}+\cdots=2^{n-1}.$$
下面我们将给出该式的组合解释。令$S={x_1,x_2,\ldots,x_n}$为$n$元集合,我们需要计数的是$S$的偶子集$X$的个数,
我们对$S$中元素逐个考虑,首先考虑$x_1$,我们有放不放入$X$中两种选择,考虑$x_2$,我们也有放不放入$X$中两种选择,接着
考虑$x_3$,一直到$x_{n-1}$,均有放不放入$X$中两种选择,最后对于$x_n$,若前面放入$X$的元素个数为奇,则将$x_n$放入$X$中,
否则将$x_n$不放入$X$中。所以在整个事件中,前$n-1$步均有两种选择,最后一步只有一种选择,于是由乘法原理,
$S$的偶子集的个数为$2^{n-1}$,又${n\choose 0}+{n\choose
2}+\cdots$也计数了$n$元集合的偶子集的个数,故而
$${n\choose 0}+{n\choose 2}+\cdots=2^{n-1}.$$
\qed

同样的道理,我们可以证明$${n\choose 1}+{n\choose
3}+\cdots=2^{n-1}.$$

利用恒等式(\ref{e4})和(\ref{e5}),我们可得到
\begin{equation}\label{e7}
1{n\choose 1}+2{n\choose 2}+\cdots+n{n \choose n}=n2^n,\ \ (n\geq
1).
\end{equation}
式(\ref{e7})也可以根据二项式定理而得到,在二项式公式
$$(1+x)^n={n\choose 0}+{n\choose 1}x+{n\choose 2}x^2+\cdots+{n\choose n}x^n$$
两边同时对$x$求导,我们有
$$n(1+x)^{n-1}={n\choose 1}+2{n\choose 2}x+\cdots+n{n\choose n}x^{n-1},$$
最后,令$x=1$,即可得到式(\ref{e7})。








\end{document} 
