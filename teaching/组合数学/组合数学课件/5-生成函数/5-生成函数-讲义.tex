\documentclass{report}
\usepackage{amsthm}
\usepackage{amsmath}
\usepackage{newtxmath}
\usepackage{ctex} %显示中文的宏包
\usepackage{graphicx}
\usepackage{geometry}
%\numberwithin{equation}
\geometry{left=3cm,right=3cm,top=3cm,bottom=2cm}

\linespread{1.5}
%\renewcommand\theequation{\thechapter-\arabic{equation}}


\newcommand{\e}{\mathrm{e}}


\begin{document}
\setcounter{chapter}{4}

\chapter{生成函数}

\section{引论}
生成函数是一种既简单又有用的数学方法,它最早出现于19世纪初.  对于组合计数问题,生成函数是一种最重要的一般性处理方法. 它的中心思想是:对于一 个有限或无限数列
用幂级数
$$
\left\{a_{0}, a_{1}, a_{2}, \cdots\right\}
$$
$$
A(x)=a_{0}+a_{1} x+a_{2} x^{2}+\cdots
$$
使之成为一个整体,然后通过研究幂级数 $A(x)$, 导出数列 $\left\{a_{0}, a_{1}, a_{2}, \cdots\right\}$ 的构造 和性质.我们称 $A(x)$ 为序列 $\left\{a_{0}, a_{1}, a_{2}, \cdots\right\}$ 的生成函数,并记为 $G\left\{a_{n}\right\}$.
实际上,在第 3 章中我们已经使用过生成函数方法. 组合数序列
$$
\left(\begin{array}{l}
n \\ 0
\end{array}\right),\left(\begin{array}{l}
n \\ 1
\end{array}\right), \cdots,\left(\begin{array}{l}
n \\ n
\end{array}\right)
$$
的生成函数为
$$
f_{n}(x)=\left(\begin{array}{l}
n \\ 0
\end{array}\right)+\left(\begin{array}{l}
n \\ 1
\end{array}\right) x+\left(\begin{array}{l}
n \\ 2
\end{array}\right) x^{2}+\cdots+\left(\begin{array}{l}
n \\ n
\end{array}\right) x^{n}
$$
由二项式定理知
$$
f_{n}(x)=(1+x)^{n}
$$
通过对 $(1+x)^{n}$ 的运算,可以导出一系列组合数的关系式,例如
$$
\sum_{i=0}^{n}\left(\begin{array}{l}
n \\ i
\end{array}\right)=2^{n}$$

$$
\sum_{i=1}^{n} i\left(\begin{array}{l}
n \\ i
\end{array}\right) = n \cdot 2^{n-1}
$$
等等.
由恒等式
$$
(1+x)^{m+n}=(1+x)^{m}(1+x)^{n}
$$
可以推导出 Vandermonde 恒等式
$$
\left(\begin{array}{c}
m+n \\ r
\end{array}\right)=\sum_{k=0}^{r}\left(\begin{array}{c}
m \\ k
\end{array}\right)\left(\begin{array}{c}
n \\ r-k
\end{array}\right)
$$
下面再看一个例子.

例 $\mathbf{1}$ 投掷一次骰子,出现点数 $1,2, \cdots, 6$ 的概率均为 $\frac{1}{6} .$ 问连续投掷两次,出 现的点数之和为 10 的概率有多少?连续投掷 10 次,出现的点数之和为 30 的概率又 是多少?
解一次投掷出现的点数有 6 种可能. 连续两次投掷得到的点数构成二元数 组 $(i, j)(1 \leqslant i, j \leqslant 6)$, 共有 $6^{2}=36$ 种可能.由枚举法,两次出现的点数之和为 10 的有 3 种可能: $(4,6),(5,5),(6,4)$, 所以概率为 $\frac{3}{36}=\frac{1}{12}$.
如果问题是连续投掷 10 次,其点数之和为 30 的概率有多少,这时就不那么简 单了.这是由于 10 个数之和为 30 的可能组合方式很多,难以一一列举,要解决这个 问题,只能另辟新径.
我们用多项式
$$
x+x^{2}+x^{3}+x^{4}+x^{5}+x^{6}
$$
表示投掷一次可能出现点数 $1,2, \cdots, 6$, 观察
$$
\left(x+x^{2}+x^{3}+x^{4}+x^{5}+x^{6}\right)\left(x+x^{2}+x^{3}+x^{4}+x^{5}+x^{6}\right)
$$
从两个括号中分别取出 $x^{m}$ 和 $x^{n}$,使
$$
x^{m} \cdot x^{n}=x^{10}
$$
即是两次投掷分别出现点数 $m, n$, 且 $m+n=10 .$ 由此得出,展开式中 $x^{10}$ 的系数 就是满足条件的方法数.
同理,连续投掷 10 次,其和为 30 的方法数为
$$
\left(x+x^{2}+x^{3}+x^{4}+x^{5}+x^{6}\right)^{10}
$$
中 $x^{30}$ 的系数.而
$$
\begin{array}{l}
\left(x+x^{2}+x^{3}+x^{4}+x^{5}+x^{6}\right)^{10} \\
\quad=x^{10}\left(1-x^{6}\right)^{10}(1-x)^{-10} \\
\quad=x^{10} \cdot \sum_{i=0}^{10}(-1)^{i}\left(\begin{array}{c}
10 \\ i
\end{array}\right) x^{6 i} \cdot \sum_{i=0}^{\infty}\left(\begin{array}{c}
10-1+i \\ i
\end{array}\right) x^{i}
\end{array}
$$

所以, $x^{30}$ 的系数为
$$
\left(\begin{array}{c}
29 \\ 20
\end{array}\right)-\left(\begin{array}{c}
23 \\ 14
\end{array}\right)\left(\begin{array}{c}
10 \\ 1
\end{array}\right)+\left(\begin{array}{c}
17 \\ 8
\end{array}\right)\left(\begin{array}{c}
10 \\ 2
\end{array}\right)-\left(\begin{array}{c}
11 \\ 2
\end{array}\right)\left(\begin{array}{c}
10 \\ 3
\end{array}\right)=2930455
$$
故所求概率为
$$
\frac{2930455}{6^{10}} \approx 0.0485
$$

\section{形式幂级数}
数列
$$
\left\{a_{0}, a_{1}, a_{2}, \cdots\right\}
$$
的生成函数是幂级数
$$
A(x)=a_{0}+a_{1} x+a_{2} x^{2}+\cdots
$$
由于只有收敛的幂级数才有解析意义,并可以作为函数进行各种运算,这样就有了 级数收敛性的问题. 为了解决这个问题,我们从代数的观点引入形式幂级数的 概念.
我们称幂级数 $(5.2 .2)$ 是形式幂级数,其中的 $x$ 是末定元,看作是抽象符号.对 于实数域 $\mathbf{R}$ 上的数列
$$
\left\{a_{0}, a_{1}, a_{2}, \cdots\right\}
$$
$x$ 是 $\mathbf{R}$ 上的末定元,表达式
$$
A(x)=a_{0}+a_{1} x+a_{2} x^{2}+\cdots
$$
称为 $\mathbf{R}$ 上的形式幂级数.
一般情况下,形式幂级数中的 $x$ 只是一个抽象符号,并不需要对 $x$ 赋予具体数 值,因而就不需要考虑它的收敛性.
$\mathbf{R}$ 上的形式幂级数的全体记为 $\mathbf{R}[[x]] .$ 在集合 $\mathbf{R}[[x]]$ 中适当定义加法和乘 法运算,便可使它成为一个整环,任何一个形式幂级数都是这个环中的元素.
\noindent
\textbf{定义\ 5.2.1}\textsl {设 $A(x)=\sum_{k=0}^{\infty} a_{k} x^{k}$ 与 $B(x)=\sum_{k=0}^{\infty} b_{k} x^{k}$ 是 $\mathbf{R}$ 上的两个形式幂 级数,若对任意 $k \geqslant 0$,有 $a_{k}=b_{k}$, 则称 $A(x)$ 与 $B(x)$ 相等,记作 $A(x)=B(x)$.}

\noindent
\textbf{定义\ 5.2.2}\textsl {设 $\alpha$ 为任意实数, $A(x)=\sum_{k=0}^{\infty} a_{k} x^{k} \in \mathbf{R}[[x]]$, 则将
$$
\alpha A(x) \equiv \sum_{k=0}^{\infty}\left(\alpha a_{k}\right) x^{k}
$$
叫作 $\alpha$ 与 $A(x)$ 的数乘积. }

\noindent
\textbf{定义\ 5.2.3}\textsl{ 设 $A(x)=\sum_{k=0}^{\infty} a_{k} x^{k}$ 与 $B(x)=\sum_{k=0}^{\infty} b_{k} x^{k}$ 是 $\mathbf{R}$ 上的两个形式幂 级数,将 $A(x)$ 与 $B(x)$ 相加定义为
$$
A(x)+B(x) \equiv \sum_{k=0}^{\infty}\left(a_{k}+b_{k}\right) x^{k}
$$
并称 $A(x)+B(x)$ 为 $A(x)$ 与 $B(x)$ 的和,把运算“+” 叫作加法.
将 $A(x)$ 与 $B(x)$ 相乘定义为
$$
A(x) \cdot B(x) \equiv \sum_{k=0}^{\infty}\left(a_{k} b_{0}+a_{k-1} b_{1}+\cdots+a_{0} b_{k}\right) x^{k}
$$
并称 $A(x) \cdot B(x)$ 为 $A(x)$ 和 $B(x)$ 的积,把运算$“\cdot”$ 叫作乘法. }


\noindent
\textbf{定理\ 5.2.1} \textsl{集合 $\mathbf{R}[[x]]$ 在上述加法和乘法运算下构成一个整环. }

\noindent
\textbf{定理\ 5.2.2} \textsl{对 $\mathrm{R}[[x]]$ 中的任意一个元素 $A(x)=\sum_{k=0}^{\infty} a_{k} x^{k}, A(x)$ 有乘法逆 元当且仅当 $a_{0} \neq 0$.若 $\widetilde{A}(x)=\sum_{k=0}^{\infty} \tilde{a}_{k} x^{k}$ 是 $A(x)$ 的乘法逆元,则有
$$
\begin{array}{l}
\tilde{a}_{0}=a_{0}{ }^{-1}, \\
\tilde{a}_{k}=(-1)^{k} a_{0}^{-(k+1)}\left|\begin{array}{cccccc}
a_{1} & a_{2} & a_{3} & \cdots & a_{k-1} & a_{k} \\
a_{0} & a_{1} & a_{2} & \cdots & a_{k-2} & a_{k-1} \\
0 & a_{0} & a_{1} & \cdots & a_{k-3} & a_{k-2} \\
\vdots & \vdots & \vdots & \vdots & \vdots & \vdots \\
0 & 0 & 0 & \cdots & a_{1} & a_{2} \\
0 & 0 & 0 & \cdots & a_{0} & a_{1}
\end{array}\right| \quad(k \geqslant 1) .
\end{array}
$$ }

在整环 $\mathbf{R}[[x]]$ 上还可以定义形式导数.

\noindent
\textbf{定理\ 5.2.2} \textsl{对于任意 $A(x)=\sum_{k=0}^{\infty} a_{k} x^{k} \in \mathbf{R}[[x]]$,规定$$
\mathrm{D} A(x) \equiv \sum_{k=1}^{\infty} k a_{k} x^{k-1}
$$
称 $\mathrm{D} A(x)$ 为 $A(x)$ 的形式导数.  }

$A(x)$ 的 $n$ 次形式导数可以递归地定义为
$$
\left\{\begin{array}{l}
\mathrm{D}^{0} A(x) \equiv A(x) \\
\mathrm{D}^{n} A(x) \equiv \mathrm{D}\left[\mathrm{D}^{n-1} A(x)\right] \quad(n \geqslant 1)
\end{array}\right.
$$
形式导数满足如下规则:

(1) $\mathrm{D}[\alpha A(x)+\beta B(x)]=\alpha \mathrm{D} A(x)+\beta \mathrm{D} B(x)$

(2) $\mathrm{D}[A(x) \cdot B(x)]=A(x) \mathrm{D} B(x)+B(x) \mathrm{D} A(x)$

(3) $\mathrm{D}\left[A^{n}(x)\right]=n A^{n-1}(x) \mathrm{D} A(x)$

证明 规则(1) 由定义可以直接得出,而规则 (3) 则是规则 $(2)$ 的推论. 现证明规则 $(2)$. 显然有
$$
\begin{aligned}
\mathrm{D}[A(x) \cdot B(x)] &=\mathrm{D} \sum_{k=0}^{\infty}\left(\sum_{i+j=k} a_{i} b_{j}\right) x^{k}=\sum_{k=1}^{\infty} k\left(\sum_{i+j=k} a_{i} b_{j}\right) x^{k-1} \\
&=\sum_{k=1}^{\infty} \sum_{i+j=k}(i+j) a_{i} b_{j} x^{i+j-1} \\
&=\sum_{k=1}^{\infty} \sum_{i+j=k}\left(i a_{i} x^{i-1}\right) b_{j} x^{j}+\sum_{k=1}^{\infty} \sum_{i+j=k}\left(a_{i} x^{i}\right)\left(j b_{j} x^{j-1}\right) \\
&=\left(\sum_{i=1}^{\infty} i a_{i} x^{i-1}\right)\left(\sum_{j=0}^{\infty} b_{j} x^{j}\right)+\left(\sum_{i=0}^{\infty} a_{i} x^{i}\right)\left(\sum_{j=1}^{\infty} j b_{j} x^{j-1}\right) \\
&=A(x) \mathrm{D} B(x)+B(x) \operatorname{D} A(x) .
\end{aligned}
$$
由此可知,形式导数满足微积分中求导运算的规则, 当某个形式幂级数在某个 范围内收敛时,形式导数就是微积分中的求导运算.为了书写方便, 以后用 $A^{\prime}(x)$, $A^{\prime \prime}(x), \cdots$ 分别代表 $\mathrm{D} A(x), \mathrm{D}^{(2)} A(x), \cdots .$

\section{生成函数的性质}
生成函数与数列之间是一一对应的. 因此, 若两个生成函数之间存在某种关 系,那么相应的两个数列之间也必然存在一定的关系;反之亦然.
设数列 $\left\{a_{0}, a_{1}, a_{2}, \cdots\right\}$ 的生成函数为 $A(x)=\sum_{k=0}^{\infty} a_{k} x^{k}$, 数列 $\left\{b_{0}, b_{1}, b_{2}, \cdots\right\}$
的生成函数为 $B(x)=\sum_{k=0}^{\infty} b_{k} x^{k}$, 我们可以得到生成函数的如下一些性质:

\noindent
\textbf{性质\ 1} \textsl{若
$$
b_{k}=\left\{\begin{array}{ll}
0 & (k<l) \\ a_{k-1} & (k \geqslant l)
\end{array}\right.
$$
则
$$
B(x)=x^{l} \cdot A(x)
$$
证明:}由假设条件,有
$$
\begin{aligned}
B(x) &=\sum_{k=0}^{\infty} b_{k} x^{k} \\
&=a_{0} \cdot x^{l}+a_{1} \cdot x^{l+1}+\cdots+a_{n} \cdot x^{l+n}+\cdots \\
&=x^{l} \cdot\left(a_{0}+a_{1} x+\cdots+a_{n} x^{m}+\cdots\right) \\
&=x^{l} \cdot A(x)
\end{aligned}
$$

\noindent
\textbf{性质\ 2} \textsl{若 $b_{k}=a_{k+l}$, 则
$$
B(x)=\frac{1}{x^{i}}\left[A(x)-\sum_{k=0}^{l-1} a_{k} x^{k}\right]
$$
}
\textsl{证明:}类似于性质 1 的证明.

\noindent
\textbf{性质\ 3} \textsl{若 $b_{k}=\sum_{i=0}^{k} a_{i}$, 则
$$
B(x)=\frac{A(x)}{1-x}
$$
证明:}由假设条件,有
$$
\begin{array}{l}
b_{0}=a_{0}, \\
b_{1} x=a_{0} x+a_{1} x \\
b_{2} x^{2}=a_{0} x^{2}+a_{1} x^{2}+a_{2} x^{2} \\
\cdots, \\
b_{k} x^{k}=a_{0} x^{k}+a_{1} x^{k}+a_{2} x^{k}+\cdots+a_{k} x^{k}, \\
\cdots .
\end{array}
$$
把以上各式的两边分别相加,得
$$
\begin{aligned}
B(x)=& a_{0}\left(1+x+x^{2}+\cdots\right)+a_{1} x\left(1+x+x^{2}+\cdots\right) \\
&+a_{2} x^{2}\left(1+x+x^{2}+\cdots\right)+\cdots \\
=&\left(a_{0}+a_{1} x+a_{2} x^{2}+\cdots\right)\left(1+x+x^{2}+\cdots\right) \\
=& \frac{A(x)}{1-x}
\end{aligned}
$$

\noindent
\textbf{性质\ 4} \textsl{若 $b_{k}=\sum_{i=k}^{\infty} a_{i}$, 则
$$
B(x)=\frac{A(1)-x A(x)}{1-x}
$$
这里, $\sum_{i=0}^{\infty} a_{i}$ 是收敛的. }

\noindent
\textsl{证明:}因为$A(1)=\sum_{k=0}^{\infty} a_{k}$ 收敛,所以 $b_{k}=\sum_{i=k}^{\infty} a_{i}$ 是存在的.于是有
$$
\begin{array}{l}
b_{0}=a_{0}+a_{1}+a_{2}+\cdots=A(1) \\
b_{1} x=a_{1} x+a_{2} x+\cdots=\left[A(1)-a_{0}\right] x \\
b_{2} x^{2}=a_{2} x^{2}+a_{3} x^{2}+\cdots=\left[A(1)-a_{0}-a_{1}\right] x^{2} \\
\cdots, \\
b_{k} x^{k}=a_{k} x^{k}+a_{k+1} x^{k}+\cdots=\left[A(1)-a_{0}-a_{1}-\cdots-a_{k-1}\right] x^{k} \\
\cdots
\end{array}
$$
把以上各式的两边分别相加,得
$$
\begin{aligned}
B(x)=& A(1)+\left[A(1)-a_{0}\right] x+\left[A(1)-a_{0}-a_{1}\right] x^{2}+\cdots \\
&+\left[A(1)-a_{0}-\cdots-a_{k-1}\right] x^{k}+\cdots \\
=& A(1)\left(1+x+x^{2}+\cdots\right)-a_{0} x\left(1+x+x^{2}+\cdots\right) \\
&-a_{1} x^{2}\left(1+x+x^{2}+\cdots\right)-\cdots-a_{n-1} x^{n}\left(1+x+x^{2}+\cdots\right)-\cdots \\
=&\left[A(1)-x\left(a_{0}+a_{1} x+a_{2} x^{2}+\cdots\right)\right] \cdot\left(1+x+x^{2}+\cdots\right) \\
=& \frac{A(1)-x A(x)}{1-x}
\end{aligned}
$$

\noindent
\textbf{性质\ 5} \textsl{若 $b_{k}=k a_{k}$, 则
$$
B(x)=x A^{\prime}(x)
$$}

\noindent
\textsl{证明:}由 $A^{\prime}(x)$ 的定义知
$$
x A^{\prime}(x)=x \sum_{k=1}^{\infty} k a_{k} x^{k-1}=\sum_{k=0}^{\infty} k a_{k} x^{k}=\sum_{k=0}^{\infty} b_{k} x^{k}=B(x)
$$


\noindent
\textbf{性质\ 6} \textsl{若 $b_{k}=\frac{a_{k}}{k+1}$, 则
$$
B(x)=\frac{1}{x} \int_{0}^{x} A(t) \mathrm{d} t
$$}
\textsl{证明:}由假设条件,有
$$
\int_{0}^{x} A(t) \mathrm{d} t=\sum_{k=0}^{\infty} \int_{0}^{x} a_{k} t^{k} \mathrm{~d} t=\sum_{k=0}^{\infty} \int_{0}^{x} b_{k}(k+1) t^{k} \mathrm{~d} t
$$
$$
=\sum_{k=0}^{\infty} b_{k} x^{k+1}=x \cdot B(x)
$$

\noindent
\textbf{性质\ 7} \textsl{若 $c_{k}=\alpha a_{k}+\beta b_{k}$,则
$$
C(x) \equiv \sum_{k=0}^{\infty} c_{k} x^{k}=\alpha A(x)+\beta B(x)
$$}

\noindent
\textbf{性质\ 8} \textsl{若 $c_{k}=a_{0} b_{k}+a_{1} b_{k-1}+\cdots+a_{k} b_{0}$, 则
$$
C(x) \equiv \sum_{k=0}^{\infty} c_{k} x^{k}=A(x) \cdot B(x)
$$}

性质 7 和性质 8 可由形式幂级数的数乘、加法及乘法运算的定义直接得出.

利用这些性质,我们可以求某些数列的生成函数,也可以计算数列的和. 下面 列出常见的几个数列的生成函数:

(1) $G\{1\}=\frac{1}{1-x} ;$

(2) $G\left\{a^{k}\right\}=\frac{1}{1-a x} ;$

(3) $G\{k\}=\frac{x}{(1-x)^{2}}$

(4) $G\{k(k+1)\}=\frac{2 x}{(1-x)^{3}}$

(5) $G\left\{k^{2}\right\}=\frac{x(1+x)}{(1-x)^{3}}$

(6) $G\{k(k+1)(k+2)\}=\frac{6 x}{(1-x)^{4}}$

(7) $G\left\{\frac{1}{k !}\right\}=\mathrm{e}^{x}$;

(8) $G\left\{\left(\begin{array}{l}\alpha \\ k\end{array}\right)\right\}=(1+x)^{\alpha}$

(9) $G\left\{\left(\begin{array}{c}n+k \\ k\end{array}\right)\right\}=\frac{1}{(1-x)^{n+1}}$

下面证明其中的几个生成函数,而生成函数(8) 和(9) 可参见定理 $3.1.2$ 及其 分析.

\textsl{证明:}(3)易知
$$
\begin{aligned}
G\{k\} &=\sum_{k=1}^{\infty} k x^{k}=x \sum_{k=1}^{\infty} k x^{k-1} \\
&=x\left(\sum_{k=0}^{\infty} x^{k}\right)^{\prime}=x\left(\frac{1}{1-x}\right)^{\prime}
\end{aligned}
$$
$$
=\frac{x}{(1-x)^{2}}
$$
(5) 易知
$$
\begin{aligned}
G\left\{k^{2}\right\} &=\sum_{k=1}^{\infty} k^{2} x^{k} \\
&=\sum_{k=1}^{\infty}(k+1) k x^{k}-\sum_{k=1}^{\infty} k x^{k} \\
&=\frac{2 x}{(1-x)^{3}}-\frac{x}{(1-x)^{2}} \\
&=\frac{x(1+x)}{(1-x)^{3}}
\end{aligned}
$$
(6) 设
$$
G\{k(k+1)(k+2)\}=A(x)
$$
则
$$
\begin{aligned}
\int_{0}^{x} t A(t) \mathrm{d} t &=\sum_{k=1}^{\infty} \int_{0}^{x} k(k+1)(k+2) t^{k+1} \mathrm{~d} t \\
&=\sum_{k=1}^{\infty} k(k+1) x^{k+2} \\
&=x^{2} \cdot \frac{2 x}{(1-x)^{3}}
\end{aligned}
$$
所以
$$
x A(x)=\left[\frac{2 x^{3}}{(1-x)^{3}}\right]^{\prime}=\frac{6 x^{2}}{(1-x)^{4}}
$$
故
$$
A(x)=\frac{6 x}{(1-x)^{4}}
$$
利用生成函数的性质,可以求出一些序列以及一些序列的和,下面的两个例子 说明了一些求解方法.

\noindent
\textbf{例\ 1}已知 $\left\{a_{n}\right\}$ 的生成函数为
$$
A(x)=\frac{2+3 x-6 x^{2}}{1-2 x}
$$
求 $a_{n}$.

\textbf{解  }用部分分式的方法得
$$
A(x)=\frac{2+3 x-6 x^{2}}{1-2 x}=\frac{2}{1-2 x}+3 x
$$
而
$$
\frac{2}{1-2 x}=2 \sum_{n=0}^{\infty} 2^{n} x^{n}=\sum_{n=0}^{\infty} 2^{n+1} x^{n}
$$
所以有
$$
a_{n}=\left\{\begin{array}{ll}
2^{n+1} & (n \neq 1) \\
2^{2}+3=7 & (n=1)
\end{array}\right.
$$

\textbf{例\ 2}计算级数
$$
1^{2}+2^{2}+\cdots+n^{2}
$$
的和.

\textbf{解  }由前面列出的第(5) 个数列的生成函数知,数列 $\left\{n^{2}\right\}$ 的生成函数为
$$
A(x)=\frac{x(1+x)}{(1-x)^{3}}=\sum_{k=0}^{\infty} a_{k} x^{k}
$$
此处, $a_{k}=k^{2}$. 令
$$
b_{n}=1^{2}+2^{2}+\cdots+n^{2}
$$
则
$$
b_{n}=\sum_{k=1}^{n} a_{k}
$$
由性质 3 即得数列 $\left\{b_{n}\right\}$ 的生成函数为
$$
\begin{aligned}
B(x) &=\sum_{n=0}^{\infty} b_{n} x^{n}=\frac{A(x)}{1-x}=\frac{x(1+x)}{(1-x)^{4}} \\
&=\left(x+x^{2}\right) \sum_{k=0}^{\infty}\left(\begin{array}{c}
k+3 \\ k
\end{array}\right) x^{k}
\end{aligned}
$$
比较等式两边 $x^{n}$ 的系数,便得
$$
\begin{aligned}
1^{2}+2^{2}+\cdots+n^{2} &=b_{n} \quad\left(\begin{array}{c}
n+2 \\ 3
\end{array}\right)+\left(\begin{array}{c}
n-1 \\ 3
\end{array}\right) \\
&=\left(\begin{array}{l}
n+2 \\ n-1
\end{array}\right)+\left(\begin{array}{l}
n+1 \\ n-2
\end{array}\right) \\
&=\frac{n(n+1)(2 n+1)}{6}
\end{aligned}
$$

\section{组合型分配问题的生成函数}
本节介绍组合数序列的生成函数, 进而介绍如何用生成函数来求解组合型分配问题.

\textbf{1.4.1}组合数的生成函数

我们在前面几章中讨论过三种不同类型的组合问题:


(1) 求 $\left\{a_{1}, a_{2}, \cdots, a_{n}\right\}$ 的 $k$ 组合数;

(2) 求 $\left\{\infty \cdot a_{1}, \infty \cdot a_{2}, \cdots, \infty \cdot a_{n}\right\}$ 的 $k$ 组合数;

(3) 求 $\{3 \cdot a, 4 \cdot b, 5 \cdot c\}$ 的 10 组合数.

其中,问题(1) 是普通集合的组合问题; 问题 $(2)$ 转化为不定方程 $x_{1}+x_{2}+\cdots+x_{n}$ $=k$ 的非负整数解的个数问题;问题(3) 是利用容斥原理在 $M=\{\infty \cdot a, \infty \cdot b,$, $\infty \cdot c\}$ 中求不满足下述三个性质:

$P_{1}: 10$ 组合中 $a$ 的个数大于或等于 4 ;

$P_{2}: 10$ 组合中 $b$ 的个数大于或等于 5 ;

$P_{3}: 10$ 组合中 $c$ 的个数大于或等于 6

的 10 组合数,它们在解题方法上各不相同.下面我们将看到,引入生成函数的概念 后,上述三类组合问题可以统一地处理.

我们先从问题(2) 开始.令
$$
M=\left\{\infty \cdot a_{1}, \infty \cdot a_{2}, \cdots, \infty \cdot a_{n}\right\}
$$
的 $k$ 组合数为 $b_{k}$. 考虑 $n$ 个形式筸级数的乘积
$$
(1+x \underbrace{\left.+x^{2}+\cdots\right)\left(1+x+x^{2}+\cdots\right) \cdots\left(1+x+x^{2}+\cdots\right)}_{n \text { 组 }}
$$
它的展开式中,每一个 $x^{k}$ 均为
$$
x^{m_{1}} x^{m_{2}} \cdots x^{m_{n}}=x^{k} \quad\left(m_{1}+m_{2}+\cdots+m_{n}=k\right)
$$
其中, $x^{m_{1}}, x^{m_{2}}, \cdots, x^{m}$. 分别取自代表 $a_{1}$ 的第一个括号, 代表 $a_{2}$ 的第二个括号, $\cdots \cdots$, 代表 $a_{n}$ 的第 $n$ 个括号 $; m_{1}, m_{2}, \cdots, m_{n}$ 分别表示取 $a_{1}, a_{2}, \cdots, a_{n}$ 的个数. 于 是,每个 $x^{k}$ 都对应着多重集合 $M$ 的一个 $k$ 组合. 因此
$$
\left(1+x+x^{2}+\cdots\right)^{n}
$$
中 $x^{k}$ 的系数就是 $M$ 的 $k$ 组合数 $b_{k}$. 由此得出序列 $\left\{b_{k}\right\}$ 的生成函数为
$$
\left(1+x+x^{2}+\cdots\right)^{n}=\frac{1}{(1-x)^{n}}
$$
从而
$$
b_{k}=\left(\begin{array}{c}
n-1+k \\
k
\end{array}\right)
$$
这时,我们再次得到了第 2 章中多重集合 $M$ 的 $k$ 组合数的公式,只不过现在是用生成函数获得的.

用生成函数方法解问题 $(3)$ 尤为简单. 将 $\{3 \cdot a, 4 \cdot b, 5 \cdot c\}$ 的 $k$ 组合数记为 $b_{k},\left\{b_{k}\right\}$ 的生成函数就是
$$
\left(1+x+x^{2}+x^{3}\right)\left(1+x+x^{2}+x^{3}+x^{4}\right)\left(1+x+x^{2}+x^{3}+x^{4}+x^{5}\right)
$$
其原因是展开式中的 $x^{k}$ 必定为
$$
x^{m_{i}} x^{m_{2}} x^{m_{3}}=x^{k} \quad\left(m_{1}+m_{2}+m_{3}=k\right)
$$
由于 $x^{m_{i}}, x^{m_{2}}, x^{m_{3}}$ 分别取自第一、第二.第三个括号,故 $0 \leqslant m_{1} \leqslant 3,0 \leqslant m_{2} \leqslant 4$, $0 \leqslant m_{3} \leqslant 5$, 于是每个 $x^{k}$ 对应集合 $\{3 \cdot a, 4 \cdot b, 5 \cdot c\}$ 的一个 $k$ 组合.
特别令 $k=10$,则
$$
\begin{array}{l}
\left(1+x+x^{2}+x^{3}\right) \cdot\left(1+x+x^{2}+x^{3}+x^{4}\right) \cdot\left(1+x+x^{2}+x^{3}+x^{4}+x^{5}\right) \\
\quad=\left(1-x^{4}\right) \cdot\left(1-x^{5}\right) \cdot\left(1-x^{6}\right) \cdot \frac{1}{(1-x)^{3}} \\
\quad=\left(1-x^{4}-x^{5}-x^{6}+x^{9}+x^{10}+x^{11}-x^{15}\right) \cdot \sum_{n=0}^{\infty}\left(\begin{array}{c}
n+2 \\
n
\end{array}\right) x^{n}
\end{array}
$$
所以, $x^{10}$ 的系数 $b_{10}$ 为
$$
\begin{aligned}
b_{10}=&\left(\begin{array}{c}
10+2 \\
10
\end{array}\right)-\left(\begin{array}{c}
6+2 \\
6
\end{array}\right)-\left(\begin{array}{c}
5+2 \\
5
\end{array}\right) \\
&-\left(\begin{array}{c}
4+2 \\
4
\end{array}\right)+\left(\begin{array}{c}
1+2 \\
1
\end{array}\right)+\left(\begin{array}{c}
0+2 \\
0
\end{array}\right) \\
=& 6,
\end{aligned}
$$
与第 4 章中用容斥原理得到的结果相同.



在普通集合 $\left\{a_{1}, a_{2}, \cdots, a_{n}\right\}$ 的 $k$ 组合中, $a_{i}(1 \leqslant i \leqslant n)$ 或者出现或者不出 现,故该集合的 $k$ 组合数序列 $\left\{b_{k}\right\}$ 的生成函数为
$$
(1+x)^{n}=\sum_{k=0}^{n}\binom{n}{k} x^{k}
$$
从而
$$
b_{k}=\binom{n}{k}.
$$

综合以上分析,我们得到:

\noindent
\textbf{定理\ 5.4.1} 设从 $n$ 元集合 $S=\left\{a_{1}, a_{2}, \cdots, a_{n}\right\}$ 中取 $k$ 个元素的组合数为 $b_{k}$, 若限定元素 $a_{i}$ 出现的次数集合为 $M_{i}(1 \leqslant i \leqslant n)$, 则该组合数序列的生成函数为
$$
\prod_{i=1}^{n}\left(\sum_{m \in M_{i}} x^{m}\right)
$$

\noindent
\textbf{例\ 1}  求多重集合 $M=\left\{\infty \cdot a_{1}, \infty \cdot a_{2}, \cdots, \infty \cdot a_{n}\right\}$ 的每个 $a_{i}$ 至少出现一次的 $k$ 组合数 $b_{k}$.

\textbf{解  }由定理 $5.4.1$ 知
$$
M_{i}=\{1,2,3, \cdots\} \quad(1 \leqslant i \leqslant n)
$$
于是
$$
\begin{aligned}
G\left\{b_{k}\right\} &=\left(x+x^{2}+x^{3}+\cdots\right)^{n} \\
&=x^{n} \cdot \frac{1}{(1-x)^{n}} \\
&=\sum_{i=0}^{\infty}\left(\begin{array}{c}
n-1+i \\
i
\end{array}\right) x^{n+i} \\
&=\sum_{k=n}^{\infty}\left(\begin{array}{c}
k-1 \\
n-k
\end{array}\right) x^{k} \\
&=\sum_{k=n}^{\infty}\left(\begin{array}{l}
k-1 \\
n-1
\end{array}\right) x^{k}
\end{aligned}
$$
所以
$$
b_{k}=\left\{\begin{array}{ll}
0 & (k<n) \\
\left(\begin{array}{l}
k-1 \\
n-1
\end{array}\right) & (k \geqslant n)
\end{array}\right.
$$

\textbf{1.4.2}组合型分配问题的生成函数

\noindent
\textbf{定理\ 5.4.2}\textsl{把 $k$ 个相同的球放人 $n$ 个不同的盒子 $a_{1}, a_{2}, \cdots, a_{n}$ 中, 限定盒子 $a_{i}$ 的容量集合为 $M_{i}(1 \leqslant i \leqslant n)$, 则其分配方案数的生成函数为
$$
\prod_{i=1}^{n}\left(\sum_{m \in M_{i}} x^{m}\right)
$$}

\noindent
\textbf{证明:}不妨设盒子 $a_{1}, a_{2}, \cdots, a_{n}$ 中放人的球数分别为 $x_{1}, x_{2}, \cdots, x_{n}$,则
$$
x_{1}+x_{2}+\cdots+x_{n}=k \quad\left(x_{i} \in M_{i}, 1 \leqslant i \leqslant n\right)
$$
一种符合要求的放法相当于 $\boldsymbol{M}=\left\{\infty \cdot a_{1}, \infty \cdot a_{2}, \cdots, \infty \cdot a_{n}\right\}$ 的一个 $k$ 组合,前 面关于盒子 $a_{i}$ 容量的限制转变成 $k$ 组合中 $a_{i}$ 出现次数的限制. 由定理 $5.4 .1$ 知,组 合型分配问题方案数的生成函数为
$$
\prod_{i=1}^{n}\left(\sum_{m \in M,} x^{m}\right)
$$


\textbf{例\ 2} 求不定方程
$$
x_{1}+x_{2}+x_{3}+x_{4}+x_{5}=20
$$
满足
$x_{1} \geqslant 3, x_{2} \geqslant 2, x_{3} \geqslant 4, x_{4} \geqslant 6, x_{5} \geqslant 0$
的整数解的个数.

\textbf{解  }本问题相当于把 20 个相同的球放人 5 个不同的盒子中,盒子的容量集合 分别为
$$
\begin{array}{l}
\boldsymbol{M}_{1}=\{3,4, \cdots\} \\
\boldsymbol{M}_{2}=\{2,3, \cdots\} \\
\boldsymbol{M}_{3}=\{4,5, \cdots\} \\
\boldsymbol{M}_{4}=\{6,7, \cdots\} \\
\boldsymbol{M}_{5}=\{0,1,2, \cdots\}
\end{array}
$$
该组合型分配问题的生成函数为
$$
\begin{array}{l}
\left(x^{3}+x^{4}+\cdots\right)\left(x^{2}+x^{3}+\cdots\right)\left(x^{4}+x^{5}+\cdots\right) \\
\cdot\left(x^{6}+x^{7}+\cdots\right)\left(1+x+x^{2}+\cdots\right) \\
\quad=x^{15} \cdot\left(1+x+x^{2}+\cdots\right)^{5} \\
\quad=x^{15} \cdot \frac{1}{(1-x)^{5}} \\
\quad=x^{15} \cdot \sum_{n=0}^{\infty}\left(\begin{array}{c}
n+4 \\
n
\end{array}\right) x^{n}
\end{array}
$$
其中, $x^{20}$ 的系数 $\left(\begin{array}{c}5+4 \\ 5\end{array}\right)=126$ 就是满足条件的整数解的个数.

\textbf{补充题\ }
求方程 $$x_{1}+x_{2}+x_{3}+x_{4}=18$$ 满足
\begin{align*}
    & 1 \leq x_{1} \leq 5, \quad  -2 \leq x_{2} \leq 4,\\
    & 0 \leq x_{3} \leq 5, \quad \,  \,3 \leq x_{4} \leq 9
\end{align*}
的整数解个数.

令 $y_{1}=x_{1}-1, y_{2}=x_{2}+2, y_{3}=x_{3}, y_{4}=x_{4}-3$, 则方程转化为
$$
y_{1}+y_{2}+y_{3}+y_{4}=16
$$
相应的条件即 $0 \leq y_{1} \leq 4$, $0 \leq y_{2} \leq 6$, $0 \leq y_{3} \leq 5$, $0 \leq y_{4} \leq 6$.

对应的生成函数为
\begin{align*}
    & (1+x+x^2+x^3+x^4)
    (1+x+x^2+x^3+x^4+x^5)
    (1+x+x^2+x^3+x^4+x^5+x^6)^2\\
    = & \,  \frac{(1-x^5)(1-x^6)(1-x^7)^2}{(1-x)^4}\\
    = & \,
    (1 - x^5 - x^6 - 2 x^7 + x^{11} + 2 x^{12} + 2 x^{13} + x^{14} -
    2 x^{18} - x^{19} - x^{20} + x^{25}) \sum_{k=0}^{\infty} \binom{k+3}{3} x^k
\end{align*}
所以它的$x^{16}$的系数为
\begin{align*}
    & \binom{16+3}{3}
    -\binom{11+3}{3}
    -\binom{10+3}{3}
    -2\binom{9+3}{3}
    + \binom{5+3}{3}
    +2\binom{4+3}{3}
    +2\binom{3+3}{3}
    +\binom{2+3}{3} \\
    = & \,969 -  364 - 286 -2 \times 220 + 56  +2 \times 35 +2 \times 20 + 10 \\
    = & \,55
\end{align*}


\textbf{补充题\ 1.}设有2红球,1黑球,3白球,若每次从中任取3个,有多少种不同的取法?

\textbf{解  }方法1:

$\begin{aligned}\left(1+x+x^{2}\right)(1+x)\left(1+x+x^{2}+x^{3}\right) &=\frac{1-x^{3}}{1-x} \cdot \frac{1-x^{2}}{1-x} \cdot \frac{1-x^{4}}{1-x} \\ &=\left(1-x^{2}-x^{3}+x^{5}\right)\left(1-x^{4}\right) \sum_{k=0}^{\infty}\left(\begin{array}{c}k+2 \\ 2\end{array}\right) x^{k} \end{aligned}$

$x^{3}$ 的系数为 $\left(\begin{array}{l}5 \\ 2\end{array}\right)-\left(\begin{array}{l}3 \\ 2\end{array}\right)-\left(\begin{array}{l}2 \\ 2\end{array}\right)=10-3-1=6$

方法2:$\left(x_{1}, x_{2}, x_{3}\right)=(0,0,3)(0, 1, 2)(1,0,2)(1,1,1)(2,0,1)(2,1,0)$

\textbf{补充题\ 2.}设有1g,2g,3g,4g的砝码各一枚. 若要求各砝码只能放在天平的一边,问能称出多少种重量?(0g不计入)

\textbf{解  }$(1+x)\left(1+x^{2}\right)\left(1+x^{3}\right)\left(1+x^{4}\right)=1+x+\cdots+x^{10}$,故十种.

\textbf{补充题\ 3.}用1分,2分,3分的邮票可贴出多少种总面值为4分的方案.

\textbf{解  }$1+1+1+1 \quad 1+2+1 \quad 1+3 \quad 2+2$,故四种.


\textbf{补充题\ 4.}求用苹果、香蕉、橘子、梨组成的有$n$个水果的不同水果篮的个数,其中要求苹果有偶数个(包括0个),香蕉有5的倍数个(包括0个),橘子不超过4个,梨最多一个.

\textbf{解  }$\left(1+x^{2}+\cdots+x^{2 n}+\cdots\right)\left(1+x^{5}+\cdots+x^{5 n}+\cdots\right)\left(1+x+x^{2}+x^{3}+x^{4}\right)(1+x)$
$=\frac{1}{1-x^{2}} \cdot \frac{1}{1-x^{5}} \cdot \frac{1-x^{5}}{1-x} \cdot(1+x)=\frac{1}{(1-x)^{2}}=\sum_{k=0}^{\infty}(k+1) x^{k}$

故所求为$n+1$种.


\section{排列型分配问题的指数型生成函数}
本节首先指出生成函数在求解排列型分配问题时的不足,然后引人指数型生成函数以及在排列数中的应用.

\textbf{1.5.1排列数的指数型生成函数}

$n$ 元集合的 $k$ 排列数为 $n(n-1) \cdots(n-k+1)$,按 $5.4 .1$ 小节中方法构成的 生成函数
$$
\sum_{k=0}^{\infty} n(n-1) \cdots(n-k+1) x^{k}
$$
没有简单的解析表达式. 但如果把基底函数 $x^{k}$ 改换成 $\frac{x^{k}}{k !}$,则
$$
\sum_{k=0}^{\infty} n(n-1) \cdots(n-k+1) \cdot \frac{x^{k}}{k !}=(1+x)^{n}
$$
这启发人们引入指数型生成函数的概念.

数列 $\left\{a_{0}, a_{1}, a_{2}, \cdots\right\}$ 的指数型生成函数定义为形式幂级数
$$
\sum_{k=0}^{\infty} a_{k} \frac{x^{k}}{k !}
$$

\noindent
\textbf{定理\ 5.5.1}\textsl{多重集合 $M=\left\{\infty \cdot a_{1}, \infty \cdot a_{2}, \cdots, \infty \cdot a_{n}\right\}$ 的 $k$ 排列中,若限 定元素 $a_{i}$ 出现的次数集合为 $M_{i}(1 \leqslant i \leqslant n)$, 则排列数的指数型生成函数为
$$
\prod_{i=1}^{n}\left(\sum_{m \in M_{i}} \frac{x^{m}}{m !}\right)
$$}

特别地,数列 $\{1,1, \cdots\}$ 的指数型生成函数 $e(x)=\sum_{n=0}^{\infty} \frac{x^{n}}{n !}$ 具有与指数函数相似的性质:
$$
e(x) e(y)=e(x+y)
$$
这是因为
$$
\begin{aligned}
e(x) e(y) &=\sum_{i=0}^{\infty} \frac{x^{i}}{i !} \cdot \sum_{j=0}^{\infty} \frac{y^{j}}{j !} \\
&=\sum_{i=0}^{\infty} \frac{1}{i !} x^{i} \cdot \sum_{j=0}^{\infty} \frac{1}{j !}\left(\frac{y}{x}\right)^{j} x^{j} \\
&=\sum_{n=0}^{\infty}\left[\sum_{k=0}^{n} \frac{1}{k !(n-k) !}\left(\frac{y}{x}\right)^{k}\right] x^{n} \\
&=\sum_{n=0}^{\infty}\left(1+\frac{y}{x}\right)^{n} \frac{x^{n}}{n !} \\
&=\sum_{n=0}^{\infty} \frac{(x+y)^{n}}{n !} \\
&=e(x+y)
\end{aligned}
$$
特别有
$$
e(x) e(-x)=e(0)=1
$$
从而
$$
e(-x)=\frac{1}{e(x)}
$$

\noindent
\textbf{例\ 1}多重集合 $M=\left\{\infty \cdot a_{1}, \infty \cdot a_{2}, \cdots, \infty \cdot a_{n}\right\}$ 的 $k$ 排列数序列 $\left\{b_{k}\right\}$ 的 指数型生成函数为
$$
\prod_{i=1}^{n}\left(1+\frac{x}{1 !}+\frac{x^{2}}{2 !}+\cdots\right)=e^{n}(x)=e(n x)=\sum_{k=0}^{\infty} n^{k} \frac{x^{k}}{k !}
$$
从而
$$
b_{k}=n^{k}
$$

\noindent
\textbf{例\ 2}由数字 $0,1,2,3$ 组成的长为 $k$ 的序列中,要求含有偶数个 0 , 问这样的 序列有多少个?

\textbf{解  }根据题意,有
$$
\begin{array}{l}
M_{1}=M_{2}=M_{3}=\{0,1,2, \cdots\} \\
M_{0}=\{0,2,4, \cdots\}
\end{array}
$$
由定理 $5.5.1$ 知,该排列数的指数型生成函数为
$$
\left(1+\frac{x}{1 !}+\frac{x^{2}}{2 !}+\cdots\right)^{3}\left(1+\frac{x^{2}}{2 !}+\frac{x^{4}}{4 !}+\cdots\right)
$$
$$
\begin{array}{l}
=e^{3}(x) \cdot \frac{e(x)+e(-x)}{2} \\
=\frac{1}{2}[e(4 x)+e(2 x)]
\end{array}
$$
所以 $\frac{x^{k}}{k !}$ 的系数为
$$
b_{k}=\frac{1}{2}\left(4^{k}+2^{k}\right)
$$
当 $k=2$ 时,满足题意的序列有 10 个, 它们是
$$
00,11,12,13,21,22,23,31,32,33
$$

\noindent
\textbf{例\ 3} 由 $1,2,3,4$ 能组成多少个 1 出现 两 次或 三 次, 2 最多出现一次, 4 出现偶数次的五位数?

\textbf{解  }根据题意,有
$$
\begin{array}{l}
{M}_{1}=\{2,3\} \\
{M}_{2}=\{0,1\} \\
{M}_{3}=\{0,1,2, \cdots\} \\
{M}_{4}=\{0,2,4, \cdots\}
\end{array}
$$
由定理 5. $5.1$ 知,该排列数的指数型生成函数为
$$
\begin{array}{l}
\left(\frac{x^{2}}{2 !}+\frac{x^{3}}{3 !}\right)\left(1+\frac{x}{1 !}\right)\left(1+\frac{x}{1 !}+\frac{x^{2}}{2 !}+\cdots\right)\left(1+\frac{x^{2}}{2 !}+\frac{x^{4}}{4 !}+\cdots\right) \\
\quad=\frac{x^{2}}{6}\left(3+4 x+x^{2}\right) \cdot e(x) \cdot \frac{e(x)+e(-x)}{2} \\
\quad=\frac{x^{2}}{12}\left(3+4 x+x^{2}\right)[e(2 x)+1]
\end{array}
$$
所以 $\frac{x^{5}}{5 !}$ 的系数为
$$
5 ! \times \frac{1}{12}\left(3 \times \frac{2^{3}}{3 !}+4 \times \frac{2^{2}}{2 !}+1 \times \frac{1}{1 !}\right)=140
$$
即满足题意的五位数有 140 个.

\noindent
\textbf{例\ 4}$\ $求$M=\{\infty\cdot a_{1},\infty\cdot a_{2},\cdots,\infty\cdot a_{n} \}$的$k$排列中每个$a_{i}$至少出现一次的排列数$P_{k}$的指数型生成函数。

\textbf{解  }$\ $根据题意,有$$M_{i}=\{1,2,3,\cdots\} \quad\quad (1\leq i\leq n).$$
由定理5.5.1知,排列数序列$\{P_{k}\}$的指数型生成函数为
$$\begin{aligned}
	\left(\frac{x}{1 !}+\frac{x^{2}}{2 !}+\frac{x^{3}}{3 !}+\cdots\right)^{n} &=[e(x)-1]^{n} \\
	&=\sum_{i=0}^{n}(-1)^{i}\binom{n}{i} e((n-i) x) \\
	&=\sum_{i=0}^{n}(-1)^{i}\binom{n}{i}\sum_{k=0}^{\infty} \frac{(n-i)^{k} x^{k}}{k !} \\
	&=\sum_{k=0}^{\infty}\left[\sum_{i=0}^{n}(-1)^{i}\binom{n}{i}(n-i)^{k}\right] \frac{x^{k}}{k !} 
\end{aligned}$$
所以
$$	P_{k}=\sum_{i=0}^{n}(-1)^{i}\binom{n}{i}(n-i)^{k} \quad(k \geq n).$$

\noindent
\textbf{例\ 5}$\quad$用红、白、蓝3种颜色给$1\times n$棋盘着色,要求白色方格数是偶数,问有多少种着色方案?

\textbf{解  }$\quad$ 将$1\times n$棋盘的$n$个方格分别用$1,2,\cdots,n$标记,第$i$个方格着$c$色看作把第$i$个物体放入$c$盒中.这时,问题转化为:把$n$个不同的球放入3个不同的盒子中,各盒的容量集合分别为
\[
\begin{aligned}
&M_{r}=M_{b}=\{0,1,2, \cdots\}, \\
&M_{w}=\{0,2,4, \cdots\} .
\end{aligned}
\]于是,分配方案数的指数型生成函数为\[
\begin{aligned}
&(1+x+\frac{x^{2}}{2 !}+\cdots)^{2}\left(1+\frac{x^{2}}{2 !}+\frac{x^{4}}{4 !}+\cdots\right) \\
=&e(2 x) \cdot \frac{e(x)+e(-x)}{2} \\
=&\frac{1}{2}[e(3 x)+e(x)],
\end{aligned}
\]其中,$\dfrac{x^{n}}{n!}$的系数$ \dfrac{1}{2}(3^{n}+1) $就是满足要求的着色方案数。

\textbf{命题\ 1}若$f(x)=\sum_{k=0}^{\infty} a_{k} \frac{x^{k}}{k !}$,则$f^{\prime}(x)=\sum_{k=1}^{\infty} a_{k} \cdot k \cdot \frac{x^{k-1}}{k !}=\sum_{l=0}^{\infty} a_{l+1} \frac{x^{l}}{l !}$
即$\left\{a_{k+1}\right\}_{k=0}^{\infty}$的指数型生成函数为$f(x)$.

\textbf{命题\ 2}若$f(x)=\sum_{k=0}^{\infty} a_{k} \frac{x^{k}}{k !}$,则$\left\{a_{k+i}\right\}_{k=0}^{\infty}$
的e.g.f为$f^{ (i) }(x)$

\textbf{命题\ 3}若$f(x)=\sum_{k=0}^{\infty} a_{k} \frac{x^{k}}{k !}$,则$\left\{ka_{k}\right\}_{k=0}^{\infty}$
的e.g.f为
$$
\begin{aligned}
\sum_{k=1}^{\infty} k a_{k} \frac{x^{k}}{k !} &=\sum_{k=1}^{\infty} a_{k} \frac{x^{k}}{(k-1) !}=x \sum_{k=1}^{\infty} a_{k} \frac{x^{k-1}}{(k-1) !} \\
&=x f^{\prime}(x)=x \frac{d}{d x}(f(x))
\end{aligned}
$$

\textbf{命题\ 4}若$f(x)=\sum_{k=0}^{\infty} a_{k} \frac{x^{k}}{k !}$,$P(k)$为一个关于$k$的多项式,则
$\left\{P(k)a_{k}\right\}_{k=0}^{\infty}$的e.g.f为$P\left(x \frac{d}{d x}\right)(f(x))$.

$$
\begin{aligned}
\sum_{R=0}^{\infty} k^{2} a_{k} \frac{x^{k}}{k !} &=x\left(\sum_{k} k a_{k} \frac{x^{k}}{k !}\right)^{\prime}=x\left(x f^{\prime}(x)\right)^{\prime} \\
&=x \frac{d}{d x}\left(x \frac{d}{d x}(f(x))=\left(x \frac{d}{d x}\right)^{2}(f(x))\right.
\end{aligned}
$$

\textbf{命题\ 5}若$f(x)=\sum_{k=0}^{\infty} a_{k} \frac{x^{k}}{k !}$,$g(x)=\sum_{k=0}^{\infty} b_{k} \frac{x^{k}}{k !}$,则

$f(x) g(x)=\left(\sum_{i=0}^{\infty} a_{i} \frac{x^{i}}{i !}\right)\left(\sum_{j=0}^{\infty} b_{j} \frac{x^{i}}{j !}\right)$
$=\sum_{n=0}^{\infty} n !\left(\sum_{i=1}^{n} \frac{a_{i}}{i !} \cdot \frac{b_{n-i}}{(n-i) !}\right) \frac{x^{n}}{n !}$
$=\sum_{n=0}^{\infty}\left(\sum_{i=0}^{n}\left(\begin{array}{l}n \\ i\end{array}\right) a_{i} b_{n-i}\right) \frac{x^{n}}{n !}$

即$\left\{\sum_{i=0}^{n}\left(\begin{array}{l}n \\ i\end{array}\right) a_{i}b_{n-i} \right\}_{n=0}^{\infty}$的
e.g.f为$f(x)g(x)$

\textbf{命题\ 6}若$f(x)=\sum_{k=0}^{\infty} a_{k} \frac{x^{k}}{k !}$,$g(x)=\sum_{k=0}^{\infty} b_{k} \frac{x^{k}}{k !}$,$h(x)=\sum_{k=0}^{\infty} c_{k} \frac{x^{k}}{k !}$则
$$
\begin{aligned}
f(x) g(x) h(x) &=\left(\sum_{n=0}^{\infty}\left(\sum_{i=0}^{n}\left(\begin{array}{l}
n \\ i
\end{array}\right) a_{i} b_{n-i}\right) \frac{x^{n}}{n !}\right) h(x) \\
&=\sum_{m=0}^{\infty}\left(\sum_{n=0}^{\infty}\left(\begin{array}{l}
m \\ n
\end{array}\right) d_{n} \cdot C_{m-n}\right) \frac{x^{m}}{m !}.\\
&=\sum_{m=0}^{\infty}\left(\sum_{n=0}^{m}\left(\begin{array}{l}
m \\ n
\end{array}\right) \sum_{i=0}^{n}\left(\begin{array}{l}
n \\ i
\end{array}\right) a_{i} b_{n-i}\right)C_{m-n} \frac{x^{m}}{m !} \\
&=\sum_{m=0}^{\infty}\left(\sum_{n=0}^{m} \sum_{i=0}^{n}\left(\begin{array}{c}
m \\i, n-i, m-n
\end{array}\right) a_{i} b_{n-i}C_{m-n} \right) \frac{x^{m}}{m !}.
\end{aligned}
$$

即$\left.\left\{\sum_{m=0}^{\infty} \sum_{i+j+k=m}\left(\begin{array}{c}m \\ i ,j, k\end{array}\right) a_{i} b_{j} c_{k}\right) \frac{x^{m}}{m !}\right\}$的e.g.f为$f(x)g(x)h(x)$.

\textbf{补充题\ 1}确定每位数字都是奇数,且1,3出现偶数次的n位数的个数.

\textbf{解  }

$\begin{aligned} &\left(1+\frac{x^{2}}{2 !}+\frac{x^{4}}{4 !}+\cdots\right)^{2}\left(1+x+\frac{x^{2}}{2 !}+\frac{x^{3}}{3 !}+\cdots\right)^{3} \\ =&\left(\frac{e^{x}+e^{-x}}{2}\right)^{2} \cdot e^{3 x} \\=& \frac{1}{4}\left(e^{2 x}+e^{-2 x}+2\right) \cdot e^{3 x} \end{aligned}$

$=\frac{1}{4}\left(e^{5 x}+2 e^{3 x}+e^{x}\right)=\frac{1}{4}\left(\sum_{k=0}^{\infty} 5^{k} \frac{ x^{k}}{k !}+2 \cdot \sum_{k=0}^{\infty} 3^{k} \frac{ x^{k}}{k !}+\sum_{k=0}^{\infty} \frac{x^{k}}{k !}\right)$


$=\frac{1}{4} \sum_{k=0}^{\infty}\left(5^{k}+2 \cdot 3^{k}+1\right) \frac{x^{k}}{k !}$

$\therefore\left[x^{n}\right]=\frac{1}{4}\left(5^{n}+2 \cdot 3^{n}+1\right)$

\end{document}