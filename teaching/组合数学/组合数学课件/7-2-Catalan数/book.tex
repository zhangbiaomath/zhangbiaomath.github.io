\documentclass[a4paper,11pt]{article}
\usepackage{ctex}



\usepackage{amsmath,amssymb}             % AMS Math
\usepackage[T1]{fontenc}



\usepackage{graphicx}
% \usepackage{epstopdf}
\usepackage{tikz}
\usepackage[left=1.5in,right=1.3in,top=1.1in,bottom=1.1in,includefoot,includehead,headheight=13.6pt]{geometry}
\renewcommand{\baselinestretch}{1.05}



\usepackage{minitoc}
\newtheorem{thm}{定理}[section]
\newtheorem{prop}[thm]{命题}
\newtheorem{coro}[thm]{推论}
\newtheorem{defi}[thm]{定义}
\newtheorem{lem}[thm]{引理}
\newtheorem{exa}[thm]{例}
\newtheorem{ex}[thm]{习题}
\newtheorem{conj}[thm]{猜想}

\def\qed{\nopagebreak\hfill{\rule{4pt}{7pt}}\medbreak}
\def\pf{{\bf 证明~~ }}
\def \sg{\sigma}
\def \asc{\mathrm{asc}}
\def \des{\mathrm{des}}
\def \fix{\mathrm{fix}}
\def \lef{\mathrm{lef}}
\def \one{\mathrm{one}}
\def \Des{\mathrm{Des}}
\def \maj{\mathrm{maj}}
\def \exc{\mathrm{exc}}
\def \inv{\mathrm{inv}}
\def \roots{\mathrm{roots}}
\def \sgn{\mathrm{sgn}}
% Table of contents for each chapter

\usepackage{color}
\definecolor{linkcol}{rgb}{0,0,0.4}
\definecolor{citecol}{rgb}{0.5,0,0}

  \usepackage{graphicx}
  \DeclareGraphicsExtensions{.eps}
  \usepackage[a4paper,pagebackref,hyperindex=true,pdfnewwindow=true]{hyperref}

% \usepackage{chapterbib}
\begin{document}




%\pagenumbering{roman}
%\cleardoublepage
%\dominitoc
% \tableofcontents
%\mainmatter
%

%\chapter{基本组合数}
%\label{zuhejiegou} \minitoc





\clearpage




%-------------------------------------------------------------
\section{Catalan数}
\subsection{格路}
\begin{defi}
平面上从 $(0, 0)$ 到 $(2n, 0)$ 的一条路径,如果每步只能是向 $(1, 1)$
方向或向 $(1, -1)$ 方向前进(只走格点),并且保证不穿越到 $x$
轴的下方,这样的路径被称为 Dyck path。从 $(0, 0)$ 到 $(2n, 0)$ 的
Dyck path 可简记为 $n$-Dyck path。
\end{defi}
我们知道,$n$-Dyck path 的总数是第 $n$ 项 Catalan 数
$C_{n}=\frac{1}{n+1}{2n\choose n}$。容易计算,$C_{2}=2, C_{3}=5,
C_{4}=14, C_{5}=42$。Catalan
数在组合计数中有着十分广泛的应用,许多计数问题都可以直接或间接地用
Catalan 数解决。比如乘法排序问题 ($multiplication\ orderings$):将
$n+1$
个数通过不同的顺序乘起来的方式数,注意要求保持数本身的顺序不变。

\begin{ex}
将 $xyz$ 按照不同顺序乘起来有 $C_{2}=2$ 种方式:$((xy)z), (x(yz))$。

将 $xyzw$ 按照不同顺序乘起来有 $C_{3}=5$ 种方式:$((xy)(zw)),
(((xy)z)w), \\((x(yz))w), (x((yz)w)), (x(y(zw)))$。
\end{ex}
%-------------------------------------------------------------
\subsection{有禁置换}

\begin{defi}
若置换 $\pi=\pi_1\pi_2\cdots\pi_n\in\mathfrak{S}_n$ 中不存在 $1\le
i<j<k\le n$ 使得 $\pi_i\pi_j\pi_k$ 是模式 (pattern) 312(即
$\pi_j<\pi_k<\pi_i$),则称置换 $\pi=\pi_1\pi_2\cdots\pi_n$ 是
312-禁排置换。类似可定义 123-禁排置换,321-禁排置换。
\end{defi}

由 312-禁排置换的定义容易得到下面的引理:

\begin{lem}
置换 $\pi\in\mathfrak{S}_n$ 是 312-禁排的当且仅当对 $\forall\
i\in[n]$,$i$ 右边比 $i$ 小的所有元素构成的子列是递减的。
\end{lem}

\begin{thm}
$\mathfrak{S}_n$ 中 312-禁排置换数目是第 $n$ 项 Catalan 数
$C_n=\frac{1}{n+1}{2n\choose n}$。
\end{thm}
{\bf{证明}} 我们知道 $n$-Dyck path 的数目是第 $n$ 项 Catalan 数
$C_{n}$,所以一个比较自然的想法就是试图在 $n$-Dyck paths 和
$\mathfrak{S}_n$ 中
312-禁排置换间构造一个双射。为此,我们介绍一种标号方法,这里称之为
312-禁排标号。对任意一条 $n$-Dyck path
$D$,我们按如下方式来标号:将向 (1, 1) 方向走的 $n$ 步用
$1,2,\ldots,n$ 从左到右依次标号;对向 $(1, -1)$ 方向走的 $n$
步我们这样来处理:对任意一个峰不妨设为 $ud$ (这里 $u$ 是向(1, 1)
方向走的一步,$d$ 是紧接的向 $(1, -1)$ 方向走的一步),对 $d$ 标与
$u$ 相同的标号;对剩下的尚未标号的向 $(1, -1)$ 方向走的每步
$d$,我们用 $max\{l^u\backslash l^d\}$ 从左到右依次给它标号,这里
$l^u(l^d)$ 是 $d$ 左边向(1, 1)($(1, -1)$)
方向走的每步标号的集合。从左到右依次记下向 $(1, -1)$ 方向走的 $n$
步的标号得与 $D$ 对应的置换记为
$L_{312}(D)$,由我们的标号原则不难看出 $L_{312}(D)$ 是
312-禁排置换。反之,对任意一个 312-禁排置换
$\pi$,据之前的原则很容易构造出与之对应的 $n$-Dyck
path。从而,我们就得到了 $n$-Dyck paths 和 $\mathfrak{S}_n$ 中
312-禁排置换之间的一个双射。\qed

\begin{ex}
对如下图所示的 Dyck path 用上面所叙述的原则来标号得到对应的
$\pi=324651798$。
\end{ex}


\begin{picture}(320,50)
\put(5,0){\line(1,1){50}}
\put(3,-2){$\bullet$}%
\put(8,10){1}%
\put(20,14){$\bullet$}%
\put(22,26){2}%
\put(37,31){$\bullet$}%
\put(40,42){3}%
\put(68,42){3}%
\put(83,27){2}%
\put(94,26){4}%
\put(120,25){4}%
\put(130,25){5}%
\put(145,42){6}%
\put(171,40){6}%
\put(186,25){5}%
\put(200,10){1}%
\put(213,10){7}%
\put(235,9){7}%
\put(248,10){8}%
\put(263,25){9}%
\put(290,24){9}%
\put(307,10){8}%
\put(54,48){$\bullet$}%
\put(56,51){\line(1,-1){35}}%
\put(71,31){$\bullet$}%
\put(88,14){$\bullet$}%
\put(90,17){\line(1,1){17}}%
\put(107,32){$\bullet$}%
\put(110,35){\line(1,-1){18}}%
\put(125,14){$\bullet$}%
\put(125,15){\line(1,1){35}}%
\put(142,31){$\bullet$}%
\put(157,46){$\bullet$}%
\put(160,49){\line(1,-1){50}}%
\put(172,31){$\bullet$}%
\put(190,14){$\bullet$}%
\put(206,-2){$\bullet$}%
\put(208,0){\line(1,1){17}}%
\put(224,15){$\bullet$}%
\put(226,17){\line(1,-1){17}}%
\put(242,-2){$\bullet$}%
\put(244,0){\line(1,1){35}}%
\put(260,15){$\bullet$}%
\put(277,31){$\bullet$}%
\put(280,33){\line(1,-1){35}}%
\put(294,15){$\bullet$}%
\put(313,-5){$\bullet$}%
\put(123,-25){图 1}
\end{picture}
\\
\\



若将置换 $\pi=\pi_1\pi_2\cdots\pi_n\in\mathfrak{S}_n$ 看作一个字
(word),我们可以定义它的子字 (subword)
$\sigma=\pi_{i_1}\cdots\pi_{i_k}$($1\le i_1<\cdots< i_k\le n$)。记
$\pi-\sigma$ 为子字 (subword)
$\pi_{j_1}\cdots\pi_{j_{n-k}}$,这里,$\{i_1,\ldots,i_k\}\cup\{j_1,\cdots,j_{n-k}\}=[n]$
且 $\{i_1,\ldots,i_k\}\cap\{j_1,\cdots,j_{n-k}\}=\phi$。

由 321-禁排置换的定义我们容易得到下面的引理:

\begin{lem}
置换 $\pi\in\mathfrak{S}_n$ 是 321-禁排的当且仅当 $lrm(\pi) =
\{\pi_i | i=1$ 或 $\pi_i>\pi_j (j<i)\}$ 和 $[n]\backslash lrm(\pi)$
都是递增的。
\end{lem}

\begin{thm}
$\mathfrak{S}_n$ 中 312-禁排置换的数目与 123-禁排置换的数目相等。
\end{thm}
{\bf{证明}} 我们这里仍然给出一个类似于定理 2.3
的组合证明,但这次介绍的是 321-禁排标号。对任意一条 $n$-Dyck path
$D$,我们按如下方式来标号:将向 (1, 1) 方向走的 $n$ 步用
$1,2,\ldots,n$ 从左到右依次标号;对向 $(1, -1)$ 方向走的 $n$
步我们这样来处理:对任意一个峰不妨设为 $ud$ (这里 $u$ 是向(1, 1)
方向走的一步,$d$ 是紧接的向 $(1, -1)$ 方向走的一步),对 $d$ 标与
$u$ 相同的标号;对剩下的尚未标号的向 $(1, -1)$ 方向走的每步
$d$,我们用 $min\{l^u\backslash l^d\}$ 从左到右依次给它标号,这里
$l^u(l^d)$ 是 $d$ 左边向(1, 1)($(1, -1)$)
方向走的每步标号的集合。从左到右依次记下向 $(1, -1)$ 方向走的 $n$
步的标号得与 $D$ 对应的置换记为 $L_{321}(D)$。


\begin{ex}
对如下图所示的 Dyck path 用上面所叙述的原则来标号得到对应的
$\pi=314625798$。
\end{ex}

\begin{picture}(320,50)
\put(5,0){\line(1,1){50}}%
\put(3,-2){$\bullet$}%
\put(8,10){1}%
\put(20,14){$\bullet$}%
\put(22,26){2}%
\put(37,31){$\bullet$}%
\put(40,42){3}%
\put(68,42){3}%
\put(83,27){1}%
\put(94,26){4}%
\put(120,25){4}%
\put(130,25){5}%
\put(145,42){6}%
\put(171,40){6}%
\put(186,25){2}%
\put(200,10){5}%
\put(213,10){7}%
\put(235,9){7}%
\put(248,10){8}%
\put(263,25){9}%
\put(290,24){9}%
\put(307,10){8}%
\put(54,48){$\bullet$}%
\put(56,51){\line(1,-1){35}}%
\put(71,31){$\bullet$}%
\put(88,14){$\bullet$}%
\put(90,17){\line(1,1){17}}%
\put(107,32){$\bullet$}%
\put(110,35){\line(1,-1){18}}%
\put(125,14){$\bullet$}%
\put(125,15){\line(1,1){35}}%
\put(142,31){$\bullet$}%
\put(157,46){$\bullet$}%
\put(160,49){\line(1,-1){50}}%
\put(172,31){$\bullet$}%
\put(190,14){$\bullet$}%
\put(206,-2){$\bullet$}%
\put(208,0){\line(1,1){17}}%
\put(224,15){$\bullet$}%
\put(226,17){\line(1,-1){17}}%
\put(242,-2){$\bullet$}%
\put(244,0){\line(1,1){35}}%
\put(260,15){$\bullet$}%
\put(277,31){$\bullet$}%
\put(280,33){\line(1,-1){35}}%
\put(294,15){$\bullet$}%
\put(313,-5){$\bullet$}%
\put(123,-25){图 2}
\end{picture}
\\
\\
\\




下面我们来证明刚刚的 321-禁排标号事实上给出了$n$-Dyck paths 和
$\mathfrak{S}_n$ 中 321-禁排置换间的一个双射。设 $A$
是由从左至右的峰中向 $(1, -1)$ 方向走的那些步的标号所构成的字,$B$
是由剩余的从左至右的向 $(1, -1)$
方向走的那些步的标号所构成的字。需要注意的是 $A$ 和 $B$
都是递增的。考虑到向 $(1, -1)$
方向走的任意三步的标号中必定有两步的标号或者包含于 $A$ 或者包含于
$B$ 中,从而易知 $L_{321}(D)$ 是 321-禁排置换。反之,对任意一个
321-禁排置换 $\pi$,作如下分解:$\pi=a_1b_1a_2b_2\cdots
a_lb_l$,这里 $1\le l\le n$,$a_i$($1\le i\le l$) 是 $\pi$
中左到右最大的元素(left-to-right maximum),$b_j$($1\le j\le l-1$) 是
$a_j$ 和 $a_{j+1}$ 间的元素构成的子字,$b_l$ 是 $a_l$
右边的元素所构成的子字。让 $a_i$ ($1\le i\le l$) 对应向(1, 1)
方向走的 $a_i-a_{i-1}$ 步(习惯上令 $a_0=0$),让 $b_i$ ($1\le i\le
l$) 对应向 $(1, -1)$ 方向走的 $|b_i|+1$ 步,结合引理 2.5
知最后所得到的路是一条 $n$-Dyck path。另外,很容易得到
$\mathfrak{S}_n$ 中 321-禁排置换与 123-禁排置换间的一一对应。\qed

%-------------------------------------------------------------
\subsection{Catalan 数}

\begin{thm}第 $n$ 项 Catalan 数
$$C_n={2n\choose n}-{2n\choose n-1}.$$
\end{thm}
{\bf{证明}} 定理中这个等式可以借助 Dyck path
通过组合方法来证明。这里,我们给出的是一种应用了反射原理的证明方法。

令 $A$ 是平面上从 (0, 0) 到 $(2n, 0)$ 且每步只能取 (1, 1) 或 $(1,
-1)$ 的路的集合。易知 $|A|={2n\choose n}$。令 $B$ 是 $n$-Dyck path
的集合,则 $|B|=C_n$ (第 $n$ 项 Catalan 数)。令 $C$ 是包含于 $A$
且穿越了 $x$ 轴的那些路的集合,则 $|C|=|A|-|B|$。令 $D$ 是平面上从
(0, 0) 到 $(2n, -2)$ 且每步只能取 (1, 1) 或 $(1, -1)$
的路的集合。易知 $|D|={2n\choose n-1}$。因此我们只需要说明 $|C|=|D|$
即可。对 $C$ 中任意一条路 $p$,假定 $(2i-1, -1)$ 是 $p$ 与 $y=-1$
所交的第一个点,容易知道,将 $p$ 中从 $(2i-1, -1)$ 到 $(2n, 0)$
的那段关于 $y=-1$ 作反射,$(0, 0)$ 到 $(2i-1,
-1)$那段不变所得到的是从 $(0, 0)$ 到 $(2n, -2)$ 属于 $D$ 的路
$p'$,很容易证明这样的反射事实上给出了集合 $C$ 和 $D$
间的一个一一对应。\qed

\begin{thm}
设 $G(x)$ 是 Catalan 数的生成函数即
$G(x)=\sum\limits_{n=0}^{\infty}C_nx^n$,则
$G(x)=\frac{1-\sqrt{1-4x}}{2x}$。
\end{thm}
{\bf{证明}} 令 $u$($d$) 表示向 (1, 1)($(1,-1)$)
方向走的一步,这样我们就可以把每条 Dyck path 对应于 $\{u, d\}$
上的一个字。比如,图 1 中的 Dyck path 对应于字
$uuudduduuddduduudd$。可以观察到每条非空的 Dyck path
可以被唯一的分解成 $\delta=u\alpha d\beta$,这里 $\alpha$ 和 $\beta$
都是 Dyck path。假定 $\delta$ 和 $\alpha$ 分别长为 $2n$ 和
$2i$($0\le i\le n-1$),从而易知 $\beta$ 长为 $2n-2i-2$,因此有
$C_n=\sum\limits_{i=0}^{n-1}C_iC_{n-1-i}$,所以
\begin{align*}
G(x)-1%
&=\sum_{n\ge1}C_nx^n\\
&=\sum_{n\ge1}\left(\sum_{i=0}^{n-1}C_iC_{n-1-i}\right)x^n\\%
&=x\sum_{n\ge0}\left(\sum_{i=0}^{n}C_iC_{n-i}\right)x^{n}\\
&=xG^2(x).%
\end{align*}

解上面的函数方程得

\[
G(x)=\frac{1-\sqrt{1-4x}}{2x}.
\]

注意我们在应用求根公式时取了负号,这是因为
$\frac{1+\sqrt{1-4x}}{2x}$ 的展开式中包含了项
$\frac{2}{2x}=x^{-1}$。\qed






\end{document} 
