% !Mode:: "TeX:UTF-8"
\documentclass[13pt]{beamer}
\usepackage[utf8]{inputenc}


\usepackage{amsmath,amssymb,amsthm}             % AMS Math
\usepackage[T1]{fontenc}

\usepackage{graphicx}
\usepackage{epstopdf}
\usepackage{tikz}
\linespread{1.3}

\usepackage{mathrsfs}  %花写字母
 

%%%=== theme ===%%%
% \usetheme{Warsaw}
%\usetheme{Copenhagen}
%\usetheme{Singapore}
\usetheme{Madrid}
%\usefonttheme{professionalfonts}
%\usefonttheme{serif}
% \usefonttheme{structureitalicserif}
%%\useinnertheme{rounded}
%%\useinnertheme{inmargin}
\useinnertheme{circles}
%\useoutertheme{miniframes}
\setbeamertemplate{navigation symbols}{}
\setbeamertemplate{footline}[page number]




\usepackage{ctex}
% \usepackage{CJK,CJKnumb,CJKulem}
\usepackage{minitoc}


\setbeamertemplate{theorems}[numbered]
\newtheorem{thm}{定理}
\newtheorem{lem}{引理}
\newtheorem{exa}{例}
\newtheorem*{defi}{定义}
\newtheorem*{coro}{推论}
\newtheorem*{ex}{练习}
\newtheorem*{rem}{注}
\newtheorem*{prop}{性质}

\def\qed{\nopagebreak\hfill{\rule{4pt}{7pt}}\medbreak}
\def\pf{{\bf 证明~~ }}
\def\R{\mathbb{R}}
\def\Rn{\mathbb{R}^n}
\def\A{\mathscr{A}}
\def\a{\alpha}
\def\b{\beta}
\def\r{\gamma}
\def\0{\mathbf{0}}


\usepackage{color}
\definecolor{linkcol}{rgb}{0,0,0.4}
\definecolor{citecol}{rgb}{0.5,0,0}



  \usepackage{graphicx}
  \DeclareGraphicsExtensions{.eps}
%   \usepackage[a4paper,pagebackref,hyperindex=true,pdfnewwindow=true]{hyperref}


\begin{document}

\titlegraphic{\includegraphics[width=2cm]{../tjnu.jpg}} 

\title[]{第九章 \quad  欧几里得空间}
\author[]{{\large 张彪}\\  }
\institute[]{{\normalsize
		天津师范大学\\[6pt]
		zhang@tjnu.edu.cn}}

\date{}


\AtBeginSection[]
{
	\setcounter{exa}{0}
	\setcounter{equation}{0}
}


\begin{frame}
\maketitle
\end{frame}

\begin{frame}{Outline}
\tableofcontents
\end{frame}

\begin{frame}{}
\begin{itemize}
\item 前面主要介绍了向量的线性运算,向量组的线性相关 与线性无关性,并讨论了向量空间中的基、维数以及向量 的坐标等概念. 

\item 但在向量空间中还没有涉及度量性质,即还没有考虑 向量空间中的向量的大小、向量间的夹角等问题. 


%在解析几何中, 考虑几何空间$\mathbb{R}^2$, $\mathbb{R}^3$, 向量的长度,夹角等度量性质
%都可以通过内积反映出来:
%\[
%\begin{aligned}
%	&\text { 长度 } \quad|\alpha|=\sqrt{\alpha \cdot \alpha}\\
%	&\text { 角度 }<\alpha, \beta>  \quad \cos <\alpha, \beta>\, =\frac{\alpha \cdot \beta}{|\alpha||\beta|}
%\end{aligned}
%\]



\item 本章将在向量空间中引入内积的概念,并赋予相应的 度量性质. 

\end{itemize}

\end{frame}


\begin{frame}
\begin{itemize}
	\item 
在几何空间中两个向量 $a, b$ 的内积 (数量积)定义为:
\[
a \cdot b=|a| \cdot|b| \cos \theta,
\]
其中  $|{a}|$, $| {b} |$ 是向量 $a, b$ 的长度, ${\theta}$ 是向量 ${a}, {b}$ 的夹角. 
 
 \item 
在建立空间直角坐标系后,有了向量的坐标表示,即
\[
a=\left(a_{1}, a_{2}, a_{3}\right), b=\left(b_{1}, b_{2}, b_{3}\right)
\]
相应地,内积的计算公式为 $a \cdot b=\sum_{i=1}^{3} a_{i} b_{i}$

\item 下面仿照该计算公式,在空间 $\mathbb{R}^{n}$ 引入中的内积概念. 
\end{itemize}
\end{frame}

\section{欧氏空间}
\begin{frame}{\S 1 欧氏空间}
\begin{defi}
设V是实数域 $\mathbb{R}$上的线性空间,对V中任意两个向量 $\alpha, \beta$
定义一个二元实函数,记作   $(\alpha, \beta)$, 
它具有满足以下性质
\begin{enumerate}
\item 	$(\alpha, \beta)=(\beta, \alpha)$
	(对称性)
\item 	$(k \alpha, \beta)=k(\alpha, \beta)$
	(左数乘性)
\item 	$(\alpha+\beta, \gamma)=(\alpha, \gamma)+(\beta, \gamma)$  (左可加性)
\item 	$(\alpha, \alpha) \geq \mathbf{0},$ 当且仅当 $\alpha=\mathbf{0}$ 时 $(\alpha, \alpha)=\mathbf{0} . \quad$ (正定性)
\end{enumerate}
这里 $\alpha, \beta, \gamma$ 是 $V$ 中任意的向量, $k$ 是任意实数,
则称 $(\alpha, \beta)$为$\alpha$和  $\beta$  的\alert{内积},并称这种定义了内积的
实数域 R上的线性空间V为\alert{欧几里得空间}, 简称 \alert{欧氏空间}.

\end{defi}
\end{frame}


\begin{frame}
\begin{exa}
在  $\mathbb{R}^n$中,对于向量 
$\alpha=\left( a_{1}, a_{2}, \cdots, a_{n}\right)$, $\beta=\left(b_{1}, b_{2}, \cdots, b_{n}\right)$

定义$$(\alpha, \beta)=a_{1} b_{1}+a_{2} b_{2}+\cdots+a_{n} b_{n}$$

验证 $(\alpha, \beta)$满足定义中的4个性质. 
\begin{enumerate}
	\item   $(\alpha, \beta)=\sum_{i=1}^{n} a_{i} b_{i}=\sum_{i=1}^{n} b_{i} a_{i}=(\beta, \alpha)$
	\item  $(k \alpha, \beta)=\sum_{i=1}^{n}\left(k a_{i}\right) b_{i}=\sum_{i=1}^{n} k\left(a_{i} b_{i}\right)=k(\alpha, \beta)$
	\item 如果 $\gamma=\left(c_{1}, c_{2}, \ldots, c_{n}\right), \alpha + \beta =\left(a_{1}+b_{1}\right.$
	$\left.a_{2}+b_{2}, \ldots, a_{n}+b_{n}\right),$ 则
	$(\alpha+\beta, \gamma)=\sum_{i=1}^{n}\left(a_{i}+b_{i}\right) c_{i}=\sum_{i=1}^{n} a_{i} c_{i}+\sum_{i=1}^{n} b_{i} c_{i}=(\alpha, \gamma)+(\beta, \gamma)$
	\item $(\alpha, \alpha)=\sum_{i=1}^{n} a_{i} a_{i}=\sum_{i=1}^{n} a_{i}^{2} \geq 0$ 
	当且仅当 $a_{i}=0(i=1,2, \dots, n)$ 时, $(\alpha, \alpha)=0$
\end{enumerate}

因此,$\mathbb{R}^n$对于内积$(\alpha, \beta)$ 就成为一个欧氏空间.
\end{exa}

\end{frame}


\begin{frame}
\begin{exa}
${C}({a}, {b})$ 为闭区间 $[{a}, {b}]$ 上的所有实连续函数
所成线性空间,对于函数 $f(x), g(x),$ 定义
\[
(f, g)=\int_{a}^{b} f(x) g(x) d x
\]
则 $C(a, b)$ 作成一个欧氏空间.
\end{exa}
\end{frame}
 
\begin{frame}
\begin{prop}
设$V$为欧氏空间, 
$\forall \alpha, \beta, \gamma \in {V}$, $\forall k \in \mathbb{R}$
\begin{enumerate}
	\item  $(\alpha, k \beta)=k(\alpha, \beta)$, 
	\item $(\alpha, \beta+\gamma)=(\alpha, \beta)+(\alpha, \gamma)$
	\item   $(\mathbf{0}, \beta)=0$
	\item $\left(\sum_{i=1}^{n} k_{i} \alpha_{i}, \sum_{j=1}^{m} l_{j} \beta_{j}\right)=\sum_{i=1}^{n} \sum_{j=1}^{m} k_{i} l_{j}\left(\alpha_{i}, \beta_{j}\right)$
\end{enumerate}
\end{prop}


\begin{rem}
\begin{itemize}
\item 在欧几里得空间的定义中,对它作为线性空间的维数并无要 求,可以是有限维的,也可以是无限维的. 
\item 内积满足齐次性、可加性,这两条性质合在一
起称为内积的双线性性. 即内积是实线性空间中的一个正定
对称双线性函数. 
\end{itemize}
\end{rem}
\end{frame}
 
 
 \begin{frame}{ 二、欧氏空间中向量的长度}
 
 1. 引入长度概念的可能性
 
 1)在 ${\R}^{3}$ 向量 $\alpha$ 的长度模 $$|\alpha|=\sqrt{\alpha \cdot \alpha}$$
 
 2) 欧氏空间V中,  对任意的$ \alpha \in V, \quad(\alpha, \alpha) \geq \mathbf{0}$ 使得 $\sqrt{\alpha \cdot \alpha}$ 有意义.
 
 2. 向量长度的定义

\begin{defi}
 在欧氏空间$V$中,对任意向量$\alpha \in V$,
称 $$|\alpha|=\sqrt{(\alpha, \alpha)}$$ 为向量 $\alpha$ 的\alert{长度}.
特别地,当 $|\alpha |=1$ 时,称 $\alpha$ 为\alert{单位向量}.
\end{defi}

 \end{frame}


\begin{frame}{向量长度的简单性质}
\begin{prop}
\begin{enumerate}
	\item $|\alpha| \geq {0}; $  \qquad $|\alpha|={0} \Leftrightarrow \alpha=\mathbf{0}$
	
	\item  $|k \alpha|=|k||\alpha|$
	
	\item 如果$\alpha \neq 0$,  则 $\frac{1}{|\alpha|} \alpha$ 是一个\alert{单位向量}. 
\end{enumerate}
\end{prop}



通常称此过程为把 $\alpha$ \alert{单位化}.
\end{frame}


\begin{frame}{三、欧氏空间中向量的角度}
1. 引入夹角概念的可能性与困难

1)在 $\mathbb{R}^{3}$ 中向量 $\alpha$ 与 $\beta$ 的夹角
\[
<\alpha, \beta>=\arccos \frac{\alpha \cdot \beta}{|\alpha||\beta|}
\]

2)在一般欧氏空间中推广
上面形式,首先应证明不等式: 
$$\quad\left|\frac{ ( \alpha, \beta)}{|\alpha \| \beta|}\right| \leq 1$$
\end{frame}


\begin{frame}{柯西-布涅柯夫斯基不等式(又称``柯西-施瓦兹不等式'')}
\begin{prop}
对欧氏空间V中任意两个向量 $\alpha, \beta,$ 有
\[
| \left(\alpha, \beta \right) | \leq |\alpha|   \cdot | \beta|
\]
当且仅当 $\alpha, \beta$ 线性相关时等号成立.
\end{prop}

\begin{itemize}
\item 对于欧氏空间$\mathbb{R}^n$
$$\left|a_{1} b_{1}+a_{2} b_{2}+\dots+a_{n} b_{n}\right|
\leq \sqrt{a_{1}^{2}+a_{2}^{2}+\dots+a_{n}^{2}}   \sqrt{b_{1}^{2}+b_{2}^{2}+\dots+b_{n}^{2}}.$$
\item 对于欧氏空间$C(a,b)$
$$\left|\int_{a}^{b} f(x) g(x) d x\right| \leq \sqrt{\int_{a}^{b} f^{2}(x) d x}  \cdot  \sqrt{\int_{a}^{b} g^{2}(x) d x}.$$
\end{itemize}
\end{frame}

\begin{frame}
\[
|(\alpha, \beta)| \leq |\alpha|   \cdot | \beta|
\]
\pf
当 $\beta=\mathbf{0}$ 时 $, \quad(\alpha, \mathbf{0})=0, \quad|\beta|=0$

因此,$ \quad(\alpha, \beta)=|\alpha||\beta|=0 . \quad$ 结论成立.

当 $\beta \neq \mathbf{0}$ 时,作向量 $\quad \gamma=\alpha+ \alert{t} \beta, \quad t  \in \mathbb{R}$

由内积的正定性,对 $\forall t \in \mathbb{R}, \quad$ 皆有
$$(\gamma, \gamma)=(\alpha+ \alert{t} \beta, \alpha+\alert{t} \beta)
=(\alpha, \alpha)+2(\alpha, \beta) \alert{t}+(\beta, \beta) \alert{t}^{2} \geq 0 $$
取 $t=-\frac{(\alpha, \beta)}{(\beta, \beta)} $
 代入上式, 得  
$$(\alpha, \alpha)-2(\alpha, \beta) \frac{(\alpha, \beta)}{(\beta, \beta)}+(\beta, \beta) \frac{(\alpha, \beta)^{2}}{(\beta, \beta)^{2}} \geq 0$$
即 $\quad(\alpha, \beta)^{2} \leq(\alpha, \alpha)(\beta, \beta)$
两边开方,

即得 $\quad|(\alpha, \beta)| \leq|\alpha|  \cdot| \beta|.$
\end{frame}

\begin{frame}
\begin{center}
$| \left(\alpha, \beta \right) | = |\alpha|    | \beta|
$
当且仅当 $\alpha, \beta$ 线性相关.
\end{center}


\begin{itemize}
\item 
当
$\alpha, \beta$ 线性相关时,不妨设$\alpha=k \beta$.
于是, $$|(\alpha, \beta)|=|(k \beta, \beta)|=|k(\beta, \beta)|=|k \| {\beta}|^{2}$$
$$
|\alpha\|\beta|=| k \beta\| \beta|=|k \| \beta|^{2}
$$
因此$|(\alpha, \beta)|=|\alpha||\beta| . \quad$ 等号成立.

\item 
反之,若等号成立,
由以上证明过程知

或者 $\beta=\0,$ 或者 $\alpha-\frac{(\alpha, \beta)}{(\beta, \beta)} \beta=0$

也即 $\alpha, \beta$ 线性相关.
\qed
\end{itemize}
\end{frame}


\begin{frame}
\begin{coro}
对欧氏空间中的任意两个向量 $\alpha, \beta,$ 有
$$|\alpha+\beta| \leq|\alpha|+|\beta|. $$
\end{coro}
\pf 
\[
\begin{aligned}
|\alpha+\beta|^{2}& =(\alpha+\beta, \alpha+\beta) \\
\quad& =(\alpha, \alpha)+2(\alpha, \beta)+(\beta, \beta) \\
\quad & \leq|\alpha|^{2}+2|\alpha||\beta|+|\beta|^{2}=(|\alpha|+|\beta|)^{2}
\end{aligned}
\]
两边开方,证毕.
\qed
\end{frame}




\begin{frame}
\begin{defi}
	设V为欧氏空间, $\alpha, \beta$ 为V中任意两非零
向量, 
$\alpha, \beta$ 的夹角定义为
\[
\langle\alpha, \beta\rangle=\arccos \frac{(\alpha, \beta)}{|\alpha||\beta|},
\quad
(0 \leq\langle\alpha, \beta\rangle \leq \pi)
\]
\end{defi}
%\end{frame}
%
%\begin{frame}
\begin{defi}
	设 $\alpha, \beta$ 为欧氏空间中两个向量,若内积
$
(\alpha, \beta)=0
$
则称 $\alpha$ 与 $\beta$ 正交或互相垂直,记作 $\alpha \perp \beta$.
\end{defi}
\begin{rem}
	\begin{itemize}
		\item 零向量与任意向量正交.
		 \item $\alpha \perp \beta \Longleftrightarrow\langle\alpha, \beta\rangle=\frac{\pi}{2},$ 即 $\cos \langle\alpha, \beta\rangle=0$.
	\end{itemize}
\end{rem}

\end{frame}

\begin{frame}
\begin{prop}[勾股定理]
设V为欧氏空间, 对任意的 $ \alpha, \beta \in V$
$$\alpha \perp \beta \Longleftrightarrow|\alpha+\beta|^{2}=|\alpha|^{2}+|\beta|^{2}. $$
\end{prop} 

\pf
因为$$
\begin{aligned}
|\alpha+\beta|^{2}& =(\alpha+\beta, \alpha+\beta) \\ 
	& =(\alpha, \alpha)+2(\alpha, \beta)+(\beta, \beta) 
\end{aligned}$$

所以$\quad|\alpha+\beta|^{2}=|\alpha|^{2}+|\beta|^{2} \Longleftrightarrow(\alpha, \beta)=0$
$\Longleftrightarrow \alpha \perp \beta$. \qed

\begin{coro}
若欧氏空间V中向量 $\alpha_{1}, \alpha_{2}, \cdots, \alpha_{m}$ 两两正交,

即 $\left(\alpha_{i}, \alpha_{j}\right)={0}$, \quad $i \neq j$, \quad $i, j=1,2, \cdots, m$, 有
$$\left|\alpha_{1}+\alpha_{2}+\cdots+\alpha_{m}\right|^{2}=\left|\alpha_{1}\right|^{2}+\left|\alpha_{2}\right|^{2}+\cdots+\left|\alpha_{m}\right|^{2}.$$
\end{coro}
%{\bf 推广~~}  

\end{frame}

\begin{frame}
	\begin{exa}己知 $\alpha=(2,1,3,2), \quad \beta=(1,2,-2,1)$
	在通常的内积定义下, 求 $|\alpha|,(\alpha, \beta),\langle\alpha, \beta\rangle,|\alpha-\beta| .$
	\end{exa}
	\pause
{\bf 解~~} 
\begin{itemize}
\item 
$|\alpha|=\sqrt{(\alpha, \alpha)}=\sqrt{2^{2}+1^{2}+3^{2}+2^{2}}=\sqrt{18}=3 \sqrt{2}$.

\item  $(\alpha, \beta)=2 \times 1+1 \times 2+3 \times(-2)+2 \times 1=0$. 

\item   $\langle\alpha, \beta\rangle=\frac{\pi}{2}$.

\item    因为 $\alpha-\beta=(1,-1,5,1)$,

所以$|\alpha-\beta|=\sqrt{1^{2}+(-1)^{2}+5^{2}+1^{2}}=\sqrt{28}=2 \sqrt{7}.$
\end{itemize}
\end{frame}
\begin{frame}
%	通常称 $|\alpha-\beta|$ 为 $\alpha$ 与 $\beta$ 的距离,记作 $d(\alpha, \beta)$.
%\begin{ex}[{课后习题第3题}]
%证明
%$$d(\alpha, {\gamma}) \leqslant d({a}, {\beta})+d({\beta}, {\gamma})$$
%\end{ex}

在解析几何中,两个点 $\alpha$ 和 $\beta$ 间的距离等于向量 ${\alpha}-{\beta}$ 的长
度.

在欧氏空间中我们同样可引入 
\begin{defi}
长度 $|{\a}-{\beta} |$ 称为向量 ${\alpha}$ 和 ${\beta}$ 的距离,记为 $d({\alpha}, {\beta}).$
\end{defi}

\begin{prop}
距离的三条基本性质:
\begin{enumerate}
\item  $d({\alpha}, {\beta})=d({\beta}, {\alpha})$
\item  $d({\alpha}, {\beta}) \geqslant 0,$ 并且仅当 ${\a}={\beta}$ 时等号才成立
\item  $d({\alpha}, {\beta}) \leqslant d({\alpha}, {\gamma})+d({\gamma}, {\beta})\text { (三角形不等式) }.$
\end{enumerate}
\end{prop}


\end{frame}


\begin{frame}
{四、$n$ 维欧氏空间中内积的矩阵表示}
设$V$为欧氏空间, $\varepsilon_{1}, \varepsilon_{2}, \cdots, \varepsilon_{n}$ 为$V$的一组基,对$V$中
任意两个向量
\begin{align*}
\alpha & = x_{1} \varepsilon_{1}+x_{2} \varepsilon_{2}+\cdots+x_{n} \varepsilon_{n},\\
\beta  & = y_{1} \varepsilon_{1}+y_{2} \varepsilon_{2}+\cdots+y_{n} \varepsilon_{n},
\end{align*}
有
$$
(\alpha, \beta)
=\left(\sum_{i=1}^{n} x_{i} \, \varepsilon_{i}, \sum_{j=1}^{n} y_{j} \, \varepsilon_{j}\right)
=\sum_{i=1}^{n} \sum_{j=1}^{n}\left(\varepsilon_{i}, \varepsilon_{j}\right) x_{i}\, y_{j}.$$
\end{frame}


\begin{frame}

令$ a_{i j}=\left(\varepsilon_{i}, \varepsilon_{j}\right), \quad i, j=1,2, \cdots n$, 

令
$A=\left(a_{i j} \right)_{n \times n}$,
$X= \left(x_1, x_2, \dots, x_n \right)^{\, \prime}$, 
$Y=\left(y_1, y_2, \dots, y_n \right)^{\, \prime}$,

于是,
$(\alpha, \beta)=\sum_{i=1}^{n} \sum_{j=1}^{n} a_{i j} x_{i} y_{j}=X^{\, \prime} A Y$

%{\bf 定义~~~} 
\begin{defi}
称$$A=
\left(\begin{array}{cccc}
\left(\varepsilon_{1}, \varepsilon_{1}\right) & \left(\varepsilon_{1}, \varepsilon_{2}\right) & \cdots & \left(\varepsilon_{1}, \varepsilon_{n}\right) \\ 
\left(\varepsilon_{2}, \varepsilon_{1}\right) & \left(\varepsilon_{2}, \varepsilon_{2}\right) & \cdots & \left(\varepsilon_{2}, \varepsilon_{n}\right) \\
\vdots & \vdots &  & \vdots \\ 
\left(\varepsilon_{n}, \varepsilon_{1}\right)  & \left(\varepsilon_{n}, \varepsilon_{2}\right) & \cdots & \left(\varepsilon_{n}, \varepsilon_{n}\right)
\end{array}\right)$$
为基 $\varepsilon_{1}, \varepsilon_{2}, \cdots, \varepsilon_{n}$ 的度量矩阵.
\end{defi}


\end{frame}


\begin{frame}{}

$$A=
\left(\begin{array}{cccc}
\left(\varepsilon_{1}, \varepsilon_{1}\right) & \left(\varepsilon_{1}, \varepsilon_{2}\right) & \cdots & \left(\varepsilon_{1}, \varepsilon_{n}\right) \\ 
\left(\varepsilon_{2}, \varepsilon_{1}\right) & \left(\varepsilon_{2}, \varepsilon_{2}\right) & \cdots & \left(\varepsilon_{2}, \varepsilon_{n}\right) \\
\vdots & \vdots &  & \vdots \\ 
 \left(\varepsilon_{n}, \varepsilon_{1}\right)  & \left(\varepsilon_{n}, \varepsilon_{2}\right) & \cdots & \left(\varepsilon_{n}, \varepsilon_{n}\right)
\end{array}\right)$$

\begin{rem}
	\begin{itemize}
		\item 度量矩阵A是实对称矩阵.
		\item 由内积的正定性,度量矩阵A还是正定矩阵.
		事实上,对 $\forall \alpha \in V, \alpha \neq \mathbf{0},$ 即 $X \neq 0$
		有 $(\alpha, \alpha)=X^{\, \prime} A X>0$.
		因此,$ A$ 为正定矩阵.
		\item 在基 $\varepsilon_{1}, \varepsilon_{2}, \cdots, \varepsilon_{n}$ 下,向量的内积
		由度量矩阵A完全确定.
	\end{itemize}
\end{rem}

\end{frame}

\begin{frame}

\begin{rem}
	\begin{itemize}
	\item 对同一内积而言,不同基的度量矩阵是合同的. 
	\end{itemize}
\end{rem}
\pf
\small{ 设 $\varepsilon_{1}, \varepsilon_{2}, \cdots, \varepsilon_{n} ; \eta_{1}, \eta_{2}, \cdots, \eta_{n}$ 为欧氏空间V的两组
基,它们的度量矩阵分别为A、B,且
$$\left(\eta_{1}, \eta_{2}, \cdots, \eta_{n}\right)=\left(\varepsilon_{1}, \varepsilon_{2}, \cdots, \varepsilon_{n}\right) C,$$
其中 $C=\left(c_{i j}\right)_{n \times n}=\left(C_{1}, C_{2}, \cdots, C_{n}\right)$

于是,$\eta_{i}=\sum_{k=1}^{n} c_{k i} \varepsilon_{k}, i=1,2, \cdots, n$

因此, 
$\left(\eta_{i}, \eta_{j}\right) =\left(\sum_{k=1}^{n} c_{k i} \varepsilon_{k}, \sum_{l=1}^{n} c_{l j} \varepsilon_{l}\right)=\sum_{k=1}^{n} \sum_{l=1}^{n}\left(\varepsilon_{k}, \varepsilon_{l}\right) c_{k i} c_{l j}$\\
	~~~~~~~~~~~~~~~~~
	$ =\sum_{k=1}^{n} \sum_{l=1}^{n} a_{k l} c_{k i} c_{l j}=C_{i}^{\, \prime} A C_{j}$

所以 $B=\left(\left(\eta_{i}, \eta_{j}\right)\right)=\left(C_{i}^{\, \prime} A C_{j}\right)$
$=\left(\begin{array}{c}C_{1}^{\, \prime} \\ C_{2}^{\, \prime} \\ \vdots \\ C_{n}^{\, \prime}\end{array}\right) A\left(C_{1}, C_{2}, \cdots, C_{n}\right)=C^{\, \prime} A C$
}
\end{frame}

\section{标准正交基的定义与求法}
\begin{frame}{\S 2 标准正交基的定义与求法}

\begin{defi}[正交向量组]
设 $\alpha_{1}, \alpha_{2}, \ldots, \alpha_{s}$ 是一组非零向量, 如果它们两两正交, 则称为\alert{正交向量组}.
\end{defi}
%\begin{rem}
%\begin{itemize}
%	\item 
%\end{itemize}
%\end{rem}
\end{frame}

\begin{frame}
\begin{prop}
	正交向量组是线性无关的.  
\end{prop}
\pf 设正交向量组 ${\alpha}_{1}, {\alpha}_{2}, \cdots, {\alpha}_{m}$ 有一线性关系
\[
k_{1} {\alpha}_{1}+k_{2} {\alpha}_{2}+\cdots+k_{m} {\alpha}_{m}=\mathbf{0}
\]
用 ${\alpha}_{i}$ 与等式两边作内积,即得
$k_{i}\left({\alpha}_{i}, {\alpha}_{i}\right)=0$

由 ${\alpha}_{i} \neq \mathbf{0},$ 有 $\left({\alpha}_{i}, {\alpha}_{i}\right)>0,$ 从而 $k_{i}=0(i=1,2, \cdots, m) .$ 

这就证明了
${\alpha}_{1}, {\alpha}_{2}, \cdots, {\alpha}_{m}$ 是线性无关的.
\qed
\begin{coro}
	$n$维欧氏空间V中, 两两正交的非零 向量的个数不会超过$n$. 
\end{coro}

这个事实的几何意义是清楚的.
例如 , 在平面上找不 到三个两两垂直的非零向量; 在空间中,找不到四个两两垂直的非零向量.

\end{frame}

\begin{frame}
\begin{defi}[正交基]
	在$n$维欧氏空间中, 由$n$个两两正交的非零向量构成的向量组称为 \alert{正交基}. 
	由单位向量组成的正交基称为 \alert{标准正交基}.
	
\end{defi}


\begin{prop}
	\begin{itemize}
		\item 向量组 $\alpha_{1}, \alpha_{2}, \ldots, \alpha_{s}$ 是一个
		标准正交向量组
		$\Longleftrightarrow$ 
		\begin{center}
		$\left(\alpha_{i}, \alpha_{j}\right)=
		\left\{\begin{array}{ll} 
		{1} & {i}={j} \\ 
		{0} & {i} \neq {j}
		\end{array}\right.$
		\end{center}

		\item 	一组基是标准正交基$\Longleftrightarrow$ 它的度量矩阵是单位矩阵. 
	\end{itemize}
\end{prop}

\end{frame}
\begin{frame}
\begin{prop}
	设 $\varepsilon_{1}, \varepsilon_{2}, \dots, \varepsilon_{n}$ 是 $n$ 维欧氏空间 $V$ 的
	一组标准正交基, 对 $\alpha, \beta \in V,$ 设向量 $\alpha, \beta$ 的 坐标分别是 $X=\left(x_{1}, x_{2}, \ldots, x_{n}\right)^{\, \prime}, Y=\left(y_{1}, y_{2}, \ldots, y_{n}\right)^{\, \prime}$, 则
	\begin{itemize}
		\item $x_{i}=\left(\alpha, \alpha_{i}\right) \quad i=1,2, \dots, n$
		\item  $(\alpha, \beta)=X^{\, \prime} Y=x_{1} y_{1}+x_{2} y_{2}+\ldots+x_{n} y_{n}$
	\end{itemize}
\end{prop}

\pf 
\small{
由题设可以
	\[
	{\alpha}=x_{1} {\varepsilon}_{1}+x_{2} {\varepsilon}_{2}+\cdots+{x}_{n} {\varepsilon}_{n}
	\]
	用 ${\varepsilon}_{i}$ 与等式两边作内积, 即得
	\[
	x_{i}=\left({\varepsilon}_{i}, {\alpha}\right) \quad(i=1,2, \cdots, n)
	\]
因为
	\[
	\begin{array}{l}
	{\alpha}=x_{1} {\varepsilon}_{1}+x_{2} {\varepsilon}_{2}+\cdots+x_{n} {\varepsilon}_{n} \\
	{\beta}=y_{1} {\varepsilon}_{1}+y_{2} {\varepsilon}_{2}+\cdots+y_{n} {\varepsilon}_{n}
	\end{array}
	\]
所以
	\[
	({\alpha}, {\beta})=x_{1} y_{1}+x_{2} y_{2}+\cdots+x_{n} y_{n}={X}^{\, \prime} {Y}
	\]}
\end{frame}


\begin{frame}
{三. 求标准正交基的办法--Schmidt正交化方法}
\begin{thm}
$n$维欧氏空间中任一个正交向量组都能扩充成一组正交基.
\end{thm}
\pf 
\small{
设 ${\alpha}_{1}, {\alpha}_{2}, \cdots, {\alpha}_{m}$ 是一正交向量组,我们对 $n-m$ 作
数学归纳法.
\begin{itemize}
\item 当 $n-m=0$ 时, ${\alpha}_{1}, {\alpha}
_{2}, \cdots, {\alpha}_{m}$ 就是一组正交基了.

\item 假设 $n-m=k$ 时定理成立,也就是说,可以找到向量 ${\beta}_{1}$
${\beta}_{2}, \cdots, {\beta}_{k},$ 使得
\[
{\alpha}_{1}, {\alpha}_{2}, \cdots, {\alpha}_{m}, {\beta}_{1}, {\beta}_{2}, \cdots, {\beta}_{k}
\]
成为一组正交基. 

\item 现在来看 $n-m=k+1$ 的情形.因为 $m<n$,所以一定有向 量不能被 ${\alpha}_{1}, {\alpha}_{2}, \cdots, {\alpha}_{m}$ 线性表出,作向量
\[
{\alpha}_{m+1}={\beta}-k_{1} {\alpha}_{1}-k_{2} {\alpha}_{2}-\cdots-k_{m} {\alpha}_{m}
\]
这里 $k_{1}, k_{2}, \cdots, k_{m}$ 是待定的系数.
\end{itemize}
}
\end{frame}


\begin{frame}
用 ${\alpha}_{i}$ 与 ${\alpha}_{m+1}$ 作内积,得
\[
\left({\alpha}_{i}, {\alpha}_{m+1}\right)=\left({\beta}, {\alpha}_{i}\right)-k_{i}\left({\alpha}_{i}, {\alpha}_{i}\right) \quad(i=1,2, \cdots, m)
\]
取
\[
k_{i}=\frac{\left({\beta}, {\alpha}_{i}\right)}{\left({\alpha}_{i}, {\alpha}_{i}\right)} \quad(i=1,2, \cdots, m)
\]
有
\[
\left({a}_{i}, {\alpha}_{m+1}\right)=0 \quad(i=1,2, \cdots, m)
\]
由 ${\beta}$ 的选择可知 $, {\alpha}_{m+1} \neq \mathbf{0} .$ 因此 ${\alpha}_{1}, {\alpha}_{2}, \cdots, {\alpha}_{m}, {\alpha}_{m+1}$ 是一正交向
量组 ,根据归纳法假定 , ${\alpha}_{1}, {w}_{2}, \cdots, {\alpha}_{m}, {\alpha}_{m+1}$ 可以扩充成一正交
基.于是定理得证. 
\qed
\end{frame}


\begin{frame}
在求欧氏空间的正交基时,常常是已经有了空间的一组基.对 于这种情形,有下面的结果:

\begin{thm}
对于 n 维欧氏空间中任意一组基 ${\varepsilon}_{1}, {\varepsilon}_{2}, \cdots, {\varepsilon}_{n},$ 都
可以找到一组标准正交基 ${\eta}_{1}, {\eta}_{2}, \cdots, {\eta}_{n},$ 使
\[
L\left(\varepsilon_{1}, \varepsilon_{2}, \cdots, \varepsilon_{i}\right)=L\left(\eta_{1}, \eta_{2}, \cdots, \eta_{i}\right), i=1,2, \cdots, n
\]
\end{thm}
\pf 
 设 ${\varepsilon}_{1}, {\varepsilon}_{2}, \cdots, {\varepsilon}_{n}$ 是一组基, 我们来逐个地求出向量
${\eta}_{1}, {\eta}_{2}, \cdots, {\eta}_{n}$.

\begin{itemize}
	\item 首先,可取 ${\eta}_{1}=\frac{1}{\left|{\varepsilon}_{1}\right|} {\varepsilon}_{1} .$ 
	
	\item 一般地,假定已经求出 ${\eta}_{1}, {\eta}_{2}, \cdots$
	${\eta}_{m},$ 它们是单位正交的,具有性质
	\[
	L\left({\varepsilon}_{1}, {\varepsilon}_{2}, \cdots, {\varepsilon}_{i}\right)=L\left({\eta}_{1}, {\eta}_{2}, \cdots, {\eta}_{i}\right), i=1,2, \cdots, m
	\]
	
	\item 下一步求 ${\eta}_{m+1}$. 
	因为 $L\left({\varepsilon}_{1}, {\varepsilon}_{2}, \cdots, {\varepsilon}_{m}\right)=L\left({\eta}_{1}, {\eta}_{2}, \cdots, {\eta}_{m}\right),$ 所以 ${\varepsilon}_{m+1}$ 不能
	被 ${\eta}_{1}, {\eta}_{2}, \cdots, {\eta}_{m}$ 线性表出.
\end{itemize}
\end{frame}


\begin{frame}	
	按定理 1 证明中的方法 $,$ 作向量
	\[
	\xi_{m+1}=\varepsilon_{m+1}-\sum_{i=1}^{m}\left(\varepsilon_{m+1}, \eta_{i}\right) \eta_{i}
	\]
	显然
	\[
	\xi_{m+1} \neq 0, \text { 且 }\left(\xi_{m+1}, \eta_{i}\right)=0, i=1,2, \cdots, m
	\]
	令
	\[
	{\eta}_{m+1}=\frac{{\xi}_{m+1}}{\left|{\xi}_{m+1}\right|}
	\]
	${\eta}_{1}, {\eta}_{2}, \cdots, {\eta}_{m}, {\eta}_{m+1}$ 就是一单位正交向量组.
	同时
	\[
	L\left({\varepsilon}_{1}, {\varepsilon}_{2}, \cdots, {\varepsilon}_{m+1}\right)=L\left({\eta}_{1}, {\eta}_{2}, \cdots, {\eta}_{m+1}\right)
	\]
	由归纳法原理,定理得证.  
\qed

\end{frame}

\begin{frame}
\begin{rem}
\begin{itemize}
	\item 定理中的要求
	\[
	L\left(\varepsilon_{1}, \varepsilon_{2}, \cdots, \varepsilon_{i}\right)=L\left(\eta_{1}, \eta_{2}, \cdots, \eta_{i}\right), i=1,2, \cdots, n
	\]
	就相当于由基 ${\varepsilon}_{1}, {\varepsilon}_{2}, \cdots, {\varepsilon}_{n}$ 到基 ${\eta}_{i}, {\eta}_{2}, \cdots, {\eta}_{n}$ 的过渡矩陈是上三角矩阵.
\end{itemize}
\end{rem}
\end{frame}


\begin{frame}{施密特(Schmidt) 正交化过程}
 \begin{itemize}
	\item $n$ 维欧氏空间 $V$必存在正交基与标准正交基.
 
	\item 对 $n$ 维欧氏空间 V的任
一组基$\alpha_1, \alpha_2,..., \alpha_n$ , 都可以用\alert{施密特(Schmidt) 正交化过程}化为正交基
$\beta_1, \beta_{2}, \cdots, \beta_{n}$.

% 施密特正交化过程如下
 
$\beta_{1}=\alpha_{1}$

$\beta_{2}=\alpha_{2}-\frac{\left(\alpha_{2}, \beta_{1}\right)}{\left(\beta_{1}, \beta_{1}\right)} \beta_{1}$

$\cdots \cdots$

$\beta_{n}=\alpha_{n}-\frac{\left(\alpha_{n}, \beta_{1}\right)}{\left(\beta_{1}, \beta_{1}\right)} \beta_{1}-\frac{\left(\alpha_{n}, \beta_{2}\right)_{\beta}}{\left(\beta_{2}, \beta_{2}\right)} \ldots \ldots-\frac{\left(\alpha_{n}, \beta_{n-1}\right)}{\left(\beta_{n-1}, \beta_{n-1}\right)} \beta_{n-1}$

	\item 如果再把每个$\beta_i$ 单位化,即得到 V 的一组标准正交基
\end{itemize}

\end{frame}


\begin{frame}
设 ${\varepsilon}_{1}, {\varepsilon}_{2}, \cdots, {\varepsilon}_{n}$ 与 ${\eta}_{1}, {\eta}_{2}, \cdots, {\eta}_{n}$ 是欧氏空间 $V$ 中的两组标
准正交基,它们之间的过渡矩阵显 ${A}=\left(a_{i j}\right),$ 即
\[
\left(\eta_{1}, \eta_{2}, \cdots, \eta_{n}\right)=\left(\varepsilon_{1}, \varepsilon_{2}, \cdots, \varepsilon_{n}\right)\left(\begin{array}{cccc}
a_{11} & a_{12} & \cdots & a_{1 n} \\
a_{21} & a_{22} & \cdots & a_{2 n} \\
\vdots & \vdots & & \vdots \\
a_{n 1} & a_{n 2} & \cdots & a_{n n}
\end{array}\right)
\]
因为 ${\eta}_{1}, {\eta}_{2}, \cdots, {\eta}_{n}$ 是标准正交基,所以
\[
\left({\eta}_{i}, {\eta}_{j}\right)=\left\{\begin{array}{l}
1, \text { 当 } i=j \\
0, \text { 当 } i \neq j
\end{array}\right.
\]
矩阵 A 的各\alert{列}就是 ${\eta}_{1}, {\eta}_{2}, \cdots, {\eta}_{n}$ 在标准正交基 ${\varepsilon}_{1}, {\varepsilon}_{2}, \cdots, {\varepsilon}_{n}$ 下的坐标.
上式可以表示为
\[
a_{1 i} a_{1 j}+a_{2 i} a_{2 j}+\cdots+a_{j i} a_{n j}=\left\{\begin{array}{l}
1, \text { 当 } i=j \\
0, \text { 当 } i \neq j
\end{array}\right.
\]

\end{frame}


\begin{frame}
相当于一个矩阵的等式
\[
A^{\, \prime} A=E
\]
或者
\[
{A}^{-1}={A}^{\, \prime}
\]
我们引入: 
\begin{defi}
$\quad n$ 级实数矩阵 A 称为正交矩阵, 如果 A $^{\, \prime} {A}={E}$
\end{defi}
 因此,以上分析表明,
 \begin{itemize}
	\item 由标准正交基到标准正交基的过渡矩阵是正交矩阵;
 	\item 如果第一组基是标准正交基, 同时过渡矩阵是正交矩阵, 那么第二组基一定也是标准正交基.
 \end{itemize}

\end{frame}


\begin{frame}
\begin{rem}
根据逆矩阵的性质,由
\[
{A}^{\, \prime} {A}={E}
\]
即得
\[
A A^{\, \prime}=E
\]
写出来就是$A$的各\alert{行}满足
\[
a_{i 1} a_{j 1}+a_{i 2} a_{j 2}+\cdots+a_{i n} a_{j n}=\delta_{i j}, 
\]
其中 \[
\delta_{i j}=
\left\{\begin{array}{l}
1, \text { 当 } i=j \\
0, \text { 当 } i \neq j
\end{array}\right.
\]
\end{rem}

\end{frame}


\begin{frame}{正交矩阵之等价定义}
实矩阵 $A=\left(a_{i j}\right)_{n n}$ 为正交矩阵

 $\Leftrightarrow  \sum_{k=1}^{n} a_{k i} a_{k j}=\delta_{i j} $
 
 $\Leftrightarrow A^{-1}=A^{\, \prime}$
 
$\Leftrightarrow A A^{\, \prime}=E $

$\Leftrightarrow \sum_{k=1}^{n} a_{i k} a_{j k}=\delta_{i j}$

$\Leftrightarrow A$ 的行(列)向量组是 $\Rn$ 的一组标准正交基
\end{frame}


\begin{frame}{正交矩阵之性质}
\begin{itemize}
	\item 如果 $A$ 是正交矩阵,则 $|A|=\pm 1$.
	\item 如果 A 是正交矩阵,则 $A^{\, \prime}, A^{-1}, A^{*}, A^{k}$ 均是正交矩阵.
%	;而 A 是正交矩
%	阵的充分必要条件是 $l=\pm 1$
	\item 如果 $A, B$ 是 $n$ 级正交矩阵,则 $A B$ 也是正交矩阵.
	\item $n$级实矩阵 A 是正交矩阵的充分必要条件是,$A$的 $n$个列(或行) 向量是 两两正交的单位向量.
\end{itemize}

%\pf 设A是一个正交矩阵. 
%因为|A| = 土1,所以
%\[
%A^{*}=|A| A^{-1}=|A| A^{T}=\pm A^{T}
%\]
%那么
%\[
%A^{*}\left(A^{*}\right)^{T}=\left(\pm A^{T}\right)(\pm A)=A^{T} A=E
%\]
%故A*也是一个正交矩阵.
\end{frame}

\begin{frame}
标准正交基的有关结果总结如下:

设 $V$ 是 $n$ 维欧氏空间 $, \varepsilon_{1}, \varepsilon_{2}, \cdots, \varepsilon_{n}$ 是 $V$ 的一组标准正交基,则

1) 标准正交基的度量矩阵是单位矩阵

2) 设 $\alpha, \beta \in V,$ 且 $\alpha, \beta$ 在基$\varepsilon_1, \varepsilon_{2}, \cdots, \varepsilon_{n}$ 下的坐标分别为
\[
x=\left(x_{1}, x_{2}, \cdots, x_{n}\right)^{\, \prime}, \quad y=\left(y_{1}, y_{2}, \cdots, y_{n}\right)^{\, \prime}
\]
则
\[
(\alpha, \beta)=x_{1} y_{1}+x_{2} y_{2}+\cdots+x_{n} y_{n}=x^{\, \prime} y
\]

3) $V$ 中任一元素 $\alpha$ 在基$\varepsilon_{1}, \varepsilon_{2}, \cdots, \varepsilon_{n}$ 下的坐标为
\[
\left(\left(\alpha, \varepsilon_{1}\right),\left(\alpha, \varepsilon_{2}\right), \cdots,\left(\alpha, \varepsilon_{n}\right)\right)^{\, \prime}
\]

4) 由标准正交基到标准正交基的过渡矩阵是正交矩阵(即满足 $A^{\, \prime} A=E$ 的
$n$ 级实矩阵). 
又若两组基之间的过渡矩阵是正交矩阵,且其中一组基是标 准正
交基,则另一组基也是标准正交基 .
\end{frame}

\section{欧氏空间的同构}
\begin{frame}{$\S 3$ 欧氏空间的同构}
我们来建立欧氏空间同构的概念.  

\begin{defi}
 实数域 $\mathbb{R}$ 上欧氏空间 $V_1$与 $V_2$称为同构的,如果由 $V_1$ 到 $V_2$ 有一个双射 $\sigma$,满足
\begin{enumerate}
	\item $\sigma({\alpha}+{\beta})=\sigma({\alpha})+\sigma({\beta})$
	\item $\sigma(k {\alpha})=k \sigma({\alpha})$
	\item $(\sigma({\alpha}), \sigma({\beta}))=({\alpha}, {\beta})$
\end{enumerate}
这里 ${\alpha}, {\beta} \in V_1, k \in {\R},$ 这样的映射 $\sigma$ 称为 $V_1$ 到 $V_2$ 的同构映射. 
\end{defi}
由定义可知,如果 $\sigma$ 是欧氏空间 $V_1$ 到$V_2$的一个同构映射,那么 $\sigma$ 也是 $V_1$ 到 $V_2$作为线性空间的同构映射.
因此,同构的 欧氏空间必有相同的维数.
\end{frame}



\begin{frame}
设 $V_1$ 是一个 $n$ 维欧氏空间,在 $V_1$ 中取一组标准正交基 ${\varepsilon}_{1}, {\varepsilon}_{2}, \cdots, {\varepsilon}_{n} \cdot$ 在这组基下, $V_1$ 的每个向量 ${\alpha}$ 都可表成
\[
{\alpha}= {x}_{1} {\varepsilon}_{1}+x_{2} {{\varepsilon}}_{2}+\cdots+x_{n} {\varepsilon}_{n}
\]
令
\[
\sigma({a})=\left(x_{1}, x_{2}, \cdots, x_{n}\right) \in \mathbb{R}^n
\]

这是 V 到 $\mathbb{R}^n$ 的一个双射,并且适合定义中条件 1), 2).

上一节可知, $\sigma$ 也适合定义中条件 3).

因而 $\sigma$ 是 V 到 $\mathbb{R}^n$ 的一个同构映射.

由此可知,每个 $n$ 维的欧氏空间都与$\mathbb{R}^n$ 同构. 

\end{frame}


\begin{frame}
下面来证明,同构作为欧氏空间之间的关系具有反身性、对称性与传递性.

\begin{itemize}
\item  首先,每个欧氏空间到自身的恒等映射显然是一同构 映射 . 这就是说,同构关系是反身的.

\item 其次, 设 $\sigma$ 是 $V_1$ 到 $V_2$的一同 构映射,我们知道,逆映射 $\sigma^{-1}$ 也适合定义中 1)与 2) , 而且对于 ${\alpha}, {\beta} \in V_2$, 有
$$
\begin{aligned}
	({\alpha}, {\beta}) &=\left(\sigma\left(\sigma^{-1}({\alpha})\right), \sigma\left(\sigma^{-1}({\beta})\right)\right) \\
	&=\left(\sigma^{-1}({\alpha}), \sigma^{-1}({\beta})\right)
\end{aligned}
$$
这就是说, $\sigma^{-1}$ 是 V' 到 $V$ 的一同构映射, 因而同构关系是对称的. 

\item 第三,设 $\sigma, \tau$ 分别是 $V_1$到 $V_2$, $V_2$ 到 $V_3$ 的同构映射. 不难证明 $\tau \sigma$ 是 $V_1$ 到$V_3$的同构映射,因而同构关系是传递的.
\end{itemize}

\end{frame}


\begin{frame}

 既然每个 $n$ 维欧氏空间都与 $\mathbb{R}^n$同构, 按对称性与传递性即
得,任意两个$n$维欧氏空间都同构.
综上所述,就有 
\begin{thm}
两个有限维欧氏空间同构$\Leftrightarrow$ 它们的维数相同.
\end{thm}
这个定理说明, 抽象的观点看,欧氏空间的结构完全被它的 维数决定.

\end{frame}


\section{正交变换}
\begin{frame}{$\S 4$ 正交变换}
在解析几何中,我们有正交变换的概念.
正交变换就是保持点 之间的距离不变的变换.在一般的欧氏空间中,我们有 
\begin{defi}
欧氏空间 $V$ 的线性变换 $\mathscr{A}$
称为正交变换, 如果它保 持向量的内积不变,
即对于任意的 ${\alpha}, {\beta} \in V$ 都有 $$(\mathscr{A} {\alpha}, \mathscr{A} {\beta})=({\alpha}, {\beta})$$
\end{defi}
\end{frame}



\begin{frame}
正交变换可以从几个不同的方面来加以刻画.  

\begin{thm}
设 $\mathscr{A}$是 $n$ 维欧邸空间 $V$ 的一个线性变换, 于是下面 四个命题是相互等价的:
\begin{enumerate}
	\item $\mathscr{A}$是正交变换.
	\item $\mathscr{A}$保持向量的长度不变, 即对于 ${\alpha} \in V,|\mathscr{A} {\alpha}|=| {\alpha} |$.
	\item 如果 ${\varepsilon}_{1}, {\varepsilon}_{2}, \cdots, {\varepsilon}_{n}$ 是标准正交基,那么 $\mathscr{A} {\varepsilon}_{1}, \mathscr{A} {\varepsilon}_{2}, \cdots, \mathscr{A}{\varepsilon}_{n}$
	也是标准正交基.
	\item $\mathscr{A}$在任一组标准正交基下的矩阵是正交矩阵.
\end{enumerate}
\end{thm}

%证明 首先证明 1)与 2 等价.  如果 人是正交变换,那么
\end{frame}

\begin{frame}
因为正交矩阵是可逆的,所以正交变换是可逆的.

由定义不难 看出,正交变换实际上就是一个欧氏空间到它自身的同构映射, 因而
\begin{prop}
正交变换的乘积与正交变换的逆变换还是正交变换. 
\end{prop}


 在标准正交基下,正交变换与正交矩阵对应,因此,
\begin{prop}
正交矩阵的乘 积与正交矩阵的逆矩阵也是正交矩阵.  
\end{prop}
\end{frame}

\begin{frame}
如果 A 是正交矩阵,那么由
$
{A A}^{\, \prime}={E}
$
可知
$
|{A}|^{2}=1 \text { 或者 }|{A}|=\pm 1
$

因此,
\begin{prop}
正交变换的行列式等于 $+ 1$ 或者 $- 1$.
\end{prop}

\begin{itemize}
\item 行列式等于 $+1$ 的正交变换通常称为旋转,或者称为\alert{第一类}的;
\item 行列式等于 $- 1$ 的正交变 换称为\alert{第二类}的.  
\end{itemize}



例如,在欧氏空间中任取一组标准正交基 ${\varepsilon}_{1}, {\varepsilon}_{2}, \cdots, {\varepsilon}_{n},$ 定义
线性变换 $\mathscr{A}$为:
\[
\mathscr{A} {\varepsilon}_{1}=-{\varepsilon}_{1}, \mathscr{A} {\varepsilon}_{i}={\varepsilon}_{i}, i=2, \cdots, {n}
\]
那么, $\mathscr{A}$ 就是一个第二类的正交变换. 从几何上看,这是一个镜面 反射(参看本章习题 15). 
\end{frame}




\section{正交子空间}
\begin{frame}{\S 5 正交子空间}
我们来讨论欧氏空间中子空间的正交关系. 
\begin{defi}
设 $V_1$,$V_2$是欧氏空间 V 中两个子空间.如果对于 任意的 ${\alpha} \in V_{1}, {\beta} \in V_{2},$ 恒有
\[
({\alpha}, {\beta})= {0}
\]
则称 $V_{1}, V_{2}$ 为正交的,记为 $V_{1} \perp V_{2} .$ 一个向量 ${\alpha},$ 如果对于任意 的 ${\beta} \in V_{1},$ 恒有
\[
({\alpha}, {\beta})=0
\]
则称 ${\alpha}$ 与子空间 $V_{1}$ \alert{正交},记为 ${\alpha} \perp V_{1}$
\end{defi}
\end{frame}


\begin{frame}
因为只有零向量与它自身正交,所以由 $V_{1} \perp V_{2}$ 可知 $V_{1} \cap$ $V_{2}=\{\mathbf{0}\} ;$ 由 ${\alpha} \perp V_{1}, {\alpha} \in V_{1}$ 可知 ${\alpha}=\mathbf{0}$
关于正交的子空间,我们有: 
\begin{thm}
	如果子空间 $V_{1}, V_{2}, \cdots, V_{s}$ 两两正交,那么和 $V_{1}+$ $V_{2}+\cdots+V,$ 是直和. 
\end{thm}
\pf 设 ${\alpha}_{i} \in V_{i}, i=1,2, \cdots, s,$ 且
\[
{\alpha}_{1}+{\alpha}_{2}+\cdots+{\alpha}_{s}=\mathbf{0}
\]
 
我们来证明 ${\alpha}_{i}=\mathbf{0}$.  事实上,用 ${\alpha}_{i}$ 与等式两边作内积, 利用正交性,得
\[
\left({\alpha}_{i}, {\alpha}_{i}\right)=0
\]
从而 ${\alpha}_{i}=\mathbf{0}\, (i=1,2, \cdots, s) .$ 这就是说 $,$ 和
\[
V_{1}+V_{2}+\cdots+V_{s}
\]
是直和 . \qed
\end{frame}


\begin{frame}
\begin{defi}
子空间 $V_{2}$ 称为子空间 $V_{1}$ 的一个正交补,如果 $V_{1} \perp V_{2},$ 并且 $V_{1}+V_{2}=V$
\end{defi}
显然,如果 $V_{2}$ 是 $V_{1}$ 的正交补,那么 $V_{1}$ 也是 $V_{2}$ 的正交补.
\end{frame}


\begin{frame}
\begin{thm}
$n$ 维欧氏空间 V 的每一个子空间 $V_{1}$ 都有唯一的正 交补. 
\end{thm}
\pf 如果 $V_{1}=\{\mathbf{0}\},$ 那么它的正交补就是 $V$,唯一性是显
然的.

设 $V_{1} \neq\{\mathbf{0}\} .$ 欧氏空间的子空间在所定义的内积之下也显
下个欧氏空间.

在 $V_1$中取一组正交基 ${\varepsilon}_{1}, {\varepsilon}_{2}, \cdots, {\varepsilon}_{m}$, 由定理1,它
可以扑充成 $V$ 的一组正交基
\[
{\varepsilon}_{1}, {\varepsilon}_{2}, \cdots, {\varepsilon}_{m}, {\varepsilon}_{m+1}, \cdots, {\varepsilon}_{n}
\]
子空间 L $\left({\varepsilon}_{m+1}, \cdots, {\varepsilon}_{n}\right)$ 就是 $V_{1}$ 的正交补.
\end{frame}


\begin{frame}
再来证唯一性.设 $V_{2}, V_{3}$ 都是 $V_{1}$ 的正交补,于是
\[
\begin{array}{l}
V=V_{1} \oplus V_{2} \\
V=V_{1} \oplus V_{3}
\end{array}
\]
令 ${\alpha} \in V_{2},$ 由第二式即有
\[
{\alpha}={\alpha}_{1}+{\alpha}_{3}
\]
其中 ${\alpha}_{1} \in V_{1}, {\alpha}_{3} \in V_{3} .$ 因为 ${\alpha} \perp {\alpha}_{1}$ 所以
\[
\begin{aligned}
\left({\alpha}, {\alpha}_{1}\right) &=\left({\alpha}_{1}+{\alpha}_{3}, {\alpha}_{1}\right)=\left({\alpha}_{1}, {a}_{1}\right)+\left({\alpha}_{3}, {\alpha}_{1}\right) \\
&=\left({\alpha}_{1}, {\alpha}_{1}\right)=0
\end{aligned}
\]
即 ${\alpha}_{1}=\mathbf{0}$. 
由此得知 ${\alpha} \in V_{3},$ 即 $V_{2} \subset V_{3}$
同理可证 $V_{3} \subset V_{2} .$ 因此 $V_{2}=V_{3},$ 唯一性得证. 
\qed

\end{frame}


\begin{frame}
\begin{rem}
\begin{itemize}
\item $V_{1}$ 的正交补记为 $V_{1}^{\perp} .$ 
\item 由定义可知
$$
\text{dim}\left(V_{1}\right)+\text{dim} \left(V_{1}^{\perp}\right)=n
$$

\item $V_{1}^{\perp}$ 恰由所有与 $V_{1}$ 正交的向量组成.

\item  由分解式
\[
V=V_{1} \oplus V_{1}^{\perp}
\]
可知,V 中任一向量 ${\alpha}$ 都可以唯一地分解成
\[
{\alpha}={\alpha}_{1}+{\alpha}_{2}
\]
其中 ${\alpha}_{1} \in V_{1}, {\alpha}_{2} \in V_{1}^{\perp}$. 

我们称 ${\alpha}_{1}$ 为向量 ${\alpha}$ 在子空间 $V_{1}$ 上的\alert{内射影}.
\end{itemize}
\end{rem}

\end{frame}



\begin{frame}{${\xi} {6} $ 实对称知阵的惊准形}
\begin{itemize}
\item 在第五章我们得到,任意一个对称矩阵都合同于一个对角矩 阵,换句话说,都有一个可逆矩阵 C 使 ${C}^{\, \prime} {A} {C}$
成对角形.
现在利用欧氏空间的理论,第五章中关于实对称矩阵的结果可以加强.

\item 这一节的主要结果: 对于任意一个 n 级实对称矩阵 A,都存在一个 n 级正交矩陈
${T},$ 使
\[
{T}^{\, \prime} {A} {T}={T}^{-1} {A} {T}
\]
成对角形. 

\item 先讨论对称矩阵的一些性质,它们本身在今后也是非常有用
的. 我们把它们归纳成下面几个引理.
\end{itemize}





\end{frame}

\begin{frame}

\begin{lem}
设 A 是实对称矩阵,则 A 的特征值皆为实数. 
\end{lem}
\pf  设 $\lambda_{0}$ 是 $\mathbf{A}$ 的特征值,于是有非零向量
$\xi=\left(x_{1}, x_{2}, \ldots, x_{n}\right)^{\, \prime}$
满足
${A} {\xi}=\lambda_{0} {\xi}$, 
令
$\xi = \left( {x}_{1}, {x}_{2}, \ldots, {x}_{n} \right)^{\, \prime}$
其中 $\bar{x}_{i}$ 是 $x_{i}$ 的共轭复数,则 $\overline{A \xi}=\bar{\lambda}_{0} \bar{\xi}$
考察等式
\[
\bar{\xi}^{\, \prime}(A \xi)=\bar{\xi}^{\, \prime} A^{\, \prime} \xi=(A \bar{\xi})^{\, \prime} \xi=(\overline{A \xi})^{\, \prime} \xi
\]
其左边为 $\lambda_{0} \bar{\xi}^{\, \prime} \xi$, 右边为 $\bar{\lambda}_{0} \bar{\xi}^{\, \prime} \xi$. 故
\[
\lambda_{0} \bar{\xi}^{\, \prime} \xi = \bar{\lambda}_{0} \bar{\xi}^{\, \prime} \xi
\]
又因 $\xi$ 是非零向量,
\[
\bar{\xi}^{\, \prime} \xi=\bar{x}_{1} x_{1}+\bar{x}_{2} x_{2}+\cdots+\bar{x}_{n} x_{n} \neq 0
\]
故 $\lambda_{0}=\bar{\lambda}_{0},$ 即 $\lambda_{0}$ 是一个实数. 
\qed
\end{frame}


\begin{frame}

对应于实对称矩阵 $A$,在 $n$ 维欧氏空间 $\Rn$ 上定义一个线性 变换$\mathscr{A}$如下:
\begin{align}\label{eq-1}
\mathscr{A}\left(\begin{array}{c}
x_{1} \\
x_{2} \\
\vdots \\
x_{n}
\end{array}\right)={A}\left(\begin{array}{c}
x_{1} \\
x_{2} \\
\vdots \\
x_{n}
\end{array}\right)
\end{align}
于是,$\mathscr{A}$在标准正交基
\[
{\varepsilon}_{1}=\left(\begin{array}{c}
1 \\
0 \\
\vdots \\
0
\end{array}\right), {\varepsilon}_{2}=\left(\begin{array}{c}
0 \\
1 \\
\vdots \\
0
\end{array}\right), \cdots, {\varepsilon}_{n}=\left(\begin{array}{c}
0 \\
0 \\
\vdots \\
1
\end{array}\right)
\]
下的矩阵就是$A$. 
\end{frame}


\begin{frame}
\begin{lem}
设 $A$ 是实对称矩阵,公的定义如上,则对任意 ${\alpha}, {\beta}$ $\in \mathbf{R}^{n}$,有
\begin{align}\label{eq-3}
(\mathscr{A} {a}, {\beta})=({\alpha}, \mathscr{A} {\beta})
\end{align}
或
\[
{\beta}^{\, \prime}({A} {\alpha})={\alpha}^{\, \prime} {A} {\beta}
\]
\end{lem}
\pf 只要证明后一等式就行了.实际上\\
$\qquad \qquad \qquad \qquad 
{\beta}^{\, \prime}({A} {\alpha})={\beta}^{\, \prime} {A}^{\, \prime} {\alpha}=({A} {\beta})^{\, \prime} {\alpha}={\alpha}^{\, \prime}({A} {\beta}).
$
\qed 
等式\eqref{eq-3}把实对称矩阵的特性反映到线性变换上. 我们引入
\begin{defi}
欧氏空间中满足等式$(\mathscr{A} {a}, {\beta})=({\alpha}, \mathscr{A} {\beta})$的线性变换称为对称变
换. 
\end{defi}
对称变换在标准正交基下的矩阵是实对称矩阵. 
用对称变换来反映实对称矩阵,一些性质可以看得更清楚.
\end{frame}


\begin{frame}
\begin{lem}
设$\mathscr{A}$是对称变换,$V_1$ 是 $\mathscr{A}$-子空间,则 $V_{1}^{\perp}$ 也是
$\mathscr{A}$-子空间. 
\end{lem}

\pf 
设 ${a} \in V_{1}^{\perp},$ 要证 $\mathscr{A} {\alpha} \in V_{1}^{\perp},$ 即 $\mathscr{A} {\alpha} \perp V_{1} .$ 任取 ${\beta} \in$
$V_{1},$ 都有 $\mathscr{A} {\beta} \in V_{1} .$ 因 ${\alpha} \perp V_{1},$ 故 $({\alpha}, \mathscr{A} {\beta})=0$
因此
\[
(\mathscr{A} {a}, {\beta})=({\alpha}, \mathscr{A} {\beta})=0
\]
即 $\mathscr{A} {\alpha} \perp V_{1}, \mathscr{A} {\alpha} \in V_{1}^{\perp}, V_{1}^{\perp}$ 也是 $\mathscr{A}-$ 子空间.
\qed
\end{frame}

\begin{frame}
\begin{lem}
设 $\mathscr{A} $ 是实对称短阵,则 $\Rn$ 中属于$\mathscr{A} $的不同特征值 的特征向量必正交.
\end{lem}
\pf 
设 $\lambda, \mu$ 是 矩阵$A$ 的两个不同的特征值, ${\alpha}, {\beta}$ 分别是属于
$\lambda, \mu$ 的特征向量 $ A {\alpha}=\lambda {\alpha}, {A} {\beta}=\mu {\beta}$. 
定义$\Rn$上的线性变换 $\A$如下: $$\A X=AX,$$ 其中$X \in \Rn$. 
于是,
$\mathscr{A} {\alpha}=\lambda {\alpha}, \mathscr{A} {\beta}=\mu {\beta}$.
由 $(\mathscr{A} {\alpha}, {\beta})=({\alpha}, \mathscr{A} {\beta})$, 有
\[
\lambda({\alpha}, {\beta})=\mu({\alpha}, {\beta}).
\]
因为 $\lambda \neq \mu,$ 所以 $({\alpha}, {\beta})=0$, 即 ${\alpha}, {\beta}$ 正交.
\qed
\end{frame}


\begin{frame}
现在来证明主要定理.
\begin{thm}
对于任意一个 $n$ 级实对称矩阵 $A$ ,都存在一个 $n$ 级 正交矩阵 $T$,使$T^{\, \prime}AT = T ^{-1} {A T}$ 成对角形. 
\end{thm}
\pf 
由于实对称矩阵和对称变换的关系, 只要证明对称变 换 $\A$有 $n$ 个特征向量做成标准正交基就行了. 我们对空间的维数 $n$ 作归纳法.
\begin{itemize}
\item $n=1,$ 显然定理的结论成立. 
\item 设 $n-1$ 时定理的结论成立.
\item 对 $n$ 维欧氏空间 $\Rn$ ,线性变换$\A$有一特征向量 ${\alpha}_{1},$ 其特征值为实数$\lambda_{1} .$  把 ${\alpha}_{1}$ 单位化,还用 ${\alpha}_{1}$
代表它.作 $L ( {\a}_{1}$ )的正交补,设为 $V_{1}.$ 
\end{itemize}

\end{frame}


\begin{frame}
由引理 3, $V_{1}$ 是 $\mathscr{A}$ 的不变子空间, 其维数为 $n-1$.

因为$$(\A|_{V_{1}} {\a}, {\beta})=(\A {\a}, {\beta})=({\alpha}, \A {\beta})=({\alpha}, \A|_{V_{1}} {\beta}),$$ 其中 $\a, \b \in V_1$, 所以 $\A|_{V_{1}}$ 仍是对称变换.

据归纳法假设, $\A|_{V_{1}}$, 有 $n-1$ 个特征向量 ${\alpha}_{2}, \cdots, {\alpha}_{n}$ 作成 $V_{1}$ 的
标准正交基.
从而 ${\alpha}_{1}, {\alpha}_{2}, \cdots, {\alpha}_{n}$ 是 $\mathbb{R}^{n}$ 的标准正交基, 又是 $\mathscr{A}$ 有
$n$ 个特征向量. 定理得证. 
\qed
\begin{itemize}
\item  下面来看看在给定了一个实对称矩阵 $\A$ 之后,按什么办法求 正交矩阵 $T$ 使$T^{\, \prime}AT$ 成对角形. 
\item  在定理的证明中我们看到, 矩阵 A 在 $\Rn$中定义了一个线性变换.
\item  求正交矩阵 $T$ 的问题就 相当于在 $\Rn$ 中求一组由$A$ 的\alert{特征向量}构成的标准正交基.
\end{itemize}
\end{frame}


\begin{frame}


事实上,设
\[
\eta_{1}=\left(\begin{array}{c}
t_{11} \\
t_{21} \\
\vdots \\
t_{n 1}
\end{array}\right), \eta_{2}=\left(\begin{array}{c}
t_{12} \\
t_{22} \\
\vdots \\
t_{n 2}
\end{array}\right), \cdots, \eta_{n}=\left(\begin{array}{c}
t_{1 n} \\
t_{2 n} \\
\vdots \\
t_{n n}
\end{array}\right)
\]
是 $\Rn$ 的一组标准正交基, 它们都是 $A$ 的特征向量.
显然,由 ${\varepsilon}_{1}$
${\varepsilon}_{2}, \cdots, {\varepsilon}_{n}$ 到 ${\eta}_{1}, {\eta}_{2}, \cdots, {\eta}_{n}$ 的\alert{过渡矩阵}就是
\[
{T}=\left(\begin{array}{cccc}
t_{11} & t_{12} & \cdots & t_{1 n} \\
t_{21} & t_{22} & \cdots & t_{2 n} \\
\vdots & \vdots & & \vdots \\
t_{n 1} & t_{n 2} & \cdots & t_{n n}
\end{array}\right)
\]
$T$ 是一个正交矩阵,而
\[
{T}^{-1} {A} {T}={T}^{\, \prime} {A} {T}
\]
就是对角形.
\end{frame}

\begin{frame}
正交矩阵$T$的求法可以按以下步骤进行:



\begin{enumerate}
	\item 求出 $A$ 的特征值. 设 $\lambda_{1}, \cdots, \lambda,$ 是 ${A}$ 的全部不同的特征
	值. 
	\item 对于每个 $\lambda_{i},$ 解齐次线性方程组$\left(\lambda_{i} {E}-{A}\right)X=\mathbf{0}$
	求出一个基础解系,这就是 A 的特征子空间 $V_{\lambda_{i}}$ 的一组基.由这 组基出发,按定理 2 的方法求出 $V_{\lambda_{i}}$ 的一组标准正交基 ${\eta}_{i 1}, \cdots,{\eta}_{i k}$
	\item 因为 $\lambda_{1}, \cdots, \lambda_{r}$ 两两不同,所以根据这一节引理 4, 向量组
	${\eta}_{11}, \cdots, {\eta}_{1 k_{1}}, \cdots, {\eta}_{r 1}, \cdots, {\eta}_{r_{r}}$ 还是两两正交的.又根据定理 7 以及
	第七章 $\S$5 的讨论,它们的个数就等于空间的维数.因此,它们就 构成 $\Rn$ 的一组标准正交基,并且也都是 $A$ 的特征向量.这样,正 交矩阵 $T$ 也就求出了. 
\end{enumerate}
\end{frame}


\begin{frame}
\small{如果线性替换
\[
\left\{\begin{array}{l}
x_{1}=c_{11} y_{1}+c_{12} y_{2}+\cdots+c_{1 n} y_{n} \\
x_{2}=c_{21} y_{1}+c_{22} y_{2}+\cdots+c_{2 n} y_{n} \\
\cdots \cdots \cdots \cdots \\
x_{n}=c_{n 1} y_{1}+c_{n 2} y_{2}+\cdots+c_{n n} y_{n}
\end{array}\right.
\]
的矩阵 ${C}=\left(c_{i j}\right)$ 是正交的,那么它就称为正交的线性替换.
正交的线性替换当然是非退化的. 

用二次型的语言,定理 7 可以叙述为: 
\begin{thm}
任意一个实二次型
\[
\sum_{i=1}^{n} \sum_{j=1}^{n} a_{i j} x_{i} x_{j},  \quad  a_{i j}=a_{j i}
\]
都可以经过正交的线性替换变成平方和
\[
\lambda_{1} y_{1}^{2}+\lambda_{2} y_{2}^{2}+\cdots+\lambda_{n} y_{n}^{2}
\]
其中平方项的系数 $\lambda_{1}, \lambda_{2}, \cdots, \lambda_{n}$ 就是矩阵 ${A}$ 的特征多项式全部 的根. 
\end{thm}
}
\end{frame}

\end{document} 