% !Mode:: "TeX:UTF-8"
\documentclass[13pt]{beamer}
\usepackage[utf8]{inputenc}


\usepackage{amsmath,amssymb,amsthm}             % AMS Math
\usepackage[T1]{fontenc}

\usepackage{graphicx}
\usepackage{epstopdf}
\usepackage{tikz}
\linespread{1.3}

\usepackage{mathrsfs}  %花写字母
 

%%%=== theme ===%%%
% \usetheme{Warsaw}
%\usetheme{Copenhagen}
%\usetheme{Singapore}
\usetheme{Madrid}
%\usefonttheme{professionalfonts}
%\usefonttheme{serif}
% \usefonttheme{structureitalicserif}
%%\useinnertheme{rounded}
%%\useinnertheme{inmargin}
\useinnertheme{circles}
%\useoutertheme{miniframes}
\setbeamertemplate{navigation symbols}{}
\setbeamertemplate{footline}[page number]




\usepackage{ctex}
% \usepackage{CJK,CJKnumb,CJKulem}
\usepackage{minitoc}


\setbeamertemplate{theorems}[numbered]
\newtheorem{thm}{定理}
\newtheorem{lem}{引理}
\newtheorem{exa}{例}
\newtheorem*{defi}{定义}
\newtheorem*{coro}{推论}
\newtheorem*{ex}{练习}
\newtheorem*{rem}{注}
\newtheorem*{prop}{性质}

\def\qed{\nopagebreak\hfill{\rule{4pt}{7pt}}\medbreak}
\def\pf{{\bf 证明~~ }}
\def\sol{{\bf 解~~ }}
\def\R{\mathbb{R}}
\def\Rn{\mathbb{R}^n}
\def\A{\mathscr{A}}
\def\a{\alpha}
\def\b{\beta}
\def\r{\gamma}
\def\0{\mathbf{0}}
\def\dim{\operatorname{dim}}

\usepackage{color}
\definecolor{linkcol}{rgb}{0,0,0.4}
\definecolor{citecol}{rgb}{0.5,0,0}



  \usepackage{graphicx}
  \DeclareGraphicsExtensions{.eps}
%   \usepackage[a4paper,pagebackref,hyperindex=true,pdfnewwindow=true]{hyperref}


\begin{document}

\titlegraphic{\includegraphics[width=2cm]{../tjnu.jpg}} 

\title[]{第九章 \quad  欧几里得空间}
\author[]{{\large 张彪}\\  }
\institute[]{{\normalsize
		天津师范大学\\[6pt]
		zhang@tjnu.edu.cn}}

\date{}


\AtBeginSection[]
{
	\setcounter{exa}{0}
	\setcounter{equation}{0}
}


%\begin{frame}
%\maketitle
%\end{frame}

%\begin{frame}{Outline}
%\tableofcontents
%\end{frame}


\setcounter{exa}{0}
\begin{frame}
\begin{exa}
	设矩阵 $A, B$ 满足 $A B A^{*}=2 A B-8 E,$ 其中 $A=\left(\begin{array}{ccc}1 & 0 & 0 \\ 0 & -2 & 0 \\ 0 & 0 & 1\end{array}\right),$ 求 $B$.
\end{exa}

\begin{exa}
	设矩阵 $A, B$ 满足 $A^{*} B A=2 B A-8 E,$ 其中 $A=\left(\begin{array}{ccc}1 & 0 & 0 \\ 0 & -2 & 0 \\ 0 & 0 & 1\end{array}\right),$ 求 $B$.
\end{exa}
\end{frame}


\setcounter{exa}{0}

\begin{frame}
\begin{exa}
求非退化线性替换,化二次型 $$f\left(x_{1}, x_{2}, x_{3}\right)=x_{1}^{2}-3 x_{2}^{2}-2 x_{1} x_{2}+2 x_{1} x_{3}-6 x_{2} x_{3}$$ 为标准形.
\end{exa}
\begin{exa}
求非退化线性替换,化二次型 $$f\left(x_{1}, x_{2}, x_{3}\right)=4 x_{1} x_{2}-2 x_{1} x_{3}-2 x_{2} x_{3}$$ 为标准形.
\end{exa}
\end{frame}

\begin{frame}
\begin{exa}
若实二次型 $f\left(x_{1}, x_{2}, x_{3}\right)=2 x_{1}^{2}+x_{2}^{2}+x_{3}^{2}-2 t x_{1} x_{2}+2 t x_{1} x_{3}$ 正定,求 $t$ 的取值范围.
\end{exa}
\end{frame}


\setcounter{exa}{0}
\begin{frame}

\begin{exa}
已知 $$f_{1}=1-x, f_{2}=1+x^{2}, f_{3}=x+2 x^{2}$$ 与 $$g_{1}=x, g_{2}=1-x^{2}, g_{3}=1-x+x^{2}$$ 是 $P[x]_{3}$ 中的两个向量组.
	\begin{enumerate}
		\item 证明 $f_{1}, f_{2}, f_{3}$ 和 $g_{1}, g_{2}, g_{3}$ 都是 $P[x]_{3}$ 的基.
		\item 求由基 $f_{1}, f_{2}, f_{3}$ 到基 $g_{1}, g_{2}, g_{3}$ 的过渡矩阵.
		\item 求 $f=1+2 x+3 x^{2}$ 在基 $f_{1}, f_{2}, f_{3}$ 下的坐标.
	\end{enumerate}
\end{exa}
\end{frame}



\begin{frame}
\small{\begin{exa}
	\begin{enumerate}
		\item 设 $A=\left(\begin{array}{ccc}1 & 0 & 0 \\ 1 & 1 & 1 \\ 0 & 0 & 2\end{array}\right),$ 记 $W=\left\{B\, | \, A B=B A, B \in P^{3 \times 3}\right\},$ 求 $W$ 的维数和一组基.
		\item 设 $A=\left(\begin{array}{ccc}1 & 1 & 0 \\ 1 & 1 & 1 \\ 0 & 1 & 1\end{array}\right),$ 记 $W=\left\{B \, |\, A B=B A, B \in P^{3 \times 3}\right\},$ 求 $W$ 的维数和一组基.
	\end{enumerate}
\end{exa}}
\end{frame}

\begin{frame}
\begin{exa}已知两个齐次线性方程组
\begin{center}
\text {(I)} $\left\{\begin{array}{c}x_{1}+2 x_{2}+x_{3}=0 \\ 2 x_{1}+2 x_{2}+x_{4}=0\end{array}\right.$ \qquad
%\end{center}
%\begin{center}
\text {(II)} $\left\{\begin{array}{l}-2 x_{1}+x_{2}+6 x_{3}-x_{4}=0 \\ -x_{1}+2 x_{2}+5 x_{3}-x_{4}=0\end{array}\right.$
\end{center}
\begin{enumerate}
\item 分别求(I)和(II)的解空间 $V_{1}$ 和 $V_{2}$ 的维数和一组基.

\item 求 $V_{1} \cap V_{2}$ 的维数和一组基.
\end{enumerate}
\end{exa}
\end{frame}


\setcounter{exa}{0}

\begin{frame}
\begin{exa}
	取 $P^{3}$ 的线性变换 $\sigma(a, b, c)=(a-b, b-c, a+b)$
	\begin{enumerate}
		\item  求 $\sigma$ 在基 $\varepsilon_{1}=(1,0,0), \varepsilon_{2}=(0,1,0), \varepsilon_{3}=(0,0,1)$ 下的矩阵
		\item  求 $\sigma$ 在基 $\eta_{1}=(1,0,0), \eta_{2}=(1,1,0), \eta_{3}=(1,1,1)$ 下的矩阵
		\item  求向量 $\alpha=(1,2,3)$ 的像 $\sigma \alpha$ 分别在基 $\varepsilon_{1}, \varepsilon_{2}, \varepsilon_{3}$ 和 $\eta_{1}, \eta_{2}, \eta_{3}$ 下的坐标.
	\end{enumerate}
\end{exa}
\end{frame}


\begin{frame}
\begin{exa}
	在线性空间 $P^{3}$ 中,定义线性变换 $$\sigma(a, b, c)=(a+2 b-c, b+c, a+b-2 c),$$ 分别求 $\sigma$ 的值域与核的维数与一组基.
\end{exa}
\end{frame}


	\setcounter{exa}{0}

\begin{frame}
\begin{exa}
设 $A=\left(\begin{array}{ccc}2 & -1 & -1 \\ -1 & 2 & -1 \\ -1 & -1 & 2\end{array}\right)$.

(1) 求 $A$ 的特征值和对应的特征向量.

(2) 求正交阵 $Q,$ 使得 $Q^{T} A Q$ 是对角阵.
\end{exa}

\begin{exa}
 用正交线性替换化实二次型为标准形 $f\left(x_{1}, x_{2}, x_{3}\right)=2 x_{1}^{2}+2 x_{2}^{2}+2 x_{3}^{2}-2 x_{1} x_{2}-2 x_{1} x_{3}-2 x_{2} x_{3}$.
\end{exa}
\end{frame}

\begin{frame}
\begin{exa}
	设1,1,-3是3阶实对称矩阵 $A$ 的特征值,  $(1,-1,0)'$ 是 $A$ 属于-3的特征向量,求 $A$.
\end{exa}
\end{frame}





\end{document} 