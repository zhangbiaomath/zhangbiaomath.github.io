% !Mode:: "TeX:UTF-8"
\documentclass[13pt]{beamer}
\usepackage[utf8]{inputenc}


\usepackage{amsmath,amssymb,amsthm}             % AMS Math
\usepackage[T1]{fontenc}

\usepackage{graphicx}
\usepackage{epstopdf}
\usepackage{tikz}
\linespread{1.3}

\usepackage{mathrsfs}  %花写字母
 

%%%=== theme ===%%%
% \usetheme{Warsaw}
%\usetheme{Copenhagen}
%\usetheme{Singapore}
\usetheme{Madrid}
%\usefonttheme{professionalfonts}
%\usefonttheme{serif}
% \usefonttheme{structureitalicserif}
%%\useinnertheme{rounded}
%%\useinnertheme{inmargin}
\useinnertheme{circles}
%\useoutertheme{miniframes}
\setbeamertemplate{navigation symbols}{}
\setbeamertemplate{footline}[page number]




\usepackage{ctex}
% \usepackage{CJK,CJKnumb,CJKulem}
\usepackage{minitoc}


\setbeamertemplate{theorems}[numbered]
\newtheorem{thm}{定理}
\newtheorem{lem}{引理}
\newtheorem{exa}{例}
\newtheorem*{theo}{定理}
\newtheorem*{defi}{定义}
\newtheorem*{coro}{推论}
\newtheorem*{ex}{练习}
\newtheorem*{rem}{注}
\newtheorem*{prop}{性质}
\newtheorem*{qst}{问题}

\def\qed{\nopagebreak\hfill{\rule{4pt}{7pt}}\medbreak}
\def\pf{{\bf 证明~~ }}
\def\sol{{\bf 解~~ }}



\def\R{\mathbb{R}}
\def\Rn{\mathbb{R}^n}
\def\A{\mathscr{A}}
\def\B{\mathscr{B}}
\def\D{\mathscr{D}}
\def\E{\mathscr{E}}
\def\O{\mathscr{O}}

\def\rank{\operatorname{rank}}
\def\dim{\operatorname{dim}}
\def\0{\mathbf{0}}
\def\a{\alpha}
\def\b{\beta}
\def\r{\gamma}

\usepackage{color}
\definecolor{linkcol}{rgb}{0,0,0.4}
\definecolor{citecol}{rgb}{0.5,0,0}

\definecolor{blue}{rgb}{0,0.08,1}
\newcommand{\blue}{\textcolor{blue}}

  \usepackage{graphicx}
  \DeclareGraphicsExtensions{.eps}
%   \usepackage[a4paper,pagebackref,hyperindex=true,pdfnewwindow=true]{hyperref}


\begin{document}



\title[]{第五章 \quad 二次型}
\author[]{{\large 张彪}\\  }
\institute[]{{\normalsize
		天津师范大学\\[6pt]
		zhang@tjnu.edu.cn}}

\date{}


\AtBeginSection[]
{
\setcounter{exa}{0}
\setcounter{equation}{0}
}


\begin{frame}
\maketitle
\end{frame}

\begin{frame}{Outline}
	\tableofcontents
\end{frame}


\begin{frame}
二次型就是二次齐次多项式. 在解析几何中讨论的有心二次曲线,
当中心与坐标原点重合时,其一般方程为:
\[
a x^{2}+2 b x y+c y^{2}=f
\]
方程的右端就是关于 x, y 的一个二次齐次多项式. 为了便于研究
这个二次曲线的几何性质,通过选取合适的角度,把坐标轴作逆
时针旋转,则相应的坐标变换为:
\[
\left\{\begin{array}{l}
x=x^{\, \prime} \cos \theta-y^{\, \prime} \sin \theta \\
y=x^{\, \prime} \sin \theta+y^{\, \prime} \cos \theta
\end{array}\right.
\]
在新坐标下二次曲线的方程可化为标准方程:
\[
a^{\, \prime} x^{\, \prime 2}+c^{\, \prime} y^{\, \prime 2}=f
\]
这是一个只含有平方项的标准方程. 
\end{frame}

\begin{frame}
考察方程:
\[
\frac{13}{72} x^{2}+\frac{10}{72} x y+\frac{13}{72} y^{2}=1
\]
该方程表示 xy 平面上怎样的一条二次曲线?

将 $x y$ 坐标系逆时针旋转 $\frac{\pi}{4}$, 即令
\[
\left\{\begin{array}{l}
x=\frac{\sqrt{2}}{2} x^{\, \prime}-\frac{\sqrt{2}}{2} y^{\, \prime} \\
y=\frac{\sqrt{2}}{2} x^{\, \prime}+\frac{\sqrt{2}}{2} y^{\, \prime}
\end{array}\right.
\]
在新坐标下二次曲线的方程可化为标准方程:
\[
\frac{x^{\, \prime 2}}{4}+\frac{y^{\, \prime 2}}{9}=1
\]
\end{frame}

\section{ 二次型及其矩阵表示}
\begin{frame}{\S 1   二次型及其矩阵表示}

\begin{defi}
一个系数在数域 $P$上的 $x_{1}, x_{2}, \cdots, x_{n}$ 的二次齐次多项式
\[
\begin{aligned}
f\left(x_{1}, x_{2}, \cdots, x_{n}\right) &=a_{11} x_{1}^{2}+2 a_{12} x_{1} x_{2}+\cdots+2 a_{1 n} x_{1} x_{n} \\
& \quad +a_{22} x_{2}^{2}+\cdots+2 a_{2 n} x_{2} x_{n}+\cdots+a_{n n} x_{n}^{2}
\end{aligned}
\]
称为数域 $P$ 上的一个$n$元二次型,简称为二次型. 
\end{defi}
注意:

(1) 二次型就是 $n$ 元二次齐次多项式;

(2) 交叉项的系数采用$2a_{ij}$,主要是为了矩阵表示的方便. 
\end{frame}


\begin{frame}
若在 $n$ 元二次型中令 $a_{i j}=a_{j i},$ 由于 $x_{i} x_{j}=x_{j} x_{i},$ 则二次型可表示为
\[
f\left(x_{1}, x_{2}, \cdots, x_{n}\right)=\sum_{i=1}^{n} \sum_{j=1}^{n} a_{i j} x_{i} x_{j}
\]
若记
\[
A=\left(\begin{array}{cccc}
a_{11} & a_{12} & \cdots & a_{1 n} \\
a_{21} & a_{22} & \cdots & a_{2 n} \\
\cdots & \cdots & \cdots & \cdots \\
a_{n 1} & a_{n 2} & \cdots & a_{n n}
\end{array}\right), \quad
X=\left(\begin{array}{c}
x_{1} \\
x_{2} \\
\vdots \\
x_{n}
\end{array}\right)
\]
其中 $a_{i j}=a_{j i}, i, j=1,2, \cdots, n,$ 则二次型可用矩阵的乘积表示为
\[
f\left(x_{1}, x_{2}, \cdots, x_{n}\right)=X^{\, \prime} A X
\]
其中 $A$ 称为该\alert{二次型的矩阵},$A$ 的秩称为该\alert{二次型的秩}. 
\end{frame}

\begin{frame}
对于二次型的矩阵表示方法,需注意如下几点:

(1) 由于 $a_{i j}=a_{j i},$ 故 $A$ 为对称矩阵

(2) 矩阵 A 中 $a_{i i}$ 为 $x_{i}^{2}$ 项的系数, $a_{i j}$ 为交叉项 $x_{i} x_{j}$ 系数的

(3) $n$ 元二次型 $f$ 与  $n$ 阶\alert{对称矩阵} $A$ 一一对应

\begin{exa}
	写出二次型 $f=x_{1}^{2}+2 x_{2}^{2}-3 x_{3}^{2}+4 x_{1} x_{2}-6 x_{2} x_{3}$
	的矩阵.
\end{exa}
\pause
%$a_{11}=1, a_{22}=2, a_{33}=-3, $
%
%\qquad $a_{12}=a_{21}=2, a_{13}=a_{31}=0, a_{23}=a_{32}=-3$\\[6pt]

$$ A=\left(\begin{array}{ccc}1 & 2 & 0 \\ 2 & 2 & -3 \\ 0 & -3 & -3\end{array}\right)$$

%$ f\left(x_{1}, x_{2}, x_{3}\right)=\left(x_{1}, x_{2}, x_{3}\right) A\left(\begin{array}{c}x_{1} \\ x_{2} \\ x_{3}\end{array}\right)

\end{frame}

\begin{frame}
\begin{exa}
写出下列对称矩阵的二次型

(1) $\left(\begin{array}{lll}0 & 0 & 1 \\ 0 & 1 & 0 \\ 1 & 0 & 0\end{array}\right)$
(2) $\left(\begin{array}{lll}1 & 1 & 2 \\ 1 & 2 & 3 \\ 2 & 3 & 3\end{array}\right)$
\end{exa}

\begin{exa}
写出二次型 $f\left(x_{1}, x_{2}\right)=X^{\,\prime}\left(\begin{array}{cc}2 & 1 \\ 3 & 1\end{array}\right) X$ 的矩阵. 
\end{exa}
\end{frame}


\begin{frame}{线性替换} 
\begin{defi}
系数在数域 P中的一组关系式:
\[
\left\{\begin{array}{cc}
x_{1}=c_{11} y_{1}+c_{12} y_{2}+\cdots+c_{1 n} y_{n} \\
x_{2}=c_{21} y_{1}+c_{22} y_{2}+\cdots+c_{2 n} y_{n} \\
\cdots & \cdots \\
x_{n}=c_{n 1} y_{1}+c_{n 2} y_{2}+\cdots+c_{n n} y_{n}
\end{array}\right.
\]
称为由向量 $x_{1}, x_{2}, \cdots, x_{n}$ 到 $y_{1}, y_{2}, \cdots, y_{n}$ 的一个\alert{线性替换}.

令
$
\quad X=\left(x_{1}, x_{2}, \cdots, x_{n}\right)^{\, \prime}, 
\quad Y=\left(y_{1}, y_{2}, \cdots, y_{n}\right)^{\,\prime}
$,

则线性替换可以表示为 $X=C Y$.  

若系数矩阵 $C$ 的行列式 $|C| \neq 0,$ 则称
该线性替换是\alert{非退化}的.
\end{defi}
\end{frame}

\begin{frame}
二次型 $f=X^{\, \prime} A X$ 经可逆变换 $X := C Y$ 后, 有
\[
f=(C Y)^{\prime} A(C Y\, )=Y^{\, \prime}\left(C^{\, \prime} A C \right) Y
\]
得到一个新二次型,矩阵由$A$变为 $B=C^{\, \prime} A C$.

\begin{defi}
设$A$, $B$是$n$阶矩阵,如果存在一个$n$阶可逆矩
阵$C$,使得 $\quad {B}={C}^{\, \prime} {A} {C}$
则称 $B$ 与 $A$是合同的. 
\end{defi}

\begin{rem}
\begin{itemize}
\item 	合同是矩阵之间的一个等价关系, 这时因为合同关系满足
\begin{enumerate}
\item  反身性 $: A=E^{\, \prime} A E$
\item  对称性:由 ${B}={C}^{\, \prime} {A C}$ 即得 ${A}=\left({C}^{-1}\right)^{\prime} {B} {C}^{-1}$
\item  传递性:由 ${A}_{1}={C}_{1}^{\, \prime} {A C}_{1}$ 和 ${A}_{2}={C}_{2}^{\, \prime} {A}_{1} {C}_{2}$ 即得
${A}_{2}=\left({C}_{1} {C}_{2}\right)^{\prime} {A}\left({C}_{1} {C}_{2}\right)$
\end{enumerate}
\item 合同矩阵具有相同的秩. 
\item 若$A$为对称矩阵,则$B$也为对称矩阵. 
\end{itemize}
\end{rem}
\end{frame}



\section{标准形}
\begin{frame}{\S 2  标准形}
\begin{defi}
	一个只含有平方项的 n 元二次型
	\[
	f\left(x_{1}, x_{2}, \cdots, x_{n}\right)=d_{1} x_{1}^{2}+d_{2} x_{2}^{2}+\cdots+d_{n} x_{n}^{2}
	\]
	称为\alert{标准二次型}.
\end{defi}

要使二次型$f$经非退化线性变换 $X=C Y$ 变成标准形 就是要使
\[
\begin{array}{l}
f=Y^{\, \prime}\left(C^{\, \prime} A C\right) Y=k_{1} y_{1}^{2}+k_{2} y_{2}^{2}+\cdots+k_{n} y_{n}^{2} \\
=\left(y_{1}, y_{2}, \cdots, y_{n}\right)\left(\begin{array}{cccc}
k_{1} & &&\\
&k_{2} &&\\
& & \ddots &\\
& &&k_{n}\\
\end{array}\right)\left(\begin{array}{c}
y_{1} \\
y_{2} \\
\vdots \\
y_{n}
\end{array}\right)
\end{array}
\]
也就是要使 $C^{\, \prime} A C$ 成为对角矩阵.
\end{frame}

\begin{frame}
\begin{thm}
数域 $P$ 上任意一个二次型都可以经过非退化的线性 替换变成标准形.
\end{thm}

\begin{thm}
数域 $P$ 上任意一个对称矩阵都合同于一个对角矩阵. 
\end{thm}

\end{frame}

\begin{frame}
\pf 下面的证明实际上是一个具体地把二次型化成平方和 的方法,这就是中学里学过的“配方法” 我们对变量的个数 $n$ 作归纳法. 
\begin{itemize}
\item 对于 $n=1,$ 二次型就是
\[
f\left(x_{1}\right)=a_{11} x_{1}^{2}
\]
已经是平方和了.
\item  现假定对 $n-1$ 元的二次型,定理的结论成立.  
\item 再设 \[
f\left(x_{1}, x_{2}, \cdots, x_{n}\right)=\sum_{i=1}^{n} \sum_{j=1}^{n} a_{i j} x_{i} x_j, \quad\left(a_{i j}=a_{j i}\right)
\]
分三种情形来讨论:

1) $a_{i i}\, (1 \le i\le n)$ 中至少有一个不为零.

2) 所有 $a_{i i}=0,$ 但是至少有一 $a_{1 j} \neq 0(2 \le j \le n)$.

3) $a_{11}=a_{12}=\dots=a_{1 n}=0$.
\end{itemize}
\end{frame}

\begin{frame}
1) $a_{i i}(i=1,2, \cdots, n)$ 中至少有一个不为零, 例如 $a_{11} \neq 0 .$ 这
时
\begin{align*}
f &=a_{11} x_{1}^{2}+\sum_{j=2}^{n} a_{1j} x_{1} x_{j}+\sum_{i=2}^{n} a_{i 1} x_{i} x_{1} +\sum_{i=2}^{n} \sum_{j=2}^{n} a_{n} x_{i} x_{j} \\
&=a_{11} x_{1}^{2}+2 \sum_{j=2}^{n} a_{1j} x_{1} x_{j}+\sum_{i=2}^{n} \sum_{j=2}^{n} a_{i, j} x_{i} x_j \\
&=a_{11}\left(x_{1}+\sum_{j=2}^{n} a_{11}^{-1} a_{1 j} x_{j}\right)^{2}-a_{11}^{-1}\left(\sum_{j=2}^{n} a_{1 j} x_{j}\right)^{2}  +\sum_{i=2}^{n} \sum_{j=2}^{n} a_{i j} x_{i} x_{j} \\
& =a_{11}\left(x_{1}+\sum_{j=2}^{n} a_{11}^{-1} a_{1j }x_{j}\right)^{2}+\sum_{i=2}^{n} \sum_{j=2}^{n} b_{i j} x_{i} x_{j}
\end{align*}
这里
\[
\sum_{i=2}^{n} \sum_{j=2}^{n} b_{i j} x_{i} x_{j}
=-a_{11}^{-1}\left(\sum_{j=2}^{n} a_{1 j} x_j \right)^{2}+\sum_{i=2}^{n} \sum_{j=2}^{n} a_{i j} x_{i} x_j
\]
是一个 $x_{2}, x_{3}, \cdots, x_{n}$ 的二次型.
\end{frame}

\begin{frame}
\begin{center}
令
$
\left\{\begin{array}{l}
y_{1}=x_{1}+\sum_{j=2}^{n} a_{11}^{-1} a_{1}, x_{j} \\
y_{2}=x_{2} \\
\cdots \cdots \cdots \cdots \\
y_{n}=x_{n}
\end{array}\right.
$
即
$
\left\{\begin{array}{l}
x_{1}=y_{1}-\sum_{j=2}^{n} a_{11}^{-1} a_{1, y} \\
x_{2}=y_{2} \\
\cdots \cdots \cdots \cdots \\
x_{n}=y_{n}
\end{array}\right.
$
\end{center}

这是一个非退化线性替换,它使
\[
f\left(x_{1}, x_{2}, \cdots, x_{n}\right)=a_{11} y_{1}^{2}+\sum_{i=2}^{n} \sum_{j=2}^{n} b_{ij}  y_i y_{j}
\]
由归纳法假定,对 $\sum_{i=2}^{n} \sum_{j=2}^{n} b_{i j} y_{i} y_j,$ 有非退化线性替换
\[
\left\{\begin{array}{l}
z_{2}=c_{22} y_{2}+c_{23} y_{3}+\cdots+c_{2 n} y_{n} \\
z_{3}=c_{32} y_{2}+c_{33} y_{3}+\cdots+c_{3 n} y_{n} \\
\cdots \ldots \ldots . \\
z_{n}=c_{n 2} y_{2}+c_{n 3} y_{3}+\cdots+c_{n n} y_{n}
\end{array}\right.
\]
能使它变成平方和
$
d_{2} z_{2}^{2}+d_{3} z_{3}^{2}+\cdots+d_{n} z_{n}^{2}.
$
\end{frame}


\begin{frame}
于是非退化线性替换
\[
\left\{\begin{array}{l}
z_{1}=y_{1} \\
z_{2}=c_{22} y_{2}+\cdots+c_{2 n} y_{n} \\
\cdots \cdots \cdots \cdots \\
z_{n}=c_{n 2} y_{2}+\cdots+c_{n n} y_{n}
\end{array}\right.
\]

就使 $f\left(x_{1}, x_{2}, \cdots, x_{n}\right)$ 变成
\[
f\left(x_{1}, x_{2}, \cdots, x_{n}\right)=a_{11} z_{1}^{2}+d_{2} z_{2}^{2}+\cdots+d_{n} z_{n}^{2}
\]
即变成平方和了.根据归纳法原理, 定理得证. 
\end{frame}

\begin{frame}
2) 所有 $a_{i i}=0,$ 但是至少有一 $a_{1 j} \neq 0(j>1),$ 不失普遍性, 设 $a_{12} \neq 0 .$ 令
\[
\left\{\begin{array}{l}
x_{1}=z_{1}+z_{2} \\
x_{2}=z_{1}-z_{2} \\
x_{3}=z_{3} \\
\dots \dots \dots \\
x_{n}=z_{n}
\end{array}\right.
\]
它是非退化线性替换,且使
\[
\begin{aligned}
f\left(x_{1}, x_{2}, \cdots, x_{n}\right) &=2 a_{12} x_{1} x_{2}+\cdots \\
&=2 a_{12}\left(z_{1}+z_{2}\right)\left(z_{1}-z_{2}\right)+\cdots \\
&=2 a_{12} z_{1}^{2}-2 a_{12} z_{2}^{2}+\cdots
\end{aligned}
\]
这时上式右端是 $z_{1}, z_{2}, \cdots, z_{n}$ 的二次型,且 $z_{1}^{2}$ 的系数不为零, 属 于第一种情况, 定理成立. 
\end{frame}


\begin{frame}
3) $a_{11}=a_{12}=\dots=a_{1 n}=0$.
由于对称性,有
\[
a_{21}=a_{31}=\dots=a_{n 1}=0
\]
这时
\[
f\left(x_{1}, x_{2}, \cdots, x_{n}\right)=\sum_{i=2}^{n} \sum_{j=2}^{n} a_{i j} x_{i} x_{j}
\]
是 $n-1$ 元二次型.

根据归纳法假定,它能用非退化线性替换变成
平方和. 
\qed
\end{frame}


\begin{frame}{配方法}
用配方法化二次型为标准形的关键是消去交叉项

\blue{情形 1}~~ 如果二次型 $f\left(x_{1}, x_{2}, \cdots, x_{n}\right)$ 含$x_1$ 的平方项,而 $a_{11} \neq 0$
则集中二次型中含 $x_1$ 的所有交叉项, 然后与 $x_i$ 配方,并作非退化线性替换
\[
\left\{\begin{array}{l}
y_{1}=a_{1} x_{1}+c_{12} x_{2}+\cdots+a_{n} x_{n} \\
y_{2}= \quad x_2\\
\cdots \cdots \\
y_{n}= \quad x_n
\end{array} \right.
\]
得 $f=d_{1} y_{1}^{2}+g\left(y_{2}, \cdots, y_{n}\right),$ 其中 $g\left(y_{2}, \cdots, y_{n}\right)$ 是 $y_{2}, \cdots, y_{n}$ 的二次型 $\cdot$ 对
$g\left(y_{2}, y_{3}, \cdots, y_{n}\right)$ 重复上述方法直到化二次型 $f$ 为标准形为止.
\end{frame}
\begin{frame}
\blue{情形 2}~~如果二次型 $f\left(x_{1}, x_{2}, \cdots, x_{n}\right)$ 不含平方项, 即 $a_{i i}=0(i=1,2,$
$\cdots, n)$,但含某一个 $a_{i j} \neq 0(i \neq j),$ 则可先作非退化线性替换
$$\left\{\begin{aligned} x_{i} &=y_{i}+y_{j} \\ x_{j} &=y_{i}-y_{j} \\ x_{k} &=y_{k} \quad(k=1,2, \cdots, n ; k \neq i, j) \end{aligned}\right.$$

把 f化为一个含平方项 $y_{i}^{2}$ 的二次型,再用情形 1 的方法化为标准形.
\end{frame}

%\begin{enumerate}
%\item  若二次型含有 $x_{i}$ 的平方项,则先把含有
%${x}_{i}$ 的乘积项集中,然后配方,再对其余的变量同 样进行,直到都配成平方项为止,经过非退化线 性变换,就得到标准形;
%\item 若二次型中不含有平方项,但是 $a_{i j} \neq {0}$ $(i \neq j)$,则先做可逆线性变换
%\[
%\left\{\begin{array}{ll}
%x_{i}& =y_{i}-y_{j} \\
%x_{j}& =y_{i}+y_{j} \\
%x_{k}& =y_{k}
%\end{array}\quad (k=1,2, \cdots, n \text { 且  } k \neq i, j)\right.
%\]
%化二次型为含有平方项的二次型,然后再按1中方 法配方.
%\end{enumerate}

\begin{frame}
\begin{exa}
用非退化线性变换化二次型 $$f\left(x_{1}, x_{2}, x_{3}\right)=2 x_{1}^{2}+4 x_{1} x_{2}+6 x_{1} x_{3}+3 x_{2}^{2}-\frac{1}{2} x_{3}^{2}$$ 为标准形, 并写出所用的非退化线性变换.
\end{exa}
\sol (配方法)
\begin{align*}
f\left(x_{1}, x_{2}, x_{3}\right)
& = 2 \blue{x_{1}^{2}}+ \blue{x_{1}} \left(\alert{4 x_{2}+6 x_{3}}\right) +3 x_{2}^{2}-\frac{1}{2} x_{3}^{2} \\
&=2\left(\blue{x_{1}}+\alert{x_{2}+\frac{3}{2} x_{3}}\right)^{2}+3 x_{2}^{2}-\frac{1}{2} x_{3}^{2}-2\left(\alert{x_{2}+\frac{3}{2} x_{3}}\right)^{2}\\
& =2\left(x_{1}+x_{2}+\frac{3}{2} x_{3}\right)^{2}+\blue{x_{2}^{2}}-6 \blue{x_{2}} \alert{x_{3}} -5 x_{3}^{2}\\
&=2\left(x_{1}+x_{2}+\frac{3}{2} x_{3}\right)^{2}+\left(x_{2}-3 x_{3}\right)^{2}-14 x_{3}^{2} 
\end{align*}
\end{frame}


\begin{frame}
令
$\left\{\begin{array}{l}y_{1}=x_{1}+x_{2}+\frac{3}{2} x_{3} \\ y_{2}=x_{2}-3 x_{3} \\ y_{3}=x_{3}\end{array}\right.$
即
$\left\{\begin{array}{l}x_{1}=y_{1}-y_{2}-\frac{9}{2} y_{3} \\ x_{2}=y_{2}+3 y_{3} \\ x_{3}=y_{3}\end{array}\right.$
\\[6pt]
则原二次型化为 
$g(Y)=2 y_{1}^{2}+y_{2}^{2}-14 y_{3}^{2}$.

\end{frame}

\begin{frame}
\begin{exa}
 化二次型
\[
f=2 x_{1} x_{2}+2 x_{1} x_{3}-6 x_{2} x_{3}
\]
成标准形,并求所用的变换矩阵. 
\end{exa}
\sol 由于所给二次型中无平方项,所以

%\begin{center}
令 $\left\{\begin{array}{l}x_{1}=y_{1}+y_{2}, \\ x_{2}=y_{1}-y_{2}, \\ x_{3}=y_{3}, \end{array}\right.$\quad 
 即  
	$\left(
	\begin{array}{l}
	x_{1} \\ x_{2} \\ x_{3}
	\end{array}\right)
	=\left(\begin{array}{ccc}1 & 1 & 0 \\ 1 & -1 & 0 \\ 0 & 0 & 1\end{array}\right)
	\left(\begin{array}{l}y_{1} \\ y_{2} \\ y_{3}\end{array}\right)$,
%\end{center}

代入 $$f=2 x_{1} x_{2}+2 x_{1} x_{3}-6 x_{2} x_{3}$$

得 $$f=2 y_{1}^{2}-2 y_{2}^{2}-4 y_{1} y_{3}+8 y_{2} y_{3}.$$

\end{frame}

\begin{frame}
再配方,得
\[
f=2\left(y_{1}-y_{3}\right)^{2}-2\left(y_{2}-2 y_{3}\right)^{2}+6 y_{3}^{2} 
\]
令 
$\left\{\begin{array}{l}
z_{1}=y_{1}-y_{3}, \\
z_{2}=y_{2}-2 y_{3}, \\
z_{3}=y_{3},
\end{array}\right. $

则 
$\left\{ \begin{array}{l}
y_{1}=z_{1}+z_{3}, \\
y_{2}=z_{2}+2 z_{3}, \\
y_{3}=z_{3},
\end{array}\right.$
即
$\left(\begin{array}{l}
y_{1} \\
y_{2} \\
y_{3}
\end{array}\right)=\left(\begin{array}{rrr}
1 & 0 & 1 \\
0 & 1 & 2 \\
0 & 0 & 1
\end{array}\right)\left(\begin{array}{l}
z_{1} \\
z_{2} \\
z_{3}
\end{array}\right)$

得 $$ f=2 z_{1}^{2}-2 z_{2}^{2}+6 z_{3}^{2}.$$


所用变换矩阵为
\[
C=\left(\begin{array}{ccc}
1 & 1 & 0 \\
1 & -1 & 0 \\
0 & 0 & 1
\end{array}\right)\left(\begin{array}{ccc}
1 & 0 & 1 \\
0 & 1 & 2 \\
0 & 0 & 1
\end{array}\right)=\left(\begin{array}{ccc}
1 & 1 & 3 \\
1 & -1 & -1 \\
0 & 0 & 1
\end{array}\right).
\]
\end{frame}


\begin{frame}{合同变换法}
\begin{enumerate}
\item 写出二次型 $f$ 的矩阵 $A$,并构造 $2 n \times n$ 矩阵 $\left(\begin{array}{c}A \\ E\end{array}\right)$.

\item  对 $A$ 进行初等行变换和同样的初等列变换,把 $A$ 化为对角矩 阵
$D$,并对 $E$ 施行与 $A$ 相同的初等列变换化为矩阵 $C,$ 此时 $C A C=D$.

\item  写出非退化线性替换 $X=C Y$ 化二次型为标准形 $f=Y^{\, \prime} D Y$
这个方法可示意如下:
\[
\left(\begin{array}{c}
A \\
E
\end{array}\right) 
\xrightarrow[\text{ 对 } E \text{ 只进行其中的初等列变换 }]{\text{ 对 } A \text{ 进行同样 的初等行变换和初 等列变换 }\quad }
\left(\begin{array}{c}
D \\
C
\end{array}\right)
\]
\end{enumerate}
\end{frame}


\begin{frame}
\sol 用合同变换法 化二次型
$
f=2 x_{1} x_{2}+2 x_{1} x_{3}-6 x_{2} x_{3}
$
成标准形.
\begin{align*}
\left(\begin{array}{c}
A \\
E
\end{array}\right) 
= & 
\left(\begin{array}{ccc}0 & 1 & 1 \\ 1 & 0 & -3 \\ 1 & -3 & 0 \\ 1 & 0 & 0 \\ 0 & 1 & 0 \\ 0 & 0 & 1\end{array}\right) 
\rightarrow
\left(\begin{array}{ccc}2 & 1 & -2 \\ 1 & 0 & -3 \\ -2 & -3 & 0 \\ 1 & 0 & 0 \\ 1 & 1 & 0 \\ 0 & 0 & 1\end{array}\right)\\
\rightarrow & 
\left(\begin{array}{ccc}2 & 0 & 0 \\ 0 & -\frac{1}{2} & -2 \\ 0 & -2 & -2 \\ 1 & -\frac{1}{2} & 1 \\ 1 & \frac{1}{2} & 1 \\ 0 & 0 & 1\end{array}\right)
\rightarrow
\left(\begin{array}{ccc}2 & 0 & 0 \\ 0 & -\frac{1}{2} & 0 \\ 0 & 0 & 6 \\ 1 & -\frac{1}{2} & 3 \\ 1 & \frac{1}{2} & -1 \\ 0 & 0 & 1\end{array}\right)=\left(\begin{array}{c}D \\ C\end{array}\right)
\end{align*}

则$C^{\, \prime} AC = D$.
\end{frame}

\begin{frame}{\S 3 唯一性}
标准形中的系数不是唯一确定的. 例如:对二次型
\[
2 x_{1} x_{2}-6 x_{2} x_{3}+2 x_{1} x_{3}
\]
做线性替换
\[
\left(\begin{array}{l}
x_{1} \\
x_{2} \\
x_{3}
\end{array}\right)=\left(\begin{array}{ccc}
1 & 1 & 3 \\
1 & -1 & -1 \\
0 & 0 & 1
\end{array}\right)\left(\begin{array}{l}
w_{1} \\
w_{2} \\
w_{3}
\end{array}\right)
\]
得到标准形
\[
2 w_{1}^{2}-2 w_{2}^{2}+6 w_{3}^{2}
\]
\end{frame}

\begin{frame}
进一步做替换
\[
\left(\begin{array}{l}
w_{1} \\
w_{2} \\
w_{3}
\end{array}\right)=\left(\begin{array}{lll}
1 & 0 & 0 \\
0 & \frac{1}{2} & 0 \\
0 & 0 & \frac{1}{3}
\end{array}\right)\left(\begin{array}{l}
y_{1} \\
y_{2} \\
y_{3}
\end{array}\right)
\]
得到另一个标准形
\[
2 y_{1}^{2}-\frac{1}{2} y_{2}^{2}+\frac{2}{3} y_{3}^{2}
\]
合同不改变矩阵的秩. 


共同点:标准形中系数不为零的平方项的个数是唯一确定的. 
\end{frame}

\begin{frame}{复数域上的  二次型}
\small{
\begin{thm}
	任意一个秩为 $r$ 的复系数的 $n$ 元二次型,可经过适当的非退化线性
	替换化为复规范型:
	\[
	z_{1}^{2}+z_{2}^{2}+\cdots+z_{r}^{2}
	\]
	而且这个规范型是唯一的.
\end{thm}
\begin{coro}
\begin{itemize}
	\item 任意一个复对称矩阵 $A$ 都合同于对角矩阵:
	\[
	\left(\begin{array}{cc}
	E_r&\\
	& O\\
	\end{array}\right)
	\]
	其中对角线上 1 的个数 $r$ 等于矩阵 $A$ 的秩.
	
	\item 两个复对称矩阵合同$\Leftrightarrow$它们的秩相等.
\end{itemize}
\end{coro}


}
\end{frame}

\begin{frame}{ 实数域上的 二次型}
\begin{thm}
任意一个秩为 $r$ 的实系数的 $n$ 元二次型,可经过适当的非退化线性
替换化为实规范型:
\[
z_{1}^{2}+z_{2}^{2}+\cdots+z_{p}^{2}-z_{p+1}^{2}-\cdots-z_{r}^{2}
\]
而且这个规范型是唯一的.
\end{thm}
\begin{defi}
	实二次型 $f$ 的规范型中,
	\begin{itemize}
		\item 正平方项的个数 $p$ 称为 $f$ 的\alert{正惯性指数};
		\item 负平方项的个数 $r-p$ 称为 $f$ 的\alert{负惯性指数};
		\item 它们的差 $p-(r-p)$ 称为$f$ 的\alert{符号差}.
	\end{itemize}
\end{defi}
\end{frame}


\begin{frame}
\begin{exa}
对标准二次型
\[
2 w_{1}^{2}-2 w_{2}^{2}+6 w_{3}^{2}, 
\]
\begin{itemize}
\item 令$z_1=\sqrt{2}w_1, \, z_2=\sqrt{2}w_2 i, \,  z_3=\sqrt{6}w_3 i$,

得到复数域上的规范形 $$z_1^2+z_2^2+z_3^2;$$

\item 令$y_1=\sqrt{2}w_1, \,  y_2=\sqrt{6}w_3, \,   y_3=\sqrt{2}w_2$,

得到实数域上的规范形 $$y_1^2+y_2^2-y_3^2.$$
\end{itemize}



\end{exa}

\end{frame}
\begin{frame}
\begin{coro}
任意一个实对称矩阵 $A$ 都合同于对角矩阵:
\[
\left(\begin{array}{ccc}
E_p & & \\
& E_{r-p} & \\
& & O
\end{array}\right)
\]
其中对角线上 1 和 -1 的个数是唯一确定的,且其和 $r$ 等于矩阵 $A$ 的秩. 
\end{coro}

问题:试给出两个实对称矩阵合同的充要条件. 
\pause
\begin{coro}
两个
实对称阵合同$\Leftrightarrow$它们的秩相等且正惯性指数也相等.
\end{coro}
\end{frame}


\section{正定二次型}
\begin{frame}{\S 4 正定二次型}
%正定二次型的定义和判定
\begin{defi}
实二次型 $f\left(x_{1}, x_{2}, \cdots, x_{n}\right)$ 是正定的,如果对任意一组不全为零的
的实数 $c_{1}, c_{2}, \cdots, c_{n}$ 都有 $f\left(c_{1}, c_{2}, \cdots, c_{n}\right)>0$
\end{defi}

\begin{thm}
实二次型 $f\left(x_{1}, x_{2}, \cdots, x_{n}\right)=d_{1} x_{1}^{2}+d_{2} x_{2}^{2}+\cdots+d_{n} x_{n}^{2}$ 是正定二次型
的充要条件是 $d_{i}>0, i=1,2, \cdots, n$
\end{thm}

非退化的线性替换不改变二次型的正定性.
\begin{thm}
 ${n}$ 元实二次型 $f\left(x_{1}, x_{2}, \cdots, x_{n}\right)$ 正定的充要条件是它的正惯性指数为 $n$.
\end{thm}
\end{frame}

\begin{frame}{正定矩阵}
\begin{defi}
如果实二次型 $f\left(x_{1}, x_{2}, \cdots, x_{n}\right)=X^{\, \prime} A X$ 是正定的,则称实对称矩阵
$A$ 为正定矩阵.
\end{defi}
\begin{thm}
实对称矩阵 $A$ 正定 $\Leftrightarrow$ 它与单位矩阵合同. 

实对称矩阵 $A$ 正定 $\Leftrightarrow$  存在非奇异矩阵 ${C},$ 使得 $A=C^{\, \prime} C$
\end{thm}



\begin{coro}
正定矩阵的行列式大于零. 
\end{coro}

正定矩阵是可逆的,且其逆矩阵仍为正定矩阵. 
\end{frame}

\begin{frame}
直接利用矩阵的元素来判断它的正定性. 

\begin{defi}
$n$ 阶实对称矩阵 ${A}=\left({a}_{i j}\right)$ 的左上角的 ${k}$ 阶子式
\[
\left|\begin{array}{cccc}
a_{11} & a_{12} & \cdots & a_{1 k} \\
a_{21} & a_{22} & \cdots & a_{2 k} \\
\vdots & \vdots &  & \vdots \\
a_{k 1} & a_{k 2} & \cdots & a_{k k}
\end{array}\right|, \quad k=1,2, \cdots, n
\]
称为矩阵 $A$ 的 $k$ 阶顺序主子式. 
\end{defi}
\begin{thm}
实二次型 $f(X)=X^{\, \prime} A X$ 正定
$\Leftrightarrow$
矩阵 $A$ 的各阶顺序主子式全大于零. 
\end{thm}
\end{frame}

\begin{frame}
\begin{theo}
	设实二次型 $f\left(x_{1}, x_{2}, \cdots, x_{n}\right)=X^{\,\prime} A X,$ 下列命题等价:
	\begin{enumerate}
		\item $n$ 元实二次型 $f\left(x_{1}, x_{2}, \cdots, x_{n}\right)=X^{\, \prime} A X$ 是正定的
		\item 它的正惯性指数为 $n$
		\item $A$ 与单位矩阵 $E$ 合同
		\item 存在 n 级实可逆矩阵 $C$
		使 $A=C^{\,\prime} C$.
		\item 矩阵 $A$ 的各阶顺序主子式全大于零. 
	\end{enumerate}
\end{theo}
\end{frame}

\begin{frame}
\begin{exa}
判别二次型 $f\left(x_{1}, x_{2}, x_{3}\right)=5 x_{1}^{2}+x_{2}^{2}+5 x_{3}^{2}+4 x_{1} x_{2}-8 x_{1} x_{3}-4 x_{2} x_{3}$
是否正定。
\end{exa}
\sol 
$f\left(x_{1}, x_{2}, x_{3}\right)$ { 的矩阵为 } 
\[
\begin{array}{l}

\qquad\left(\begin{array}{rrr}
5 & 2 & -4 \\
2 & 1 & -2 \\
-4 & -2 & 5
\end{array}\right)
\end{array}
\]
它的顺序主子式
\[
5>0,\left|\begin{array}{cc}
5 & 2 \\
2 & 1
\end{array}\right|>0,\left|\begin{array}{rrr}
5 & 2 & -4 \\
2 & 1 & -2 \\
-4 & -2 & 5
\end{array}\right|>0
\]
因之, $f\left(x_{1}, x_{2}, x_{3}\right)$ 正定.
\end{frame}


\begin{frame}
\begin{exa}
当$t$ 为何值时, 下面的二次型是正定的? $$f\left(x_{1}, x_{2}, x_{3}\right)=x_{1}^{2}+4 x_{2}^{2}+4 x_{3}^{2}+2 t x_{1} x_{2}-2 x_{1} x_{3}+4 x_{2} x_{3}$$

\end{exa}
%\pf 分析 $\quad$ 应填 : - $2<t<1$ 
\sol
二次型的矩阵为 
$A=\left(\begin{array}{rrr}1 & t & -1 \\ t & 4 & 2 \\ -1 & 2 & 4\end{array}\right)$

要使 $f$ 为正定二次型, 其充分必要
条件是 $A$的各阶顺序主子式均大于零,
\begin{align*}
& \Delta_{1}=1>0, \Delta_{2}=\left|\begin{array}{cc}
1 & t \\
t & 4
\end{array}\right|=4-t>0, \\
&
\Delta_{3}=\left|\begin{array}{ccc}
1 & t & -1 \\
t & 4 & 2 \\
-1 & 2 & 4
\end{array}\right| =- 4( t + 2) ( t - 1) > 0
\end{align*}


%由 $\Delta_{2}>0$ 解得 $-2<t<2 ;$ 由 $\Delta_{3}>0$ 解得 $-2<t<1.$ 

 解得 当 $-2<t<1$ 时
二次型 $f$ 正定 .
\end{frame}

\begin{frame}{二次型的分类}
\begin{defi}
设实二次型 $f\left(x_{1}, x_{2}, \cdots, x_{n}\right),$ 若对于任意一组不全为零的实数
$c_{1}, c_{2}, \cdots, c_{n}$ 都有
\begin{enumerate}
\item $f\left(c_{1}, c_{2}, \cdots, c_{n}\right)>0,$ 则称 $f\left(c_{1}, c_{2}, \cdots, c_{n}\right)$ 是正定的。
\item $f\left(c_{1}, c_{2}, \cdots, c_{n}\right) \geq 0,$ 则称 $f\left(c_{1}, c_{2}, \cdots, c_{n}\right)$ 是半正定的。
\item $f\left(c_{1}, c_{2}, \cdots, c_{n}\right)<0,$ 则称 $f\left(c_{1}, c_{2}, \cdots, c_{n}\right)$ 是负定的。
\item $f\left(c_{1}, c_{2}, \cdots, c_{n}\right) \leq 0,$ 则称 $f\left(c_{1}, c_{2}, \cdots, c_{n}\right)$ 是半负定的。
\item $f\left(c_{1}, c_{2}, \cdots, c_{n}\right)$ 不确定,则称 $f\left(c_{1}, c_{2}, \cdots, c_{n}\right)$ 是不定的。
\end{enumerate}
\end{defi}
\end{frame}

\begin{frame}
\begin{thm}
设实二次型 $f\left(x_{1}, x_{2}, \cdots, x_{n}\right)=X^{\,\prime} A X,$ 下列命题等价:
\begin{enumerate}
	\item $f\left(x_{1}, x_{2}, \cdots, x_{n}\right)=X^{\, \prime} A X$ 是半正定的
	\item 它的负惯性指数与秩相等,
	\item 有可逆实矩阵 $C,$ 使得
\[
C^{\, \prime} A C=\left(\begin{array}{cccc}
d_{1} & & & \\
& d_{2} & & \\
& & \ddots & \\
& & & d_{n}
\end{array}\right), d_{i} \geq 0, i=1,2, \cdots, n
\]
	\item 有实矩阵 $C$,使得 $A=C^{\, \prime} C$
 	\item 矩阵$A$ 的所有主子式大于或等于零.
\end{enumerate}
\end{thm}
\end{frame}


\setcounter{exa}{0}
\begin{frame}
\begin{exa}
设矩阵 $A, B$ 满足 $A B A^{*}=2 A B-8 E,$ 其中 $A=\left(\begin{array}{ccc}1 & 0 & 0 \\ 0 & -2 & 0 \\ 0 & 0 & 1\end{array}\right),$ 求 $B$
\end{exa}

\begin{exa}
设矩阵 $A, B$ 满足 $A^{*} B A=2 B A-8 E,$ 其中 $A=\left(\begin{array}{ccc}1 & 0 & 0 \\ 0 & -2 & 0 \\ 0 & 0 & 1\end{array}\right),$ 求 $B$
\end{exa}
\end{frame}


\begin{frame}
\begin{exa}
求非退化线性替换,化二次型 $f\left(x_{1}, x_{2}, x_{3}\right)=x_{1}^{2}-3 x_{2}^{2}-2 x_{1} x_{2}+2 x_{1} x_{3}-6 x_{2} x_{3}$ 为标准形.
\end{exa}
\begin{exa}
求非退化线性替换,化二次型 $f\left(x_{1}, x_{2}, x_{3}\right)=4 x_{1} x_{2}-2 x_{1} x_{3}-2 x_{2} x_{3}$ 为标准形.
\end{exa}

\begin{exa}
若实二次型 $f\left(x_{1}, x_{2}, x_{3}\right)=2 x_{1}^{2}+x_{2}^{2}+x_{3}^{2}-2 t x_{1} x_{2}+2 t x_{1} x_{3}$ 正定,求 $t$ 的取值范围.
\end{exa}
\end{frame}

\begin{frame}
\begin{exa}
设 $A$ 是 $n$ 阶方阵,证明存在一个 $n$ 阶非零方阵 $B$ 使得 $A B=0$ 的充要条件是 $|A|=0$.
\end{exa}

\begin{exa}
设 $A$ 是 $n$ 阶实方阵,证明 $r\left(A A^{T}\right)=r(A)$.
\end{exa}
\end{frame}
\end{document} 