% !Mode:: "TeX:UTF-8"
\documentclass{ctexbeamer}


\usepackage{amsmath,amssymb,amsthm}             % AMS Math
\usepackage{graphicx}
\usepackage{epstopdf}
\usepackage{tikz}
\linespread{1.3}
\usepackage{amsfonts}


\usepackage{mathrsfs}  %花写字母


%%%=== theme ===%%%
% \usetheme{Warsaw}
%\usetheme{Copenhagen}
%\usetheme{Singapore}
\usetheme{Madrid}
%\usefonttheme{professionalfonts}
%\usefonttheme{serif}
% \usefonttheme{structureitalicserif}
%%\useinnertheme{rounded}
%%\useinnertheme{inmargin}
\useinnertheme{circles}
%\useoutertheme{miniframes}
\setbeamertemplate{navigation symbols}{}
%\setbeamertemplate{footline}[page number]
\setbeamertemplate{footline}[frame number]


\titlegraphic{\includegraphics[width=2cm]{tjnu.jpg}}


%\usepackage[fontset=mac]{ctex}
%\usepackage{ctex}
\usepackage{xcolor}
\newcommand{\red}[1]{\textcolor{red}{#1}}
\newcommand{\blue}[1]{\textcolor{blue}{#1}}
\newcommand{\green}[1]{\textcolor{green}{#1}}



\setbeamertemplate{theorems}[numbered]
\newtheorem{thm}{定理}
\newtheorem{lem}{引理}
\newtheorem{exa}{例}
\newtheorem*{theo}{定理}
\newtheorem*{conj}{猜想}
\newtheorem*{defi}{定义}
\newtheorem*{coro}{推论}
\newtheorem*{ex}{练习}
\newtheorem*{rem}{注}
\newtheorem*{prop}{性质}
\newtheorem*{qst}{问题}

\def\qed{\nopagebreak\hfill{\rule{4pt}{7pt}}\medbreak}
\def\pf{{\bf 证明~~ }}
\def\sol{{\bf 解~~ }}



\def\R{\mathbb{R}}
\def\Rn{\mathbb{R}^n}
\def\A{\mathscr{A}}
\def\B{\mathscr{B}}
\def\D{\mathscr{D}}
\def\E{\mathscr{E}}
\def\O{\mathscr{O}}

\def\rank{\operatorname{rank}}
\def\dim{\operatorname{dim}}
\def\0{\mathbf{0}}
\def\a{\alpha}
\def\b{\beta}
\def\r{\gamma}


\begin{document}



\title[]{论文写作指导 --- 数学文章的词句}
\author[]{{\large 张彪} }
\institute[]{{\normalsize
		天津师范大学\\[6pt]
		zhang@tjnu.edu.cn}}

\date{}



\AtBeginSection[]
{
\begin{frame}
	\frametitle{Outline}
	\tableofcontents[currentsection]
\end{frame}
}



\begin{frame}
\maketitle
\end{frame}


\begin{frame}
	\tableofcontents
\end{frame}



\section{证明}



\begin{frame}{证明的常用句型}
	下面是一些可以用的各种各样的短语例子。
	\begin{itemize}
		\item The aim/idea is to
		\item Our first goal is to show that
		\item Now for the harder part.
		\item The trick of the proof is to find
		\item \dots ~is the key relation.
		\item The only, but crucial use of  \dots~ is that
		\item To obtain \dots ~a little manipulation is needed.
		\item THe essential observation is that
	\end{itemize}
\end{frame}


\begin{frame}{证明}
	当你省略一个证明的一部分是,最好是通过短语表示省略地方的性质和长度,有如下内容:
	\begin{itemize}
		\item It is easy/simple/straightforward to show that
		\item Some tedious manipulation yields
		\item An easy/obvious induction gives
		\item After two applications of  \dots ~we find
		\item An argument similar to the one used in  \dots ~shows that
	\end{itemize}
\end{frame}

\begin{frame}{证明}
	你也应该努力让读者了解你证明到了那里以及还剩下什么需要证明。有用的短语有
	\begin{itemize}
		\item First, we establish that
		\item Our task is now to
		\item Our problem reduces to
		\item It remains to show that
		\item We are almost ready to invoke
		\item Finally, we have to show that
	\end{itemize}
\end{frame}

\begin{frame}{证明}
	\begin{itemize}
		\item
		一个证明的结束通常用哈尔莫斯符号$\square$标记的。
		\item
		有时用缩写QED  (拉丁语: quod erat demonstrandum = 这就是被证明了)代替。
	\end{itemize}
\end{frame}


\section{优劣比较}

\begin{frame}{符号放置}
	避免以数学表达式开始一个句子,特别是如果前面的句子以数学表达式结束,否则读者难以分析句子。

	例如, ``$A$ is  an ill-conditioned matrix''
	(可能与单词``A''混淆)
	可改为\pause

	 \qquad 	 \ ``The matrix $A$ is   ill-conditioned''.
\vspace{10pt}

如果可能的话,为了同样的理由,用标点符号或者文字将数学符号隔开。

	\begin{itemize}
\item 差: If $x>1 \, f(x)<0$.\pause

\item 中: If $x>1, f(x)<0$. \pause

\item  好: If $x>1$ then $f(x)<0$.
	\end{itemize}

\vspace{10pt}

	\begin{itemize}
	\item
差: since $p^{-1}+q^{-1}=1,\|\cdot\|_{p}$ and $\|\cdot\|_{q}$ are dual norms. \pause

\item  好: since $p^{-1}+q^{-1}=1,$ the norms $\|\cdot\|_{p}$ and $\|\cdot\|_{q}$ are dual.
	\end{itemize}
\vspace{10pt}
\end{frame}

\begin{frame}
	\begin{itemize}
	\item
差: It suffices to show that $\|H\|_{p}=n^{1 / p}, 1 \leqslant p \leqslant 2$. \pause
	\item

\item
好: It suffices to show that $\|H\|_{p}=n^{1 / p}(1 \leqslant p \leqslant 2)$.

好: It suffices to show that $\|H\|_{p}=n^{1 / p}$ for $1 \leqslant p \leqslant 2$.
	\end{itemize}
\vspace{10pt}

\begin{itemize}
	\item
差: For $n=r$  (2.2) holds with $\delta_{r}=0$. \pause
\item
好: For $n=r$, (2.2) holds with $\delta_{r}=0 .$
\item
好: For $n=r,$ inequality (2.2) holds with $\delta_{r}=0$.
	\end{itemize}

\end{frame}
\section{{冠词}的使用}
\begin{frame}{``The''或``A''}

    \begin{itemize}
        \item  当一个\blue{普通名词}以\red{复数形式}第一次出现时,它的前面不加the,除非它带有一限定性形容词,比如writing for sciences 和 writing for the mathematical sciences.
        \item 当一个\blue{普通名词}以\red{单数形式}出现时,除了一些抽象名词外,前面要加a或the。
        比如   Let $M$ be a matroid on a ground set $E$.

    \end{itemize}
\end{frame}


\begin{frame}{冠词}
冠词的使用规则是 复杂的.


{Swan} 解释了他认为两种最重要的规则:

\begin{itemize}
    \item \blue{通常不要使用 $t h e$ (复数或不可数名词) 描述事情. }

    例: ``Mathematics is interesting''

    (不是 ``The mathematics is interesting'');

    ``Indefinite integrals do not always have closed form solutions''

    (不是 ``The indefinite integrals do not always have the closed form solutions'' ).

    \item \blue{没有冠词就不要使用单数可数名词.}

    例: ``the derivative is'' ``a derivative is'' , 而不是 ``derivative is''.
\end{itemize}
\end{frame}

\begin{frame}{a/the}
    By using \alert{a}  computer algebra system,  Gao,   Lu,  Xie,  Yang, and  Zhang~[15] proved that the polynomial  $P_{U_{m,d}}(t)$ has only negative zeros for $2\leq m\leq 15$.
    It is worth mentioning {that \alert{the} computer algebra system} is also used  to  prove  real-rootedness of  other combinatorial polynomials, see~[10].
\end{frame}


\begin{frame}
\begin{itemize}
	\item 在特定环境下冠词是可选择的.

``A matrix with the property (3.2) is well conditioned''
和

 ``A matrix with property (3.2) is well conditioned''
都是正确的.

 \item 使用冠词的错误是令人反感的, 但是它们通常不会模糊句子的含义.
\end{itemize}

\end{frame}


\begin{frame}{``The''或``A''}
在数学写作中,当宾语指代(可能)\blue{不是独一无二的}或者\blue{不存在的事物}时,使用冠词``the''是不恰当地。
改写句子,或者将冠词改为``a'', 通常能解决问题。
%\vspace{10pt}
%  Let \alert{a / the} Schur decomposition of $A$ be $Q T Q^{*}$.
{

\begin{itemize}
\item 差: Let the Schur decomposition of $A$ be $Q T Q^{*}$.
\item 好: Let a Schur decomposition of $A$ be $Q T Q^{*}$.
	\end{itemize}
}

\pause
\red{
Although every square matrix has a Schur decomposition, in general this decomposition is not unique.  Similarly, }


\begin{itemize}
\item 差: Under what conditions does the iteration converge to the solution of $f(x)=$
$0 ?$  \pause

\item  好: Under what conditions does the iteration converge to a solution of $f(x)=0 ?$
\end{itemize}
\end{frame}

\begin{frame}{翻译}
\begin{itemize}
\item
    令$a$为常数。不失一般性我们可把这个常数取成非零。
\pause

Let $a$ be \alert{a} constant. Without loss of generality we may take \alert{the} constant to be nonzero.
\item
    令$A$为满秩方阵并记$A^{-1}$为$A$的逆。

    \pause
Let $A$ be \alert{a} square matrix with full rank and let $A^{-1}$ be \alert{the} inverse of $A$.
\end{itemize}
\end{frame}


\end{document}