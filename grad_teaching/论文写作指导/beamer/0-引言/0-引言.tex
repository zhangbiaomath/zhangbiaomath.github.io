\documentclass[13pt]{ctexbeamer}


\usepackage{amsmath,amssymb,amsthm}             % AMS Math
\usepackage{graphicx}
\usepackage{epstopdf}
\usepackage{tikz}
\linespread{1.3}

\usepackage{mathrsfs}  %花写字母
 

%%%=== theme ===%%%
%\usetheme{Warsaw}
%\usetheme{Copenhagen}
%\usetheme{Singapore}
\usetheme{Madrid}
%\usefonttheme{professionalfonts}
%\usefonttheme{serif}
% \usefonttheme{structureitalicserif}
%%\useinnertheme{rounded}
%%\useinnertheme{inmargin}
\useinnertheme{circles}
%\useoutertheme{miniframes}
\setbeamertemplate{navigation symbols}{}
%\setbeamertemplate{footline}[page number]
\setbeamertemplate{footline}[frame number] 


\titlegraphic{\includegraphics[width=2cm]{tjnu.jpg}} 

\setbeamertemplate{theorems}[numbered]
\newtheorem{thm}{定理}
\newtheorem{lem}{引理}
\newtheorem{exa}{例}
\newtheorem*{theo}{定理}
\newtheorem*{conj}{猜想}
\newtheorem*{defi}{定义}
\newtheorem*{coro}{推论}
\newtheorem*{ex}{练习}
\newtheorem*{rem}{注}
\newtheorem*{prop}{性质}
\newtheorem*{qst}{问题}

\def\qed{\nopagebreak\hfill{\rule{4pt}{7pt}}\medbreak}
\def\pf{{\bf 证明~~ }}
\def\sol{{\bf 解~~ }}



\def\R{\mathbb{R}}
\def\Rn{\mathbb{R}^n}
\def\A{\mathscr{A}}
\def\B{\mathscr{B}}
\def\D{\mathscr{D}}
\def\E{\mathscr{E}}
\def\O{\mathscr{O}}

\def\rank{\operatorname{rank}}
\def\dim{\operatorname{dim}}
\def\0{\mathbf{0}}
\def\a{\alpha}
\def\b{\beta}
\def\r{\gamma}

\usepackage{color}
\definecolor{linkcol}{rgb}{0,0,0.4}
\definecolor{citecol}{rgb}{0.5,0,0}

\definecolor{blue}{rgb}{0,0.08,1}
\newcommand{\blue}{\textcolor{blue}}

  \usepackage{graphicx}
  \DeclareGraphicsExtensions{.eps}
%   \usepackage[a4paper,pagebackref,hyperindex=true,pdfnewwindow=true]{hyperref}


\begin{document}



\title[]{论文写作指导 }
\author[]{{\large 张彪} }
\institute[]{{\normalsize
		天津师范大学\\[6pt]
		zhang@tjnu.edu.cn}}

\date{}


%
%\AtBeginSection[]
%{
%\begin{frame}
%	\frametitle{Outline}
%	\tableofcontents[currentsection]
%\end{frame}
%\setcounter{exa}{0}
%\setcounter{equation}{0}
%}



\begin{frame}
\maketitle
\end{frame}



\begin{frame}{参考书目}

	\begin{itemize}
	\item 汤涛,丁玖,数学之英文写作,高等教育出版社,2019. 

	\item  数学论文写作(原书第二版), [英] 尼古拉斯,J.海厄姆 著,贾志刚,常亮,李建波 译, 科学出版社, 2016. 

	\item \alert{数学论文英文写作实用模板(汉英对照)}, [波兰] 耶日 泰锡斯克(Jerzy Trzeciak) 著,张文彪,刘晓敏 译, 机械工业出版社, 2016. 

	\item LaTeX入门, 刘海洋 著, 电子工业出版社, 2013.
	\end{itemize}

	文档密码:po21
\end{frame}

\begin{frame}
	通过
		\begin{itemize}
		\item  系统的训练,
		\item 用心去探索\alert{规律}
		\item 并吸取教训,反复提高
	\end{itemize}
研究者完全可以写出语言表达上乘的学术论文。


\vspace{15pt}

\begin{itemize}
	\item 在多阅读、多练习的基础上,
	\item 掌握\alert{技巧},
	\item 熟练一些\alert{典型句型}和\alert{结构}
\end{itemize}
是写出好的科技论文的第一步。
\end{frame}


\begin{frame}
	数学在语言表达上的特点
	\begin{itemize}
		\item 数学词汇的意义经久不衰,不为时代所动,
		\item 数学概念的定义严密准确,无懈可击,
		\item 数学定理的证明服从逻辑规律,以三段论推理为其宗旨,
		\item 数学写作的方式技巧,有章可循
	\end{itemize}
\vspace{15pt}
\pause
	数学写作的要求
\begin{itemize}
	\item 如何能让我们的写作体现数学之美?
	\item 如何能让我们的文章结构、遣词造句、思想流动、动机结论让人读之犹如行云流水?
	\item 如何能让表达之美和推导之美并驾齐驱、相辅相成?
\end{itemize}
\end{frame}




\begin{frame}
	写文章前的准备工作及写作中应该遵循的几项基本原则
	\begin{itemize}
		\item  写作目的是让别人清楚知道你想叙述和表达的东西。
		\item 要收集一些与写作有关的材料,特别要有几篇关键的文献。
		\item 文章要精确、清楚、简洁地表达你想说的东西。
		\item 初稿完成后,要反复修改,不要冀望一次完成。
		\item 平生第一篇英文文章写作时,应该找有英文写作经验的人修改。
	\end{itemize}
\end{frame}

\begin{frame}{基本原则}

\begin{itemize}
	\item  写作帮助学习。写作通过强迫你将思想集中于有可能忽略的每一步,从而提出理解上的间断。
	\item 优秀的写作反映清晰的思想。好的框架来源于清晰的思考。
	\item  写作时困难的。一个在某些程度上很有效的解决方法是强迫自己先写出来。
	\item 写作的一个最基本的原则是保持你的论文简单和直接。
	\item 改进写作的最好的方法可能是接受建设性的批评并从中学习。
	\item 另外一个改进写作的办法是带着批判的眼光去读,越多越好。
\end{itemize}
\end{frame}

\begin{frame}{Halmos's principle}
	\begin{itemize}
		\item  In order to say something well you must have something to say. 
		\item When you decide to write something, ask yourself who it is that you want to reach. (为谁而写)
	\end{itemize}

\vspace{8pt}
The basic problem in writing mathematics is the same as in writing biology, writing a novel, or wrting directions for assembling a harpsichord(大键琴): 
\begin{center}
the problem is \alert{to communicate an idea}.
\end{center}
To do so, and to do it clearly, 
\begin{itemize}
	\item you must have somthing to say, and 
	\item you must arrange it in the order that you want to it said in, 
	\item you must write it, rewrite it, and re-rewrite it  several times, and 
	\item you must be willing to think hard about and work hard on mechanical details such as diction, notation, and punctuation.
	\end{itemize}
That is' all there is to it.
\end{frame}
	
	


\end{document} 