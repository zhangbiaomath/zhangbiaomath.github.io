\documentclass[10pt]{ctexbeamer}


\usepackage{amsmath,amssymb,amsthm}             % AMS Math
\usepackage{graphicx}
\usepackage{epstopdf}
\usepackage{tikz}
\usetikzlibrary{graphs, positioning, quotes, shapes.geometric}
\usepackage{tipa}
\linespread{1.3}

\usepackage{mathrsfs}  %花写字母


%%%=== theme ===%%%
%\usetheme{Warsaw}
%\usetheme{Copenhagen}
%\usetheme{Singapore}
\usetheme{Madrid}
%\usefonttheme{professionalfonts}
%\usefonttheme{serif}
% \usefonttheme{structureitalicserif}
%%\useinnertheme{rounded}
%%\useinnertheme{inmargin}
\useinnertheme{circles}
%\useoutertheme{miniframes}
\setbeamertemplate{navigation symbols}{}
%\setbeamertemplate{footline}[page number]
\setbeamertemplate{footline}[frame number]


\titlegraphic{\includegraphics[width=2cm]{tjnu.jpg}}

\usepackage{color}
\definecolor{linkcol}{rgb}{0,0,0.4}
\definecolor{citecol}{rgb}{0.5,0,0}


\usepackage{xcolor}
\newcommand{\red}[1]{\textcolor{red}{#1}}
\newcommand{\blue}[1]{\textcolor{blue}{#1}}
\newcommand{\green}[1]{\textcolor{green}{#1}}


  \usepackage{graphicx}
%  \DeclareGraphicsExtensions{.eps}
%   \usepackage[a4paper,pagebackref,hyperindex=true,pdfnewwindow=true]{hyperref}


\begin{document}



\title[]{论文写作指导 }
\author[]{{\large 张彪} }
\institute[]{{\normalsize
		天津师范大学\\[6pt]
		zhang@tjnu.edu.cn}}

\date{}


%
%\AtBeginSection[]
%{
%\begin{frame}
%	\frametitle{Outline}
%	\tableofcontents[currentsection]
%\end{frame}
%\setcounter{exa}{0}
%\setcounter{equation}{0}
%}



\begin{frame}
\maketitle
\end{frame}

\setcounter{tocdepth}{1}

\begin{frame}
    \frametitle{\textcolor{orange}{Outline}}
    \tableofcontents
\end{frame}


\section{论文写作}
\begin{frame}


\red{参考书目}
	\begin{itemize}
	\item 汤涛,丁玖,数学之英文写作,高等教育出版社,2019.

	\item  数学论文写作(原书第二版), [英] 尼古拉斯,J.海厄姆 著,贾志刚,常亮,李建波 译, 科学出版社, 2016.

	\item \blue{数学论文英文写作实用模板(汉英对照)}, [波兰] 耶日 泰锡斯克(Jerzy Trzeciak) 著,张文彪,刘晓敏 译, 机械工业出版社, 2016.

	\item LaTeX入门, 刘海洋 著, 电子工业出版社, 2013.
	\end{itemize}

%	文档密码:po21

%课程网页:

%https://zhangbiaomath.github.io/grad_teaching/writing.html

%\begin{figure}[tb]
%	\centering
%%	\includegraphics[width=0.18\textwidth]{writing.png}
%	\caption{课程网页}
%\end{figure}

\end{frame}

\begin{frame}{课程内容}
		\begin{itemize}
		\item 学术伦理及学术道德

		\item 数学文章的结构

		\item 数学文章的词句

		\item 怎样修改论文

		\item Latex使用初步

	\end{itemize}

\end{frame}


\begin{frame}
	通过
		\begin{itemize}
		\item  系统的训练,
		\item 用心去探索\blue{规律}
		\item 并吸取教训,反复提高
	\end{itemize}
研究者完全可以写出语言表达上乘的学术论文。


\vspace{15pt}

\begin{itemize}
	\item 在多阅读、多练习的基础上,
	\item 掌握\blue{技巧},
	\item 熟练一些\blue{典型句型}和\blue{结构}
\end{itemize}
是写出好的科技论文的第一步。
\end{frame}


\begin{frame}
	数学在语言表达上的特点
	\begin{itemize}
		\item 数学词汇的意义经久不衰,不为时代所动,
		\item 数学概念的定义严密准确,无懈可击,
		\item 数学定理的证明服从逻辑规律,以三段论推理为其宗旨,
		\item 数学写作的方式技巧,有章可循
	\end{itemize}
\vspace{15pt}
\pause
	数学写作的要求
\begin{itemize}
	\item 如何能让我们的写作体现数学之美?
	\item 如何能让我们的文章结构、遣词造句、思想流动、动机结论让人读之犹如行云流水?
	\item 如何能让表达之美和推导之美并驾齐驱、相辅相成?
\end{itemize}
\end{frame}




\begin{frame}
	写文章前的准备工作及写作中应该遵循的几项基本原则
	\begin{itemize}
		\item  写作目的是让别人清楚知道你想叙述和表达的东西。
		\item 要收集一些与写作有关的材料,特别要有几篇关键的文献。
		\item 文章要精确、清楚、简洁地表达你想说的东西。
		\item 初稿完成后,要反复修改,不要冀望一次完成。
		\item 平生第一篇英文文章写作时,应该找有英文写作经验的人修改。
	\end{itemize}
\end{frame}

\begin{frame}{基本原则}

\begin{itemize}
	\item  写作帮助学习。写作通过强迫你将思想集中于有可能忽略的每一步,从而提出理解上的间断。
	\item 优秀的写作反映清晰的思想。好的框架来源于清晰的思考。
	\item  写作时困难的。一个在某些程度上很有效的解决方法是强迫自己先写出来。
	\item 写作的一个最基本的原则是保持你的论文简单和直接。
	\item 改进写作的最好的方法可能是接受建设性的批评并从中学习。
	\item 另外一个改进写作的办法是带着批判的眼光去读,越多越好。
\end{itemize}
\end{frame}

\begin{frame}{Halmos's principle}
\begin{itemize}
	\item  In order to say something well you must have something to say.
\end{itemize}

\pause
哈尔莫斯把
\begin{center}
“{要想说好某件事, 一定要有某事说}”
\end{center}

作为写作的 \blue{first principle(第一原则)}。
他宣称:

\begin{itemize}
    \item  Much bad writing, mathematical and otherwise, is caused by a violation of that first principle.
\end{itemize}

\pause

数学或其它方面的许多劣质写作都是因为违反这个第一原则而 造成的。



\end{frame}

\begin{frame}{Halmos's principle}

\begin{itemize}
\item When you decide to write something, ask yourself who it is that you want to reach.
\end{itemize}
\pause
 {\small
他把 “为谁而写”列为写作的\blue{第二原则}:
\begin{center}
当你决定写东西时, 问问自己预期中的读者是谁。
\end{center}


 比方说, 你是写只让自己 看的日记? 给远方朋友的信?还是给情人的浪漫情书? 给专家读的研 究报告? 还是大学生用的教科书? 你的写作方式、考虑重点、内容布 局、行文风格等都需根据读者作出考量。

因此哈尔莫斯总结道:}

\begin{itemize}
    \item
All writing is influenced by the audience, but, given the audience, the author's problem is to communicate with it as best he can ...
\end{itemize}
\pause

 {\small
所有写作都被读者所左右。但是, 当读者是既定的, 作者的课 题就是尽他所能与之交流 $\cdots$
}
\end{frame}



\begin{frame}{Halmos's principle}


    {\small  The basic problem in writing mathematics is the same as in writing biology, writing a novel, or wrting directions for assembling a harpsichord(大键琴):     the problem is  to communicate an idea.}

  \pause

{\small  写数学的基本问题和写生物、写小说或写键琴安装指南一样: 如何交流想法。}

\end{frame}

\begin{frame}{}


    {\small To do so, and to do it clearly,
        \begin{itemize}
            \item you must have something to say, and
            \item you must arrange it in the order that you want to it said in,
            \item you must write it, rewrite it, and re-rewrite it  several times, and
            \item you must be willing to think hard about and work hard on mechanical details such as diction, notation, and punctuation.
        \end{itemize}
        That is' all there is to it.}

    \pause

    {\small 为了这样做并做得好, 你必须有述说的内容, 你必须有说话的对象, 你必须组织好你想说的一切, 你必须按 照你想说的次序来安排它, 你必须写、重写, 并重复改写几次, 你必须愿意绞尽脑汁, 在措词、记号及标点符号等细节上猛下 功夫。这就是一切的一切。 }
\end{frame}



\section{LaTeX}

\begin{frame}{LaTeX}

    \begin{itemize}
        \item 	TeX
        ($\backslash$t\textepsilon x$\backslash$,常被读作$\backslash$t\textepsilon k$\backslash$,音译“泰赫”,“泰克”)

        \item 它在学术界特别是数学、物理学和计算机科学界十分流行。
        \item
        TeX被普遍认为是一个优秀的排版工具,尤其是对于复杂数学公式的处理。
        \item
        科研工作者使用文本编辑器(WinEdt, TexStudio等)编写tex源文件(.tex),  经过LaTeX进行编译,能够排版出精美的pdf文件。
    \end{itemize}

    \begin{figure}

        \begin{tikzpicture}

            \tikzstyle{arrow} = [thick,->,>=stealth]
            \node[diamond, minimum width=3cm,
            minimum height=1cm, text width=2cm, rounded corners, text centered, fill=blue!30]     (tex)  {.tex文件};
            \node[diamond, rounded corners, minimum width=3cm,
            minimum height=1cm, text width=2cm, fill=blue!30, text centered,  right=100pt of tex]     (pdf)  {.pdf文档};
            \draw [arrow] (tex) -- node[anchor=south] {LaTeX编译} (pdf);

        \end{tikzpicture}
    \end{figure}

\end{frame}




\begin{frame}{线上Latex:  Overleaf}
    网站
    \href{https://www.overleaf.com?r=4c8832b9&rm=d&rs=b
    }{https://www.overleaf.com/
    }


    \begin{itemize}
        \item 在线编辑

        \item 免安装,免配置

        \item 多人协作

        \item 版本管理
    \end{itemize}
\end{frame}



\begin{frame}{换行和新的段落}
    \begin{itemize}
        \item 一个空行或者多个连续空行代表着前后内容是两段,会产生缩进。


        \item  $\backslash\backslash$ 也可以表示换行,但只是换行,不会产生缩进。


        \item 一般是用插入空行来实现分段,为保证源文件的清晰

    \end{itemize}
\end{frame}

\begin{frame}{输入数学公式}

    欧拉函数常用于数论中.
    \begin{align}\label{eq1}
        \varphi(n)=n\left(1-\frac{1}{p_{1}}\right)\left(1-\frac{1}{p_{2}}\right) \cdots\left(1-\frac{1}{p_{q}}\right)
    \end{align}

    例如,若 $n=12=2^{2} \cdot 3$,则由\eqref{eq1}可得
    $$
    \varphi(12)=12\left(1-\frac{1}{2}\right)\left(1-\frac{1}{3}\right)=4
    $$


\end{frame}

\begin{frame}
    \begin{itemize}
        \item 插入图片
        \begin{center}
            \includegraphics[width=2cm]{tjnu.jpg}
        \end{center}

        \item 插入表格
        \begin{table}
            \centering
            \caption{顾客损失率计算参数值}
            \begin{tabular}{|l||c|c|c|}
                \hline
                姓名 & 语文 & 数学 & 外语  \\
                \hline  \hline
                张三 & 87 & 100 & 93 \\
                \hline
                李四 & 75 & 64 & 52  \\
                \hline
                王二 & 80 & 82 & 78  \\
                \hline
            \end{tabular}
        \end{table}
    \end{itemize}



\end{frame}




\begin{frame}{推荐使用TeXLive + TeXStudio }
    推荐安装\green{最新版本}的

    \begin{itemize}
        \item \blue{TeXLive}  ---\red{发行版}

        \href{https://www.tug.org/texlive/}{https://www.tug.org/texlive/}



        \item \blue{TeXStudio}    ---\red{编辑器}

        http://www.texstudio.org/
    \end{itemize}

    {TeXLive} + {TeXStudio}   ~安装指南,可以参考
    \begin{itemize}
        \item 	\href{https://zhuanlan.zhihu.com/p/80603542}{https://zhuanlan.zhihu.com/p/80603542}
        \item \href{https://blog.csdn.net/yeler082/article/details/80665186}{https://blog.csdn.net/yeler082/article/details/80665186}
    \end{itemize}
\end{frame}

\end{document}